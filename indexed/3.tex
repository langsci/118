\chapter[Grammatical classes]{Grammatical classes}
\hypertarget{RefHeading1210801525720847}{}
Moloko has the following grammatical classes, each described in the referenced sections or chapters below:

\begin{itemize}
\item nouns, which can be simple, compound, or derived from a verb (\chapref{chap:4})
\item verbs  (Chapters~\ref{chap:6}-- \ref{chap:9})
\item pronouns, both free and bound (as prefixes, suffixes, or clitics; \sectref{sec:3.1})
\item demonstratives and demonstrationals (\sectref{sec:3.2})
\item numerals and quantifiers (\sectref{sec:3.3})
\item existentials (\sectref{sec:3.4}), which are verb-like but pattern differently than verbs
\item adverbs (see \sectref{sec:3.5.1}), which can be simple or derived from nouns or verbs
\item ideophones (\sectref{sec:3.6}), which pattern as adverbs, adjectives, or in particular cases, as verbs
\item adpositions (\sectref{sec:5.6}), 
\item discourse markers, including the presupposition marker (see \chapref{chap:11} and \chapref{chap:12}), 
\item conjunctions and conjunctive adverbs (see \sectref{sec:12.3}),
\item interjections (see \sectref{sec:3.7})
\item the negative (\sectref{sec:10.2}), which can be simple or compounded with certain adverbs
\end{itemize}

Note the absence of adjectives as a word class, since all adjectives in Moloko are derived from nouns (\sectref{sec:5.3}).

In the following sections, a detailed treatment will be given for each of these word classes and the morphological structure of each class.  An operational definition will be given for each class, so that any word in the language can be readily classified.

The first line in the examples is written in the orthography. The second line is the phonetic form for slow speech with morpheme breaks. All consonantal and vowel allophones will be indicated for the sake of the non-speaker.  Palatisation and labialisation prosodies will be discernible from the quality of the vowels and the consonants. When an  underlying form (typically identified by / / brackets) is cited, only the consonants and the full vowels will be written (i.e. not the epenthetic schwas) and the palatalisation or labialisation prosody on the form will be marked by a superscripted ‘e’ or ‘o,’ respectively, after the morpheme. 

\section{Pronouns}\label{sec:3.1}
\hypertarget{RefHeading1210821525720847}{}
Pronouns stand in the place of a noun phrase in a clause.  Pronouns are deictic elements -- their reference changes according to the context of the utterance.  The role of the speaker furnishes the basic point of reference (first person). The addressee is defined with respect to the speaker (second person).  The third person pronouns refer to people or things being talked about by the first and second persons. There are definite and indefinite third person pronouns. Definite pronouns can be used anaphorically\is{Cohesion!Anaphoric referencing}, and their reference is determined by linguistic or pragmatic elements in the textual or extratextual environment. Indefinite pronouns have a non-identified referent. Other possible types of pronouns are not discussed in this work. 

Moloko personal pronouns and proforms are illustrated in \tabref{tab:3.15}. Moloko has one set of free personal pronouns (regular, see \sectref{sec:3.1.1.1}), one set of bound pronouns (possessive, see \sectref{sec:3.1.2}), and three sets of pronominals within the verb complex for subject, direct object, and indirect object (see \sectref{sec:7.3}). All personal pronouns and pronominals are shown in \tabref{tab:3.15}. The regular free pronouns can refer to any of the subject or direct object or indirect object. An emphatic subset of free pronouns exists, formed by adding the adjectiviser \textit{ga} to the regular personal pronouns. Possessive pronouns always occur within a noun phrase or a relative clause. Special vocative pronouns that attach to nouns are honorific (\sectref{sec:3.1.3}).  There are also interrogative pronouns (\sectref{sec:3.1.4}) and unspecified pronouns (\sectref{sec:3.1.5}). 

\clearpage
In some of the pronoun sets, there is an inclusive/exclusive distinction in the first person plural. There are no dual nor gender-specific forms, nor are there logophoric pronouns.\footnote{\citet{Frajzyngier1985} describes the types of logophoric systems found in some Chadic languages. No logophoric pronouns are described for Biu-Mandara. }  

\begin{table}
\resizebox{\textwidth}{!}{\begin{tabular}{llllp{1.75cm}p{2.25cm}p{2.5cm}} 
\lsptoprule
 & \multicolumn{2}{c}{{Free pronouns}} & \multicolumn{1}{c}{{Bound}} & \multicolumn{3}{c}{{Pronominal affixes and extensions}\footnote{Pronominals are discussed in \sectref{sec:7.3}.}}\\
\cmidrule(lr){2-3}\cmidrule(lr){4-4}\cmidrule(lr){5-7}
{Person} & {Regular} & {Emphatic} & {Possessive suffix} & {Subject pronominal affixes}\footnote{Note that the \oneP and {\twoP} bound pronominals consist of both a prefix and a suffix. They are further discussed in \sectref{sec:7.3}.} & {Dedicated direct object pronominals}\footnote{Note that although \textit{na} and \textit{ta} are free in that they are phonologically separate from the verb word, they are closely bound parts of the verb complex and so are called pronominal extensions, see \sectref{sec:7.3.2}.} & {Indirect object pronominal enclitic}\\\midrule
{\oneS} & \textit{ne} & \textit{ne ga} & \textit{=əwla} & \textit{n-} &  & \textit{=aw }\\
{\twoS} & \textit{nok} & \textit{nok ga} & \textit{=ango(k)}\footnote{This pronoun is pronounced either [\textit{aŋgʷɔ}] or [\textit{aŋgʷɔk\textsuperscript{w }}] by speakers from different regions.} & \textit{k-} &  & \textit{=ok}\\
\oldstylenums{3}\textsc{s} & \textit{ndahan} & \textit{ndahan ga} & \textit{=ahan} & \textit{a-} & \textit{na} & \textit{=an }\\
\oldstylenums{1}\textsc{Pin} & \textit{loko} & \textit{loko ga} & \textit{=aloko} & \textit{m/k-…-ok} &  & \textit{=aloko}\\
\oldstylenums{1}\textsc{Pex} & \textit{ləme} & \textit{ləme ga} & \textit{=aləme} & \textit{n-  . . . -om} &  & \textit{=aləme}\\
{\twoP} & \textit{ləkwəye} & \textit{ləkwəye ga} & \textit{=aləkwəye} & \textit{k- . . .  {}-om} &  & \textit{=aləkwəye}\\
\oldstylenums{3}\textsc{p} & \textit{təta} & \textit{təta ga} & \textit{=atəta} & \textit{ta- } & \textit{ta} & \textit{=ata}\\
\lspbottomrule
\end{tabular}}
\caption{\label{tab:3.15}Moloko personal pronouns and pro-forms}\is{Deixis!Pronouns and pro-forms}
\end{table}

\subsection{Free personal pronouns}\label{sec:3.1.1}
\hypertarget{RefHeading1210841525720847}{}
Free pronouns express subject, direct object, and indirect object. They are relatively rare in texts since participants are generally tracked by the bound verbal pronominals.  Free pronouns are found in cases of switch reference, at the peak\is{Focus and prominence!Discourse peak} of a story where the verbal pronominals disappear, or in cases of emphasis (see \sectref{sec:3.1.1.2}).  

\subsubsection{Regular pronouns}\label{sec:3.1.1.1}
\largerpage[2]
When free subject, direct object, or indirect object pronouns do occur, they are in the same place within a clause or noun phrase where one would expect the full noun phrase to be (see Sections \ref{sec:5.1} and \ref{sec:10.1}). 

The clause in \REF{ex:3:1} has subject (\textit{Mala}, a male proper name), direct object (\textit{dalay} ‘girl’), and indirect object (\textit{Arsakay}, another male proper name).  Note that the subject is also indicated on the verb by the subject pronominal \textit{à-} and the indirect object is indicated on the verb by the indirect object pronominal enclitic \textit{=an} (see \sectref{sec:7.3.2}). The noun phrase representing the indirect object is within a prepositional phrase (see \sectref{sec:5.6.1}). 

\ea \label{ex:3:1}
Mala  avəlan  dalay  ana  Arsakay.\\
\gll  Mala   à-vəl=aŋ     dalaj   ana   Arsakaj\\
      Mala   \oldstylenums{3}\textsc{s}+{\PFV}-give=\oldstylenums{3}\textsc{s}.{\IO}  girl  {\DAT} Arsakay\\
\glt ‘Mala gave the girl to Arsakay.’
\z

When the subject is replaced by a free pronoun \REF{ex:3:2}, the pronoun must be marked as presupposed in the clause (see \sectref{sec:11.2}). Note that since subject is pronominalised in the verb word a subject noun phrase is not required (see \sectref{sec:7.3.1}); the presence of any noun phrase or free pronoun is for pragmatic purposes.
\ea \label{ex:3:2}
\textbf{Ndahan} na,    avəlan  dalay  ana  Arsakay.\\
\gll  \textbf{ndahaŋ}  na    à-vəl=aŋ     dalaj   ana   Arsakaj\\
      \oldstylenums{3}\textsc{s}    {\PSP}  \oldstylenums{3}\textsc{s}+{\PFV}-give=\oldstylenums{3}\textsc{s}.{\IO}  girl  {\DAT} Arsakay\\
\glt ‘He [for his part], he gave the girl to Arsakay.’
\z

When the direct object is replaced by a free pronoun (compare \ref{ex:3:1} and \ref{ex:3:3}), the pronoun \textit{ndahan} (replacing \textit{dalay}) occurs in the normal direct object slot in the clause.\footnote{The dedicated direct object pronominal \textit{na} is can also replace a direct object noun phrase in the case of an inanimate object, \sectref{sec:7.3.2}.}  

\ea \label{ex:3:3}
Mala  avəlan  \textbf{ndahan} ana  Arsakay.\\
\gll  Mala à-vəl=aŋ     \textbf{ndahaŋ}   ana   Arsakaj\\
      Mala  \oldstylenums{3}\textsc{s}+{\PFV}-give=\oldstylenums{3}\textsc{s}.{\IO}    \oldstylenums{3}\textsc{s}  {\DAT} Arsakay\\
\glt ‘Mala gave her to Arsakay.’
\z

When the indirect object is replaced by a free pronoun, the pronoun occurs in a prepositional phrase \REF{ex:3:4}. The prepositional phrase is delimited by square brackets. Note that the indirect object pronominal enclitic =\textit{an} co-occurs on the verb complex (see \sectref{sec:7.3.1.1}).


\ea \label{ex:3:4}
Mala  avəl\textbf{an} dalay  [ana  \textbf{ndahan.}]\\
\gll  Mala   à-vəl\textbf{=aŋ}     dalaj   [ana   \textbf{ndahaŋ}]\\
      Mala  \oldstylenums{3}\textsc{s}+{\PFV}-give=\oldstylenums{3}\textsc{s}.{\IO}    girl  {\DAT} \oldstylenums{3}\textsc{s}\\
\glt  ‘Mala gave the girl to him.’
\z

The indirect object pronominal enclitic can entirely stand in the place of the prepositional phrase expressing indirect object with no loss in meaning (\ref)({ex:3:5}, see \sectref{sec:7.3.1.1}).

\ea \label{ex:3:5}
Mala  avəl\textbf{an} dalay.\\
\gll  Mala   à-vəl\textbf{=aŋ}     dalaj\\
       Mala  \oldstylenums{3}\textsc{s}+{\PFV}-give=\oldstylenums{3}\textsc{s}.{\IO}    girl\\
\glt  ‘Mala gave the girl to him.’
\z

\subsubsection{Emphatic pronouns}\label{sec:3.1.1.2}

Emphatic pronouns are formed by adding either the adjectiviser \textit{ga} (\sectref{sec:5.3}) or the third person singular possessive pronoun form \textit{=ahan} to the free pronoun (\ref{ex:3:6}--\ref{ex:3:8}). 

\ea \label{ex:3:6}
\textbf{Ne  ga}  nege.\\
\gll  \textbf{nɛ}  \textbf{ga}     n\`{ɛ}-g-ɛ\\
      {\oneS}  {\ADJ}    {\oneS}+{\PFV}-do-{\CL}\\
\glt  ‘It was me, I did it.’ (lit. me, I did)
\z

\ea \label{ex:3:7}
\textbf{Ne  ga}  aməgəye.\\ 
\gll  \textbf{nɛ}  \textbf{ga}     amɪ-g-ijɛ \\
      {\oneS}   {\ADJ}    {\DEP}-do-{\CL}\\
\glt  ‘It was me who did it.’ (lit. me, the one that did)
\z

\ea \label{ex:3:8}
\textbf{Ne  ahan}  nege.\\
\gll  \textbf{nɛ}  \textbf{=ahaŋ}     n\`{ɛ}-g-ɛ\\
      {\oneS}  =\oldstylenums{3}\textsc{s}.{\POSS}  {\oneS}+{\PFV}-do-{\CL}\\
\glt  ‘It was me, I did it.’ (lit. me, I did)
\z

\subsection{Possessive pronouns}\label{sec:3.1.2}\is{Clitics!Possessive pronoun}
\hypertarget{RefHeading1210861525720847}{}
\largerpage[2]
Another set of Moloko pronouns occurs only within noun phrases and among its primary uses, indicates a possessive relationship, i.e. these pronouns relate the possessor referent to the person or thing that is possessed. Possessive pronouns immediately follow the noun or noun phrase they modify (\ref{ex:3:9}--\ref{ex:3:11}) and occur before the plural \REF{ex:3:12}.\footnote{\citet{Bow1997c} postulated that the set of possessive pronouns did not include the plural possessive pronouns. Rather, she proposed that the plural possessive was actually an associative noun phrase formed by the preposition /a/ and the free pronoun (\textit{a} \textit{loko, a ləme, a ləkwəye}, and \textit{a təta}). We found that possessives are viewed as a set in the minds of speakers, and that there is no difference in distribution between singular and plural possessives. Therefore we will treat the possessive pronouns as a set of in Moloko (\textit{aloko, aləme, aləkwəye, } and \textit{ atəta}). }  

\ea \label{ex:3:9}
hor  ahan\\
\gll  hʷɔr   =\textbf{ahaŋ}\\
      woman  =\oldstylenums{3}\textsc{s}.{\POSS}\\
\glt  ‘his wife’
\z

\ea \label{ex:3:10}
məgəye  \textbf{ango}\\
\gll  mɪ-g-ijɛ   =\textbf{aŋgʷɔ}\\
      {\NOM}{}-do-{\CL}  ={\twoS}.{\POSS}\\
\glt  ‘your doings’
\z

\ea \label{ex:3:11}
war   dalay  \textbf{ahan}\\
\gll  war     dalaj    =\textbf{ahaŋ}\\
      child  girl  =\oldstylenums{3}\textsc{s}.{\POSS}\\
\glt  ‘his daughter’
\z

\ea \label{ex:3:12}
anjakar  \textbf{ata}  ahay\\
\gll  anzakar   =\textbf{atəta}     =ahaj\\
      chicken  =\oldstylenums{3}\textsc{p}.{\POSS}  =Pl\\
\glt  `their chickens’
\z

We consider the possessive pronouns to be noun clitics. They are phonologically bound to the noun. Even though prosodies on the possessive pronouns do not spread to the noun (\ref{ex:3:9}--\ref{ex:3:10}), \citet{Bow1997c} demonstrated that word-final changes indicating a word break do not occur (\tabref{tab:3.16}). They are clitics, not affixes, since they bind to the right edge of the head of the noun phrase, binding to the final noun where the head is composed of more than one noun, yet modifying the entire structure (\ref{ex:3:11}, see \sectref{sec:5.4.2}).

\begin{table}
\caption{Possessive cliticising to nouns with word-final /h/\label{tab:3.16}}
\resizebox{\textwidth}{!}{\begin{tabular}{lllll}\lsptoprule
 & {Underlying form} & \multicolumn{2}{l}{{Surface forms of isolated words}} & {Gloss}\\\midrule
{Neutral} & /g v h/ & [gəvax]   [uwla] \hspace{3pt}  → & [gəvəhuwla] & ‘my field’\\

& & ‘field’     ={\oneS}.{\POSS} \\
\midrule
{Labialised} & /hamb h \textsuperscript{o}/ & [hɔmbɔx] [uwla]   → & [hɔmbʊhuwla] & ‘my pardon’\\

& & ‘pardon’  ={\oneS}.{\POSS} \\
\midrule
{Palatalised} & /ta z h \textsuperscript{e}/ & [tɛʒɛx]   [uwla] \hspace{3pt}  → & [tɛʒɛhuwla] & ‘my snake’\\

& & ‘snake’  ={\oneS}.{\POSS} \\
\lspbottomrule
\end{tabular}}
\end{table}

\subsubsection{Semantic range of possessive constructions}\label{sec:3.1.2.1}

The semantic relation between the possessor and possessed can be flexible and covers the same range of possibilities as the associative construction (see \sectref{sec:5.4.1}). These semantic categories include ownership (\ref{ex:3:13}--\ref{ex:3:15}),\footnote{Examples \ref{ex:3:13}--\ref{ex:3:15} show that alienable and inalienable is not a relevant distinction for Moloko.} kinship relationships \REF{ex:3:16}, part-whole \REF{ex:3:17} and other associations (\ref{ex:3:18}--\ref{ex:3:19}).  

\ea \label{ex:3:13}
awak  \textbf{əwla}\\
\gll  awak =\textbf{uwla}\\
      goat    ={\oneS}.{\POSS}\\
\glt  ‘my goat’ (i.e. the goat I own)
\z

\ea \label{ex:3:14}
hay   \textbf{əwla}\\
\gll  haj     =\textbf{uwla}\\
      house  ={\oneS}.{\POSS}\\
\glt  ‘my house’ (i.e. the house I own/live in)
\z

\ea \label{ex:3:15}
gəvah  \textbf{əwla}\\
\gll  gəvax  =\textbf{uwla}\\
      field  ={\oneS}.{\POSS}\\
\glt  ‘my field’ (i.e. the field I own)
\z

\ea \label{ex:3:16}
baba  \textbf{əwla}\\
\gll  baba   =\textbf{uwla}\\
      father  ={\oneS}.{\POSS}\\
\glt  ‘my father’ (also, an older man in my father’s family)
\z

\ea \label{ex:3:17}
asak  \textbf{əwla}\\
\gll  asak   =\textbf{uwla}\\
      foot    ={\oneS}.{\POSS}\\
\glt  ‘my foot’ 
\z

\ea \label{ex:3:18}
məgəye  \textbf{əwla}\\
\gll  mɪ-g-ijɛ   =\textbf{uwla}\\
      {\NOM}{}-do-{\CL}  ={\oneS}.{\POSS}\\
\glt  ‘my doings’ (i.e. the things I do)
\z

\clearpage
\ea \label{ex:3:19}
məzəme  \textbf{əwla}\\
\gll  mɪ-ʒʊm-ɛ     =\textbf{uwla}\\
      {\NOM}{}-eat-{\CL}  ={\oneS}.{\POSS}\\
\glt  ‘my food’ (i.e. the food I grew/ the food that I am eating)
\z

\subsubsection{Tone of possessive pronouns}\label{sec:3.1.2.2}

\citet{Bow1997c} concluded that the underlying tone melody for possessive pronouns is HLH. \tabref{tab:3.17} (from \citealt{Bow1997c}) shows the surface tonal melodies and underlying tone pattern for all the possessive pronouns with the noun [\textit{ɗ\={a}f}]  ‘loaf.’\footnote{In Moloko, \textit{ɗaf} is the basic starch form consumed by the people, a millet porridge eaten with various sauces. The word can refer to one loaf of the porridge, and can also simply mean ‘food’.}. The singular forms with only two syllables drop the final high tone. All forms but the {\twoS} have the HM(H) surface pattern; the {\twoS} form contains the depressor consonant /ŋg/ and so the second syllable is low tone. 

\begin{table}
\begin{tabular}{llll}
\lsptoprule
& {Possessive pronoun in \textsc{np}} & {Surface tone} & {Underlying tone}\\
\midrule
\oneS & \textit{ɗ\={a}f} \textit{úwl\={a}} & HM & HL\\
& ‘my loaf’ \\
\twoS & \textit{ɗ\={a}f \'{ɔ}ŋgʷ\`{ɔ}} & HL & HL\\
& ‘your loaf’ \\
\SSS & \textit{ɗ\={a}f áh\={a}ŋ}  & HM & HL\\
& ‘your loaf’\\
\oldstylenums{1}\textsc{Pin} & \textit{ɗ\={a}f ál\={ɔ}kʷ\'{ɔ}} & HMH & HLH\\
& ‘our (inclusive) loaf’\\
\oldstylenums{1}\textsc{Pex} & \textit{ɗ\={a}f ál\={ɪ}m\'{ɛ}} & HMH & HLH\\
& ‘our (exclusive) loaf’\\
\oldstylenums{2}\textsc{p} & \textit{ɗ\={a}f ál\={ʊ}kǿj\'{ɛ}} & HMH & HLH\\
& ‘your (P) loaf’\\
\oldstylenums{3}\textsc{p} & \textit{ɗ\={a}f át\={ə}tá} & HMH & HLH\\
& ‘their loaf’\\
\lspbottomrule
\end{tabular}
\caption{Possessive pronoun paradigm with tone marked\label{tab:3.17}}
\end{table}

\tabref{tab:3.18} (from \citealt{Bow1997c}) gives examples of nouns with each underlying tone melody combined with {\twoS}, \oldstylenums{3}\textsc{s} and \oldstylenums{1}\textsc{Pex} possessive pronouns.  Some of the rules governing variations in the surface form are considered in \sectref{sec:2.4.2}. The possessive pronoun maintains its tonal melody in every environment. Note that the low surface tone of [dàndàj] ‘intestines’ (due to the depressor consonant) lowers the first high tone of the \oldstylenums{3}\textsc{s} and \oldstylenums{1}\textsc{Pex} possessive.

\begin{table}
\resizebox{\textwidth}{!}{%
\begin{tabular}{llllll}
\lsptoprule
 & {Example} & {Gloss} & {\twoS} & {\SSS} & {\oldstylenums{1}}{\textsc{Pex}}\\\midrule
H & [tsáf] & ‘shortcut’ & [ts\'{ə}f \'{ɔ}ŋgʷ\`{ɔ}] & [ts\'{ə}f áh\={a}ŋ] & [ts\'{ə}f ál\={ɪ}m\'{ɛ}]\\
& [b\'{ɔ}ɮ\'{ɔ}m] & ‘cheek’ & [b\'{ɔ}ɮ\'{ʊ}m \'{ɔ}ŋgʷ\`{ɔ}] & [b\'{ɔ}ɮ\'{ʊ}m áh\={a}ŋ] & [b\'{ɔ}ɮ\'{ʊ}m ál\={ɪ}m\'{ɛ}]\\\midrule
L & [ɗ\={a}f] & ‘loaf’ & [ɗ\={ə}f \'{ɔ}ŋgʷ\`{ɔ}] & [ɗ\={ə}f áh\={a}ŋ] & [ɗ\={ə}f ál\={ɪ}m\'{ɛ}]\\
& [dàndàj] & ‘intestines’ & [dànd\`{i}j \'{ɔ}ŋgʷ\`{ɔ}] & [dànd\`{i}j \={a}h\={a}ŋ] & [dànd\`{i}j \={a}l\={ɪ}m\'{ɛ}]\\\midrule
HL & [m\'{ɛ}k\={ɛ}tʃ] & ‘knife’ & [m\'{ɛ}k\={ɪ}tʃ \'{ɔ}ŋgʷ\`{ɔ}] & [m\'{ɛ}k\={ɪ}tʃ áh\={a}ŋ] & [m\'{ɛ}k\={ɪ}tʃ ál\={ɪ}m\'{ɛ}]\\
& [m\'{ɔ}gʷ\`{ɔ}d\`{ɔ}kʷ] & ‘hawk’ & [m\'ɔgʷ\`{ɔ}d\`{ʊ}kʷ \'ɔŋgʷ\`{ɔ}] & [m\'{ɔ}gʷ\`{ɔ}d\`{ʊ}kʷ \={a}h\={a}ŋ] & [m\'{ɔ}gʷ\`{ɔ}d\`{ʊ}kʷ \={a}l\={ɪ}m\'{ɛ}]\\\midrule
LH & [ɬ\={ə}máj] & ‘ear’ & [ɬ\={ə}m\'{i}j \'{ɔ}ŋgʷ\`{ɔ}] & [ɬ\={ə}m\'{i}j áh\={a}ŋ] & [ɬ\={ə}m\'{i}j ál\={ɪ}m\'ɛ]\\
& [b\`{ɔ}gʷ\={ɔ}m] & ‘hoe’ & [b\`{ɔ}g\={ʊ}m \'{ɔ}ŋgʷ\`ɔ] & [b\`ɔg\={ʊ}m áh\={a}ŋ] & [b\`ɔg\={ʊ}m ál\={ɪ}m\'{ɛ}]\\\midrule
HLH & [ák\={ʊ}f\'{ɔ}m] & ‘mouse’ & [ák\={ʊ}f\'{ʊ}m \'{ɔ}ŋgʷ\`{ɔ}] & [ák\={ʊ}f\'{ʊ}m áh\={a}ŋ] & [ák\={ʊ}f\'{ʊ}m ál\={ɪ}m\'{ɛ}]\\
& [d\'{ɛ}d\`{ɪ}l\={ɛ}ŋ] & ‘black’ & [d\'{ɛ}d\`{ɪ}l \'{ɔ}ŋgʷ\`{ɔ}] & [d\'{ɛ}d\`{ɪ}l \={a}h\={a}ŋ] & [d\'{ɛ}d\`{ɪ}l \={a}l\={ɪ}m\'{ɛ}]\\\midrule
LHL & [s\={ə}sáj\={a}k] & ‘wart’ & [s\={ə}sáj\={ə}kʷ \'{ɔ}ŋgʷ\`{ɔ}] & [s\={ə}sáj\={ə}k áh\={a}ŋ] & [s\={ə}sáj\={ə}k ál\={ɪ}m\'{ɛ}]\\
& [m\={ə}ŋgáhàk] & ‘crow’ & [m\={ə}ŋgáh\`{ə}kʷ \'{ɔ}ŋgʷ\`{ɔ}] & [m\={ə}ŋgáh\`{ə}k \={a}h\={a}ŋ] & [m\={ə}ŋgáh\`{ə}k \={a}l\={ɪ}m\'{ɛ}]\\
\lspbottomrule
\end{tabular}}
\caption{Tonal melodies in possessive constructions}\label{tab:3.18}
\end{table}

\subsection{Honorific possessive pronouns}\label{sec:3.1.3}
\hypertarget{RefHeading1210881525720847}{}
There are two special possessive pronouns used within vocative expressions to give honour to the person addressed.  The honorific pronouns are grammatically bound to the noun they follow.  They are used to honour people both within and outside the family. For men and women, whether married or not, to address one another with honour, \textit{golo} ‘dear/honourable’ follows the noun (\ref{ex:3:20}--\ref{ex:3:21}); for other relationships (mother, father, grandmother)  \textit{ya} ‘dear/honourable’ follows the noun (\ref{ex:3:22}--\ref{ex:3:24}). 


\ea \label{ex:3:20}
hor  \textbf{golo} \\
\gll  hʷɔr \textbf{gʷɔlɔ}\\
      woman  \textsc{honour}\\
\glt  `my dear wife'
\z

\ea \label{ex:3:21}
zar  \textbf{golo}\\
\gll  \ zar     \textbf{gʷɔlɔ} \\
      man    \textsc{honour}\\
\glt  `my dear husband'
\z

\ea \label{ex:3:22}
baba  \textbf{ya} \\
\gll  baba  \textbf{ja}\\
      father  \textsc{honour}\\
\glt  `my dear father'
\z

\ea \label{ex:3:23}
 dede  \textbf{ya}\\
      grandmother  \textsc{honour}
\glt  `my dear grandmother'
\z

\ea \label{ex:3:24}
Mama  \textbf{ya}  asaw  ɗaf.\\
\gll  mama  \textbf{ja}     a-s=aw      ɗaf\\
      mother  \textsc{honour}  \oldstylenums{3}\textsc{s}-please={\oneS}.{\IO}    {millet loaf}\\
\glt  `My dear mother, I want millet loaf!' (lit. millet loaf is pleasing to me)
\z

\subsection{Interrogative pronouns}\label{sec:3.1.4}
\hypertarget{RefHeading1210901525720847}{}
Interrogative pronouns request content information about an event, state, or participant (who, what, when, where, why, how). The basic interrogative words in Moloko are shown in \tabref{tab:3.19}.\footnote{Table adapted from \citealt{Boyd2003}.} 

\begin{table}
\resizebox{\textwidth}{!}{\begin{tabular}{llll}
\lsptoprule
{Element questioned} & {Interrogative} & {Gloss} & {Example}\\ 
		     &	pronoun        &         &  numbers\\\midrule
{Clause constituent} & \textit{way} & ‘who’ (human) & 25 and 26\\
& \textit{almay} & ‘what’ (non-human) & 27 and 28\\
& \textit{epeley} & ‘when’ & 29\\
& \textit{amtamay} & ‘where’ & 30\\
& \textit{kamay} & ‘why’ & 31\\
& \textit{memey} & ‘how/ explain’ & 32 and 33\\
& \textit{malmay} & ‘what is this’ & 35 and 34\\\midrule
{Noun phrase constituent} & \textit{mətəmey} & ‘how much’ & 36\\
& \textit{weley} & ‘which one’ & 37\\
\lspbottomrule
\end{tabular}}
\caption{Interrogative pronouns\label{tab:3.19}}
\end{table}

\clearpage
The normal position for interrogative pronouns is clause or noun phrase final (\ref{ex:3:25}--\ref{ex:3:38}).\footnote{See interrogative constructions in Moloko, \sectref{sec:10.3}.}  Two of the interrogative pronouns (\textit{memey} ‘how,’ and \textit{malmay} ‘what’) can question a clause in and of themselves (\ref{ex:3:33}--\ref{ex:3:35}). In each example, the interrogative pronoun is bolded. 

\ea \label{ex:3:25}
Aməvəlok  baskor  na  \textbf{way?}\\
\gll  amə-vəl=ɔkʷ baskʷɔr   na   \textbf{waj}\\
      {\DEP}-give={\twoS}.{\IO}  bicycle  {\PSP}  who\\
\glt  ‘Who gave you the bicycle?’ (lit. the one that gave you the bicycle [is] who?)
\z

\ea \label{ex:3:26}
Mana  amənjar  \textbf{way?}\\
\gll  Mana   à-mənzar   \textbf{waj}\\
      Mana  \oldstylenums{3}\textsc{s}+{\PFV}-see  who\\
\glt  ‘Whom did Mana see?’
\z

\ea \label{ex:3:27}
Kənjakay  \textbf{almay?}\\
\gll  k\`{ə}-nzak-aj    \textbf{almaj}\\
      {\twoS}+{\PFV}-find-{\CL}  what\\
\glt  ‘What did you find?’
\z

\ea \label{ex:3:28}
Kəzom  \textbf{almay?}\\
\gll  k\`{ə}-zɔm  \textbf{almaj}\\
      {\twoS}+\PFV-eat  what\\
\glt  ‘What did you eat?’
\z

\ea \label{ex:3:29}
Kálala  \textbf{epeley?}~\\
\gll  ká-l=ala     \textbf{ɛpɛlɛj}~\\
      {\twoS}+{\IFV}-go=to   when\\
\glt  ‘When are you coming?’
\z

\largerpage
\ea \label{ex:3:30}
Kólo  \textbf{amtamay?}~\\
\gll  k\'{ɔ}-lɔ     \textbf{amtamaj}\\
      {\twoS}+\IPV-go    where\\
\glt  ‘Where are you going?’
\z

\ea \label{ex:3:31}
Kólo  a  Lalaway  \textbf{kamay?}\\
\gll  k\'{ɔ}-lɔ   a  Lalawaj    \textbf{kamaj}\\
      {\twoS}+\IFV-go  at  Lalaway    why\\
\glt  ‘Why are you going to Lalaway?’
\z

\ea \label{ex:3:32}
Kəlala  na  \textbf{memey?}\\
\gll  k\`{ə}-l=ala  na  \textbf{mɛmɛj}\\
      {\twoS}+{\PFV}-go=to  {\PSP}  how\\
\glt  ‘Why  did you come?’
\z

\ea \label{ex:3:33}
\textbf{Memey?}\\
\gll  \textbf{mɛmɛj}\\
      how\\
\glt  ‘Explain?’ (what do you mean?, lit. how?)
\z

\ea \label{ex:3:34}
Nehe  na  \textbf{malmay?}\\
\gll  nɛhɛ   na   \textbf{malmaj}\\
      {\DEM}  {\PSP}  what\\
\glt  ‘What is this here?’
\z

\ea \label{ex:3:35}
\textbf{Malmay?}\\
\gll  \textbf{malmaj}\\
      what\\
\glt  ‘What is it?’
\z

\ea \label{ex:3:36}
Dala  \textbf{mətəme?}\\
\gll  dala    \textbf{mɪtɪmɛ}\\
      money  {how much}\\
\glt  ‘How much money [is that]?’
\z

\ea \label{ex:3:37}
Məlama  ango  na  \textbf{weley?}\\
\gll  məlama     =aŋgʷɔ     na   \textbf{wɛlɛj}\\
      brother    ={\twoS}.{\POSS}  {\PSP}  which\\
\glt  ‘Which (one among these) is your brother?’ (lit. your brother [is] which one?)
\z


\ea \label{ex:3:38}\corpussource{Cicada, S. 26}\\
Albaya  ahay  \textbf{weley} təh  anan  dəray  na  abay.\\
\gll  albaja   =ahaj   \textbf{wɛlɛj}  təx     an=aŋ         dəraj   na  abaj\\
      youth    =Pl    which   \textsc{id}put   {\DAT}=\oldstylenums{3}\textsc{s}.{\IO}   head   {\PSP}   {\EXT}+{\NEG}\\
\glt  ‘No one could lift it.’ (lit. whichever young man put his head [to the tree in order to lift it], there was none)
\z

In an emphatic question, a reduced interrogative pronoun both commences and finishes the clause (\ref{ex:3:39}--\ref{ex:3:42}). The interrogative pronouns \textit{way} ‘who,’ \textit{malmay} ‘what is this,’ \textit{memey} ‘why,’ and \textit{almay} ‘what’ are reduced, (without a change in meaning), to \textit{wa} \REF{ex:3:39}, \textit{malma} \REF{ex:3:40}, \textit{meme} \REF{ex:3:41}, and \textit{alma} \REF{ex:3:42}, respectively. These reduced forms occur at the beginning of an emphatic question. At the end of the clause, some of these same pronouns are reduced in a different manner. The interrogative pronoun \textit{memey} becomes \textit{mey} \REF{ex:3:41} and \textit{almay} becomes \textit{may} (\ref{ex:3:40}, \ref{ex:3:42}).

\ea \label{ex:3:39}
\textbf{Wa}  andaɗay  \textbf{way?}\\
\gll  \textbf{wa}    a-ndaɗ-aj   \textbf{waj}\\
      who    \oldstylenums{3}\textsc{s}-love-{\CL}  who\\
\glt  ‘No one loves him.’ (lit. who loves him?)
\z

\ea \label{ex:3:40}
\textbf{Malma}  awəlok \textbf{may?}\\
\gll  \textbf{malma}   a-wəl=ɔkʷ   \textbf{maj}\\
      what  \oldstylenums{3}\textsc{s}-hurt={\twoS}.{\IO}  what\\
\glt  ‘What is bothering (hurting) you?’
\z

\ea \label{ex:3:41}
\textbf{Meme}  ege  \textbf{mey?}\\
\gll  \textbf{mɛmɛ}   ɛ{}-g-ɛ     \textbf{mɛj}\\
      how    \oldstylenums{3}\textsc{s}-do-{\CL}  how\\
\glt  ‘What is going on here? [when something is wrong]’/ ‘What are you doing?’ (lit. how is it doing?)
\z

\ea \label{ex:3:42}\corpussource{Snake, S. 7}\\
\textbf{Alma}  amədəvala  okfom  na  \textbf{may?}\\
\gll  \textbf{alma}  amə-dəv=ala    ɔkʷfɔm  na  \textbf{maj}\\
      what  {\DEP}-trip=to    mouse  {\PSP}  what\\
\glt  ‘What was it that made that mouse fall?’
\z

\subsection{Unspecified pronouns}\label{sec:3.1.5}
\hypertarget{RefHeading1210921525720847}{}
A few pronouns refer to unspecified referents.  \textit{Meslenen} is a negative indefinite ‘no one’ \REF{ex:3:43} and must occur in a clause that is negated (see \sectref{sec:10.3}). \textit{Mana} is purposefully indefinite, referring to a person ‘who shall remain nameless’ \REF{ex:3:44}.  \textit{ Enen} ‘another’ \REF{ex:3:45} is an indefinite determiner, used to introduce new participants or things not previously mentioned.

\clearpage
\ea \label{ex:3:43}
Nəmənjar  \textbf{meslenen} bay.\\
\gll  n\`{ə}-mənzar     \textbf{mɛɬɛnɛŋ}    baj\\
      {\oneS}+{\PFV}-see    {no one}    {\NEG}\\
\glt  ‘I didn’t see anyone.’
\z

\ea \label{ex:3:44}
Anjaka  aməɓezlata  azla  \textbf{mana}  \textbf{mana}  \textbf{mana.}  \\
\gll  a-nz=aka  amə-ɓɛɮ    =ata   aɮa    \textbf{mana}    \textbf{mana}    \textbf{mana}  \\
      \oldstylenums{3}\textsc{s}-left=on  {\DEP}-count  =\oldstylenums{3}\textsc{p}.{\IO}  now  {so and so} {so and so} {so and so}\\
\glt  ‘He started telling their names: so and so, and so and so, and so on.’  
\z

\ea \label{ex:3:45}{}
{}[Nafat  \textbf{enen}] aba\\
\gll  {}[nafat  \textbf{ɛnɛŋ}]  aba\\
      day    another  {\EXT}\\
\glt  ‘One day\ldots’  (a usual way to start a story)
\z

\section{Demonstratives and demonstrationals}\label{sec:3.2}\is{Deixis!Demonstratives and demonstrationals|(}
\hypertarget{RefHeading1210941525720847}{}
Moloko has three main types of demonstratives: nominal demonstratives (\sectref{sec:3.2.1}) which point to a person or object and modify a noun in a noun phrase, local adverbial demonstratives (\sectref{sec:3.2.2}) which point to a place and modify a noun in a noun phrase, and manner adverbal demonstratives (\sectref{sec:3.2.3}), which point to an action and modify a verb.\footnote{\citet{Dixon2003} describes three types of demonstratives: nominal, local adverbial, and verbal. Verbal demonstratives do not occur in Moloko. Dixon considers manner adverbial demonstratives to be a subtype of nominal demonstratives.} Manner adverbials are derived from local adverbial denonstratives. 

\tabref{tab:3.20} shows all of the demonstratives in Moloko. All demonstratives have the same form for both singular and plural referents.  All are anaphoric in their basic use in that the referent must be known from the preceding context. Place/time adverbs are also shown for comparison. The proximal demonstratives are morphologically similar to the locational adverb \textit{ehe} ‘here/now’ (shown for comparison in \tabref{tab:3.20}.)

It can be seen that the near speaker and distant from speaker demonstratives are morphologically derived from the corresponding adverbs. Note that there are no non-visible demonstratives or place/time adverbs.

\begin{landscape}
\begin{table}
\begin{tabular}{lllll} 
\lsptoprule
& \textbf{Nominal demonstratives}  & \textbf{Local adverbial}  & \textbf{Manner adverbial} & \textbf{Place/time}\\ 
& &  \textbf{demonstratives}  & \textbf{demonstratives} & \textbf{adverbs}\\
\midrule
\textbf{Proximal } & \textit{ngehe / }\textit{nəngehe / nengehe }\footnote{The demonstrative \textit{ngehe} is a contraction of \textit{nəngehe}.} & \textit{nehe} & \textit{ka nehe } & \textit{ehe} \\ 
\textbf{(near speaker)} & ‘this’  & ‘here’ & ‘like this’ & ‘here’\\
& & & \textit{kəygehe} & \textit{cəcəngehe}\\
& & & ‘this way’ & ‘now’\\
\midrule
\textbf{Distal } & \textit{ngəndəye / ngəndəge}\footnote{This demonstrative is pronounced either [\textit{nɪŋgɪndijɛ}] or [\textit{nɪŋgɪndɪgɛ}] by speakers from different regions.} & \textit{nəndəye / nendəge}\footnote{Likewise, dialect differences account for the different pronunciations.} & & \\
\textbf{(away from}  & ‘that’ & ‘there’ &  & \\
\textbf{speaker)} &  &  & &\\
\midrule
\textbf{Distant from}  &  & \textit{toho}\footnote{In a genitive or possessive construction.} &  & \textit{toho}\\
\textbf{speaker} & & ‘over there’ & & ‘over there’\\
\midrule
\textbf{Anaphoric} &  & \textit{ndana} & \textit{ka ndana } & \\
& & ‘that previously mentioned’ & ‘like what was described’ & \\
& & & \textit{kəyga} & \\
& & & ‘like that’ & \\
\lspbottomrule
\end{tabular}
\caption{\label{tab:3.20} Demonstratives in Moloko}
\end{table}\end{landscape}

\subsection{Nominal demonstratives}\label{sec:3.2.1}
\hypertarget{RefHeading1210961525720847}{}
Nominal demonstratives (\ref{ex:3:46}--\ref{ex:3:48}) have a referent that is a person or object. They modify a noun within a noun phrase to specify or point out the referent. Moloko has two nominal demonstratives: proximal (near the speaker) and distal (away from the speaker). There is no nominal demonstrative to indicate a referent that is far away from the speaker. In the examples in this section,\footnote{The first line in each example is the orthographic form. The second is the phonetic form (slow speech) with morpheme breaks.} the demonstrative is bolded and the noun phrase is marked by square brackets. In \REF{ex:3:55} from \sectref{sec:3.2.2.1}, the demonstrative is head of the noun phrase, suggesting that it can act as a demonstrative pronoun. 

\ea \label{ex:3:46}
Náskom  [zana \textbf{ngehe.}]\\
\gll  ná-sʊkʷɔm  [zana  \textbf{ŋgɛhɛ}]\\
      {\oneS}+{\IFV}-buy  cloth  {\DEM}\\
\glt  ‘I will buy this particular cloth here.’ (pointing to or holding a particular cloth among others)
\z

\ea \label{ex:3:47}
Asaw  [awak \textbf{ngəndəye}.]\\
\gll  a-s=aw    [awak  \textbf{ŋgɪndijɛ}]\\
      \oldstylenums{3}\textsc{s}-please={\oneS}.{\IO}  goat  {\DEM}\\
\glt  ‘I want that particular goat there.’ (pointing to a particular goat among others)
\z

\ea \label{ex:3:48}
[Babəza  ahay  \textbf{ngəndəye}]  anga  əwla  ahay.\\
\gll  [babəza  =ahaj   \textbf{ŋgɪndijɛ}]  aŋga  =uwla     =ahaj\\
      children  =Pl  {\DEM}     {\POSS}  ={\oneS}.{\POSS}  =Pl  \\
\glt  ‘These particular children here [are] belonging to me.’ 
\z

Besides their use to point out specific referents, the nominal demonstratives can also be used anaphorically\is{Cohesion!Anaphoric referencing} in discourse.\footnote{Moloko has one specifically anaphoric demonstrative used in discourse (\textit{ndana}, \sectref{sec:3.2.2.2}). Also, two other particles function in cohesion as discourse anaphoric referent markers. They are \textit{ga} (\sectref{sec:5.3}) and \textit{na} (\chapref{chap:11}).} The distal nominal demonstrative \textit{ngəndəye} in line S. 14 of the Cicada story \REF{ex:3:49} identifies the tree as being that particular previously mentioned one that the men wanted the chief to have. 

\clearpage
\ea\label{ex:3:49}\corpussource{Cicada, S. 14}\\\relax
      [Agwazla  \textbf{ngəndvəye}]  ágasaka  ka  mahay  ango  aka.\\
\gll  [agwazla \textbf{ŋgɪndijɛ}]  á-gas=aka  ka  mahaj  =aŋgʷɔ    aka\\
      {spp. of tree}  {\DEM}              \oldstylenums{3}\textsc{s}+{\IFV}-catch=on    at    door  ={\twoS}.{\POSS}  on\\
\glt  ‘That particular (previously mentioned) tree would be pleasing by your door.’ 
\z

At the conclusion of the Disobedient Girl story, nominal demonstratives are used anaphorically to mark two different referents -- the suffering brought to the Moloko people and the young girl whose disobedience resulted in the suffering. Both are shown in \REF{ex:3:50}. The beginning of the Disobedient Girl story describes the blessing -- that Moloko people could make an entire meal for a whole family from one grain of millet. The blessing occurred because the millet would multiply during its grinding. The story describes how a young, newly-married non-Moloko girl hears how to handle the millet yet disobeys the rules on how to handle it. As a result, the disobedient girl was killed by the millet. The story tells how the Creator was offended by her act and withdrew his blessing from the Moloko people such that millet would not multiply any more and the Moloko had to work hard to even get enough food to feed their families. The suffering that the Moloko people experienced as a result of the withdrawal of God’s blessing is described in lines 33-37 but it is not named as such until line S. 38. In that line, the particular suffering of the Moloko people that was brought on by the girl is marked by the proximal nominal demonstrative \textit{avəya nengehe} ‘this particular previously mentioned suffering.' Also, the young woman who, by her disobedience, brought suffering to the entire Moloko population is marked in lines 33 and 38 by the distal nominal demonstrative. Line 33 contains \textit{war dalay na} \textit{amecen sləmay bay} \textit{ngəndəye} ‘the young woman, the previously mentioned disobedient one’ and line 38 contains \textit{war dalay ngəndəye} ‘that previously mentioned young woman.’

\ea \label{ex:3:50} \corpussource{Disobedient Girl, S. 33}\\
      Məloko  ahay  tawəy,  Hərmbəlom  ága  ɓərav  va  \\
\gll  Mʊlɔkʷɔ =ahaj tawij Hʊrmbʊlɔm á-g-a ɓərav =va     \\
      Moloko =Pl \oldstylenums{3}\textsc{p}+said God \oldstylenums{3}\textsc{s}+{\IFV}-do-{\CL} heart ={\PRF}      \\
      \glt  ‘The Molokos say, God got angry (lit. God did heart)’ \\
      
\medskip
\largerpage[2]
kəwaya  war  dalay  na,  amecen  sləmay  bay \textbf{ngəndəye}. \\   
\gll kuwaja        war    dalaj     na   amɛ-tʃɛŋ      ɬəmaj  baj \textbf{ŋgɪndijɛ}\\
     {because of}  child    girl    {\PSP}  {\DEP}-hear   ear      {\NEG}  {\DEM}\\
\glt ‘because of the girl, the particular previously mentioned one that was disobedient.’\\

\medskip
\corpussource{Disobedient Girl, S. 34}\\
Waya  ndana  Hərmbəlom  ázata  aka  barka  ahan  va.\\
\gll waja     ndana   Hʊrmbʊlɔm   á-z     =ata     =aka  barka     =ahaŋ   \\
      because  {\DEM}   God      \oldstylenums{3}\textsc{s}+{\IFV}-take  =\oldstylenums{3}\textsc{p}.{\IO}  =on   blessing  =\oldstylenums{3}\textsc{s}.{\POSS}  \\
      
\medskip
\gll   =va  \\
={\PRF}\\
\glt ‘Because of that, God had taken back his blessing from them.’\\

\medskip
\corpussource{Disobedient Girl, S. 35}\\
     Cəcəngehe  na,  war  elé  háy  bəlen  na,  ásak  asabay.\\
\gll tʃɪtʃɪŋgɛhɛ  na,  war  ɛlɛ        haj  bɪlɛŋ     na      á-sak                                        asa-baj\\
      now       {\PSP}  child   eye   millet   one   {\PSP}  \oldstylenums{3}\textsc{s}+{\IFV}-multiply    again-{\NEG}\\
\glt ‘And now, one grain of millet, it doesn’t multiply anymore.’

\medskip
\corpussource{Disobedient Girl, S. 36}\\
        Talay war elé  háy bəlen kə ver aka na, ásak asabay.\\
\gll    talaj          war     ɛlɛ    haj      bɪlɛŋ  kə     vɛr     aka  na        á-sak                                      asa-baj\\
       \textsc{id}put  child   eye   millet    one    on    stone    on    {\PSP}  \oldstylenums{3}\textsc{s}+{\IFV}-multiply  again-{\NEG}\\
\glt ‘[If] one puts one grain of millet on the grinding stone, it doesn’t multiply anymore.’

\medskip
 \corpussource{Disobedient Girl, S. 37}\\
 {Səy  kádəya  gobay.}\\
\gll {sij} ká-d    =ija  gʷɔbaj\\
only    {\twoS}+{\IFV}-prepare  ={\PLU}   {a lot}\\
\glt ‘You must put on a lot.’

\medskip
\corpussource{Disobedient Girl, S. 38}\\
Ka  nehe  tawəy,  metesle  anga  war  dalay \textbf{ngəndəye}\\
\gll ka  nɛhɛ  tawij  mɛ-tɛɬ-ɛ      aŋga  war    dalaj  \textbf{ŋgɪndijɛ}\\
like  {\DEM}   \oldstylenums{3}\textsc{p}+said  {\NOM}{}-curse-{\CL}   {\POSS}   child  girl       {\DEM}      \\
\glt ‘It is like this they say, “The curse [is] belonging to that particular (previously mentioned) young woman’\\

\medskip
     amazata  aka  ala    [avəya \textbf{nengehe}] ana  məze  ahay  na.\\
\gll ama-z        =ata                                =aka  =ala      avija  \textbf{nɛŋgɛhɛ} ana    mɪʒɛ  =ahaj   na\\
     {\DEP}-take  =\oldstylenums{3}\textsc{p}.{\IO}   =on     =to  suffering  {\DEM}      {\DAT} person    =Pl  {\PSP}\\
\glt ‘that brought this particular (previously mentioned)  suffering onto the people.”’  
\z

\subsection{Local adverbial demonstratives}\label{sec:3.2.2}
\hypertarget{RefHeading1210981525720847}{}
Local adverbial demonstratives point to a referent that is a place (physical or metaphorical). They commonly occur with a noun but can also occur as the only element in a noun phrase. Moloko has three local adverbial demonstratives: proximal (near the speaker), distal (away from the speaker) (\sectref{sec:3.2.2.1}), and an anaphoric demonstrative used only in discourse (\sectref{sec:3.2.2.2}). There is no demonstrative to indicate a place far away from the speaker. However the adverb \textit{toho} ‘over there’ is used within noun phrases where such a place needs to be indicated. 

\subsubsection{Proximal and distal local adverbial demonstratives}\label{sec:3.2.2.1}

Proximal and distal local adverbial demonstratives refer to a physical place (here or there). In a noun phrase, the position for the local adverbial demonstrative is different than for a nominal demonstrative. The local adverbial demonstrative occurs as a separate final element (\ref{ex:3:51}--\ref{ex:3:54}).\footnote{Note that nominal demonstratives do not occur after the adjectiviser, \sectref{sec:5.1}.} In the examples in this section, the demonstrative is bolded and the noun phrase is marked by square brackets.

\ea \label{ex:3:51}
[Ɗaf  \textbf{nehe}]  acar.\\
\gll  [ɗaf  \textbf{nɛhɛ}]  a-tsar\\
      {millet loaf}    {\DEM}  {\oldstylenums{3}\textsc{s}-taste good}\\
\glt  ‘This millet loaf here (in this place) tastes good.’
\z

\ea \label{ex:3:52}
      Nazalay      [awak  ahay  \textbf{nəndəye}]  a  kosoko  ava.\\
\gll  na-z-alaj    [awak  =ahaj  \textbf{nɪndijɛ}]  a  kɔsɔkʷɔ  ava\\
      {\oneS}-carry-away  goat  =Pl  {\DEM}    at  market  in\\
\glt  ‘I take the goats there (in that place) to the market.’
\z

\ea \label{ex:3:53}\corpussource{Disobedient Girl, S. 13}\\\relax
      [War  elé  háy  bəlen  ga  \textbf{nəndəye}]  [nok amɛzəɗe  na,]\\
\gll  [war  ɛlɛ  haj  bɪlɛŋ  ga  \textbf{nɪndijɛ}]  [nɔkʷ   amɛ-zɪɗ-ɛ     na]\\
      child   eye   millet  one     {\ADJ}   {\DEM}    {\twoS}    {\DEP}-take-{\CL}   {\PSP}\\
\glt  ‘That one grain there (highlighted\footnote{See below for the discourse function of local adverbial demonstratives.}), the one that you have taken,’\\
\medskip
\largerpage
káhaya  na  kə  ver  aka.\\
\gll ká-h  =aja     na       kə       vɛr        aka\\
{\twoS}+{\IFV}-grind ={\PLU}  \oldstylenums{3}\textsc{s}.{\DO}  on   {grinding stone}  on\\
\glt ‘grind it on the grinding stone.’
\z

\ea \label{ex:3:54}\corpussource{Values, S. 3}\\ 
Səwat  na, [təta  a  məsəyon  na  ava  \textbf{nəndəye}  na,]  pester  áhata.\\
\gll  suwat  na   [təta  a  mʊsijɔŋ   na   ava  \textbf{nɪndijɛ}  na]  pɛʃtɛr  á-h=ata\\
      \textsc{id}disperse  {\PSP}  \oldstylenums{3}\textsc{p}      at   mission  {\PSP}  in  {\DEM}  {\PSP}  pastor   \oldstylenums{3}\textsc{s}-tell   =\oldstylenums{3}\textsc{p}.{\IO}\\
\glt  ‘As the people go home from church, the pastor tells them, (lit. disperse, they in the mission there), 
\z

The local adverbial demonstrative can be the head of a noun phrase. In \REF{ex:3:55} the demonstrative is modified by the plural. 

\ea \label{ex:3:55}
Nde  [\textbf{nehe}   ahay   na]  sla  ango ahay  ɗaw?\\
\gll  ndɛ    [\textbf{nɛhɛ}  =ahaj   na]     ɬa   =aŋgʷɔ =ahaj  ɗaw\\
      so     {\DEM}  =Pl  {\PSP}  cow  ={\twoS}.{\POSS}  =Pl {\QUEST}\\
\glt  ‘So, these [cows] here (in this place), are they your cows?’
\z

For locations far away from the speaker, the locational adverb \textit{toho } is used in a possessive or genitive construction with the noun it modifies, (\textit{anga toho}, \REF{ex:3:56} see \sectref{sec:5.6.1}; or  \textit{a toho}, \REF{ex:3:57}, see \sectref{sec:5.4.1}).

\ea \label{ex:3:56}
[Hay  əwla  \textbf{anga} \textbf{toho} na,]  eleməzləɓe  tanday  tozom  na.\\
\gll  [haj   =uwla             \textbf{aŋga} \textbf{tɔhʷɔ}  na]     ɛlɛmɪɮɪɓɛ  ta-ndaj                             tɔ-zɔm  na\\
      house  ={\oneS}.{\POSS}  {\POSS}       {\DEM}                            {\PSP}  termites   \oldstylenums{3}\textsc{p}-{\PROG}  \oldstylenums{3}\textsc{p}-eat  \oldstylenums{3}\textsc{s}.{\DO}\\
\glt  ‘My house way over there (pointing to a particular house among others in the distance), termites are eating it.’ (lit. my house, the one that belongs to over there, termites are eating it)
\z

\ea \label{ex:3:57}
[Awak  ahay  \textbf{a}  \textbf{toho}]  anga  əwla.\\
\gll  [awak  =ahaj  \textbf{a}  \textbf{tɔhʷɔ}]  aŋga  =uwla\\
      goat    =Pl  {\GEN}  {\DEM}  {\POSS}  ={\oneS}.{\POSS}\\
\glt  ‘The goats over there (in that place) belong to me.’ (lit. the goats over there [are] belonging to me)
\z

The function of local adverbial demonstratives to point out a place can be seen in the Cicada text (\ref{ex:3:58}--\ref{ex:3:59}, found in its entirety in \sectref{sec:1.6}). In the story, a beautiful tree is found in the bush and the chief decides that he wants to have it moved to his yard. The tree is first mentioned as being \textit{a ləhe} ‘in the bush’ in line S. 5 \REF{ex:3:58}. The tree is mentioned again in line S. 12 marked by the local adverbial demonstrative \textit{nəndəye}\textit{ }‘that one there’ \REF{ex:3:59}. 


\ea \label{ex:3:58}\corpussource{Cicada, S. 5}\\
Təlo  tənjakay  agwazla  malan  ga  a  ləhe.\\
\gll  t\`{ə}-lɔ            t\`{ə}-nzak-aj           agʷaɮa    malaŋ     ga   a  lɪhɛ\\
      \oldstylenums{3}\textsc{p}+{\PFV}-go   \oldstylenums{3}\textsc{p}+{\PFV}-find-{\CL}  {spp. of tree}  large    {\ADJ}  at  bush\\
\glt  ‘They went and found a large tree (of a particular species) in the bush.’
\z

\ea \label{ex:3:59}\corpussource{Cicada, S. 12}\\
Təlo  tamənjar  na            ala           [mama  agwazla  \textbf{nəndəye} ]\\
\gll  t\`{ə}-lɔ tà-mənzar     na   =ala     [mama  agʷaɮa    \textbf{nɪndijɛ} ]\\
      \oldstylenums{3}\textsc{p}+{\PFV}-go    \oldstylenums{3}\textsc{p}+{\HOR}-see  \oldstylenums{3}\textsc{s}.{\DO}  =to  mother  {spp. of tree}  {\DEM}\\
\glt  ‘They went to see the mother tree there.’
\z

Sometimes local adverbial demonstratives\is{Focus and prominence!Local adverbial demonstratives|(} have a highlighting function for new information in a narrative, drawing attention to their referent.\footnote{\citet{Dixon2003} mentions that demonstratives can function to introduce new information. Note that in Moloko, all new information need not be marked with a demonstrative.} In the ‘Cows in the Field’ story (not illustrated in its entirety in this work), \textit{ɗerəywel nendəge} ‘this paper here’ \REF{ex:3:60} was not with the speaker when he told the story; neither was it previously mentioned in the discourse. According to the discourse, the paper should have helped to bring justice to the men whose cotton was destroyed, but didn’t. Its marking with a demonstrative therefore has the function to highlight the paper at that moment of the eventline. 

\ea \label{ex:3:60}
Alala  na,  ta  anaw  [ɗerəywel  \textbf{nendəge.}]\\
\gll  a-l=ala  na      ta   an  =aw   [ɗɛrijwɛl   \textbf{nɛndɪgɛ}]\\
      \oldstylenums{3}\textsc{s}-go=to  {\PSP}   \oldstylenums{3}\textsc{p}  {\DAT}  ={\oneS}.{\IO}  paper        {\DEM}\\
\glt  ‘Later, they [gave] me this here paper.’ 
\z
\largerpage
In the Values exhortation (\ref{ex:3:61}, shown in its entirety in \sectref{sec:1.7}) the local adverbial demonstrative \textit{nehe}  ‘this here’ is used to draw attention to new information. In the exhortation, the phrase \textit{ele nehe}  ‘these things here’ introduces information not previously mentioned in the discourse.\footnote{Note that the local adverbial demonstrative \textit{nəndəye} ‘here’ in the same example functions to simply point out a place in the phrase \textit{təta a mʊsəyon na ava nəndəye} ‘the ones in church there’. Also, compare the function of the proximal local adverbial demonstrative \textit{nehe}  with that of the proximal nominal demonstrative \textit{nəngehe}  in the same example. The nominal demonstrative in the phrase \textit{ele =ahay aməgəye bay nəngehe} ‘these particular things that one shouldn’t do’ points out particular things which are previously mentioned \sectref{sec:3.2.1}.} This information -- the things that people are not supposed to do -- is the main topic of the entire discourse. The demonstrative marking functions to notify the reader of the importance of the new information. Note that the demonstrative is not functioning cataphorically here. It is the narrator who specifies the things that people are not supposed to do in the discourse which follows (S. 4--5 in \ref{ex:3:61}), not the pastor in his speech. 

\ea \label{ex:3:61}\corpussource{Values, S. 3}\\
 Səwat  na,  [təta  a  məsəyon  na  ava  nəndəye  {na,}]  Pester  ahata,      \\
\gll  suwat na   [təta   a   mʊsijɔŋ   na   ava nɪndijɛ  na]   Pɛstɛr    a-h  =ata\\
      \textsc{id}disperse  {\PSP}  \oldstylenums{3}\textsc{p}  at  mission  {\PSP}  in  {\DEM}  {\PSP}  pastor  \oldstylenums{3}\textsc{s}-tell  =\oldstylenums{3}\textsc{p}.{\IO}\\
\glt  ‘As the people go home from church (lit. disperse, they in the mission there), the Pastor said, \\
\medskip
    “Ey, [ele \textbf{nehe} na] kogom  bay!”\\
\gll {ɛj} [{ɛlɛ} {\textbf{nɛhɛ}} na] kɔ-gʷ-ɔm   baj\\
      hey  thing  {\DEM}  {\PSP}  \oldstylenums{2}-do-{\twoP}    {\NEG}\\
\glt ‘“Hey! These things here, don’t do them!”’

\medskip 
\corpussource{Values, S. 4}\\
{Yawa,  war  dalay  ga ándaway  mama  ahan.}\\
\gll jawa   war   dalaj  ga  á-ndaw-aj   mama   =ahaŋ\\
     well    child  female  {\ADJ}  \oldstylenums{3}\textsc{s}+{\IFV}-insult{}-{\CL}  mother  =\oldstylenums{3}\textsc{s}.{\POSS}  \\
\glt ‘Well, the girls insult their mothers.’ 

\medskip
\corpussource{Values, S. 5}\\
War  zar  ga  ándaway  baba  ahan.\\
\gll war     zar  ga  á-ndaw-aj   baba   =ahaŋ\\
    child  male  {\ADJ}  \oldstylenums{3}\textsc{s}+{\IFV}-insult{}-{\CL}  father  =\oldstylenums{3}\textsc{s}.{\POSS}\\
\glt ‘[And] the boys insult their fathers.’ 

\medskip
\corpussource{Values, S. 6}\\ 
Yo, [ele  ahay  aməgəye  bay  nəngehe pat,]\\
\gll jɔ   [ɛlɛ  =ahaj  amɪ-g-ijɛ            baj   nɪŋgɛhɛ   pat] \\
   well    thing  =Pl  {\DEP}-go-{\CL}     {\NEG}  {\DEM}     all  \\
\glt ‘Well, all these particular things that we are not supposed to do,’ 
   
\medskip

tahata  na  va  kə dəftere  aka.\\
\gll ta-h=ata        na  =va   kə   dɪftɛrɛ  aka. \\
   \oldstylenums{3}\textsc{p}-tell=\oldstylenums{3}\textsc{p}.{\IO}    \oldstylenums{3}\textsc{s}.{\DO}  ={\PRF}   on  book  on\\
\glt ‘they have already told them in the book.’ 
\z
\largerpage
The highlighting function of local adverbial demonstratives\is{Focus and prominence!Local adverbial demonstratives|)} does not have to be associated with the introduction of new information. For example, in the Disobedient Girl story (\ref{ex:3:62}, shown in its entirety in \sectref{sec:1.5}), the one grain of millet is introduced in the first line of the husband’s speech to his wife (line S. 13 in \ref{ex:3:62}). The next mention of the one grain of millet is in the next line of his speech is where the grain is marked by the local adverbial demonstrative in \textit{war elé háy bəlen ga nəndəye} ‘that one grain there.’ In this case, \textit{nəndəye} ‘that there’ does not mark new information; the one grain of millet has already been mentioned in the previous sentence. However, the highlighting function of the demonstrative identifies the one grain of millet as being important in the developing story. It is the one grain of millet which becomes transformed and multiplied and suffocates the disobedient girl by the end of the story. 

\ea \label{ex:3:62}\corpussource{Disobedient Girl, S. 13}\\
Asa  asok aməhaya  na,  kázaɗ   war  elé  háy  bəlen.\\
\gll  asa  à-s            =ɔkʷ  amə-h       =aja        na   ká-zaɗ              war     ɛlɛ      haj   \\
      if \oldstylenums{3}\textsc{s}+{\PFV}-please ={\twoS}.{\IO} {\DEP}-grind ={\PLU} {\PSP} {\twoS}+{\IFV}-take  child  eye  millet \\
      
      \medskip
\gll bɪlɛŋ\\
     one\\
\glt  ‘If you want to grind, you take only one grain.’\\
\medskip
[War  elé háy  bəlen   ga \textbf{nəndəye}] [nok amɛzəɗe  na,]\\
\gll [war ɛlɛ  haj     bɪlɛŋ  ga    \textbf{nɪndijɛ} nɔkʷ amɛ-zɪɗ-ɛ  na]\\
     child eye millet  one   {\ADJ}   {\DEM}        {\twoS}                {\DEP}-take-{\CL}   {\PSP}\\
\glt ‘That (highlighted) one grain, the one that you have taken,’\\
\medskip
káhaya  na  kə  ver  aka.\\
\gll ká-h    =aja     na       kə       vɛr        aka\\
{\twoS}+{\IFV}-grind ={\PLU}  \oldstylenums{3}\textsc{s}.{\DO}  on   {grinding stone}  on\\
\glt ‘grind it on the grinding stone.’
\z

The distal non local demonstrative is employed in a common discourse idiom  -- \textit{a slam nendəye ava} ‘at that time.’ The idiom notifies the reader of an important pivotal moment in a story. \REF{ex:3:63} is from the ‘Cows in the Field’ story (not illustrated in its entirety in this work). The narrative concerns dealings with the owners of a herd of cows that had destroyed someone’s field of cotton. \textit{A slam nendəye ava} marks the transition point in the way that the speaker dealt with the cows. 


\ea \label{ex:3:63}
A  [slam  \textbf{nendəye}]  ava  na,  nawəy,\\
\gll  a   [ɬam   \textbf{nɛndijɛ}]   ava   na    nawij\\
      at  place      {\DEM}          in   {\PSP}   {\oneS}+said\\
\glt ‘At that moment, I said,’\\
      
 \medskip
  “Sla  ahay  na,  məmokok  ta  bay,  \\
\gll ɬa   =ahaj   na   mʊ-mɔkʷ-ɔkʷ ta baj \\
cow   =Pl  {\PSP}  \oldstylenums{1}\textsc{Pin}+{\HOR}-leave-\oldstylenums{2}\textsc{Pin}   \oldstylenums{3}\textsc{p}.{\DO}   {\NEG}   \\
\glt ‘“These cows, let’s not leave them at all,’

\medskip
golok ta  a  Kəɗəmbor,\\
\gll gʷɔl-ɔkʷ ta  a  Kʊɗʊmbɔr\\
     drive[{\IMP}]-\oldstylenums{1}\textsc{Pin} \oldstylenums{3}\textsc{p}.{\DO}   at   Tokombere\\
\glt ‘let’s drive them to Tokembere,’

\medskip
ɗeɗen  bay  na  memey?”\\
\gll ɗɛɗɛŋ   baj     na      mɛmɛj\\
truth  {\NEG}   {\PSP}   how \\   
\glt  ‘if it’s not true, then how?”’
\z

\subsubsection{Anaphoric demonstrative}\label{sec:3.2.2.2}\is{Cohesion!Anaphoric referencing}
The anaphoric demonstrative \textit{ndana} ‘that previously mentioned’ refers to a metaphorical place and is used only in discourse for anaphoric marking of a participant that is important to the message of the discourse. In the Disobedient Girl story, \textit{war dalay ndana} ‘that previously mentioned young woman’ occurs in the introduction of the major characters in the story (\ref{ex:3:64}). The three major characters in the story are the husband, the woman, and the grain of millet. The woman will, by her disobedience, bring a curse on the Moloko people. 

\ea \label{ex:3:64}\corpussource{Disobedient Girl, S. 11}\\
Azləna,  [war  dalay  \textbf{ndana}]  cezlere  ga.\\
\gll  aɮəna  [war   dalaj   \textbf{ndana}]  tʃɛɮɛrɛ         ga\\
      but  child      female    {\DEM}  disobedience   {\ADJ}\\
\glt  ‘Now, the above-mentioned young girl was disobedient.’
\z

Likewise, in the Cicada story (\ref{ex:3:65}--\ref{ex:3:67}), found in its entirety in \sectref{sec:1.6}, the demonstrative \textit{ndana} ‘previously mentioned’ is used anaphorically to mark the young men and the tree, both of which are key elements in the story. The chief desired to have a particular tree transplanted at his gate. He commissioned his people to do it. In \REF{ex:3:66} (from S. 6), \textit{albaya =ahay ndana} ‘those previously mentioned young men’ and \REF{ex:3:67} (from S. 9)  \textit{agwazla ndana} ‘that tree just mentioned,’ \textit{ndana} is used to refer back to the young men introduced in S3 and the tree introduced in S5. 

\ea \label{ex:3:65}\corpussource{Cicada, S. 3 and S. 5}\\
Albaya  ahay  aba.\ldots  Təlo  tənjakay  agwazla  malan  ga  a  ləhe.\\
\gll  albaja  =ahaj  aba.\ldots \\
      {young man}   =Pl  {\EXT}\\
\glt  ‘There were some young men\ldots\\
\medskip
\gll t\`{ə}-lɔ            t\`{ə}-nzak-aj           agʷaɮa    malaŋ     ga   a  lɪhɛ\\
     \oldstylenums{3}\textsc{p}+{\PFV}-go   \oldstylenums{3}\textsc{p}+{\PFV}-find-{\CL}  {spp. of tree}  large    {\ADJ}  at  bush\\
\glt ‘They went and found a large tree (of a particular species) in the bush.’
\z

\ea \label{ex:3:66}\corpussource{Cicada, S. 6}\\
{[}Albaya  ahay  \textbf{ndana}]  kəlen  təngalala  ma  ana  bahay.\\
\gll {[albaja}   =ahaj  \textbf{ndana}]  kɪlɛŋ  t\`{ə}-ŋgala      =ala   ma  ana   bahaj\\
     {young man}    =Pl         {\DEM}  then  {\oldstylenums{3}\textsc{p}+{\PFV}-come back}  =to  word  {\DAT} chief\\ 
\glt ‘Those above-mentioned young men then took the word (response) to the chief.’
\z

\ea \label{ex:3:67}\corpussource{Cicada, S. 9}\\
Káazaɗom  anaw  ala  [agwazla  \textbf{ndana}]  ka  mahay  əwla  aka.\\
\gll káá-zaɗ{}-ɔm    an   =aw   =ala  [agʷaɮa  \textbf{ndana}]  ka  mahaj  =uwla     \\
    {\twoP}+{\POT}-take-{\twoP}  {\DAT} ={\oneS}.{\IO} =to    {spp. of tree}  {\DEM}  on       door         ={\oneS}.{\POSS}  \\
    
    \medskip
\gll aka\\
     on\\
\glt ‘You will bring the above-mentioned tree to my door for me.’
\z

\textit{Ndana} ‘the above-mentioned’ can also replace an entire thought. \REF{ex:3:68} is from line S. 34 of the Millet story. In this sentence, \textit{ndana}  ‘the above-mentioned’ is head of the noun phrase and refers to the entire preceding story of the disobedience and death of the girl. 

\ea \label{ex:3:68}\corpussource{Disobedient Girl, S. 34}\\
Waya  \textbf{ndana}  Hərmbəlom  ázata  aka  barka  ahan  va.\\
\gll  waja   ndana  Hʊrmbʊlɔm   á-z    =ata      =aka   barka     =ahaŋ   \\
      because   {\DEM}   God             \oldstylenums{3}\textsc{s}+{\IFV}-take  =\oldstylenums{3}\textsc{p}.{\IO}  =on   blessing  =\oldstylenums{3}\textsc{s}.{\POSS}  \\
      
      \medskip
\largerpage     
\gll =va\\
     ={\PRF}\\
\glt  ‘Because of the above-mentioned, God had taken back his blessing from them.’
\z

\subsection{Manner adverbial demonstratives}\label{sec:3.2.3}
\hypertarget{RefHeading1211001525720847}{}
Manner adverbial demonstratives have been described by \citet{Dixon2003} to function as non-inflecting modifiers to verbs. There are two types in Moloko, depending on how they are derived.\footnote{\citet{Dixon2003} notes that manner adverbial demonstratives are morphologically derived from nominal demonstratives. In Moloko they are derived from the nominal demonstrative, an adverb, or the adjectiviser.} 

The first type in Moloko is derived from the demonstrative by the addition of \textit{ka} ‘like.’ The adverbial demonstrative \textit{ka nehe} ‘like this’ \REF{ex:3:69} is used when the speaker indicates with hand or body movements how a particular action is carried out. It is derived from the proximal nominal demonstrative \textit{nehe} ‘this here’ (see \sectref{sec:3.2.2.1}).

\ea \label{ex:3:69}
Enjé  ele  ahan  dəren  \textbf{ka  nehe.}\\
\gll  à-ndʒ-ɛ  ɛlɛ  =ahaŋ    dɪrɛŋ  \textbf{ka}  \textbf{nɛhɛ}\\
      \oldstylenums{3}\textsc{s}+{\PFV}-leave-{\CL}  thing  =\oldstylenums{3}\textsc{s}.{\POSS}  far  like   this\\
\glt  ‘He went (lit. took his things away) far away like this.’  
\z

The adverbial demonstrative \textit{ka ndana} ‘like what was just said’ is used  in the reply (\ref{ex:3:70}b) to the statement in (\ref{ex:3:70}a). \textit{Ka ndana} is derived from the anaphoric demonstrative \textit{ndana} ‘the above-mentioned’ (see \sectref{sec:3.2.2.2}). \textit{Ka ndana} can be negated; compare the positive and negative replies in (\ref{ex:3:70}b) and (\ref{ex:3:71}b), respectively.

\ea \label{ex:3:70}
\ea Nəvəye  ngehe  na,  ngama  aməgəye  jerne  nə  eteme. \\
\gll  nɪvijɛ  ŋgɛhɛ    na    ŋgama   amɪ-g-ijɛ  dʒɛrnɛ   nə    ɛtɛmɛ  \\
      season  {\DEM}   {\PSP}    better    {\DEP}-do-{\CL}        garden  with      onion  \\
\glt  ‘This season I think it is better to grow onions.’ \\
\ex
N\'{ə}ɗəgalay  \textbf{ka  ndana}.\\
\gll n\'{ə}-ɗəgal-aj  \textbf{ka}  \textbf{ndana}\\
     {\oneS}+{\IFV}-think-{\CL}  like  {\DEM}\\
\glt ‘I think so too.’\\     
\z\z

\ea \label{ex:3:71}
\ea Nəvəye  ngehe  na,  ngama  aməgəye  jerne  nə  eteme.       \\
\gll  nɪvijɛ   ŋgɛhɛ  na   ŋgama    amɪ-g-ijɛ  dʒɛrnɛ  nə      ɛtɛmɛ  \\
      season  {\DEM}    {\PSP}    better   {\DEP}-do-{\CL}  garden  with     onion  \\
\glt  ‘This season I think it is better to grow onions.’        \\
\ex
 N\'{ə}ɗəgalay  \textbf{ka  ndana}  bay.\\
 \gll n\'{ə}-ɗəgal-aj      \textbf{ka}  \textbf{ndana}   baj\\
      {\oneS}+{\IFV}-think-{\CL}  like {\DEM}   {\NEG}\\
 \glt ‘I don’t think so.’
 \z\z

The second type of adverbial demonstrative in Moloko is derived from the adverb \textit{ehe} by the addition of the tag \textit{kəyga} ‘like that’ (see \sectref{sec:10.3.3}). \textit{Kəygehe} ‘like this’ will be accompanied by gestures demonstrating the place where the action will occur (\ref{ex:3:72}--\ref{ex:3:73}). 

\ea \label{ex:3:72}
Adəkwalay  ana  Hərmbəlom  ton  \textbf{kəygehe.}\\
\gll  à-dʊkʷ     =alaj  ana  Hʊrmbʊlɔm  tɔŋ    \textbf{kijgɛhɛ}\\
      \oldstylenums{3}\textsc{s}+{\PFV}-arrive   =away  {\DAT} God    \textsc{id}touch     {like this}\\
\glt  ‘It touched God like this [in the eye]. (lit. it arrived to God, touching [him] like this)’
\z

\ea \label{ex:3:73}
Lo  kəygehe.\\
\gll lo    kijgɛhɛ\\
      go[{\twoS}.{\IMP}]  {like this}\\
\glt  ‘Go that way [along that pathway].’
\z
\is{Deixis!Demonstratives and demonstrationals|)}

\section{Numerals and quantifiers}\label{sec:3.3}\is{Plurality!Numerals and quantifiers|(}
\hypertarget{RefHeading1211021525720847}{}
Three systems of numerals are found in Moloko:

\begin{itemize}
\item A base ten system for counting in isolation and for cardinal numbers (counting items excluding money, \sectref{sec:3.3.1}).
\item A base five system for counting money (\sectref{sec:3.3.2}).
\item A base ten system for ordinal numbers (ordering items with respect to one another, \sectref{sec:3.3.3}). 
\end{itemize}
\subsection{Cardinal numbers for items}\label{sec:3.3.1}
\hypertarget{RefHeading1211041525720847}{}
Cardinal numbers for counting items follow a base-ten system are shown in \tabref{tab:3.21}.

\begin{table}
\begin{tabular}{rl@{\hspace{4em}}rl}
\lsptoprule
1 & \textit{bəlen} & 21 & \textit{kokər cew hər bəlen}\\
2 & \textit{cew} & 30 & \textit{kokər makar}\\
3 & \textit{makar} & 100 & \textit{səkat}\\
4 & \textit{məfaɗ / əwfaɗ}\footnote{This numeral is pronounced either [\textit{məfaɗ}] or [\textit{uwfaɗ}]  by speakers from different regions.} & 101 & \textit{səkat nə bəlen}\\
5 & \textit{zlom} & 122 & \textit{səkat nə kokər cew hər cew}\\
6 & \textit{məko} & 200 & \textit{səkat cew}\\
7 & \textit{səsəre} & 300 & \textit{səkat makar}\\
8 & \textit{slalakar} & 1,000 & \textit{dəbo}\\
9 & \textit{holombo} & 1,001 & \textit{dəbo nə bəlen}\\
10 & \textit{kəro} & 1,100 & \textit{dəbo  nə səkat}\\
11 & \textit{kəro hər bəlen} & 2,000 & \textit{dəbo cew}\\
12 & \textit{kəro hər cew} & 3,000 & \textit{dəbo makar}\\
13 & \textit{kəro hər makar} & 5,000 & \textit{dəbo zlom}\\
14 & \textit{kəro hər məfaɗ} & 10,000 & \textit{dəbo kəro}\\
15 & \textit{kəro hər zlom} & 10,001 & \textit{dəbo kəro nə bəlen}\\
16 & \textit{kəro hər məko} & 100,000 & \textit{dəbo dəbo səkat}\\
17 & \textit{kəro hər səsəre} & 100,001 & \textit{dəbo dəbo səkat nə bəlen}\\
18 & \textit{kəro hər slalakar} & 1,000,000 & \textit{dəbo dəbo dəbo}\\
19 & \textit{kəro hər holombo} & 1,000,001 & \textit{dəbo dəbo dəbo nə bəlen}\\
20 & \textit{kokər cew} &  & \\
\lspbottomrule
\end{tabular}

\caption{\label{tab:3.21}Cardinal numerals for counting items}
\end{table}

Numbers used for counting in isolation are identical to the system shown in \tabref{tab:3.21}. When modifying a noun, the numerals follow the noun in a noun phrase (\ref{ex:3:74}--\ref{ex:3:75}). The consitiutent order of the noun phrase is discussed in \sectref{sec:5.1}.\footnote{The first line in each example is the orthographic form. The second is the phonetic form (slow speech) with morpheme breaks. Examples in the tables are written in orthography unless otherwise specified.}

\ea \label{ex:3:74}
Məze  ahay  dəbo  cew  tolo  aməmənjere  məkəɗe  balon.\\
\gll  mɪʒɛ    =ahaj  dəbɔ   tʃɛw  t\`{ɔ}-lɔ    amɪ-mɪnzɛr-ɛ  mɪ-kɪɗ-ɛ    balɔŋ\\
      person  =Pl  1000  two  \oldstylenums{3}\textsc{p}+{\PFV}-go  {\DEP}-see-{\CL}  {\NOM}-kill-{\CL}  ball\\
\glt  ‘Two thousand people went to see the football game.’
\z
\clearpage
\ea \label{ex:3:75}
Nəmənjar  awak  ahay  kəro  a  kosoko  ava.\\
\gll  n\`{ə}-mənzar  awak  =ahaj  kʊrɔ  a  kɔsɔkʷɔ  ava\\
      {\oneS}+{\PFV}-see  goat  =Pl  10  at  market  in\\
\glt  ‘I saw ten goats at the market.’
\z

The numerals can stand as head of a noun phrase in a clause (\ref{ex:3:76}--\ref{ex:3:77}) but the immediate context must give the referent. In (\ref{ex:3:76}b), the response to the question in (\ref{ex:3:76}a) need only give the number. 

\ea\label{ex:3:76}
\ea
Kənjakay  awak  mətəmey?  \\
\gll  k\`{ə}-nzak-aj     awak   mɪtɪmɛj  \\
      {\twoS}+{\PFV}-find-{\CL}  goat  {how many}    \\
\glt  ‘How many goats did you find?’   \\
\ex
Nənjakay  bəlen.\\
\gll  nə-nzak-aj   bɪlɛŋ\\
      {\oneS}-find-{\CL}  one\\
\glt  ‘I found one.’
\z\z

\ea \label{ex:3:77}
Babəza  əwla  ahay  na,  cew.\\
\gll  babəza   =uwla    =ahaj  na  tʃɛw\\
      children  ={\oneS}.{\POSS}  =Pl  {\PSP}  two\\
\glt ‘I have two children.’ (lit. my children, two)
\z

\tabref{tab:3.21} shows that the numbers one to ten are unique. The numbers eleven through nineteen are composites of ten plus one, ten plus two, etc. The word to indicate ‘plus’ is \textit{hər}, which has no other meaning in the language. Twenty is \textit{kokər cew}, which is some kind of derivitave of \textit{kəro}  ‘ten.’ After 100, numbers are made of a coordinate noun phrase composed of  \textit{səkat} ‘one hundred,’ the adposition \textit{nə} ‘with,’ and a second number. One thousand is \textit{dəbo}, and higher numbers are seen as multiples of \textit{dəbo}.

There is a culturally governed exception to the use of cardinal numbers in Moloko.  To give the age of a one year old child, a Moloko speaker will say \textit{məvəye daz} (not *\textit{məvəye} \textit{bəlen} ‘year one’). \textit{Məvəye daz} means that the child has lived through one Moloko New Year (celebrated in September). We found no other meaning for the word \textit{daz}  apart from its use here. 

\subsection{Numbers for counting money}\label{sec:3.3.2}
\hypertarget{RefHeading1211061525720847}{}
Money is counted using two different systems which overlap (see \tabref{tab:3.22}). A base-five system is used for amounts under about 250 Central African Francs (Fcfa). Many languages in Cameroon use a base five system for counting money. The reason for its use is probably based on the fact that the smallest coin was worth 5 Fcfa, and it became the basic unit for monetary transactions.\footnote{The generic term for money in Moloko is \textit{dala}, possibly a borrowed term from the American dollar.} Ten francs, being two of these coins, is \textit{dal cew} ‘two coins,’ fifteen francs is \textit{dal makar} ‘three coins,’ and so on (the values for the other coins that were available are indicated in the left column of \tabref{tab:3.22}). The system becomes awkward for higher amounts (above 50 coins, or 250 Fcfa) because of the high numbers, and so a base ten system is superimposed (right column of \tabref{tab:3.22}). Between 100 Fcfa and 250 Fcfa, both the base five and base ten are used, although within the Moloko mountain region, the base five system predominates. 

\begin{table}
\resizebox{\textwidth}{!}{\begin{tabular}{r@{ }lll}
\lsptoprule
\multicolumn{2}{c}{Amount of money} & {Base five system} & {‘Base ten’ system}\\
\midrule
5 & Fcfa (coin) & \textit{səy say} & \\
10 & Fcfa (coin) & \textit{dal cew} & \\
15 & Fcfa & \textit{dal makar} & \\
50 & Fcfa (coin) & \textit{dal kəro} & \\
100 & Fcfa (coin) & \textit{dal kokər cew} & \textit{(səloy) səkat}\\
150 & Fcfa & \textit{dal kokər makar} & \textit{səloy st nə dal kəro}\\
200 & Fcfa & \textit{dal kokər məfaɗ} & \textit{səkat cew}\\
250 & Fcfa & \textit{dal kokər zlom} & \textit{səkat cew dal kəro}\\
300 & Fcfa &  & \textit{səkat makar}\\
500 & Fcfa (coin) &  & \textit{səkat zlom}\\
1,000 & Fcfa (bill) &  & \textit{ombolo}\\
2,000 & Fcfa (bill) &  & \textit{ombolo cew}\\
3,250 & Fcfa &  & \textit{ombolo makar nə səloy kokər zlom}\\
5,000 & Fcfa (bill) &  & \textit{ombolo zlom}\\
10,000 & Fcfa (bill) &  & \textit{ombolo kəro}\\
50,000 & Fcfa &  & \textit{ombolo kokər zlom}\\
100,000 & Fcfa &  & \textit{ombolo səkat}\\
1,000,000 & Fcfa &  & \textit{ombolo səkat kəro}\\
\lspbottomrule
\end{tabular}}
\caption{Numbers for money}\label{tab:3.22}
\end{table}

The basic unit for the monitary base ten system is the 100 Fcfa coin (\textit{səloy} \textit{səkat } ‘coin 100’). This system uses the same number for one hundred as the system for counting items (\textit{səkat}). Ten of these coins make the 1000 Fcfa bill, so not unexpectedly, the term for the 1000 Fcfa bill is not the same as the number ‘1000’ for counting non-money items (\textit{dəbo} see \tabref{tab:3.21}), but rather is a term specific to money -- \textit{ombolo}. 

When larger amounts of money are counted, both base ten and base five systems are used. For example, 13,250 Fcfa is \textit{ombolo kəro hər makar nə səloy kokər zlom} ‘thirteen thousand Fcfa (base ten) and fifty 5 Fcfa coins (base five)’ (lit. 13 thousand with 50 5Fcfa coins).

It is interesting that recently, a one franc coin has been made available in Cameroon. The term for this coin wasn’t in the original counting system where the 5 Fcfa coin was the basic unit. It is now called [el\'{e} bəlen] literally ‘one eye.’ 

\subsection{Ordinal numbers}\label{sec:3.3.3}
\hypertarget{RefHeading1211081525720847}{}
Only the first ordinal number is a unique vocabulary word in Moloko: \textit{cekem} ‘first’ \REF{ex:3:78}.The other ordinal expressions use a noun phrase construction using the cardinal counting numbers (\ref{ex:3:76}--\ref{ex:3:77}, cf. \tabref{tab:3.21}):

\ea \label{ex:3:78}
cekem\\
      tʃɛkɛm\\
\glt  ‘first’
\z

\ea \label{ex:3:79}
anga  baya cew\\
\gll  aŋga baja   tʃɛw\\
      {\POSS}   time     two\\
\glt  ‘second’
\z

\ea \label{ex:3:80}
anga  baya  makar\\
\gll  aŋga   baja   makar\\
      {\POSS}   time     three\\
\glt  ‘third’ 
\z

\subsection{Non-numeral quantifiers}\label{sec:3.3.4}
\hypertarget{RefHeading1211101525720847}{}
Non-numeral quantifiers\footnote{Some of these quantifiers can also pattern as adverbs, e.g., \textit{gam} ‘much.’ \REF{ex:3:107}.} include \textit{gam} ‘much' (\ref{ex:3:81}--\ref{ex:3:82}), \textit{nekwen} ‘little,’ \textit{jəyga} ‘all,’ \textit{dəyday} ‘approximately,’ and \textit{haɗa} ‘enough’ (\ref{ex:3:83}--\ref{ex:3:84}). When they occur in a noun phrase, they are the final element (\ref{ex:3:81}, the noun phrase is delimited by square brackets).  

\ea \label{ex:3:81}
[Məze  ahay  \textbf{gam}]  təlala  afa  ne.\\
\gll  [mɪʒɛ =ahaj  \textbf{gam}]  tə-l=ala    afa    nɛ\\
      people  =Pl  much  \oldstylenums{3}\textsc{p}-go=to  {at house of}  {\oneS}\\
\glt  ‘Many people came to my house.’
\z

\ea \label{ex:3:82}
 Slərele  \textbf{gam!}\\
\gll  ɬɪrɛlɛ   \textbf{gam}\\
      work  much\\
\glt  [That is] a lot of work!
\z

\ea\label{ex:3:83}\corpussource{Disobedient Girl, S. 4}\\
Ávata  [məvəye  \textbf{haɗa}].\\
\gll  á-v=ata    [mɪ-v-ijɛ     \textbf{haɗa}]\\
      \oldstylenums{3}\textsc{s}+{\IFV}-{spend time}=\oldstylenums{3}\textsc{p}.{\IO}   {\NOM}{}-{spend time}-{\CL}     enough\\
\glt  ‘It would last them enough for the whole year.’  
\z

\ea \label{ex:3:84}
Nok [\textbf{haɗa}  bay.]\\
\gll  nɔkʷ  [\textbf{haɗa}   baj]\\
      {\twoS}    enough  {\NEG}\\
\glt  ‘You [are] small.’ (lit. not enough)
\z
\is{Plurality!Numerals and quantifiers|)}
\section{Existentials}\label{sec:3.4}
\hypertarget{RefHeading1211121525720847}{}
Moloko has three positive existentials and one negative existential. The prototypical existential \textit{aba} ‘there exists’ \REF{ex:3:85}. carries the most basic idea of existence. Its negative is \textit{abay} ‘there does not exist.’\footnote{This existential is perhaps a compound of the existential \textit{aba} and the negative \textit{bay}.} The locational existential \textit{ava} ‘there exists in a particular place,’ and the possessive existential \textit{aka} ‘there exists associated with’ each carry the concept of existence along with their own specific meaning. The possessive existential must be accompanied by a indirect object pronominal.  

Existentials are verb-like and fill the verb slot in a clause, but are not conjugated for aspect or mood and do not take subject or direct object pronominals. Some of the existentials can carry verbal extensions or indirect object pronominals. The existential clause contains few elements -- most commonly just a subject and the existential. The existential clause can be in a presupposition construction (\chapref{chap:11}) or interrogative construction (\sectref{sec:10.3}). 

The prototypical existential is \textit{aba} ‘there is’ (\ref{ex:3:85}--\ref{ex:3:86}) and its negative is \textit{abay} ‘there is none’ (\ref{ex:3:87}--\ref{ex:3:88}). A clause with one of these existentials requires a subject but there are no other core participants or obliques. The existential is bolded in the examples.\footnote{The first line in each example is the orthographic form. The second is the phonetic form (slow speech) with morpheme breaks.} 

\ea \label{ex:3:85}
{Məze}  \textbf{aba.}\\
\gll mɪʒɛ \textbf{aba} \\
      person  {\EXT}\\
\glt ‘There was a man \ldots' (a common beginning to a story)
\z

\ea \label{ex:3:86}
Babəza  əwla  ahay  \textbf{aba.}\\
\gll  babəza   =uwla    =ahaj  \textbf{aba}\\
      children  ={\oneS}.{\POSS}  =Pl  {\EXT}\\
\glt  ‘I have children.’ (lit. my children exist)
\z

\ea \label{ex:3:87}
Babəza  əwla  ahay  \textbf{abay.}\\
\gll  babəza  =uwla    =ahaj  \textbf{abaj}\\
      children  ={\oneS}.{\POSS}  =Pl  {\EXT}+{\NEG}\\
\glt  ‘I have no children.’  (lit. my children do not exist)
\z

\ea \label{ex:3:88}
Dala  \textbf{abay.}\\
\gll  dala    \textbf{abaj}\\
      money  {\EXT}+{\NEG}\\
\glt  ‘I have no money.’ (lit. there is no money)
\z

The existentials \textit{aba} and \textit{abay} can also carry an extended sense to indicate the health of the person. (\ref{ex:3:89}a) and \REF{ex:3:90} are greetings, which are questions that can occur with (\ref{ex:3:89}a) or without \REF{ex:3:90} the word \textit{zay} ‘peace.’ (\ref{ex:3:89}b) and \REF{ex:3:91} are possible replies to either of these questions. Likewise, (\ref{ex:3:92}--\ref{ex:3:93}) show inquiries and possible replies as to the health of a third person.

\ea \label{ex:3:89}
\ea Nok \textbf{aba} zay ɗaw? \\
\gll nɔkʷ  \textbf{aba}    zaj  ɗaw        \\
      {\twoS}    {\EXT}    peace  {\QUEST}       \\
\glt  ‘Are you well?’ (lit. ‘Do you exist [in] peace?’)\\

\medskip
\ex
Ne \textbf{aba}. \\
\gll nɛ \textbf{aba} \\
     {\oneS} \EXT \\
\glt ‘I am well.’ (lit. I exist)\\
\z\z

\ea \label{ex:3:90}
Nok \textbf{aba}  ɗaw?\\
\gll  nɔkʷ  \textbf{aba}    ɗaw\\
      {\twoS}    {\EXT}    {\QUEST}\\
\glt  ‘Are you well?’ (lit. ‘Do you exist?’)
\z

\ea \label{ex:3:91}
Asak  əwla  \textbf{abay.}\\
\gll  asak    =uwla    \textbf{abaj}\\
      foot    ={\oneS}.{\POSS}  {\EXT}+{\NEG}\\
\glt  ‘My foot hurts.’ (lit. my foot doesn’t exist)
\z

\ea \label{ex:3:92}
\ea Baba  ango \textbf{aba} ɗaw?  \\
\gll baba   =aŋgʷɔ     \textbf{aba}   ɗaw   \\
      father  ={\twoS}.{\POSS}  {\EXT}  {\QUEST}   \\
\glt ‘Is your father well?’  (lit. does your father exist?)  \\

\medskip
 \ex
Ayaw,  ndahan \textbf{aba.}\\
 \gll ajaw   ndahaŋ  \textbf{aba}\\
      yes  \oldstylenums{3}\textsc{s}  {\EXT}\\ 
 \glt ‘Yes, he is well.’ (lit. yes, he exists)\\
\z\z

\ea \label{ex:3:93} 
Baba  əwla  na,  hərva  ahan \textbf{abay.}\\
\gll baba    =uwla      na  hərva  ahaŋ      \textbf{abaj}\\
      father  ={\oneS}.{\POSS}  {\PSP}  body  =\oldstylenums{3}\textsc{s}.{\POSS}    {\EXT}+{\NEG}\\
\glt ‘My father is sick.’ (lit. my father, his body doesn’t exist)
\z

The existential \textit{aba} is also used in presentational clauses in a narrative to introduce some major participants in the setting. \REF{ex:3:94} is the introduction to the Cicada story. 

\ea \label{ex:3:94}\corpussource{Cicada, S. 3-4}\\
Albaya  ahay  \textbf{aba.}  Tánday  t\'{ə}talay  a  ləhe.\\
\gll  albaja  =ahaj  \textbf{aba}  tá-ndaj    t\'{ə}-tal-aj    a  lɪhɛ\\
      {young man}   =Pl  {\EXT}  \oldstylenums{3}\textsc{p}+{\IFV}-{\PRG}     \oldstylenums{3}\textsc{p}+{\IFV}-walk-{\CL}  at    bush\\
\glt  ‘There were some young men. They were walking in the bush.’
\z

In some presentational clauses both the prototypical existential and the locational existential can co-occur.  \REF{ex:3:95} is from the setting of a story story. Note that this existential clause contains the adverb \textit{ete} ‘also.’

\ea \label{ex:3:95}
Albaya  \textbf{ava}  \textbf{aba}  ete.  Olo  azala  hor.\\
\gll  albaja  \textbf{ava}    \textbf{aba}  ɛtɛ   \`{ɔ}-lɔ    à-z=ala    hʷɔr\\
      {young man}  {\EXT}+in   {\EXT}  also  \oldstylenums{3}\textsc{s}+{\PFV}-go    \oldstylenums{3}\textsc{s}+{\PFV}-take=to    woman\\
\glt  ‘And so, there once was a young man (in a particular place). He went and took a wife.’
\z

The locational existential \textit{ava} ‘there is in’ (\ref{ex:3:96}--\ref{ex:3:99}) expresses existence ‘in’ a particular location. This existential is the same as the adpositional verbal extension \textit{=ava}\is{Adpositionals} ‘in’ (see \sectref{sec:7.5.1}) and locational postposition \textit{ava} ‘in’ (see \sectref{sec:5.6.2}), all of which express the location in something, either physically or figuratively. In some of the examples below, a response is included which also employs the same existential. Note that the existential in \REF{ex:3:99} carries the directional ‘away from’ (see \sectref{sec:7.5.2}).

\ea\label{ex:3:96}
\ea Sese  \textbf{ava}  ɗaw? \\
\gll  ʃɛʃɛ    \textbf{ava}  ɗaw   \\
      meat  {\EXT}+in  {\QUEST}   \\
\glt  ‘Is there any meat located here [for sale]?’  \\

\medskip
\ex
Ayaw,  sese  \textbf{ava.}\\
\gll ajaw  ʃɛʃɛ  \textbf{ava}\\
     yes  meat  {\EXT}+in\\
\glt ‘Yes, we have meat located here.’\\
\z\z

\ea\label{ex:3:97}
\ea Baba  ango,  ndahan  \textbf{ava}  ɗaw? \\
\gll  baba  =aŋgʷɔ      ndahaŋ    \textbf{ava}  ɗaw \\
      father  ={\twoS}.{\POSS}    \oldstylenums{3}\textsc{s}        {\EXT}+in   {\QUEST} \\
\glt  `Is your father located here?' (lit. your father, is he here?)     
       \\\medskip
       \ex
 Ndahan  \textbf{ava}  bay;  enjé  amətele.\\
\gll ndahaŋ  \textbf{ava}  baj    \`{ɛ}-nʒɛ    amɪ-tɛl-ɛ\\
        \oldstylenums{3}\textsc{s}       {\EXT}+in  {\NEG}   \oldstylenums{3}\textsc{s}+{\PFV}-left {\DEP}-travel-{\CL}\\
\glt   `No, he is not located here; he went somewhere.'\\
\z\z%%\todo{We need to adjust example number spacing - yes }
\clearpage
\ea\label{ex:3:98}
Ndahan  \textbf{ava.}\\
\gll  ndahaŋ  \textbf{ava}\\
      \oldstylenums{3}\textsc{s}    {\EXT}+in\\
\glt  ‘He/she is here.’
\z

\ea \label{ex:3:99}
Ndahan  \textbf{ava}  alay.\\
\gll  ndahaŋ  \textbf{ava}  =alaj\\
      \oldstylenums{3}\textsc{s}    {\EXT}+in  =away\\
\glt  ‘He/she is located at the place of reference.’ (lit. he is in away)
\z
 
The possessive existential \textit{aka} ‘there is on’ (\ref{ex:3:100}--\ref{ex:3:103}, \ref{ex:3:105}) expresses existence ‘on’ a person (indicating possession or accompaniment).  This existential is the same as the adpositional verbal extension =\textit{aka}\is{Adpositionals} ‘on’ (see \sectref{sec:7.5.1}) and locational postposition \textit{aka} ‘on’ (see \sectref{sec:5.6.2}), all of which express location on something, whether physically or figuratively. The subject of the possessive existential (the possessed item) is followed by a construction consisting of the indirect object pronominal cliticised to the particle \textit{an-}, in turn followed by the possessive existential \textit{aka} ‘on.’ The particle \textit{an- } is the same particle to which the indirect object pronominal cliticises when there is a suffix on the verb stem (see \sectref{sec:7.3.1.1}) and these elements are found in the same order as they are within the verb complex. \REF{ex:3:100} shows a question and response pair.

\ea \label{ex:3:100}
\ea Dala  anok \textbf{aka}  ɗaw? \\
\gll  dala   an=ɔkʷ     \textbf{aka}     ɗaw  \\
      money  {\DAT}={\twoS}.{\IO}  {\EXT}+on    {\QUEST}  \\
\glt  ‘Do you have any money [located] with you?’ (lit. is there money on you?)\\

\medskip
\ex
   Ayaw,  dala  anaw  \textbf{aka.}\\
\gll ajaw  dala   an  =aw   \textbf{aka}\\
   yes  money  {\DAT} ={\oneS}.{\IO}  {\EXT}+on\\
\glt ‘Yes, I have money [located] on me.’ 
\z\z

\ea \label{ex:3:101}
Hor  anan  \textbf{aka} ana  Mana.\\
\gll  hʷɔr   an=aŋ     \textbf{aka} ana   Mana\\
      woman  {\DAT}=\oldstylenums{3}\textsc{s}.{\IO}  {\EXT}+on    {\DAT} Mana\\
\glt  ‘He has a wife.’ (lit. a woman to him there is on for Mana)
\z

The existential \textit{aka} can also be used to mean accompaniment \REF{ex:3:102}.

\ea \label{ex:3:102}
 Bahay  a  sla  ahay  na,  ndahan  \textbf{aka}  ɗaw?\\
\gll  bahaj  a  ɬa  =ahaj  na  ndahaŋ  \textbf{aka}  ɗaw\\
      chief  {\GEN}  cow  =Pl  {\PSP}  \oldstylenums{3}\textsc{s}  {\EXT}  {\QUEST}\\
\glt  ‘Was the owner of the cows [located] with [you]?' (lit. the chief of the cows, was he ‘on’?)
\z

The locational existential \textit{aka}  (\ref{ex:3:103}, \ref{ex:3:105}) can also fill the same role as the verb \textit{nday} (\ref{ex:3:104}, see \sectref{sec:8.2.1}) to express an action in progress\is{Tense, mood, and aspect!Progressive}. This usage of \textit{aka} may be due to adoption of a similar particle in \ili{Fulfulde}\il{Fulfulde}, the language of wider communication in the region. The particle \textit{don} in Adamawa \ili{Fulfulde} has a present progressive and existential use similar to \textit{aka} in Moloko (%%%
%%% \citealt[58, 73]{Noye1974}; %% The author no longer wants to cite Noye here.
Edward Tong\ia{Tong, Edward}, personal communication).

\ea \label{ex:3:103}
Ndahan \textbf{aka} ózom ɗaf.\\
\gll  ndahaŋ   \textbf{aka}     \'{ɔ}-zɔm     ɗaf\\
      \oldstylenums{3}\textsc{s}    {\EXT}+on    \oldstylenums{3}\textsc{s}+{\IFV}-eat  {millet loaf}\\
\glt  ‘He/she is eating millet loaf.’
\z

\ea \label{ex:3:104}
Ánday  ózom  ɗaf.\\
\gll  á-ndaj    \'{ɔ}-zɔm     ɗaf\\
      \oldstylenums{3}\textsc{s}+{\IFV}-{\PROG}  \oldstylenums{3}\textsc{s}+{\IFV}-eat  {millet loaf}\\
\glt  ‘He/she is eating millet loaf.’
\z

\ea \label{ex:3:105}\corpussource{Disobedient Girl, S. 24}\\
Ndahan  na,  ndahan    \textbf{aka}  njəw  njəw  njəw.\\
\gll ndahaŋ   na    ndahaŋ   \textbf{aka}          {nzuw  nzuw  nzuw}\\
      \oldstylenums{3}\textsc{s}                   {\PSP}    \oldstylenums{3}\textsc{s}           {\EXT}+on       \textsc{id}grind\\
\glt  ‘And she, she is grinding some more.'
\z

\section{Adverbs}\label{sec:3.5}
\hypertarget{RefHeading1211141525720847}{}
\largerpage Some adverbs modify verbs within the verb phrase (simple or derived, Sections \ref{sec:3.5.1} and \ref{sec:3.5.2}, respectively), others modify the clause as a whole (temporal adverbs, \sectref{sec:3.5.3}), and yet others function at the discourse level (\sectref{sec:3.5.4}). Note that ideophones can function adverbially to give pictoral vividness to a clause \citep{Doke1935}. Because they pattern differently than adverbs, they are considered in their own section (\sectref{sec:3.6}).

\subsection{Simple verb phrase-level adverbs}\label{sec:3.5.1}
\hypertarget{RefHeading1211161525720847}{}
Verb phrase adverbs give information concerning the location, quality, quantity, or manner of the action expressed in the verb phrase. These adverbs occur after any adpositional phrases (\ref{ex:3:106}--\ref{ex:3:108}).\footnote{The first line in each example is the orthographic form. The second is the phonetic form (slow speech) with morpheme breaks.} The negative follows any such adverbs in negative clauses (\sectref{sec:10.2}). 


\ea \label{ex:3:106}
\corpussource{Disobedient Girl, S. 4}\\
Təwasava  \textbf{neken} kəygehe.\\
\gll t\`{ə}-was                  =ava         \textbf{nɛkʷɛŋ}  kijgɛhɛ\\
      \oldstylenums{3}\textsc{p}+{\PFV}-cultivate =in       little    {like this}\\
\glt  ‘They cultivated a little like this.’
\z

\ea \label{ex:3:107}
Hərmbəlom  andaɗay  nok  \textbf{gam.}\\
\gll  Hʊrmbʊlɔm  a-ndaɗ-aj  nɔkʷ  \textbf{gam}\\
      God    \oldstylenums{3}\textsc{s}-love-{\CL}  {\twoS}  much\\
\glt  ‘God loves you a lot.’
\z

\ea \label{ex:3:108}
Názaɗ  a  dəray  ava  \textbf{sawan.}\\
\gll  ná-zaɗ    a  dəraj  ava    \textbf{sawaŋ}\\
      {\oneS}+{\IFV}-carry    at  head  in    {without help}\\
\glt  ‘I can carry it (on my head) by myself!’
\z

Verb phrase adverbs include \textit{dəren} ‘far distance,’ \textit{nekwen} ‘a small quantity’ \REF{ex:3:106}\textit{, gam }‘a large quantity’ \REF{ex:3:107}, \textit{sawan} ‘without help’ \REF{ex:3:108} and the modal adverbs \textit{təta} ‘can,’ an adverb of ability (\ref{ex:3:109} and \ref{ex:3:110}), and \textit{dewele } ‘ought,’ an adverb of necessity \REF{ex:3:111}. 

\ea \label{ex:3:109}
Kázala  \textbf{təta.}\\
\gll  ká-z      =ala    \textbf{təta}\\
      {\twoS}+{\IFV}-carry    =to    ability\\
\glt  ‘You can carry it.’
\z

\ea \label{ex:3:110}
Bahay  ázom  sese  \textbf{təta.}\\
\gll  bahaj  á-zɔm    ʃɛʃɛ  \textbf{təta}\\
      chief  \oldstylenums{3}\textsc{s}+{\IFV}-eat  meat  ability\\
\glt  ‘The chief can eat meat.’
\z

\ea \label{ex:3:111}
Bahay  ázom  sese  \textbf{dewele.}\\
\gll  bahaj   á-zɔm    ʃɛʃɛ  \textbf{dɛwɛlɛ}\\
      chief  \oldstylenums{3}\textsc{s}+{\IFV}-eat  meat  necessary\\
\glt  ‘The chief must eat meat.’
\z

The simple adverbs expressing location, quantity, quality, and manner can be intensified by reduplication (or lengthening) of a consonant or reduplication of the entire adverb.\footnote{Adverbs of ability and necessity cannot be reduplicated, nor can adverbs which function beyond the verb phrase level. }  (\ref{ex:3:112}--\ref{ex:3:115}) show the simple adverb with its intensified counterpart. The reduplication of a consonant occurs at the onset of the final syllable (\ref{ex:3:112} and \ref{ex:3:113}). The entire adverb is reduplicated in \REF{ex:3:114} and \REF{ex:3:115}. Intensified adverbs cannot be negated. 

\ea \label{ex:3:112}
\gll dəren \hspace{20pt}  də\textbf{rr}en\\
      dɪrɛŋ  \hspace{20pt}   dɪ\textbf{rr}ɛŋ\\
\glt  ‘far’  \hspace{27pt}  ‘very far’
\z

\ea \label{ex:3:113}
\gll ɗeɗen   \hspace{20pt}   ɗe\textbf{ɗɗ}en\\
      ɗɛɗɛŋ  \hspace{20pt}    ɗɛ\textbf{ɗɗ}ɛŋ\\
\glt  ‘true’  \hspace{25pt}   ‘very true’
\z

\ea \label{ex:3:114}
gam  \hspace{30pt}    gam gam\\
\glt  ‘a lot’ \hspace{25pt}     ‘a whole lot’
\z

\ea \label{ex:3:115}
nekwen   \hspace{15pt}   nekwen nekwen\\
  nɛkʷœŋ   \hspace{15pt}   nɛkʷœŋ nɛkʷœŋ\\
\glt  ‘little’  \hspace{25pt}   ‘a little at a time’
\z

\subsection{  Derived verb phrase-level adverbs}\label{sec:3.5.2}\is{Derivational processes!Noun to adverb}
\hypertarget{RefHeading1211181525720847}{}
Verb phrase adverbs can be derived from nouns by reduplicating the final consonant of the noun and adding [a] (i.e. Ca where the C is the final consonant of the noun). The reduplicated syllable is labelled ‘adverbiser’ ({\ADV})\footnote{We have not found the term ‘adverbiser’ in the literature. Adverbiser in this work is defined as a derivational morpheme whose presence changes the grammatical class of a stem to become an adverb.} in (\ref{ex:3:116}--\ref{ex:3:117}). Compare the noun and its derived adverb in \REF{ex:3:116} and \REF{ex:3:117}. Note that the reduplicated consonant in the derived adverb in example \REF{ex:3:116} is the word-final allophone [x] rather than word-medial [h]. Likewise, example \REF{ex:3:117} shows [ŋ] rather than [n]. These word-final changes (see \sectref{sec:2.6.1}) in the reduplicated consonant indicate that the reduplication occurs after phonological word-final changes are made and that the reduplicated segment is phonologically bound to the noun (see \sectref{sec:2.6.2}). 

\ea \label{ex:3:116}
zayəh \hspace{20pt} \textbf{zayəhha}\\
\gll  zajəx \hspace{23pt}   zajəx=xa\\
      care  \hspace{23pt}    {care     ={\ADV}}\\
\glt  ‘care’  \hspace{21pt}  ‘carefully’
\z

\ea \label{ex:3:117}
deden  \hspace{20pt}  \textbf{dedenna}\\
\gll  dɛdɛŋ   \hspace{22pt} \textbf{dɛdɛŋ=ŋa}\\
      truth  \hspace{22pt}  {truth    ={\ADV}}\\
\glt  ‘truth’  \hspace{20pt}  ‘truthfully’  
\z

Note especially \REF{ex:3:118} and \REF{ex:3:119} which illustrate that the labialisation prosody on the nouns \textit{rəbok}  and \textit{hərək}  does not spread rightwards to the adverbiser (otherwise, the reduplicated /k/ would be labialised, see \sectref{sec:2.1}). 

\ea \label{ex:3:118}
zar  akar  ɗəw,  ndahan  ava  \textbf{rəbokka}\\
\gll  zar     akar  ɗuw  ndahaŋ  ava  \textbf{rʊbɔkʷ} \textbf{=ka}\\
      man    theft  also  \oldstylenums{3}\textsc{s}  {\EXT}{}+in  {hiding place}  ={\ADV}\\
\glt  ‘There was (in that place) a thief, hiding.’ 
\z

\ea \label{ex:3:119}
nege  slərwle  \textbf{hərəkka}\\
\gll  nɛ-g-ɛ  ɬɪrɛlɛ  \textbf{hʊrʊkʷ} \textbf{=ka}\\
      {\oneS}-do-{\CL}  work  {all day}  ={\ADV}\\
\glt  ‘I worked all day.’
\z

\subsection{Clause-level adverbs}\label{sec:3.5.3}
\hypertarget{RefHeading1211201525720847}{}
Temporal adverbs modify the clause as a whole and can occur clause initial or verb phrase final (\ref{ex:3:120} and \ref{ex:3:121},\footnote{Note that \textit{a kosoko ava} ‘in the market’ is a complex adpositional phrase (see \sectref{sec:5.6.2}).} respectively).\footnote{The order of constituents in the verb phrase is given in \chapref{chap:8}.} These include \textit{egəne} ‘today,’ \textit{hajan} ‘tomorrow,’ \textit{apazan} ‘yesterday.’ 

\clearpage
\ea \label{ex:3:120}
\textbf{Egəne} nólo  a  kosoko  ava.\\
\gll  \textbf{ɛgɪnɛ}  n\'{ɔ}-lɔ     a  kɔsɔkʷɔ  ava\\
      today  {\oneS}+{\IFV}-go  at  market  in\\
\glt  ‘Today I will go to the market.’
\z

\ea \label{ex:3:121}
Nólo  a  kosoko  ava \textbf{hajan.}\\
\gll  n\'{ɔ}-lɔ   a  kɔsɔkʷɔ  ava  \textbf{hadʒaŋ}\\
      {\oneS}+{\IFV}-go  at  market  in  tomorrow\\
\glt  ‘I will go to the market tomorrow.’
\z

\subsection{Discourse-level adverbs}\label{sec:3.5.4}
\hypertarget{RefHeading1211221525720847}{}
Discourse adverbs function at the clause combining level. Grammatically they are found verb phrase final. Semantically they tell something of the relation of their clause to what has happened before in the discourse. Discourse adverbs can neither be negated nor intensified by reduplication. They include \textit{ese} ‘again’ (same actor, same action, \ref{ex:3:122}), \textit{ete} ‘also’ (same action, different actor, \ref{ex:3:123}), \textit{fan} ‘already’ (expressing Perfect %%\is{Tense, mood, and aspect!Perfect} 
aspect in that the action is performed in the past with effects continuing to the present, \ref{ex:3:124}), \textit{kəlo} ‘already’ or ‘before’ (the action was performed at least once before a particular time, \ref{ex:3:125}). 

\ea \label{ex:3:122}
Nóolo  \textbf{ese.}\\
\gll  n\'{ɔ}\'{ɔ}-lɔ     \textbf{ɛʃɛ}\\
      {\oneS}+{\POT}-go    again\\
\glt  ‘I will go again.’
\z

\ea \label{ex:3:123}
Nóolo  \textbf{ete.}\\
\gll  n\'{ɔ}\'{ɔ}-lɔ     \textbf{ɛtɛ}\\
      {\oneS}+{\POT}-go    also\\
\glt  ‘I will go too.’
\z

\ea \label{ex:3:124}
Nege  na  \textbf{fan.}\\
\gll  n\`{ɛ}-g-ɛ    na  \textbf{faŋ}\\
      {\oneS}+{\PFV}-do-{\CL}  \oldstylenums{3}\textsc{s}.{\DO}  already\\
\glt  ‘I did it already.’
\z

\ea \label{ex:3:125}
Nəmənjar  ndahan  \textbf{kəlo.}\\
\gll  nə-mənzar    ndahaŋ  \textbf{kʊlɔ}\\
      {\oneS}-see  \oldstylenums{3}\textsc{s}  before\\
\glt  ‘I have seen him/her before.’
\z

The adverb \textit{əwɗe} ‘first’ \REF{ex:3:126} indicates that the event expressed in the clause occurs before something else. 

\ea \label{ex:3:126}\corpussource{Cicada, S. 20}\\
Náamənjar  na  alay  memele  ga  ndana \textbf{  əwɗe.}\\
\gll náá-mənzar    na   =alaj    mɛmɛlɛ  ga  ndana  \textbf{uwɗɛ}\\
      {\oneS}+{\POT}-see \oldstylenums{3}\textsc{s}.{\DO}   =away    tree  {\ADJ}  {\DEM}  {before something else}\\
\glt  ‘First let me go and see that tree that you spoke of.’ (lit. I would like to see that above-mentioned tree first)
\z

\textit{Azla} ‘now’ (\ref{ex:3:127} and \ref{ex:3:128}) adds tension and excitement.

\ea \label{ex:3:127}\corpussource{Disobedient Girl, S. 21}\\
Ndahan  bah  məbehe  háy  ahan  amadala  na  kə  ver  aka  \textbf{azla.}\\
\gll  ndahaŋ  bax      mɪ-bɛh-ɛ    haj     =ahaŋ\\
      \oldstylenums{3}\textsc{s}              \textsc{id}pour  {\NOM}{}-pour-{\CL}  millet    =\oldstylenums{3}\textsc{s}.{\POSS}\\
\medskip
\gll  ama-d  =ala  na        kə     vɛr         aka  \textbf{aɮa}\\
      {\DEP}-put  =to    \oldstylenums{3}\textsc{s}.{\DO}  on  stone     on     now\\
\glt  ‘She poured out her millet to prepare it on the grinding stone now.’
\z

\ea \label{ex:3:128}\corpussource{Disobedient Girl, S. 22}\\
Njəw  njəw  njəw  aməhaya  \textbf{azla.}\\
\gll  {nzu   nzu    nzu}   amə-h=aja  \textbf{aɮa}\\
      \textsc{id}grind    {\DEP}-grind={\PLU}  now\\
\glt  ‘\textit{Njəw  njəw  njəw} [she] ground [the millet] now.’
\z

\textit{Ɗəwge} ‘actual’ (to indicate that the events in the clause actually happened, \ref{ex:3:129}).


\ea \label{ex:3:129}\corpussource{Snake, S. 24}\\
Ka  nehe  ləbara  a  ma  ndana  \textbf{ɗəwge.}\\    
\gll ka  nɛhɛ  ləbara     a       ma      ndana     \textbf{ɗuwgɛ}\\     
    like  {\DEM}  news   {\GEN}    word     {\DEM}   actual\\
\glt ‘And so was that  previously mentioned story.’
\z

\textit{Re} ‘counterexpectation’ \REF{ex:3:130} indicates that the clause is the opposite to what the hearer might have expected.  
 \clearpage
\ea \label{ex:3:130}\corpussource{Values, S. 50}\\
Epele  epele  na  me,  Hərmbəlom  anday  agas  ta  a  ahar  ava  \textbf{re.}\\
\gll {ɛpɛlɛ ɛpɛlɛ}   na   mɛ  Hʊrmbʊlɔm   a-ndaj       a-gas   ta  a   ahar  ava   \textbf{rɛ}\\
      {in the future}  {\PSP}  opinion  God      \oldstylenums{3}\textsc{s}-{\PROG}    \oldstylenums{3}\textsc{s}-catch   \oldstylenums{3}\textsc{p}.{\DO}  at  hand  in     {in spite}\\
\glt  ‘In the future in my opinion, God is going to accept them [the elders] in his hands, in spite [of what anyone says].’
\z

\section{Ideophones}\label{sec:3.6}\is{Ideophone|(}
\hypertarget{RefHeading1211241525720847}{}
Ideophones are a “vivid representation of an idea in sound” \citep[118]{Doke1935}. They evoke the ``idea'' of a sensation or sensory perception (action, movement, colour, sound, smell, or shape). As such they are often onomatopoeic. 

Ideophones add vividness to texts. They are found in strategic places in narratives (both in legends and in true stories) and add vividness to major points in exhortations. At the peak moment of a story, ideophones can have a special narrative function to present the entire event expressed in a sentence. In such cases the clause may have no expressed subject or object – a transitivity of zero.

\citet{Newman1968} suggests that ideophones do not comprise a grammatical class of their own, but rather are words from several different classes (including nouns, adjectives, and adverbs) which are grouped together based on phonological and semantic similarities rather than syntax. In Moloko ideophones are treated as a separate grammatical class since although they may fill the noun, verb, or adverb slot in a clause, ideophones do not pattern as typical nouns, verbs, or adverbs. \sectref{sec:3.6.1} describes the semantic and phonological features of ideophones; \sectref{sec:3.6.2} discusses their syntax and their role in discourse, and \sectref{sec:3.6.3} discusses the fact that a clause where an ideophone fills the verb slot can carry zero transitivity.  

\subsection{Semantic and phonological features of ideophones}\label{sec:3.6.1}
\hypertarget{RefHeading1211261525720847}{}
Ideophones carry an idea of a particular state or event -- Moloko speakers can imagine the particular situation and the sensation of it when they hear a particular ideophone. The sensation may be a sound \REF{ex:3:131}, vision \REF{ex:3:132}, taste \REF{ex:3:133}, feeling \REF{ex:3:134}, or even abstract idea (for example, an insult, \ref{ex:3:135}).\footnote{The first line in each example is the orthographic form. The second is the phonetic form (slow speech) with morpheme breaks.} 

\ea \label{ex:3:131}
\gll gəɗəgəzl\\
  g\`{ə}ɗ\`{ə}g\`{ə}ɮ\\
\glt  ‘the noise of something closing or being set down’
\z

\ea \label{ex:3:132}
\gll danjəw\\
  dànzúw\\
\glt  ‘sight of someone walking balancing something on their head’
\z

\ea \label{ex:3:133}
\gll poɗococo\\
  p\`{ɔ}ɗ\'{ɔ}ts\'{ɔ}ts\'{ɔ}\\
\glt  ‘taste of sweetness’
\z

\ea \label{ex:3:134}
\gll pəyecece\\
  pìj\'{ɛ}tʃ\'{ɛ}tʃ\'{ɛ}\\
\glt  ‘feeling of coldness’
\z

\ea \label{ex:3:135}
\gll kekəf  kəf  kekəf  kəf\\
  k\`{ɛ}k\'{ɪ}f    k\'{ɪ}f      k\`{ɛ}k\'{ɪ}f  k\'{ɪ}f\\
\glt  ‘imagination of someone who hasn’t any weight’ (an insult)
\z

Ideophones have specific meanings; compare the following three ideophones in (\ref{ex:3:136}--\ref{ex:3:138}). The ideophones differ in only the final syllable.

\ea \label{ex:3:136}
\gll pəvbəw pəvbəw\\
  pə\dentalflap uw pə\dentalflap uw\\
\glt  ‘sight of rabbit hopping’
\z

\ea \label{ex:3:137}
\gll pəvba  pəvba\\
  pə\dentalflap a  pə\dentalflap a\\
\glt  ‘sound of a whip’
\z

\ea \label{ex:3:138}
\gll pəvban  pəvban\\
  pə\dentalflap aŋ  pə\dentalflap aŋ\\
\glt  ‘sight of the start of a race’
\z

Ideophones do not follow the stress rules for the language (\chapref{chap:2}) Some ideophones are stressed on the initial syllable (shown by full vowels in \ref{ex:3:135}). Some ideophones have no full vowel (\ref{ex:3:131}, \ref{ex:3:139}, \ref{ex:3:142}). 

\ea \label{ex:3:139}
\gll jəɓ  jəɓ\\
  dʒɪɓ dʒɪɓ\\
\glt  ‘completely wet’
\z

Moloko ideophones sometimes contain unusual sounds, including the labiodental flap [\dentalflap ], marked as \textit{vb} in the orthography. The labiodental flap is found only in ideophones that carry a neutral prosody. 

\ea \label{ex:3:140}
\gll vbaɓ\\
  \dentalflap àɓ\\
\glt  ‘sound of something soft hitting the ground’ (a snake, or a mud wall collapsing)
\z

Ideophones often have reduplicated segments as shown in \REF{ex:3:141} (see also \REF{ex:3:133}, \REF{ex:3:134}, \REF{ex:3:135} for additional examples).

\ea \label{ex:3:141}
\gll həɓek  həɓek\\
  hìɓ\'{ɛ}k   hìɓ\'{ɛ}k\\
\glt  ‘hardly breathing’ (almost dead)
\z

Some ideophones require a context in order for their meaning to be understood clearly; others give a clear meaning even if they are spoken in isolation. For example, if a Moloko speaker hears someone say \textit{njəw njəw} \REF{ex:3:142}, they know that the speaker is talking about someone grinding something on a grinding stone. Likewise see also \REF{ex:3:131}, (\ref{ex:3:133}--\ref{ex:3:135}), \REF{ex:3:141}, \REF{ex:3:143}, and \REF{ex:3:160}, all of which carry a distinctive lexical meaning even when spoken in isolation. 

\ea \label{ex:3:142}
\gll njəw  njəw\\
  nzùw  nzùw\\
\glt  ‘the sound of someone grinding something on a grinding stone’
\z

\ea \label{ex:3:143}
\gll pəcəkəɗək\\
  p\'{ʊ}ts\'{ʊ}k\'{ʊ}ɗ\'{ʊ}k\\
\glt  ‘the sight of a toad hopping’
\z

However, a Moloko speaker will need to understand a wider context to determine the meaning of \textit{dergwejek} \REF{ex:3:144}, which requires a context for the listener to understand the detail of the picture. In the same way, \REF{ex:3:140} also requires a context to specify its exact meaning (snake falling or wall collapsing).

\ea \label{ex:3:144}
\gll dergwecek\\
  dɛrgʷɛtʃɛk\\
\glt  ‘sight of someone lifting something onto their head’
\z

\subsection{Syntax of ideophones}\label{sec:3.6.2}\is{Attribution!Ideophones}
\hypertarget{RefHeading1211281525720847}{}
In a sentence, an ideophone can function as a noun, adverb, or verb. As a noun, the ideophone carries a descriptive picture with certain features.  Ideophones that are lexical nouns (\ref{ex:3:145}--\ref{ex:3:147}, see also \ref{ex:3:133} and \ref{ex:3:134}) can function as the head of a noun phrase, but they cannot be pluralised or modified by noun phrase constituents except with the adjectiviser \textit{ga}. In example \REF{ex:3:147}, the ideophone \textit{mbajak mbajak mbajak} ‘something big and reflective’ is the direct object of the clause. The ideophones are bolded in the examples.

\ea \label{ex:3:145}\corpussource{Values, S. 34}\\
 Ehe na, təta na, kəw na, \textbf{bəwɗere.}\\
\gll      ɛhɛ  na      təta   na  kuw      na    \textbf{buwɗɛrɛ}\\
      here   {\PSP}    \oldstylenums{3}\textsc{p}    {\PSP}    \textsc{id}take   {\PSP}    \textsc{id}foolishness\\
\glt  ‘Here, what they are taking is foolishness!’ (lit. here, they, taking,  foolishness)
\z

\ea \label{ex:3:146}\corpussource{Values, S. 48}\\
Kə  wəyen  aka  ehe  \textbf{tezl  tezlezl.}\\
\gll  kə   wijɛŋ   aka   ɛhɛ    \textbf{tɛɮ tɛɮɛɮ}\\
      on  earth  on  here    \textsc{id}hollow\\
\glt  ‘[Among the people] on earth here, [we are like] the sound of a hollow cup bouncing on the ground.’
\z

\ea \label{ex:3:147}\corpussource{Snake, S. 11}\\
Námənjar  na,  \textbf{mbajak  mbajak  mbajak}  gogolvon.\\
\gll  ná-mənzar   na  \textbf{mbadzak  mbadzak  mbadzak}  gʷɔgʷɔlvɔŋ\\
      {\oneS}+{\IFV}-see  \oldstylenums{3}\textsc{s}.{\DO}  {\textsc{id}something big and reflective}  snake\\
\glt  ‘I was seeing it, something big and reflective, a snake!’
\z

When an ideophone functions as an adverb, the ideophone gives information concerning the subject of the clause as well as the manner of the action.  \tabref{tab:3.23} illustrates 11 different adverbial ideophones that collocate with the verb \textit{həmay} ‘run’ but vary depending on the actor of the clause. Unlike most other adverbs however, ideophones cannot be negated.

\begin{sidewaystable}
\resizebox{.75\textwidth}{!}{\begin{tabular}{lll}
\lsptoprule
1 & \textit{zar   a-həm-ay} \textbf{\textit{gədo gədo gədo}} & ‘A man runs \textit{gədo gədo gədo}.’\\
 & man  \oldstylenums{3}\textsc{s}-run{}-{\CL}     \textsc{id}man running &\\\midrule
2 & \textit{war a-həm-ay} \textbf{\textit{njəɗok njəɗok}} & ‘A toddler runs \textit{njəɗok njəɗok}.’\\
 & child \oldstylenums{3}\textsc{s}-run{}-{\CL}    \textsc{id}child running and jumping \\\midrule
3 & \textit{albaya  a-həm-ay} \textbf{\textit{njəl njəl}} & ‘A young man runs \textit{njəl njəl}.’\\
 & youth   \oldstylenums{3}\textsc{s}-run{}-{\CL}      \textsc{id}youth running  & (also mice run like this)\\\midrule
4 & \textit{mədehwer  a-həm-ay təta baj;} & ‘An old person can’t run; \\
 & old person  \oldstylenums{3}\textsc{s}-run{}-{\CL}   ability   {\NEG} \\
\\
& \textit{a-həm-ay} \textbf{kərwəɗ wəɗ, kərwəɗ wəɗ} & he moves \textit{kərwəɗ wəɗ, kərwəɗ wəɗ}.’\\
& \oldstylenums{3}\textsc{s}-run{}-{\CL}      \textsc{id}insult someone with no stomach & \\\midrule
5 & \textit{zlevek  a-həm-ay} \textbf{pavbəw pavbəw} & ‘A rabbit runs \textit{pavbəw pavbəw}.’\\
 & rabbit  \oldstylenums{3}\textsc{s}-run{}-{\CL}      \textsc{id}rabbit hopping \\\midrule
6 & \textit{sla     =ahay tə-həm-ay} \textbf{gərəp gərəp} & ‘Cows run \textit{gərəp gərəp}.’\\
& cow  =Pl    \oldstylenums{3}\textsc{p}-run{}-{\CL}   \textsc{id}something heavy running \\\midrule
7 & \textit{javar   =ahay tə-həm-ay } \textbf{\textit{cərr}} & ‘Guinea fowl run \textit{cərr}.’ \\
& guinea fowl=Pl \oldstylenums{3}\textsc{p}-run{}-{\CL} \textsc{id}guinea fowl taking off & (when they are taking off)\\\midrule
8 & \textit{erkece  a-həm-ay } \textbf{\textit{yeɗ yeɗ yeɗ}} & ‘An ostrich runs \textit{yeɗ yeɗ yeɗ}.’\\
& ostrich  \oldstylenums{3}\textsc{s}-run{}-{\CL}    \textsc{id}ostrich running \\\midrule
9 & \textit{moktonok} \textit{ a-həm-ay } \textbf{pəcəkəɗək, pəcəkəɗək}  & ‘A toad runs \textit{pəcəkəɗək, pəcəkəɗək}.’\\
& toad           \oldstylenums{3}\textsc{s}-run{}-{\CL}  \textsc{id}toad hopping \\\midrule
10 & \textit{məwta a-həm-ay} \textbf{\textit{fəhh}}  & ‘A truck runs \textit{fəhh}.’\\
& truck    \oldstylenums{3}\textsc{s}-run{}-{\CL}  \textsc{id}truck humming \\\midrule
11 & \textit{həmaɗ a-həm-ay} \textbf{\textit{fowwa}}  & ‘The wind runs \textit{fowwa}.’\\ 
& wind     \oldstylenums{3}\textsc{s}-run{}-{\CL}  \textsc{id}wind blowing \\
\lspbottomrule
\end{tabular}}
\caption{Selected ideophones that co-occur with the verb həmaj  ‘to run’}\label{tab:3.23}
\end{sidewaystable}

When they act as adverbs, ideophones can occupy one of two slots in the clause. When the verb they modify is finite, ideophones will occur at the end of the clause following other adverbs (\ref{ex:3:148}--\ref{ex:3:150} and all of the examples in \tabref{tab:3.23}). In a narrative, ideophones that function as adverbs can be found wherever the language is vivid. They occur most often at the inciting moment and the peak\is{Focus and prominence!Discourse peak} section of a narrative. The ideophones in each clause are bolded  and the verb phrase is delimited by square brackets. 

\ea \label{ex:3:148}
[Azləgalay]  avəlo  \textbf{zor!}\\
\gll  [à-ɮəg        =alaj]  avʊlɔ  \textbf{z\'{ɔ}r}\\
      \oldstylenums{3}\textsc{s}+{\PFV}-throw   =away  above  \textsc{id}throwing\\
\glt  ‘She threw [the pestle] up high (movement of throwing).’
\z

\largerpage
\ea \label{ex:3:149}
[Anday  azlaɓay  ele]  \textbf{kəndal, kəndal,  kəndal.}\\
\gll  [a-ndaj    a-ɮaɓ-aj    ɛlɛ]  \textbf{k\`{ə}ndál,  k\`{ə}ndál,  k\`{ə}ndál}\\
      \oldstylenums{3}\textsc{s}-{\PRG}    \oldstylenums{3}\textsc{s}-pound-{\CL}  thing  {\textsc{id}pounding millet}\\
\glt  ‘She was pounding the [pestle] (threshing millet) pound, pound pound.’
\z

\ea \label{ex:3:150}
[Həmbo  ga  anday  asak  ele  ahan]  \textbf{wəsekeke.}\\
\gll  [hʊmbɔ  ga  a-ndaj    a-sak    ɛlɛ  =ahaŋ]    \textbf{wuʃɛkɛkɛ}\\
      flour  {\ADJ}  \oldstylenums{3}\textsc{s}-{\PRG}    \oldstylenums{3}\textsc{s}-multiply  thing  =\oldstylenums{3}\textsc{s}.{\POSS}  \textsc{id}many\\
\glt  ‘The flour was multiplying all by itself (lit. its things), sound of multiplying.’
\z

When the verb it modifes is non-inflected, the ideophone is the first element of the verb phrase, preceding the verb complex (\ref{ex:3:151} and \ref{ex:3:152}).  This is a special construction that is discussed in \sectref{sec:8.2.3}.

\ea \label{ex:3:151}
Nata  ndahan  [\textbf{pək}  mapata  aka  va  pərgom  ahay  na.]\\
\gll nata  ndahaŋ  [\textbf{pək}      ma-p =ata =aka =va  \\
      also    \oldstylenums{3}\textsc{s}  {\textsc{id}open door or bottle} {\NOM}{}-open =\oldstylenums{3}\textsc{p}.{\IO} =on ={\PRF} \\
      
      \medskip
\gll pʊrgʷɔm  =ahaj  na]\\
     trap    =Pl  {\PSP}\\
\glt ‘He opened the traps for them.’
\z

\ea \label{ex:3:152}
Dərlenge  [\textbf{pəyteɗ}  məhəme  ele  ahan]  ete.\\
\gll  dɪrlɛŋgɛ  [\textbf{píjt\'{ɛ}ɗ}    mɪ-hɪm-ɛ  ɛlɛ  =ahaŋ]    ɛtɛ\\
      hyena  \textsc{id}crawling  {\NOM}{}-run-{\CL}  thing  =\oldstylenums{3}\textsc{s}.{\POSS}  also\\
\glt  ‘The hyena, barely escaping, ran home also.’
\z

At the most vivid moments of a discourse\is{Focus and prominence!Discourse peak}, an ideophone can carry the morphosyntactic features of a verb. As a verb the ideophone syntactically fills the verb slot in the verb phrase: it takes verbal extensions and non-subject pronominals. Semantically, the main event in a clause is expressed by the ideophone. For example, the ideophone \textit{mək} ‘positioning [self] for throwing’ in line 14 of the Snake story \REF{ex:3:153} carries the verbal extensions \textit{=ava} ‘in’ and\textit{ =alay} ‘away.’ Also, the ideophone \textit{təh} ‘put on head’ in lines 26 and 27 of the Cicada story \REF{ex:3:154} carries the verbal pronominal \textit{an=an} ‘to it.’ \REF{ex:3:155} also shows an ideophone with the direct object verbal extension \textit{na}.

\ea \label{ex:3:153}\corpussource{Snake, S. 14}\\\relax
[\textbf{Mək}  ava  alay.]\\
\gll  [\textbf{mək}          =ava   =alaj]\\
      {\textsc{id}position [self] for throwing}  =in    =away \\
\glt  ‘[I] positioned [myself] \textit{mək}!’
\z

\ea \label{ex:3:154}\corpussource{Cicada, S. 26}\\
 Albaya ahay weley [\textbf{təh} anan dəray na,] abay.\\
\gll      albaja   =ahaj   wɛlɛj [\textbf{təx} an=aŋ dəraj   na]  abaj\\
      youth    =Pl    which   {\textsc{id}put on head}   {\DAT}=\oldstylenums{3}\textsc{s}.{\IO}   head   {\PSP}   {{\EXT}+{\NEG}}\\
\glt  ‘No one could lift it.’ (lit. whichever young man put his head to [the tree in order to lift it], there was none) 
\z

In an exhortation, the major points may be made more vivid by the use of ideophones. \REF{ex:3:155} expresses a major point in the Values exhortation (see \sectref{sec:1.7}). Additionally, see (\ref{ex:3:145}--\ref{ex:3:146}) which also display this device.


\ea \label{ex:3:155}\corpussource{Values, S. 22}\\
Təta  [\textbf{dəl}  na,  ma  Hərmbəlom  nendəye.]\\
\gll  təta   [\textbf{dəl}     na   ma   Hʊrmbʊlɔm   nɛndijɛ]\\
      \oldstylenums{3}\textsc{p}  \textsc{id}insults  \oldstylenums{3}\textsc{s}.{\DO}  word  God    {\DEM}\\
\glt  ‘They insult it, this word of God!’ 
\z

At the peak of a story, ideophones are found within many of the clauses. In some cases, the ideophone is the only element in the clause. In the Snake story for example, the peak episode\is{Focus and prominence!Discourse peak} (lines 8--18, see \sectref{sec:1.4} for the entire text) contains seven ideophones. The narrator tells that he took his flashlight, shone it up \textit{cəlar}, saw something \textit{mbajak} \textit{mbajak} (big and reflective), a snake! He \textit{mbət} turned off his light, \textit{kaləw} took his spear, \textit{mək} (positioned himself). Penetration \textit{mbəraɓ }! It fell \textit{vbaɓ} on the ground. Note that at the climactic moment (line 14, \ref{ex:3:156}), the entire clause is expressed by a single ideophone \textit{mək}, followed by verbal extensions.

\ea \label{ex:3:156}\corpussource{Snake, S. 14}\\
\textbf{Mək}  ava  alay.\\
\gll  \textbf{mək}  =ava   =alaj\\
      {\textsc{id}position for throwing}  =in    =away\\
\glt  ‘[I] positioned myself mək!’
\z

\largerpage
Likewise, in the peak episode of the Cicada text\is{Focus and prominence!Discourse peak} (S. 25--29, see \sectref{sec:1.6}) ideophones are frequent and at the climactic\is{Focus and prominence!Ideophones} moment (\ref{ex:3:157}--\ref{ex:3:158}), the ideophone is the only element in the clause. The cicada and young men go \textit{sen} to the tree to move it. None of the young men could {\textit{təh} ‘put' their head to the tree. Then the cicada \textit{təh} ‘put' his head to the tree. \textit{Kəwna} ‘he got it!' \textit{ Dergwejek} ‘he lifted it to his head.' In line S. 26 the ideophone \textit{təx} takes the place of the verb in the main clause and in lines S. 28 and 29 the ideophone is the only element in the clause. The entire event in each of those lines is thus expressed by that one word. 


\ea \label{ex:3:157}\corpussource{Cicada, S. 28}\\
\textbf{Kəwna}.\\
\gll  \textbf{kuwna}\\
      \textsc{id}getting\\
\glt  ‘[He] got [it].’ 
\z


\ea \label{ex:3:158}\corpussource{Cicada, S. 29}\\
\textbf{Dergwecek}.~\\
\gll  \textbf{dɛrgʷɛtʃɛk}\\
      {\textsc{id}lifting.onto.head}\\
\glt  ‘[He] lifted [it] onto [his head].’
\z

\subsection{Clauses with zero transitivity}\label{sec:3.6.3}\is{Transitivity!Clauses with zero transitivity|(}
\hypertarget{RefHeading1211301525720847}{}
\largerpage
\chapref{chap:9} discusses the semantics of Moloko verbs for different numbers of core grammatical relations. Moloko verbs can have from zero to four grammatical relations, three of which can be coded as part of the verb complex.  Similarly, in clauses where ideophones fill the verb slot, the clause can have from zero to three explicit grammatical relations. The cases where the ideophone clause requires no explicit grammatical relations presents a most interesting situation. The clause displays a grammatical transitivity of zero, even though it expresses a semantic event with participants. The use of ideophones makes the moment vivid and draws the listener into the story as if it was present before him/her so that the hearer can see and hear and participate in what is going on. This is a narrative device found in Moloko peak episodes.\is{Focus and prominence!Discourse peak}

For example, ideophones make up the entire clause in lines S. 28 and 29 at the peak of the Cicada text (example  \ref{ex:3:157} above). On hearing the ideophones \textit{kəwna} and \textit{dergwejek}, the hearer knows that someone has gotten a hold of something, and then lifted it up onto his head to carry it. Two participants are understood, but the actual number of grammatical relations in the clauses is zero. The hearer must infer from the context that it was the cicada (the unexpected participant) who was doing the lifting and carrying. The cicada being so small, the people actually watching the event would not know for sure who was moving the tree either, since it would look like the tree was moving all by itself. Thus the use of ideophones with zero grammatical relations contributes to the visualisation of the story and makes the listener more of an actual participant in the events of the story. 

Likewise, in line S. 21 of the Disobedient Girl story \REF{ex:3:159} the clause has no expressed subject, direct or indirect object. The verb /h/ is in nominalised form with no pronominals to indicate participants. If a Moloko person hears the ideophone \textit{njəw njəw}, he or she knows that someone is grinding something. In the context of the story, the woman is grinding millet, but the millet is expanding to fill the room and eventually will crush the woman. The clause only gives a picture/sound/idea of grinding with gaps in knowledge that the listener must work to fill in for himself, such as who is grinding whom. The listener is thus drawn into the story and made to be a participant in the event, creating vividness.


\ea \label{ex:3:159}\corpussource{Disobedient Girl, S. 21}\\
\textbf{Njəw  njəw  njəw}  aməhaya  azla.\\
\gll  {\textbf{nzuw  nzuw  nzuw}}           amə-h    =aja     aɮa\\
      \textsc{id}grind             {\DEP}-grind  ={\PLU}    now\\
\glt  \textit{Njəw  njəw  njəw} [she] ground [the millet] now. 
\z

A third example is found in the Snake story. In lines S. 14 and 15, both the ideophone clause (line 14) and the nominalised form plus ideophone (line 15) have zero grammatical relations \REF{ex:3:160}. The speaker is making both himself and the snake ‘invisible’ at this peak moment of his story\is{Focus and prominence!Discourse peak}. The effect would be to allow the hearer to imagine himself there right beside the speaker in the darkness, wondering where the snake was, hearing only the sounds of the events. 

\clearpage
\ea \label{ex:3:160}\corpussource{Snake, S. 14}\\
\textbf{Mək}  ava  alay.\\
\gll  \textbf{mək}            =ava   =alaj\\
      {\textsc{id}take position for throwing}  =in    =to\\
\glt  ‘[He] positions himself for throwing [the spear].’

\medskip

 \corpussource{Snake, S. 15}\\
Mecesle  \textbf{mbəraɓ}.\\
\gll mɛ-tʃɛɬ-ɛ           \textbf{mbəraɓ}\\
      {\NOM}{}-penetrate-{\CL}    \textsc{id}penetrate \\
\glt ‘[The spear] penetrates [the snake].’
\z

\is{Ideophone|)}\is{Transitivity!Clauses with zero transitivity|)}
\section{Interjections}\label{sec:3.7}
\hypertarget{RefHeading1211321525720847}{}
Interjections can form a clause of their own (\ref{ex:3:161} and \ref{ex:3:162}) or can function as a kind of ‘audible’ pause while the speaker is thinking \REF{ex:3:163}. They can also occur before or after the clause in an exclamation construction (see \sectref{sec:10.5}). Note that some interjections can be reduplicated for emphasis (compare \ref{ex:3:162} and \ref{ex:3:164}).

\ea \label{ex:3:161}
\textbf{məf}\\
  \textbf{məf}\\
\glt  ‘get away! (to put off an animal or a child from continuing to do an undesirable action).’
\z

\ea \label{ex:3:162}
\textbf{təde}\\
  \textbf{tɪdɛ}\\       
\glt  ‘good’    
\z

\ea \label{ex:3:163}
Apazan  nəmənjar,  \textbf{andakay},  Hawa.\\
\gll  apazaŋ  n\`{ə}-mənzar   \textbf{andakaj}     Hawa\\
      yesterday  {\oneS}+{\PFV}-see    {what’s her name}    Hawa\\
\glt  ‘Yesterday I saw \ldots\xspace what’s her name \ldots\xspace Hawa.’ 
\z

\ea \label{ex:3:164}
\textbf{tətəde}\\
\glt  ‘very good’
\z
