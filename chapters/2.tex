\chapter[Phonology]{Phonology}\setcounter{equation}{0}\label{chap:2}
\makeatletter\@addtoreset{equation}{part}\makeatother
\hypertarget{RefHeading1210401525720847}{}
The vowel system of Moloko is noteworthy in its simplicity - it can be analysed as having only one underlying phoneme with ten phonetic representations (see \sectref{sec:2.3}).

The phonology of Moloko has been fully discussed by \citet{Bow1997c}. The following is a summary of the aspects that are necessary to understand the grammar, with focus on the new work that has been done since her manuscript was initially published. 

\citet{Bow1997c} based her phonological outline of Moloko on a database she compiled consisting of around 1500 words, including some 400 verbs and 1000 nouns.  Bow’s database was modified and extended by \citet{Boyd2002} with a focus on nouns. Later, Mamalis built on their work to describe the tone on verbs, and Friesen discussed phonological word structure of the verb word (\citealt{FriesenMamalis2008}).  

Three inter-related phonological factors must be touched on before a discussion of any of them can be fully understood. The first is that Moloko words are built on a consonantal skeleton with only one underlying vowel /a/ (phonetically expressed as the \textit{full vowels} [a, o, œ, \ae, ɛ], see \sectref{sec:2.3}) that occurs between only some of the consonants.\footnote{\citet{Bow1997c} used the distinction +/- Low, which focused on one phonetic feature, however we have found that the most salient issue in discussing the vowel patterns of this language is the concept of \textit{full} vs. \textit{epenthetic} vowels.  For clarity, therefore, this work will use the terms \textit{full} and \textit{epenthetic} to distinguish between the two sets of vowel phones, with \textit{full} referring to /a/ and its prosodically conditioned allophones, and \textit{epenthetic} referring to schwa and its allophones.} Some consonant clusters (caused by the absence of an underlying vowel between them) are broken up by epenthetic schwa insertion when they are pronounced (and phonetically expressed as [ə, ʊ, u, ø, ɪ, i]).\footnote{Likewise in Muyang\il{Muyang}, another Central Chadic language closely related to Moloko, it can be shown that syllables are built postlexically from the consonant skeleton by regular rules. Only the low vowel /a/ is phonemic, and all high vowels can be accounted for by means of epenthesis \citep{Smith1999}.} Although syllable structure will be mentioned in this work, attention will be focussed on the underlying consonantal skeleton. \citet[15]{Roberts2001} notes for Central Chadic languages, 

\largerpage
\begin{quote}
[because] “the consonant skeleton is all-important to the phonological structure, the traditional unit of the syllable is much less useful in the description of Central Chadic languages since at the core of every syllable must be a vowel (or some syllabic segment, at least). And in fact, it can be shown for most of these languages that the syllable is a very superficial phenomenon.”
\end{quote}

And further on p. 16, 

\begin{quote}
“We conclude then that the syllable is not a unit that can be exploited as it is in other languages to elucidate the phonological  structure. It is a surface structure phenomenon whose character is completely predictable from other phonological aspects of lexemes. On the other hand, an underlying structure that is more worthy of study in Central Chadic languages is that of the consonant skeleton that can take up lexical roots; to this core are added other peripheral phonological elements such as vowels, prosodies, and tones.”
\end{quote}

The second basic phonological factor for Moloko is that all of the vowels (both full and epenthetic) and some of the consonants are affected by word-level labialisation or palatalisation prosodies\footnote{Prosodies in Chadic languages are word-level suprasegmental processes that labialise or palatalise the entire word and affect all vowels and some consonants. See \citet{Roberts2001} for a fuller discussion.} (see \sectref{sec:2.1}). These prosodies account for most of the vowel and consonant allophones in the language. Palatalisation can be part of certain morphemes, but Moloko is unlike other Chadic languages where palatalisation and labialisation alone can have morphemic status (for example in Muyang where the application of the palatalisation prosody on a noun produces a diminutive, and application of the labialisation prosody produces an augmentative, Smith, personal communication).

The third basic factor is that the final syllable before a pause is stressed in pronunciation. The stressed syllable necessitates a full vowel, meaning that any epen\-the\-tic vowel in that syllable will be changed to its full counterpart.  The following two example pairs each show the same word in unstressed and stressed environments. Compare [zij] (non-stressed with epen\-the\-tic vowel) with [zaj] (stressed with full vowel) in \REF{ex:2:1} and \REF{ex:2:2}, and [nɔ-zʊm] (non-stressed with epen\-the\-tic vowel in final syllable) with [nɔ-zɔm] (stressed with full vowel) in \REF{ex:2:3} and \REF{ex:2:4}. 


\ea \label{ex:2:1}
\gll  {[}zij  ɗaw{]}\\
      peace  \textsc{q}\\
\glt  ‘Is there peace?’ 
\z

\ea \label{ex:2:2}
\textup{[zaj]}\\
      peace\\
\glt  ‘There is peace.’ 
\z

\ea \label{ex:2:3}
\label{bkm:nozomdaf}
\gll [n\'{ɔ}-zʊm    ɗaf]\\
      {\oneS}+{\PFV}-eat  {millet loaf}\\
\glt  ‘I ate millet loaf.’
\z

\ea \label{ex:2:4}
\gll [n\'{ɔ}-zɔm]\\
      {\oneS}+{\PFV}-eat\\
\glt  ‘I ate.’
\z

Due to these interrelated factors, much of the phonological discussion will require representation of both the underlying and surface forms of lexemes. The underlying form  consists of the consonant and vowel phonemes (written between slashes) and the word prosody (written as a superscripted \textsuperscript{e} for palatalisation, \textsuperscript{o} for labialisation at the right of the morphemes). A neutral prosody has no superscript. The following examples illustrate the phonetic forms (between square brackets) and underlying forms (between slashes) of nouns that are palatalised \REF{ex:2:5}, labialised \REF{ex:2:6}, and neutral with respect to prosody \REF{ex:2:7}. All of the examples in this work will be presented in the phonetic form unless otherwise indicated. 

\ea \label{ex:2:5}
\textup{[mɪdɪgɛr] \hspace{8pt} /m d g r\textsuperscript{e}}\textup{/}\\
\glt  ‘hoe’
\z

\ea \label{ex:2:6}
\textup{[lʊhɔ] \hspace{20pt}  / l ha\textsuperscript{o}}\textup{/}\\
\glt  ‘late afternoon’
\z

\ea \label{ex:2:7}
\textup{[ɗaf] \hspace{25pt} /ɗ f/}\\
\glt  ‘millet loaf’  
\z

The phonetic forms of the examples given in this paper are all in citation form (the form of the word when it is pronounced in isolation), and therefore show each word with a stressed final syllable.  In each case, the final syllable (whether open or closed) always contains a full vowel, regardless of whether the underlying form has a full vowel or not.  

The phonology section of the present work begins with a description of the prosodies of labialisation and palatalisation and their effects (\sectref{sec:2.1}), which leads to a description of the consonant and vowel systems (Sections \ref{sec:2.2} and \ref{sec:2.3}, respectively). An examination of the tone system follows (\sectref{sec:2.4}). Finally, notes on the syllable and word breaks are discussed (see Sections \ref{sec:2.5} and \ref{sec:2.6}, respectively). Appendix~\ref{sec:13.1} includes a list of verbs used in this analysis, showing their imperative form, underlying form, and underlying tone.

\section{Labialisation and palatalisation prosodies}\label{sec:2.1}\is{Prosody (labialisation or palatalization)|(}
\hypertarget{RefHeading1210421525720847}{}
One of the most basic phonological processes in Moloko is prosody.  Chadic linguists refer to prosody as a suprasegmental process where a labialisation or palatalisation feature is applied to a phonological word. \citet{Wolff1981} refers to prosodies as suprasegmental sources of palatalisation and labiovelarisation.

\citet{Bow1997c} has discovered that labialisation and palatalisation work at the morpheme level in Moloko. Both prosodies are attached to a particular morpheme and spread leftward over the entire phonological word. Labialisation affects the back consonants (k, g, ŋg, and h) and vowels; palatalisation affects alveolar fricatives (s and z), affricates (ts and dz, see \sectref{sec:2.2}), and vowels (see \sectref{sec:2.3}).  All Moloko words are either labialised, palatalised, or are neutral with respect to prosody. Recent work demonstrates that some syllables can be affected by both labialisation and palatalisation (see \sectref{sec:2.3.1} and \sectref{sec:2.3.2}).  

As stated above, in this work prosody is indicated in the underlying form using superscript symbols included at the right edge of the word: /\textsuperscript{o}/ to represent labialisation and / \textsuperscript{e}/ to represent palatalisation.  In the phonetic form, the prosody is indicated by the quality of the full vowel in the word ([ɔ] for labialisation, [ɛ] for palatalisation, and [a] for no  prosody (see \sectref{sec:2.3}). The examples (\ref{ex:2:8}--\ref{ex:2:10}) from \citet{Bow1997c} give evidence of contrast between the prosodies in a minimal triplet:


\ea \label{ex:2:8}
\textup{/k ra/  \hspace{17pt}    [kəra]}\\
\glt  ‘dog’
\z

\ea \label{ex:2:9}
\textup{/k ra \textsuperscript{o}}\textup{/  \hspace{10pt}    [kʷʊrɔ]    }\\
\glt  ‘ten'    
\z

\ea \label{ex:2:10}
\textup{/k ra \textsuperscript{e}}\textup{/  \hspace{10pt}    [kɪrɛ]}\\
\glt  ‘stake/post’
\z

\largerpage
The effects of both prosodies on a single underlying form can be seen in the paradigm for the verb /mnzar/ ‘see’ shown in \tabref{tab:2.1} (adapted from \citealt{Bow1997c}). The verb stem is bolded in the table. The {\twoS} imperative is neutral with respect to prosody, while the {\twoP} imperative form involves a labialisation prosody and the addition of a suffix /-am \textsuperscript{o}/ (see \sectref{sec:7.3.1}). The nominalised form carries a palatalisation prosody, and involves the addition of both a prefix /m{}-/ and suffix /-a \textsuperscript{e}/. Note that vowels and some consonants are affected by the prosodies. As previously stated, the vowel /a/ is realised as [ɔ] in labialised forms, and [ɛ] in palatalised forms, while [ə] is realised as [ʊ] in labialised forms and [ɪ] in palatalised forms (see \sectref{sec:2.3.2}). The consonant /nz/ is realised as [nʒ] in palatalised forms (see \sectref{sec:2.2.3}). 

\begin{table}
\begin{tabular}{llll} 
\lsptoprule
& {Underlying form} & {Phonetic form} & {Gloss}\\
\midrule
{{\twoS} imperative form} & \textsc{/}\textbf{m nza r}/ & [\textbf{mənzar}] & ‘see! ({\twoS})’\\
{{\twoP} imperative form} & \textsc{/}\textbf{m nza r}{}-am\textsuperscript{o}/ & \textsc{[}\textbf{mʊnzɔr}ɔm\textsc{]} & ‘see! ({\twoP})’\\
{Nominalised form} & \textsc{/}m-\textbf{m nza r}{}-a\textsuperscript{e}/ & \textsc{[}mɪ\textbf{mɪnʒɛr}ɛ\textsc{]} & ‘seeing’\\
\lspbottomrule
\end{tabular}
\caption{Paradigm for /mnzar/\label{tab:2.1}}
\end{table}

Labialisation and palatalisation prosodies are lexical features that are applied to a morpheme, and can spread over an entire word. A prosody in the root will spread to a prefix. Compare the prosody in the subject prefixes of the following verbs. In \REF{ex:2:11}, the root is labialised, in \REF{ex:2:12}, the root is palatalised, and in \REF{ex:2:13}, the root is neutral. The underlying forms are given in the examples. 


\ea \label{ex:2:11}
\gll {[nɔ-zɔm]}   \hspace{10pt}   /na-  z m\textsuperscript{o}/\\
      {\oneS}-eat\\
\glt  ‘I eat.’
\z

\ea \label{ex:2:12}
\gll {[nɛ-ʃ-ɛ]}  /na-  s-j\textsuperscript{e}/\\
      {\oneS}-drink-{\CL}\\
\glt  ‘I drink.’
\z

\ea \label{ex:2:13}
\gll {[na-zaɗ]}  \hspace{10pt}    /na-  z ɗ/\\
      {\oneS}-take\\
\glt  ‘I take.’
\z

\largerpage
When initiated by a suffix carrying a prosody, the prosody spreads leftwards, affecting all morphemes within the word including prefixes.\footnote{When the prosody of the suffix is neutral, the prosody on the root is neutralised (compare examples \ref{ex:2:16} and \ref{ex:2:17}).} The effect of the prosody is shown by comparing the vowels and consonants in \REF{ex:2:14} and \REF{ex:2:15}, both forms of [kaɬ] ‘wait,’ a verb root with no underlying prosody. The prosody of the second person singular verb form remains neutral \REF{ex:2:14}. The second person plural contains the labialised suffix /-ak\textsuperscript{o}/ \REF{ex:2:15} and the prosody of the suffix spreads over the entire word. The underlying forms are given in each example. Note that the prosody does not spread to the right across word boundaries since \textit{na}, a separate word, is not affected by the prosody of the verb stem (nor does it neutralise the prosody on the verb).

\ea \label{ex:2:14}
\gll [kà-kaɬ na] {\hspace{60pt}} /ka-  ka ɬ  na/\\
      {\twoS}+{\PFV}-wait    \oldstylenums{3}\textsc{s}.\DO\\
\glt  ‘You waited [for] it.'
\z

\ea \label{ex:2:15}
\gll [m\'{ɔ}-kʷɔɬ-ɔkʷ na] {\hspace{5pt}} /ma-  ka ɬ  -ak\textsuperscript{o} na/\\
      \oldstylenums{1}\textsc{Pin}+\PFV-wait-\oldstylenums{1}\textsc{Pin}  \oldstylenums{3}\textsc{s}.\DO\\
\glt  ‘We waited [for] it.'
\z

Palatalised verbs almost always have a palatalised suffix [{}-ɛ] (see \sectref{sec:6.6}).\footnote{With the exception of verb stems whose final consonant is /n/, e.g., [tʃɛŋ], /tsan\textsuperscript{e }/, ‘know’.}  Whenever there is another suffix or enclitic attached to the verb stem, the [-ɛ] is deleted, taking with it the palatalisation prosody (see \sectref{sec:6.3}). The verb becomes neutral with respect to prosody, as is shown by (\ref{ex:2:16}--\ref{ex:2:17}). In \REF{ex:2:16}, the verb ends with [{}-ɛ] and the entire verb form is palatalised. In \REF{ex:2:17}, the enclitic [=va] has replaced the [{}-ɛ] and the entire verb form is neutral in prosody.

\ea \label{ex:2:16}
\gll nɛ{}-tʃɪk-ɛ  {\hspace{25pt}}  /n-   ts k \textsuperscript{e}/\\
      {\oneS}-move-{\CL}\\
\glt  ‘I move.’      
\z

\ea \label{ex:2:17}
\gll nə-tʃəkə=va    {\hspace{5pt}} \textup{/n-   ts k \textsuperscript{e}   =va/}\\
      {\oneS}-move={\PRF}\\
\glt  ‘I moved already.’
\z

  \citet{Bow1997c} found that prosodies seem to have the least effect on word-initial V syllables. She notes that in palatalised words, the first syllable of nouns that begin with /a/ will sometimes be completely palatalised and pronounced  [ɛ]. However, often it will have an incomplete palatalisation and be pronounced [æ] or even [a]. See the alternate pronunciations that Bow has found for the words /a- la la\textsuperscript{e}/ \REF{ex:2:18} and /a- nd ɓ\textsuperscript{e}/ \REF{ex:2:19}. Palatalisation is a stronger process than labialisation. In labialised words, the first syllable in words that begin with /a/ will often\footnote{Bow found these first syllables always unaffected by labialisation; \citet{Friesen2001} has found that some speakers do pronounce vowel-initial syllables with labialisation [ɔ].} be unaffected by the labialisation and be pronounced [a] (see the alternate pronunciations for the words /a- la ka\textsuperscript{o}/ in \ref{ex:2:20} and /a- g ra\textsuperscript{o}/ in \ref{ex:2:21}). 

\ea \label{ex:2:18}
\textup{[alɛlɛ] {\textasciitilde} [ælɛlɛ] {\textasciitilde} [ɛlɛlɛ]  }\\
\glt  ‘leaf sauce’        
\z

\ea \label{ex:2:19}
\textup{[andɛɓ] {\textasciitilde} [ændɛɓ] {\textasciitilde} [ɛndɛɓ]}\\
\glt  ‘brain’
\z

\ea \label{ex:2:20}
 \textup{[alɔkʷɔ}] {\textasciitilde} [ɔlɔkʷɔ] \\
\glt  ‘fire’        
\z

\ea \label{ex:2:21}
\textup{[agʊrɔ] {\textasciitilde} [ɔgʷʊrɔ]}\\
\glt  ‘gold’
\z

\section{Consonants}\label{sec:2.2}\is{Prosody (labialisation or palatalization)|)}
\hypertarget{RefHeading1210441525720847}{}
\citet{Bow1997c} reported 31 consonant phonemes{.}\footnote{\citet{Bow1997c} described 30 consonant phonemes although her chart of consonant phonemes included ŋɡʷ, making the total 31.} Since her work, the labiodental flap /\dentalflap / in Moloko has been noted, making the total 32 consonantal phonemes.  

The labiodental flap /\dentalflap  / was first described by \citet{OlsonHajek2004} and is typical of many of the Chadic languages in the Far North Province of Cameroon. In Moloko it is found in ideophones (\ref{ex:2:22}--\ref{ex:2:23}, see \sectref{sec:3.6}). 


\ea \label{ex:2:22}
\textup{[\dentalflap aɓ]}\\
\glt  ‘snake falling’
\z

\ea \label{ex:2:23}
\textup{[ɓa\dentalflap aw]}\\
\glt  ‘men running’ 
\z

Moloko has three sets of sequences which \citet{Bow1997c} interpreted as single units (C) rather than sequences of two consonants (CC).  These are prenasalised consonants /mb/, /nd/, /ŋg/, /nz/, affricates /ts/, /dz/, and labialised consonants /kʷ/, /gʷ/, /ŋgʷ/, /hʷ/. In the case of prenasalised consonants, the nasal is always homorganic with the following consonant.\footnote{Note that the phoneme /n/ assimilates to the point of articulation of a following consonant throughout the language.} Only voiced consonants are prenasalised. 

Allophonic variation for consonants occurs in Moloko due to prosodic conditioning (\sectref{sec:2.2.3}) and word-final variations (\sectref{sec:2.2.4}). There is a relationship between consonants and tone which is considered in \sectref{sec:2.4.1}.


\tabref{tab:2.2} (adapted from \citealt{Bow1997c}) shows place and manner of articulation of all phonetic realisations of consonants in Moloko.  Allophones are shown in parentheses. The individual phonemes and their allophones are considered in Sections \ref{sec:2.2.1}--\ref{sec:2.2.4}.

\begin{table}
\begin{tabular}{p{2cm}lllll} 
\lsptoprule
&  & {Labial} & {Alveolar} & {Velar / Glottal} & {Labio-Velar}\\
\midrule
{Stops} & {}-voice & p & t & k & kʷ\\
& +voice & b & d & g & gʷ\\
& nasal & m & n   (ŋ) &  & \\
& prenasal & mb & nd & ŋg & ŋgʷ  \\
& implosive & ɓ & ɗ &  & \\\midrule
{Affricates} & {}-voice &  & ts   (tʃ) &  & \\
& +voice &  & dz  (dʒ) &  & \\
& prenasal &  & nz  (nʒ) &  & \\\midrule
{Fricatives} & {}-voice & f & s    (ʃ) & h    (x) & hʷ\\
& +voice & v & z    (ʒ) &  & \\\midrule
\multirow{2}{2cm}{{Lateral fricatives}} & {}-voice &  & ɬ &  & \\
& +voice &  & ɮ &  & \\\midrule
{Lateral\newline approximants} &  &  & l &  & \\\midrule
{Approximants} &  &  & j &  & w\\\midrule
{Flaps} &  & \dentalflap & r &  & \\
\lspbottomrule
\end{tabular}
\caption{Consonant phonemes\label{tab:2.2}}
\end{table}



\begin{table} 
\caption{List of phonemes and allophones with phonetic description}\label{tab:2.3}
\resizebox{.975\textwidth}{!}{\begin{tabularx}{\textheight}{lllQ@{\hspace{2.5em}}lllQ}
\lsptoprule
/p/ &	p  & [p] &     voiceless bilabial unaspirated stop & /nz/&   nj  & [nz] &     prenasalised voiced alveolar nasal occurring in unpalatalised syllables\\
/b/ &    b & [b] &     voiced bilabial stop &     &       & [nʒ] &     prenasalised voiced alveopalatal nasal occurring in palatalised syllables\\
/mb/&   mb & [mb]&     prenasalised voiced bilabial stop        & /ɬ/ &   sl  & [ɬ] &     voiceless alveolar lateral fricative\\
/m/ &   m  & [m] &     voiced bilabial nasal & /ɮ/ &   zl  & [ɮ] &     voiced alveolar lateral fricative   \\
/ɓ/ &   ɓ  & [ɓ] &     voiced bilabial stop with ingressive pharynx air (implosive) & /l/ &   l  & [l] &     voiced alveolar lateral approximant\\
/f/ &   f  & [f] &     voiceless labiodental fricative & /r/ &   r  & [r] &     voiced alveolar flap\\
/v/ &   v  & [v] &     voiced labiodental fricative & /\dentalflap / &   vb & [\dentalflap ] &     voiced labiodental flap\\
/t/ &   t  & [t] &     voiceless alveolar unaspirated stop & /j/ &   y  & [j] &     voiced palatal semi-vowel\\
/d/ &   d  & [d] &     voiced alveolar stop & /k/ &   k  & [k] &     voiceless velar unaspirated stop occurring in unlabialised syllables\\
/n/ &   n  & [n] &     voiced alveolar nasal & 	&      & [kʷ] &     voiceless labialised velar stop occurring in labialised words\\
    &      & [ŋ] &     voiced velar nasal occurring word-finally & /kʷ/&  kw/wk\textsuperscript{\textdagger} &[kʷ] &   voiceless labialised velar stop          \\
/nd/&   nd & [nd]&     prenasalised voiced alveolar stop        & /g/ &   g  & [g]  &  voiced velar stop occurring in unlabialised syllables\\
/ɗ/ &   ɗ  & [ɗ] &     voiced alveolar stop with ingressive pharynx air (implosive) &     &      & [gʷ]  &  voiced labialised velar stop occurring in labialised syllables\\
/ts/&   c  & [ts]&     voiceless alveolar affricate occurring in unpalatalised syllables & /gʷ/ & gw & [gʷ]  &  voiced labialised velar stop    \\
	&      & [tʃ] &     voiceless alveopalatal affricate occurring in palatalised syllables &/ŋg/ &   ng & [ŋg] &   prenasalised voiced velar stop occurring in unlabialised syllables\\
/dz/&   j  & [dz] &     voiced alveolar affricate occurring in unpalatalised syllables &     &      & [ŋgʷ] &   voiced prenasalised labialised velar stop occurring in labialised syllables\\
    &      & [dʒ] &     voiced alveopalatal affricate occurring in palatalised syllables\ & /ŋgʷ/  & ngw &  [ŋgʷ] &   voiced prenasalised labialised velar stop\\
/s/ &   s  & [s] &     voiceless alveolar fricative occurring in unpalatalised syllables & /h/ &   h & [h]  &  voiceless glottal fricative occurring word-medially\\
    &      & [ʃ] &     voiceless alveopalatal fricative occurring in palatalised syllables &     &     & [x]  & voiceless velar fricative occurring word-finally  \\
/z/ &   z  & [z] &     voiced alveolar fricative occurring in unpalatalised syllables & /hʷ/ & hw & [hʷ]  &  voiceless labialised glottal fricative    \\    
    &      & [ʒ] &     voiced alveopalatal fricative occurring in palatalised syllables & /w/ &   w & [w]  & voiced labio-velar semi-vowel\\
   \lspbottomrule
\end{tabularx}}
\phantom{\renewcommand*{\thempfootnote}{\fnsymbol{mpfootnote}}\setcounter{mpfootnote}{1}\footnote{Orthographically, ‘kw’ is word-initial and word-medial, ‘wk’ is word-final.}\renewcommand{\thempfootnote}{\alph{mpfootnote}}}
%\end{sideways}
\end{table}

\subsection{Phonetic description}\label{sec:2.2.1}%%%\is{Clitics!Criteria for|)}
\hypertarget{RefHeading1210461525720847}{}
The list of phonemes and allophones with phonetic description shown in \tabref{tab:2.3} is adapted from \citet{Bow1997c} and includes additions from our work done since then. The phoneme (inside slashes), the phonetic form (in square brackets), and the orthographic form (non-bracketed) are shown for each consonant phone. All sounds are made with egressive lung air except where otherwise stated (i.e. implosives are made with ingressive pharynx air). The orthography is discussed in \citet{Friesen2001}. The orthography conforms to the General Alphabet for Cameroonian Languages. Examples in the grammar sections are written using both the orthography (top line) and phonetic transcription so that both speakers of Moloko and outside linguists can appreciate them. 

\subsection{Underlyingly labialised consonants}\label{sec:2.2.2}
\hypertarget{RefHeading1210481525720847}{}
\citet{Bow1997c} posited the existence of a set of underlyingly labialised consonant phonemes [kʷ, gʷ, ŋgʷ, hʷ]. She showed them to be phonemes even though each of these consonants is also the realisation in labialised words of their non-labialised counterpart (see \sectref{sec:2.2.3}). At the surface phonetic level, Bow showed that a labialised velar can have two possible sources, either a labialisation prosody across the whole word \REF{ex:2:24}, or the presence of an underlyingly labialised consonant \REF{ex:2:25}. Example \REF{ex:2:24} shows consistently labialised vowels indicating labialisation across whole word, while the palatalised vowels in \REF{ex:2:25} indicate that there is a palatalisation prosody across the whole word; with the presence of an underlyingly labialised velar consonant.


\ea \label{ex:2:24}
\textup{/dz g r \textsuperscript{o}}\textup{/ \hspace{15pt} [dzʊgʷɔr]}\\
\glt  ‘stake’ 
\z

\ea \label{ex:2:25}
\textup{/}\textup{dza gʷ r \textsuperscript{e}}\textup{/ \hspace{5pt} [dʒœgʷɛr]}\\
\glt  ‘limpness’ 
\z

\citet{Bow1997c} found underlyingly labialised consonants in words which do not have a labialisation prosody across the whole word.  She concluded that the labialisation feature was attached only to these velar consonants within a word since the prosody only affected those particular consonants and the vowels immediately adjacent to them, while other consonants and vowels within the word were unaffected by the labialisation prosody.\footnote{Another interpretive option could be positing that the labialisation prosody touches down on the velar consonant but something prevents it from spreading to the rest of the word (Smith, personal communication). For the purposes of this work, we will consider the labialised velar to be a separate phoneme rather than a supra-segmental phenomenon.}  

\tabref{tab:2.4} (adapted from \citealt{Bow1997c}) shows two pairs of words that are distinguished by the contrast between the underlyingly labialised and non-labialised velars.

\begin{table}
\resizebox{\textwidth}{!}{\begin{tabular}{llllll}
\lsptoprule
 \multicolumn{3}{c}{{Labialised consonant}} & \multicolumn{3}{c}{{Word-level prosody}}\\\cmidrule(lr){1-3}\cmidrule(lr){4-6}
{Underlying form} & {Phonetic form} & {Gloss} & {Underlying form} & {Phonetic form} & {Gloss}\\\midrule
/s l k \textsuperscript{e}/ & [ʃɪlɛk] & ‘jealousy’ 		 &             /s l kʷ  \textsuperscript{e}/ & [ʃɪlœkʷ] & ‘broom’\\
/g la \textsuperscript{o}/ & [gʷʊlɔ] & ‘left’  &        /gʷ la/ & [gʷʊla] & ‘son’\\
/ha ɗa \textsuperscript{o}/ & [hʷɔɗɔ] & ‘wall’ &       /hʷa ɗa/ & [hʷɔɗa] & ‘dregs’\\
\lspbottomrule
\end{tabular}}
\caption{Minimal pairs for word-level labialised prosody vs. labialised consonant\label{tab:2.4}}
\end{table}

\tabref{tab:2.5} illustrates words containing each of the labialised velar phonemes. The labialised velars may occur as the word-initial consonant, medial consonant in palatalised words or words of neutral prosody. Only voiceless labialised velars can occur in word-final position (see \sectref{sec:2.2.4}). It is interesting that there are no words of neutral prosody which can have a labialised velar in word-final position. Note that only the vowels that immediately surround a labialised velar consonant are affected by the prosody of the velar consonant (see Section~ \ref{sec:2.3.3}).

\begin{table}
% \resizebox{\textwidth}{!}{
\begin{tabular}{llll} 
\lsptoprule
& {Initial} & {Medial} & {Final}\\
\midrule
{Neutral prosody} & [kʷʊsaj]      & [tʊkʷʊrak]     \\
& ‘haze’ & ‘partridge’ &\\
& & [agʷɔɮak]       & \\
& & ‘rooster’ &\\
{Palatalisation} & [kʷʊtʃɛɬ]      & [mɛtʃœkʷɛɗ]   & [pɛɗœkʷ]       \\
& ‘viper’ & ‘maggot’ & ‘blade’\\
& [gʷʊdɛɗɛk]  & {}[mɛdɛlœŋgʷɛʒ]  & \\
& ‘frog’ & ‘leopard’ & \\
& & [ahʷœɗɛ]       & \\
& & ‘fingernail’ & \\
\lspbottomrule
\end{tabular}
%}
\caption{Distribution of labialised velar phonemes\label{tab:2.5}}
\end{table}

\largerpage
\citet{Bow1997c} found there are several cases in the data where it was impossible to tell whether the consonant is underlyingly labialised or there is a labialisation prosody across the word, as in \REF{ex:2:26} and \REF{ex:2:27} (from \citealt{Bow1997c}).

\ea \label{ex:2:26}
\textup{/s kʷ m/ {\textasciitilde} /s k m \textsuperscript{o}}\textup{/   \ExampleSpace \hspace{15pt}    [sʊkʷɔm]}\\
\glt  ‘buy/sell’      
\z
\clearpage
\ea \label{ex:2:27}
\textup{/ma gʷ m/ {\textasciitilde}  /ma g m \textsuperscript{o}}\textup{/  \ExampleSpace [mɔgʷɔm]}\\
\glt  ‘home’
\z

Our further work on verb conjugations clarified that \REF{ex:2:26} actually contains a labialised velar (i.e., the underlying form is /s kʷ m/). The nominalised form of the verb is palatalised, yet the labialised velar is still present \REF{ex:2:28}. If there was no underlyingly labialised velar, the nominalised form would have been *[mɪsɪkɪmɛ].

\ea \label{ex:2:28}
mɪ-sɪkʷøm-ɛ\\
      {\NOM}{}-buy-{\CL}\\
\glt  ‘buying’
\z

\subsection{Prosodic conditioning of consonant allophones}\label{sec:2.2.3}\is{Prosody (labialisation or palatalization)|(}
\hypertarget{RefHeading1210501525720847}{}
\tabref{tab:2.6} (adapted from \citealt{Bow1997c}) shows the effect of prosodic conditioning on each consonant phone.  Each consonant phone (reading down the table) is shown in three environments, one without any prosody, one with a labialisation and one with a palatalisation prosody. The table illustrates that prosody has an effect on fricatives, affricates, and back consonants (velar and glottal).

The fricatives [s, z, nz] and affricates [ts, dz] are in complementary distribution with [ʃ, ʒ, nʒ] and [tʃ, dʒ], respectively, with the second group only appearing in palatalised words.  

Labialisation affects the back consonants such that [k, g, ŋg, h] are in complementary distribution with [kʷ, gʷ, ŋgʷ, hʷ], with the second group only appearing in labialised words. Note however that there is a set of underlyingly labialised back consonant phonemes (see \sectref{sec:2.2.2}).

Note also that the labiodental flap [\dentalflap ] is found only in ideophones (\sectref{sec:3.6}) that have a neutral prosody.

\begin{table}
\resizebox{.975\textwidth}{!}{%
\begin{tabular}{lllllll}
\lsptoprule
& {Neutral} & {Gloss} & {Labialised} & {Gloss} & {Palatalised} & {Gloss}\\\midrule
\multicolumn{7}{c}{{Stops}}\\\midrule
p & [paj] & ‘open’ & [apɔŋgʷɔ] & ‘mushroom’ & [pɛmbɛʒ] & ‘blood’\\
 b & [baj] & ‘light’ & [abɔr] & ‘lust’ & [bɛkɛ] & ‘slave’\\
 ɓ & [ɓaj] & ‘hit’ & [aɓɔlɔ] & ‘yam’ & [ɓɛɮɛŋ] & ‘count’\\
 m & [maj] & ‘hunger’ & [mɔlɔ] & ‘twin’ & [amɛlɛk] & ‘bracelet’\\
 mb & [mbaj] & ‘follow’ & [ambɔlɔ] & ‘bag’ & [mbɛ] & ‘argue’\\
 t & [tar] & ‘call’ & [atɔs] & ‘hedgehog’ & [tɛʒɛh] & ‘boa’\\
 d & [dar] & ‘burn’ & [dɔkʷɔj] & ‘arrive’ & [dɛ] & ‘cook’\\
 ɗ & [ɗas] & ‘weigh’ & [ɗɔgʷɔm] & ‘nape’ & [ɗɛ] & ‘flourish’\\
 n & [nax] & ‘ripen’ & [sɔnɔ] & ‘joke’ & [ɛnɛŋ] & ‘snake’\\
  ŋ & [ɮaŋ] & ‘start’ & [tɔlɔlɔŋ] & ‘heart’ & [ɓɛɮɛŋ] & ‘count’\\
 nd & [ndar] & ‘weave’ & [ndɔɮaj] & ‘explode’ & [ndɛ] & ‘lie down’\\
 k & [kaɬ] & ‘wait’ &  &  & [bɛkɛ] & ‘slave’\\
  g & [gar] & ‘grow’ &  &  & [gɛ] & ‘do’\\
 ŋg & [ŋgaj] & ‘set’ &  &  & [fɛŋgɛ] & ‘termite mound’\\
kʷ & [kʷʊsaj] & ‘fog’ & [kʷcndɔŋ] & ‘banana’ & [ajœkʷ] & ‘ground nut’\\
 gʷ & [agʷɔɮak] & ‘cockerel’ & [gʷɔrɔ] & ‘kola’ & [dʒœgʷɛr] & ‘limpness’\\
 ŋgʷ & [ŋgʷʊdaɬaj] & ‘simmer’ & [aŋgʷɔlɔ] & ‘return’ & [adɔngʷɛrɛɗ] & ‘type of tree’\\
 \midrule\multicolumn{7}{c}{{Fricatives and Affricates}}\\\midrule
 f & [far] & ‘itch’ & [fɔkʷɔj] & ‘whistle’ & [fɛ] & ‘play instrument’\\
 v & [vaj] & ‘winnow’ & [avɔlɔm] & ‘ladle’ & [vɛ] & ‘spend (time)’\\
 s & [sar] & ‘know’ & [sɔnɔ] & ‘joke’ &  & \\
 z & [zaj] & ‘peace’ & [zɔm] & ‘eat’ &  & \\
 ts & [tsar] & ‘climb’ & [tsɔkʷɔr] & ‘fish net’ &  & \\
 dz & [dzaj] & ‘speak’ & [dzɔgʷɔ] & ‘hat’ &  & \\
 nz & [nzakaj] & ‘find’ & [nzɔm] & ‘sit down’ &  & \\
 h & [haj] & ‘millet’ &  &  & [mɛhɛr] & ‘forehead’\\
 x & [rax] & ‘satisfy’ &  &  & [tɛʒɛx] & ‘boa’\\
 hʷ & [hʷɔɗa] & ‘dregs’ & [hʷɔr] & ‘woman’ & [ahʷœɗɛ] & ‘fingernail’\\
 ʃ &  &  &  &  & [ʃɛ] & ‘drink’\\
 ʒ &  &  &  &  & [ʒɛ] & ‘smell’\\
 tʃ &  &  &  &  & [tʃɛ] & ‘lack’\\
 dʒ &  &  &  &  & [dʒɛŋ] & ‘luck’\\
 nʒ &  &  &  &  & [nʒɛ] & ‘sit down’\\
 \midrule\multicolumn{7}{c}{{Laterals}}\\\midrule
  ɬ & [ɬaj] & ‘slit’ & [ɬɔkʷɔ] & ‘earring’ & [aɬɛɬɛɗ] & ‘egg’\\
ɮ & [ɮaŋ] & ‘start’ & [bɛɮɛm] & ‘cheek’ & [aɮɛrɛ] & ‘lance’\\
 l & [laj] & ‘dig’ & [lɔ] & ‘go’ & [lɪhɛ] & ‘bush’\\
\midrule\multicolumn{7}{c}{{Flaps}}\\\midrule
 r & [rax] & ‘satisfy’ & [arɔx] & ‘pus’ & [tɛrɛ] & ‘other’\\
 \dentalflap  & [pə\dentalflap aŋ] & ‘start of race’ &  &  &  & \\
\midrule\multicolumn{7}{c}{{Semivowels}}\\\midrule
 j & [jam] & ‘water’ & [sɔkʷɔj] & ‘clan’ & [ajɛwɛɗ] & ‘whip’\\
 w & [war] & ‘child’ & [wuldɔj] & ‘devour’ & [wɛ] & ‘give birth’\\
\lspbottomrule
\end{tabular}}
\caption{Prosodic conditioning of consonant phonemes\label{tab:2.6}}
\end{table}
\is{Prosody (labialisation or palatalization)|)}
\subsection{Non-prosodic conditioning of consonants}\label{sec:2.2.4}
\hypertarget{RefHeading1210521525720847}{}

Word-final position influences the distribution of certain phonemes as well as the production of allophones. The following phonemes do not occur in word-final position: voiced stops (including prenasalised stops but excluding /m/ and the implosives), voiced affricates, and the labiodental flap i.e., [b, mb, d, nd, g, gʷ, ŋg, ŋgʷ, dz, dʒ , nz, nʒ, \dentalflap ]. Also, [x] and [ŋ] are the word-final allophones of /h/ and /n/, respectively (\sectref{sec:2.2.4.1}). In some contexts, word-final /r/ can be realised as [l] (\sectref{sec:2.2.4.2}). \tabref{tab:2.7} (adapted from \citealt{Bow1997c}) shows the distribution of each consonant phone (reading down) in different positions within the word (reading across).  

\begin{table}
\resizebox{.975\textwidth}{!}{%
\begin{tabular}{lllllll} 
\lsptoprule	
 & {Initial} &  & {Medial} &  & {Final} & \\
\midrule
\multicolumn{7}{c}{Voiceless stops and affricates}\\\midrule 
 p & [palaj] & ‘choose’ & [kapaj] & ‘roughcast’ & [dap] & ‘fake’ \\
 t & [talaj] & ‘walk’ & [fataj] & ‘descend’ & [mat] & ‘die’\\
 k & [kapaj] & ‘roughcast’ & [makaj] & ‘leave/let go’ & [sak] & ‘multiply’\\
 kʷ & [kʷʊsaj] & ‘fog’ & [tʊkʷasaj] & ‘cross/fold’ & [ajœkʷ] & ‘ground nut’\\
 ts & [tsahaj] & ‘ask’ & [watsaj] & ‘write’ & [harats] & ‘scorpion’ \\
 tʃ & [tʃɛ tʃɛ] & ‘all’ & [mɛtʃɛkʷɛɗ] & ‘worm’ & [mɛkɛtʃ] & ‘knife’\\
\midrule\multicolumn{7}{c}{Implosives}\\\midrule
 ɓ & [ɓalaj] & ‘build’ & [ndaɓaj] & ‘wet/whip’ & [haɓ] & ‘break’\\
 ɗ & [ɗakaj] & ‘indicate’ & [jaɗaj] & ‘tire’ & [zaɗ] & ‘take’ \\
\midrule\multicolumn{7}{c}{Fricatives}\\\midrule
 f & [fataj] & ‘descend’ & [dafaj] & ‘bump’ & [taf ] & ‘spit’\\
 v & [vakaj] & ‘burn’ & [ɮavaj] & ‘swim’ & [dzav] & ‘plant’\\
 s & [sakaj] & ‘sift’ & [pasaj] & ‘detatch’ & [was] & ‘farm’\\
 ʃ & [ʃɛdɛ] & ‘witness’ & [ʃɛʃɛ] & ‘meat’ & [pɪlɛʃ] & ‘horse’\\
 z & [zaɗ] & ‘take’ & [wazaj] & ‘shake’ & [baz] & ‘reap’\\
 ʒ & [ʒɛ] & ‘smell’ & [mɪʒɛ] & ‘person’ & [mɛdɪlɪŋgʷœʒ] & ‘leopard’\\
 h & [halaj] & ‘gather’ & [mbahaj] & ‘call’ &  & \\
 hʷ & [hʷʊlɛŋ] & ‘back’ & [tʃœhʷɛɬ] & ‘stalk’ &  & \\
 x &  &  &  &  & [ɓax] & ‘sew’\\
\midrule\multicolumn{7}{c}{Laterals, approximants, flap, and semivowels}\\ \midrule
 ɬ & [ɬaraj] & ‘slide’ & [tsaɬaj] & ‘pierce’ & [kaɬ] & ‘wait’\\
 ɮ & [ɮavaj] & ‘swim’ & [daɮaj] & ‘join/tie’ & [mbaɮ] & ‘demolish’ \\
 l & [lagaj] & ‘accompany’ & [balaj] & ‘wash’ & [wal] & ‘attach’\\
 r & [rax] & ‘pluck’ & [garaj] & ‘command’ & [sar] & ‘know’\\
 \dentalflap  & [\dentalflap ə\dentalflap ə\dentalflap ə] & ‘rapidly’ & [ɓa\dentalflap aw] & ‘man running’ &  & \\
 j & [jaɗaj] & ‘tire’ & [haja] & ‘grind’ & [balaj] & ‘wash’\\
 w & [watsaj] & ‘write’ & [ɮawaj] & ‘fear’ & [mahaw] & ‘snake’\\
\midrule\multicolumn{7}{c}{Voiced stops and affricates}\\ \midrule
 m & [makaj] & ‘leave/let go’ & [lamaj] & ‘touch’ & [tam] & ‘save’\\
 b & [balaj] & ‘wash’ & [abaj] & ‘there is none’ &  & \\
 mb & [mbahaj] & ‘call’ & [hambar] & ‘skin’ &  & \\
 d & [daraj] & ‘snore’ & [hadak] & ‘thorn' &  & \\
 nd & [ndavaj] & ‘finish’ & [dandaj] & ‘intestines’ &  & \\
 n & [nax] & ‘ripen’ & [zana] & ‘cloth’ &  & \\
 g & [garaj] & ‘command’ & [lagaj] & ‘accompany’ &  & \\
 gʷ & [gʷʊlɛk] & ‘small axe’ & [agʷɔɮak] & ‘rooster’ &  & \\
 ŋg & [ŋgaɮaj] & ‘introduce’ & [maŋgaɬ] & ‘fiancée’ &  & \\ ŋgʷ & [ŋgʷʊdaɬaj] & ‘simmer’ & [aŋgʷʊrɮa] & ‘sparrow’ &  & \\
 dz & [dzakaj] & ‘lean’ & [dzadzaj] & ‘dawn/light’ &  & \\
  dʒ & [dʒɛŋ] & ‘luck’ & [tʃɪdʒɛ] & ‘illness’ &  & \\
 nz & [nzakaj] & ‘find’ & [manzaw] & ‘beignet’ &  & \\
 nʒ & [nʒɛ] & ‘sit’ & [hɪrnʒɛ] & ‘quarrel’ &  & \\
 ŋ &  &  &  &  & [hadzaŋ] & ‘tomorrow’\\
\lspbottomrule
\end{tabular}}
\caption{Non-prosodic conditioning of consonant phonemes\label{tab:2.7}}
\end{table}

\subsubsection{Word-final allophones of /n/ and /h/}\label{sec:2.2.4.1}

\citet{Bow1997c} demonstrates that [n] and [ŋ] are allophones of /n/ with a distribution as shown in \figref{fig:2.2}.

\begin{figure}\caption{Word-final allophone of /n/\label{fig:2.2}}
n → ŋ / \_ \#
\end{figure}

\tabref{tab:2.8} (adapted from \citealt{Bow1997c}) illustrates [n]{ }and [ŋ] in complementary distribution (with [n]{ } initially and medially and [ŋ]  finally).

\begin{table}[H] % H option at author's request
\resizebox{\textwidth}{!}{\begin{tabular}{lllllll}
\lsptoprule
{Prosody} & \multicolumn{2}{l}{  {Initial}} & \multicolumn{2}{l}{ {Medial}} & \multicolumn{2}{l}{ {Final}}\\\midrule
{Neutral} & [\textbf{n}ax] & ‘ripen’ & [gə\textbf{n}aw] & ‘animal’ & [=aha\textbf{ŋ}] & =\SSS.{\POSS}\\
{Labialised} & [\textbf{n}ɔkʷ] & ‘you’ & [a\textbf{n}a] & ‘to’ (dative) & [tɔlɔlɔŋ] & ‘heart’\\
{Palatalised} & [\textbf{n}ɛ] & ‘me' & [mɪtɛ\textbf{n}ɛŋ] & ‘bottom’ & [mɪtɛnɛ\textbf{ŋ}] & ‘bottom’\\
\lspbottomrule
\end{tabular}}
\caption{Complementary distribution for /n/\label{tab:2.8}}
\end{table}

Likewise, \citet{Bow1997c} demonstrates that [h] and [x] are allophones of /h/ with a distribution as shown in \figref{fig:2.3}. 

\begin{figure}
\centering h → x / \_ \#
\caption{Word-final allophone of /h/\label{fig:2.3}}
\end{figure}

\largerpage
\tabref{tab:2.9} shows [x] and [h] in complementary distribution (with [h]{ }initially and medially and [x] finally).

\begin{table}[H] % H option at author's request.
\resizebox{\textwidth}{!}{\begin{tabular}{lllllll}
\lsptoprule

{Prosody} & \multicolumn{2}{l}{{Initial}} & \multicolumn{2}{l}{{Medial}} & \multicolumn{2}{l}{{Final}}\\\midrule
{Neutral} & [har] & ‘make’ & [ahar] & ‘hand’ & [rax] & ‘satisfy’\\
{Labialised} & [hʷʊdɔ] & ‘wall’ & [tɔhʷɔr] & ‘cheek’ & [hʷɔmbɔx] & ‘pardon’\\
{Palatalised} & [hɛrɛɓ] & ‘heat’ & [mɛhɛr] & ‘forehead’ & [tɛʒɛx] & ‘boa’\\
\lspbottomrule
\end{tabular}}
\caption{Complementary distribution for /h/\label{tab:2.9}}
\end{table}

\subsubsection{Word-final allophones of /r/}\label{sec:2.2.4.2}

\citet{FriesenMamalis2008} demonstrated that for some verb roots, final /r/ is realise as [l] in certain contexts.\footnote{This process does not appear to be free variation.} In \REF{ex:2:29} and \REF{ex:2:30}, which are consecutive lines from a narrative text, the final /r/ of the verb /v r/\textit{ }‘give’ is [r] in  \textit{navar}\textit{ }‘I give’ \REF{ex:2:30} but is realised as [l] when the indirect object pronominal enclitic =\textit{aw} (see \sectref{sec:7.3.1.1}) is attached \REF{ex:2:29}: 

\ea \label{ex:2:29}
\gll [vəl=aw                                  kɪndɛw    =aŋgʷɔ     na        ɛhɛ]\\
     {give[{\twoS}.{\IMP}]={\oneS}.{\IO}}      guitar    ={\twoS}.{\POSS}             {\PSP}  here\\
\glt  ‘Give me your guitar, here!’
\z

\ea \label{ex:2:30}
\gll  [na-var   na                              baj]\\
      {\oneS}-give    \oldstylenums{3}\textsc{s}.{\DO}  {\NEG}\\
\glt  ‘I won’t give it.’  
\z

Likewise, the verb /war/ ‘hurt’ exhibits similar changes, where the word-final /r/ in \REF{ex:2:31} becomes [l] when the indirect object pronominal enclitic attaches \REF{ex:2:32}. 

\ea \label{ex:2:31}
\gll  [həmaɗ   a-war   gam]\\
      wind  \oldstylenums{3}\textsc{s}-hurt  much\\
\glt  ‘It’s very cold.’ (lit. wind hurts a lot)
\z

\ea \label{ex:2:32}
\gll  [həmaɗ   a-wal                          =alɔkʷɔ]\\
      wind  \oldstylenums{3}\textsc{s}-hurt  =\oldstylenums{1}\textsc{Pin}.{\IO}\\
\glt  ‘We’re cold.’ (lit. wind hurts us)
\z

\section{Vowels}\label{sec:2.3}
\hypertarget{RefHeading1210541525720847}{}
\largerpage
There are ten surface phonetic vowels in Moloko (\tabref{tab:2.10}) but the vowel system can be analysed as having one underlying vowel /a/.\footnote{An analysis by \citet{Bow1999} using Optimality Theory allowed both a single underlying vowel system (/a/) or a two underlying vowel system (/a/ and /ə/).  For the purposes of this work, the schwa is considered as epenthetic since its presence is predictable, and /a/ is considered the only underlying vowel phoneme.}  This vowel may be either present or absent between any two consonants in the underlying form of a morpheme. \citet{Bow1997c} found that the absence of a vowel requires an epenthetic vowel to break up some consonant clusters in the surface form.\footnote{Certain consonants do not require epenthetic schwa insertion (\sectref{sec:2.5.1}).} Different environments acting on the underlying vowel and the epenthetic [ə] result in the ten allophones in Moloko (four from /a/: [a, ɛ, ɔ, œ]\footnote{\citet{Bow1997c} reported ten surface vowel forms including [æ] which she did not consider as a distinct allophone since not all speakers distinguish between [a] and [æ], leaving nine allophones. \citet{Friesen2001} added [ø].} and six from the epenthetic schwa: [ə, ɪ, ʊ, ø, i, u]).  Note the addition of the vowel [ø] not in Bow’s analysis. Bow noted “a phonetic gap left by the absence of a high vowel with both palatalisation and labialisation.” This work reports the presence of this vowel in environments affected by both prosodies (see \sectref{sec:2.3.3}). 

\begin{table}
\resizebox{\textwidth}{!}{%
\begin{tabular}{l@{\hspace{.5em}}p{4cm}l@{\hspace{.25em}}lll@{\hspace{.25em}}ll} 
\lsptoprule
&  & {/a/} & & \multicolumn{1}{l}{{Example}} & \multicolumn{2}{l}{{Epenthetic ə}} &  {Example}\\
\midrule
{1} & {No word-level process} & [a] & \textbf{a} & [awak]   \textbf{awak}  & [ə] & \textbf{ə} & [gəgəmaj]  \textbf{gəgəmay} \\
& & & & ‘goat’ &  & & ‘cotton’ \\
{2} & {Labialisation } & [ɔ] & \textbf{o} & [sɔnɔ]     \textbf{sono} & [ʊ]   & \textbf{ə} & [mʊlɔkʷɔ]  \textbf{Məloko}\\
& & & & ‘game’ & & & ‘Moloko’ \\
{3} & {Palatalisation} & [ɛ]   & \textbf{e} & [ʃɛʃɛ]      \textbf{sese} & [ɪ]  &    \textbf{ə} & [ʃɪlɛk]       \textbf{səlek}\\
& & & & ‘meat’ & & & ‘jealousy’ \\
{4} & {Adjacent to  [j]} & [a]  &  \textbf{a} & [haja]     \textbf{haya} & [i]         &      \textbf{ə} & [kija]                  \textbf{kəya}\\
& & & & ‘grind’ & & & ‘moon’ \\
{5} & {Adjacent to  [w]} & [a]   &  \textbf{a} & [mawar] \textbf{mawar} & [u]   &   \textbf{ə} & [ɗuwa]      \textbf{ɗəwa}\\
& & & & ‘tamarind’ & & & ‘milk’ \\
{6} & {Adjacent to an inherent}  & [œ] & \textbf{e} & [ʃɪlœkʷ] \textbf{səlewk} & [ø]  &  \textbf{ə} & [lʊkʷøjɛ]     \textbf{ləkwəye}\\
& {labio-velar or /j/} & & & ‘broom’ & &  & ‘you’ (Pl)\\
\lspbottomrule
\end{tabular}}
\caption{Sources of allophonic variation in vowels with orthographic representation\label{tab:2.10}}
\end{table}

\citet{Bow1997c} distinguished the vowels in Moloko using four features: height, tense (or ATR), palatalisation, and labialisation. In this work, the conditioning environments that affect the phonetic expression of a full or epenthetic vowel include the labialisation and palatalisation prosodies (\sectref{sec:2.3.2})  and adjacency of the epenthetic vowel to particular consonants (\sectref{sec:2.3.3}). 

\subsection{Vowel phonemes and allophones}\label{sec:2.3.1}
\hypertarget{RefHeading1210561525720847}{}
\tabref{tab:2.10} is a summary table showing the sources of allophonic variation and the resulting phonetic realisations and orthographic representations. In the table, the orthographic representation of each of these phonetic vowels is bolded and follows each vowel or example in the table.\footnote{The orthographic representation is not employed elsewhere in the chapter, since it is important that the reader appreciate the phonetic expression. However, in the grammar chapters, the orthography is given for each example. } For each source of allophonic variation, an example is also given. In a word which is neutral with respect to prosody (line 1), the underlying vowel is pronounced [a] and epenthetic schwa [ə]. In labialised words, (line 2), /a/ becomes [ɔ] and the epenthetic schwa becomes [ʊ].  In palatalised words (line 3), /a/ is pronounced [ɛ] and the epenthetic schwa is pronounced [ɪ]. The epenthetic vowel can also be assimilated to a neighbouring approximant: it is realised as [i] when it occurs beside [j] (line 4) and as [u] when it occurs beside a labialised velar [w, kʷ, gʷ, ŋgʷ, hʷ] (line 5). Under the influence of labialised velars and an adjacent /j/, the /a/ becomes [œ] and the epenthetic schwa becomes [ø] (line 6).

The working orthography for Moloko \citep{Friesen2001} indicates the word-level processes by the three full vowel graphemes in the word pronounced in isolation: <e> in palatalised words, <o> in labialised words, and ‘a’ in words with neutral prosody.\footnote{Even if the palatalisation or labialisation is incomplete in a word beginning with /a/, that first vowel is written <e> or <o>, respectively, in the orthography. }  Epenthetic vowels are written as <ə> in the orthographic representation regardless of the word prosody, because their pronunciation is predictable from the word prosody (discernable from the full vowel in the word) and the surrounding consonants.  This results in four orthographic vowel symbols (a, e, o, ə).

\subsection{Prosodic conditioning of vowel allophones}\label{sec:2.3.2}\is{Prosody (labialisation or palatalization)|(}
\hypertarget{RefHeading1210581525720847}{}
\citet{Bow1997c} reports that there is a clear prosodic pattern in Moloko where, with  few exceptions,\footnote{Labialisation and palatalisation in words which begin with a vowel will sometimes be incomplete, leaving the first syllable as [a] for labialised words and [æ] for palatalised words (see \sectref{sec:2.1}).} all vowels in any word will have the same prosody, be it labialised, palatalised, or neutral.  \tabref{tab:2.11} (adapted from \citealt{Bow1997c}) illustrates the three possible underlying prosody patterns in two and three syllable words.\footnote{Adjacency to certain consonants can also affect the quality of a particular vowel (\sectref{sec:2.3.3}).}


\begin{table}
\resizebox{\textwidth}{!}{\begin{tabular}{lllllll} 
\lsptoprule
& \multicolumn{3}{l}{{Two syllable stems}} & \multicolumn{3}{l}{{Three syllable stems}}\\
\midrule
{Neutral} & /ha r ts/ & [harats] & ‘scorpion’ & /ma ta b ɬ/ & [matabaɬ] & ‘cloud’\\
& /d r j/ & [dəraj] & ‘head’ & /g g m j/ & [gəgəmaj] & ‘cotton’\\
{LAB} & /ba ɮ m \textsuperscript{o}/ & [bɔɮɔm] & ‘cheek’ & /ta la l n \textsuperscript{o}/ & [tɔlɔlɔŋ] & ‘chest’\\
& /s k j \textsuperscript{o}/ & [sʊkʷɔj] & ‘clan’ & /ga g l v n \textsuperscript{o}/ & [gʷɔgʷʊlvɔŋ] & ‘snake’\\
{PAL} & /ma h r \textsuperscript{e}/ & [mɛhɛr] & ‘forehead’ & /ma ba b k \textsuperscript{e}/ & [mɛbɛbɛk] & ‘bat’\\
& /ɮ ga \textsuperscript{e}/ & [ɮɪgɛ] & ‘sow’ & /ts ka la \textsuperscript{e}/ & [tʃɪkɛlɛ] & ‘price’\\
\lspbottomrule
\end{tabular}}
\caption{Underlying prosody patterns in two and three syllable words\label{tab:2.11}}
\end{table}
\is{Prosody (labialisation or palatalization)|)}
\subsection{ Non-prosodic conditioning of vowel allophones}\label{sec:2.3.3}
\hypertarget{RefHeading1210601525720847}{}
\citet{Bow1997c} reported that, besides the prosodies of labialisation and palatalisation, the epenthetic vowel allophones are conditioned by the phonemes /j/ and /w/ as well as the underlyingly labialised consonants.  The rules governing these two conditioning environments follow, along with examples of each. Bow found that the epenthetic vowel assimilates to the palatal and labial features of an adjacent semi-vowel even when there is a prosody on the root.  \figref{fig:2.4} and \figref{fig:2.5} illustrate the rules for the influence of /j/\footnote{We found no cases of *[ji].} and /w/ with examples of each (\ref{ex:2:33}--\ref{ex:2:37}). 

\begin{figure}
\begin{centering}[ə] → [i] / \_ j\end{centering}
\caption{Influence of j on ə\label{fig:2.4}}
\end{figure}


\ea \label{ex:2:33}
\textup{/k ja/   \ExampleSpace \hspace{10pt} [kija] }\\
\glt  ‘moon’  
\z

\ea \label{ex:2:34}
\textup{/m j k }\textup{\textsuperscript{e}}\textup{/  \ExampleSpace  [mijɛk] }\\
\glt  ‘deer’
\z

\begin{figure}\caption{Influence of w on ə\label{fig:2.5}}
\begin{centering}[ə] → [u] / \_ w\end{centering}\\
\begin{centering}[ə] → [u] / w \_\end{centering}
\end{figure}


\ea \label{ex:2:35}
\textup{/ɗ wa/  \ExampleSpace \hspace{15pt}   [ɗuwa] }\\
\glt  ‘milk/breast’    
\z

\ea \label{ex:2:36}
\textup{/ɗ w r \textsuperscript{e}}\textup{ /   \ExampleSpace  \hspace{5pt} [ɗuwɛr] }\\
\glt  ‘sleep’
\z

\ea \label{ex:2:37}
\textup{/w ɗa k -j/  \ExampleSpace   [wuɗakaj]  }\\
\glt  ‘separate/share’
\z

Bow found that the vowel phoneme /a/ is not affected by semi-vowels, as demonstrated in \REF{ex:2:38} and \REF{ex:2:39}.

\ea \label{ex:2:38}
\textup{/ja ɗ -j/  \ExampleSpace    [jaɗaj]    not *[jɛɗɛj]}\\
\glt  ‘tire’   
\z

\ea \label{ex:2:39}
\textup{/g n w/   \ExampleSpace [gənaw]     not *[gənɔw]}\\
\glt  ‘animal’    
\z

Bow noted that the semi-vowels themselves do not cause morpheme-level palatalisation or labialisation to occur.  (\ref{ex:2:40}--\ref{ex:2:44}) illustrate that the presence of the labiovelar semi-vowel /w/ in any position within a word (including word-finally) does not effect a labialisation prosody across the word. In fact, the existing data lists no examples of words containing /w/ which have a word-level labialisation prosody.

\ea \label{ex:2:40}
\textup{/ma w r/    \ExampleSpace  \hspace{12pt} [mawar]}\\
\glt  ‘tamarind’   
\z

\ea \label{ex:2:41}
\textup{/da da wa  \textsuperscript{e}}\textup{/  \ExampleSpace   [dɛdɛwɛ]}\\
\glt  ‘a species of bird’
\z

Similarly with the palatal semi-vowel, Bow shows that the presence of /j/ does not effect a palatalisation prosody across the word (\ref{ex:2:42}--\ref{ex:2:44}), although it may occur within a palatalised or labialised word.

\ea \label{ex:2:42}
\textup{/la j w/   \ExampleSpace \hspace{5pt} [lajaw]}\\
\glt  ‘large squash’
\z

\ea \label{ex:2:43}
\textup{/s k j \textsuperscript{o}}\textup{/    \ExampleSpace \hspace{5pt}   [sʊkʷɔj] }\\
\glt  ‘clan’    
\z

\ea \label{ex:2:44}
\textup{/ha j w\textsuperscript{ e}}\textup{/     \ExampleSpace [hɛjɛw]}\\
\glt  ‘cricket’  
\z

This work also illustrates the rules governing the production of [œ] and the combined influence on the epenthetic vowel of adjacency to /j/ and either /w/ or /kʷ/  to produce [ø]. An underlying /a/ is realised as [œ] when it occurs before the labialised velar /kʷ/ in a palatalised word (\ref{ex:2:45}, \figref{fig:2.6}). When an epenthetic schwa occurs between /j/ and a labialised velar (/kʷ/ or /w/ in the examples),\footnote{We have not found the epenthetic vowel between /j/ and any other of the underlyingly labialised consonants (gʷ, ŋgʷ, hʷ, see \sectref{sec:2.2.2}), but we expect it to occur.  Note also that the prosody of the labialised velar affects the quality of the preceding schwa} it is realised as [ø] (\ref{ex:2:46}--\ref{ex:2:47}, \figref{fig:2.7}). It is important to note that the presence of an underlyingly labialised velar consonant also does not cause labialisation of the entire phonological word; in fact, the evidence for their existence stems from this fact (see \sectref{sec:2.2.2}). 

\begin{figure}\caption{Influence of labialised velar on /a/\label{fig:2.6}}
\begin{centering}/a/ → [œ] /   \_\_ Cʷ \textsuperscript{e}/\end{centering}
\end{figure}

\ea \label{ex:2:45}
\textup{/azɛk\textsuperscript{w  e}}\textup{/ \ExampleSpace [æʒœkʷ}\textup{]}\\
\glt  ‘sorry’
\z

\begin{figure}
\begin{centering}[ə] →  [ø] /   kʷ \_ j\end{centering}
\caption{Influence of labialised velar and j on ə\label{fig:2.7}}
\end{figure}

\ea \label{ex:2:46}
\textup{/l kʷ ja \textsuperscript{e}}\textup{/  \ExampleSpace  [lʊkʷøjɛ]}\\
\glt  ‘you (plural)’
\z

\ea \label{ex:2:47}
\textup{/w j n \textsuperscript{e}}\textup{ /  \ExampleSpace \hspace{2pt}  [wøjɛŋ}\textup{]}\\
\glt  ‘land’
\z

\section{Tone}\label{sec:2.4}
\hypertarget{RefHeading1210621525720847}{}
In addition to published manuscripts and a thesis, Bow produced a database and an extensive series of observations relating to lexical and grammatical tone in Moloko nouns and verbs.  This database was later expanded and modified, leading to an initial analysis of tone in noun phrases by \citet{Boyd2002} and later to tone in verbs by \citet{FriesenMamalis2008}.  

\citet{Bow1997c} describes three phonetic tones (H, M, and L) but only two phonemic tones. In this work, lexical tone and grammatical tone are marked when relevant.\footnote{Some data was transcribed without tone.}  The phonetic tone patterns will be indicated on the words using accent marks for H (  \'{ }), M (  \={ }) when necessary, or L tone (  \`{ }).  Because phonetic M can occur due to two causes (see below), this work carefully distinguishes \textit{underlying} tones (H or L) from \textit{phonetic} tones (H, M, and L).

\tabref{tab:2.12} (adapted from \citealt{Bow1997c} with additional data) shows minimal pairs which illustrate the underlying two tone system in Moloko. Tone does not carry a high lexical load, and so there are only a limited number of lexical items distinguished by tone.\footnote{One of each in these minimal pairs are marked in the orthography with a diacritic so that the pairs can be distinguished. } The examples in \tabref{tab:2.12} are divided into grammatical categories. Some of the minimal pairs are from different grammatical categories. 

\begin{table}
\begin{tabular}{llll} 
\lsptoprule
\multicolumn{2}{c}{{H tone}} & \multicolumn{2}{c}{{L tone}}\\
\midrule
\multicolumn{4}{c}{Nouns} \\\midrule
 {[háj]} & ‘millet’ & [hàj] & ‘house/compound’\\
 {[án\={ɛ}ŋ]} & ‘other’ & [àn\={ɛ}ŋ] & ‘snake’\\
 {[g\'{ə}láŋ]} & ‘threshing floor’ & [g\`{ə}l\={a}ŋ] & ‘kitchen/clan’\\
 {[háhàr]} & ‘bean’ & [h\={a}hár] & ‘straw granary’\\
 {[m\={ə}dár\={a}]} & ‘fire’ & [m\`{ə}d\`{ə}rà] & ‘bicep’\\
 {[m\'{ɔ}l\`{ɔ}]} & ‘twin’ & [m\`{ɔ}l\`{ɔ}] & ‘vulture’\\
 {[\={ɛ}l\'{ɛ}]} & ‘eye’ & [\={ɛ}l\`{ɛ}] & ‘thing’\\
 {[v\'{ɛ}r]} & ‘grinding stone’ & [v\`{ɛ}r] & ‘room’\\
\midrule\multicolumn{4}{c}{Verbs} \\\midrule
 {[dár]} & ‘burn’ & [dàr] & ‘withdraw/recoil’\\
 {[h\={a}r]} & ‘pick up/transport’ & [hàr] & ‘build/make’\\
 {[nʒ\'{ɛ}]} & ‘left’ (gone) & [nʒ\`{ɛ}] & ‘sit’\\
 {[tsáháj]} & ‘ask’ & [ts\={a}háj] & ‘get water’\\
 {[tsáwáj]} & ‘cut off the head’ & [tsàw\={a}j] & ‘grow’\\
 {[p\={ə}ɗ\={a}káj]} & ‘wake up’ & [p\`{ə}ɗàk\={a}j] & ‘melt’\\
\midrule\multicolumn{4}{c}{Different grammatical categories}\\\midrule
 {[ává]} & ‘there is’ ({\EXT}) & [àvà]\footnote{A third example ([áv\={a}]\textit{ }‘under’) makes this line a minimal triplet for tone.} & ‘arrow’ (noun)\\
 {[k\={ʊ}rsáj]} & ‘sweep’ (verb) & [k\`{ʊ}rs\={a}j] & ‘cucumber’ (noun)\\
 {[l\={a}lá]} & ‘come back’ (verb) & [l\={a}l\={a}] & ‘good’ (adverb)\\
 {[[\={ɛ}h\'{ɛ}]} & ‘no’ (interjection) & [\`{ɛ}h\={ɛ}] & ‘here’ (adverb)\\
 {[t\={ə}tá]} & \oldstylenums{3}\textsc{p} & [t\={ə}t\={a}] & ‘is able to’\\
 {[vá]} & Perfect extension\is{Tense, mood, and aspect!Perfect} & [và] & ‘body’ \\
 {[ndán\={a}]} & ‘therefore’ / ‘you (\textsc{s}) must’ & [nd\={a}nà] & ‘previously mentioned’\\
 {[\={a}háŋ]} & \oldstylenums{3}\textsc{p}.{\POSS} & [àh\={a}ŋ] & ‘he said’\\
\lspbottomrule
\end{tabular}
 \caption{Minimal pairs for phonetic tone\label{tab:2.12}}
 \end{table}

From an underlying two-tone system, with the influence of depressor consonants, certain melodies can be derived.  There are different melodies for nouns and verbs. These melodies will be discussed in the noun and verb sections (see Sections \ref{sec:4.1} and \ref{sec:6.7}). Bow described three different categories of verbs, those with underlying high tone, those with underlying low tone, and those with no underlying tone at all (toneless). A list of verbs showing their underlying tone is in Appendix~\ref{sec:13.1}.

Lexical tone itself is not marked in the orthography (or in examples in the morphosyntax part of this work) since there are only a few minimal pairs which are distinguished by a diacritic on one of the words in each pair. Imperfective and Perfective aspect\is{Tense, mood, and aspect!Perfective aspect} on verbs\is{Tense, mood, and aspect!Imperfective aspect} (indicated by grammatical tone) are distinguished by a diacritic on the subject pronominal verb prefix (see \sectref{sec:7.4}). 

\subsection{Depressor consonants}\label{sec:2.4.1}
\hypertarget{RefHeading1210641525720847}{}
There are certain consonants which affect tone in Moloko.  \citet{Bow1997c} discovered that the voiced obstruents [b, d, g, mb, nd, ŋg, v, z, dz, nz, ɮ]\footnote{\citet{Bow1997c} notes that the phonemes /h, w, r, l/ can appear to function as depressors.} have the effect of lowering the phonetic tone of the syllable in which they occur.  \cite[113, 158]{Yip2002} notes that: 

\begin{quote}
“The most frequent form of interaction between tone and laryngeal features in African languages is the presence of ‘depressor' consonants. This term describes a subset of consonants, usually voiced, which lower the tone of neighbouring high tones, and may also block high spreading across them. This is a departure from the usual inertness of consonants in tonal systems[…]The set of depressor consonants may include all voiced consonants, or often only non-glottalized, non-implosive voiced obstruents. In some languages, such as Ewe, we find a three-way split, with voiced obstruents most active as depressors, voiceless obstruents as non-depressors, and voiced sonorants having some depressor effects, but fewer than the obstruents.{\textquotedbl}
\end{quote}

Depressor consonants do not affect words that have an underlying high tone in Moloko. Words that are underlyingly low tone and contain no depressor consonants have phonetic mid tone, and words that are underlyingly low tone and contain depressor consonants have phonetic low tone. The phonetic low tone is triggered by the presence of depressor consonants.  \tabref{tab:2.13} demonstrates the effect of depressor consonants on the tone of the verb root in Moloko. The table shows minimal pairs of verb roots with phonetic mid and low tone with and without depressor consonants. 

\begin{table}
\resizebox{\textwidth}{!}{\begin{tabular}{cl@{ }lcl@{ }l}
\lsptoprule
\multicolumn{3}{c}{{Root with no depressor consonants}} & \multicolumn{3}{c}{{Root with depressor consonants}}\\\cmidrule(lr){1-3}\cmidrule(lr){4-6}
\multicolumn{1}{p{2.25cm}}{Phonetic tone\newline on root} & \multicolumn{2}{p{3.5cm}}{Verb in \twoS imperative form} &  \multicolumn{1}{p{2.25cm}}{Phonetic tone\newline on root} & \multicolumn{2}{p{3.5cm}}{Verb in \twoS imperative form }\\\midrule
M & \textit{fɛ} & ‘play an instrument’ &  L & \textit{v}\textit{ɛ} &  ‘spend time’\\
M & \textit{taf} & ‘spit’ & L & \textit{dav} & ‘plant’ \\
M & \textit{taɬ-aj} &  ‘curse’ & L & \textit{baɮ-aj} & ‘breathe’\\
\lspbottomrule
\end{tabular}}
\caption{Effect of depressor consonants on tone of verb root\label{tab:2.13}}
\end{table}

\subsection{Tone spreading rules}\label{sec:2.4.2}
\hypertarget{RefHeading1210661525720847}{}
At the phrase level, \citet{Bow1997c} found that a surface mid tone can have two sources: either an underlying low tone with no depressor consonants (see \sectref{sec:2.4.1}), or a surface high tone lowered by a preceding low.  Bow found no LH melodies within words, and illustrated that a noun whose final syllable is low will lower a high tone on the first syllable of any word that follows. \tabref{tab:2.14} (from \citealt{Bow1997c}) illustrates high tone lowering. Bow also describes a spreading rule which is optional across word boundaries where the mid or high final tone of a noun optionally spreads over a low tone on the first syllable of an adjective. 

\begin{table}
\resizebox{\textwidth}{!}{
\begin{tabular}{p{3cm}llll} 
\lsptoprule
& {Words in isolation} & {Words in context} & {Tone change} & {Gloss}\\
\midrule
\multirow{2}{3cm}{{Across morpheme boundary}} & [ɬàlà] +[áháj] & [ɬàlàh\={a}j] & LL+H → LLM & ‘villages’\\
& [jàm]+    [áh\={a}ŋ] & [jàm\={a}h\={a}ŋ] & L+HM → LMM & ‘his/her water’\\\midrule
\multirow{2}{3cm}{{Across word\newline boundary}} & [jàm]+  [ábá] & [jàm \={a}bá] & L+HH → LMH & ‘there is water’\\
& [áz\'{ʊ}ŋg\textsuperscript{w}\`{ɔ}]+ [ná] + [ɬ\={a}] & [áz\'{ʊ}ŋg\textsuperscript{w}\`{ɔ} n\={a} ɬ\={a}] & HHL+H+M → HHLMM & ‘donkey and cow’\\
\lspbottomrule
\end{tabular}}
\caption{High tone lowering at morpheme boundaries\label{tab:2.14}}
\end{table}

\section{Notes on the syllable}\label{sec:2.5}
\hypertarget{RefHeading1210681525720847}{}
The syllable in Moloko is a somewhat fluid entity that makes a flexible relation between the underlying structure (consonantal skeleton with optional vowels) and the phonetic surface structure (see introduction to \chapref{chap:2}). \citet{Bow1997c} has discussed the syllable in Moloko in detail. This section deals with aspects of syllable structure that pertain to the grammar (\sectref{sec:2.5.1}) and syllable restructuring when words combine in speech (\sectref{sec:2.5.2}). 

\subsection{Syllable structure}\label{sec:2.5.1}
\hypertarget{RefHeading1210701525720847}{}
Bow notes that “[t]he basic syllable in Moloko has a consonantal onset, a vocalic nucleus and an optional consonant coda: CV(C), and carries tone” \citep[1]{Bow1997c}. She found three syllable types in Moloko:  CV, CVC, and initial V.  Both CV and CVC syllables can appear anywhere within the word. V syllables occur only in word-initial position and are most likely to have come from what was once a separate morpheme -- the /\textit{a-}/ prefix in nouns (see \sectref{sec:4.1}), the third singular prefix in verbs (see \sectref{sec:7.3.1}), and an adposition (see Sections \ref{sec:5.4.1} and \ref{sec:5.6.1}).

Bow notes no restrictions on consonantal onsets.\footnote{\citet{FriesenMamalis2008} also discovered that although there are no restrictions on consonantal onsets for nouns, verb stems beginning with /n/ or /r/ are rare.} \cite{FriesenMamalis2008} noted that although nouns ending in CV can have any prosody (see \sectref{sec:4.1}), almost all verb stems phonetically ending in CV are palatalised (\ref{ex:2:48}--\ref{ex:2:49}), where the V is the [\textit{{}}-ɛ] suffix discussed in \sectref{sec:6.3}.\footnote{The only non-palatalised verb stems ending in CV end with the pluractional clitic \textit{=aya} or \textit{=iya}, e.g., [h=aja] ‘grind.’ [s=ija] ‘cut.’ see \sectref{sec:7.5.2}.  These verbs do not occur without the clitic so we do not know if they carry an underlying prosody or /-j/ suffix.} 

\ea \label{ex:2:48}
 [g-ɛ] \\
      do[{\twoS}.{\IMP}]-{\CL}\\
\glt  ‘Do!’
\z

\ea \label{ex:2:49}
[d-ɛ]\\
      prepare[{\twoS}.{\IMP}]-{\CL}\\
\glt  ‘Prepare!’
\z

The coda position carries more restrictions. Firstly, in word-medial position, the consonants that are permitted as coda are restricted. Bow reported that liquids can function as the coda to a non-word-final syllable.\footnote{\citet{Bow1997c} also reports that liquids can function as the nucleus of a syllable and also as the second component of a consonantal onset.} Further research has also shown that a semivowel /w/, /j/ or nasal /m, n/ can also function as the coda of a non word-final closed syllable (\ref{ex:2:50}--\ref{ex:2:52}).

\ea \label{ex:2:50}
duwlaj\\
\glt  ‘millet drink’
\z


\ea \label{ex:2:51}
kijga\\
\glt  ‘like this’
\z

\ea \label{ex:2:52}
amsɔkʷɔ\\
\glt  ‘sorghum’
\z

Secondly, consonants that can fill the coda position word-finally have other restrictions. Bow reported that the voiced plosives [b, d, dz, g, gʷ] and prenasalised consonants [mb, nd, nz, ŋg, ŋgʷ] do not appear in word-final position, and /n/ and /h/ have word-final allophones (see \sectref{sec:2.2.4.1}). In addition, \citet{FriesenMamalis2008} found that word-final consonants in verb stems that do not take the /-j/ suffix exclude all of the above and also exclude the voiceless affricate /ts/ and the approximants /w/ and / j/.  

\citet{FriesenMamalis2008} postulated that a function of the /-j/ suffix of verb stems (see \sectref{sec:6.3}) is to allow root-final consonants which cannot occur word-finally to surface. Verb roots that take the /-j/ suffix permit /b/, /g/, /ts/, and /w/ as final consonant (\ref{ex:2:53}--\ref{ex:2:55}), all consonants that are restricted in the coda position either in all Moloko words or in verb stems. The presence of the /-j/ suffix, another suffix, or an enclitic ensures that in context, the final consonants of /-j/ roots never occur word-finally in speech.  

\ea \label{ex:2:53}
\textup{[dab-aj]}\\
      follow[{\twoS}.{\IMP}]{}-{\CL}\\
\glt  ‘Follow!’
\z

\ea \label{ex:2:54}
\textup{[lag-aj]}\\
      accompany[{\twoS}.{\IMP}]{}-{\CL}\\
\glt  ‘Accompany!’
\z

% \clearpage
\ea \label{ex:2:55}
\textup{[ndaw-aj]}\\
      swallow[{\twoS}.{\IMP}]{}-{\CL}\\
\glt  ‘Swallow!’
\z

Schwa becomes voiceless in some contexts. Two voiceless consonants do not permit a voiced epenthetic schwa between them -- a voiceless schwa results. In some cases, speakers could assign tone to the syllable (\ref{ex:2:56}--\ref{ex:2:59}), and in other cases, they could not assign tone to the syllable (\ref{ex:2:60}--\ref{ex:2:63}).\footnote{Data from \citet{Bow1997c} show tone in every syllable for all of these words except \textit{mɔkʷtɔnɔkʷ} ‘toad,’ \textit{ɔkʷfɔm} ‘mouse,’ \textit{Ftak} ‘Ftak’ (a proper name) and \textit{dɛftɛrɛ} ‘book.’} In the example, the syllables are separated by a period in the phonetic form. The voiceless schwa is underlined. 

\ea \label{ex:2:56}
\textup{[}s\underline{\textup{ʊ.}}\textup{kʷɔm}\textup{]}\\
\glt  ‘buy/sell’
\z

\ea \label{ex:2:57}
\textup{[}t\underline{\textup{ə.}}\textup{ka.raj}\textup{]}\\
\glt  ‘taste’
\z

\ea \label{ex:2:58}
\textup{[mɪ.t}\underline{\textup{ɪ.}}\textup{fɛ}\textup{]}\\
\glt  ‘spitting’ ({\NOM})
\z

\ea \label{ex:2:59}
\textup{[mɪ.tʃ}\underline{\textup{ɪ.}}\textup{kɛ}\textup{]}\\
\glt  ‘standing’ ({\NOM})
\z

\ea \label{ex:2:60}
\textup{[}\textup{mɔ.kʷ}\underline{\textup{ʊ.}}\textup{tɔ.nɔkʷ}\textup{ ]}\\
\glt  ‘toad’
\z

\ea \label{ex:2:61}
\textup{[dɛ.f}\underline{\textup{ɪ.}}\textup{tɛ.rɛ}\textup{]}\\
\glt  ‘book’
\z

\ea \label{ex:2:62}
\textup{[}\underline{\textup{fə.}}\textup{tak}\textup{]}\\
\glt  ‘Ftak’ (a proper name)
\z

\ea \label{ex:2:63}
[ɔ.kʷ\underline{ʊ.}fɔm]\\
\glt  ‘mouse’
\z

\subsection{Syllable restructuring}\label{sec:2.5.2}
\hypertarget{RefHeading1210721525720847}{}
In fast speech, changes may happen within words or at word boundaries affecting adjacent syllables. At word boundaries, certain word-final consonants are lost and there may be vowel elision and reduction of vowels. Within the word, the segments may be restructured into new syllables, vowels may be reduced or deleted, and certain consonants may be deleted.

\citet{Bow1997c} notes vowel elision and  assimilation of semivowels at morpheme boundaries. Other changes that we have noted are illustrated in \tabref{tab:2.15}. When clitics are added or words juxtaposed within a construction, syllables within the morphemes are sometimes reorganised or deleted. Syllables in the table are separated by a period. Line 1 shows the resyllabification of /anzakr/ where [r] (originally the coda) is in the onset of a syllable that includes the first vowel of the following word. Line 2 illustrates vowel elision and loss of prosody. Lines 3--5 illustrate that in fast speech, word-final /-n/ is deleted. Note in line 5 that although /-n/ is deleted, the high tone of the suffix remains on the vowel and there is no vowel elision. Line 6 illustrates deletion of /h/.\footnote{This kind of deletion seems to be irregular and may relate to a language change, since in some neighbouring languages, ‘chief’ is [baj]. ‘Chief' is [baj] in Cuvok\il{Cuvok} \citep[120]{Ndokobai2006}, Gemzek\il{Gemzek} \citep[9]{Gravina2005}, Muyang\il{Muyang} (Smith, personal communication), Vame\il{Vame} \citep[17]{Kinnaird2006}, but [bahaj] in Mbuko\il{Mbuko} \citep[9]{Gravina2001}.} Note that stress is phrase-final necessitating a full vowel in the final syllable of an utterance (see introduction to \chapref{chap:2}).

\begin{sidewaystable}
\begin{tabular}{llll}
\lsptoprule

{Number} & {Underlying form} & {Phonetic pronunciation} & {Phonetic pronunciation}\\
& & {in isolation} & {in fast speech}\\\midrule
1 & /anzakr     wla/  & [a.nza.kar] [u.la] & [anzakrula]\\
& chicken      {\oneS}.{\POSS}\\
& ‘my chicken’\\\midrule
2 & /a-  la\textsuperscript{o}   ala\textsuperscript{e}      ahan/   & [a.lɔ] [ɛ.lɛ ] [a.haŋ] & [alɔlahaŋ]\\
& \oldstylenums{3}\textsc{s}-   go -thing    =\oldstylenums{3}\textsc{s}.{\POSS}\\
& ‘he went away’\\\midrule
3 & /n-la\textsuperscript{o}  a   ɓ r ɮ n  ava/ & [nʊ.lɔ] [a] [ɓər.ɮaŋ] [a.va] & [nʊlɔɓərɮava]\\
& {\oneS}+{\PFV}-go at  mountain in\\
& ‘I went to the mountain’ \\\midrule
4 & /gln                 =ahaj/ & [gə.laŋ] [a.haj] & [gəlahaj]\\
& threshing area   =Pl\\
& ‘threshing areas’ \\\midrule
5 &  /a-mbɗ =an =aka/  & [a.mbə.ɗaŋ] [a.ka] & [àmb\`{ə}ɗááka]\\
& \oldstylenums{3}\textsc{s}-change=\oldstylenums{3}\textsc{s}.{\IO}  =on  \\
& ‘he/she replied’ (lit. he changed on him) \\\midrule
6 & /bahj   alaka\textsuperscript{o}/ & [ba.haj]  [a.lɔ.kʷɔ] & [bajalɔkʷɔ]{ }\\
& chief    1\textsc{Pin}.{\POSS}\\
& ‘our (in.) chief’ \\
\lspbottomrule
\end{tabular}
\caption{Changes due to syllable restructuring\label{tab:2.15}}
\end{sidewaystable}

\section{Word boundaries}\label{sec:2.6}
\hypertarget{RefHeading1210741525720847}{}
\citet{Bow1997c} notes that “the phonological word in Moloko is made up of a root with the optional addition of affixes.” Further research has revealed that phonologically bound morphemes added to the root include affixes and several kinds of clitics. Specific phonological aspects of nouns and verbs will be discussed in their respective chapters (Chapters~\ref{chap:4} and \ref{chap:6}).

\largerpage Word breaks are determined in this work by the phonological criteria discussed in \sectref{sec:2.6.1} as well as using the grammatical criteria discussed in \sectref{sec:2.6.2}. Using these criteria, affixes, clitics, and extensions\footnote{Note that the term \textit{extension} for Chadic languages has a different use than for Bantu languages. In Chadic languages, \textit{extension} refers to particles or clitics in the verbal complex (\sectref{sec:7.5}).} can be distinguished from separate words in Moloko. Phonological criteria are illustrated for both nouns and verbs, when applicable (\sectref{sec:2.6.1}). \textit{Affix}, \textit{clitic}, and \textit{extension} are categorised for Moloko in \sectref{sec:2.6.2}.

\subsection{Phonological criteria for word breaks}\label{sec:2.6.1}
\hypertarget{RefHeading1210761525720847}{}
Five phonological criteria are used in this work:
\begin{itemize}
 \item Word-final /h/ is realized as [x] (\sectref{sec:2.6.1.1})
 \item Word-final /n/ is realised as [ŋ] (\sectref{sec:2.6.1.2})
 \item Prosodies spread over a word but do not cross word boundaries\\ (\sectref{sec:2.6.1.3})
 \item The -aj suffix in verbs drops off when suffixes or extensions are attached to the verb (\sectref{sec:2.6.1.4})
 \item Word-final /n/ is deleted before certain clitics and extensions\\ (\sectref{sec:2.6.1.5})
\end{itemize}
The criteria are illustrated for both nouns and verbs. Examples are given in pairs showing word breaks in the first example and phonologically bound morphemes in the second example. 

\subsubsection{Word-final /h/ realized as [x]}\label{sec:2.6.1.1}

 The presence of the word-final allophone [x] (\citealt{Bow1997c}) indicates a word break between \textit{gəvax} ‘field' and \textit{nɛhɛ} ‘this' \REF{ex:2:64}. The \oldstylenums{3}\textsc{p} possessive (=\textit{atəta}) is shown to be phonologically bound to the same noun \REF{ex:2:65} since this word-final change does not occur (\citealt{Bow1997c}, see \sectref{sec:3.1.2}).\footnote{Note that although \textit{=atəta} is not completely phonologically bound to \textit{gəvax} since the neutral prosody of /=atta/ does not neutralise the prosody of the noun (\sectref{sec:2.6.1.3}), it is a type of noun clitic since it fulfills the grammatical criteria for a clitic (\sectref{sec:2.6.2})}. 

\ea \label{ex:2:64}
\gll       {[gəvax]}  {\ExampleSpace}   {/gvah}     {naha\textsuperscript{e}/}  {→}  {[gəvaxnɛhɛ]}\\
       {‘field’}   {}  {‘field’}    {\DEM}    { }        {‘this field’}\\
\z

\ea \label{ex:2:65}
\gll [gəvax]  {\ExampleSpace}  /gvah     =atəta/  →  [gəvahatəta]\\
     ‘field’    {}  ‘field’  =\oldstylenums{3}\textsc{p}.{\POSS} {} {‘their field’}    \\
\z

\REF{ex:2:66} shows word-final changes for /h/ for the verb stem /b h/. In contrast, the {\oneS} indirect object pronominal clitic /=aw / (\ref{ex:2:67}, see \sectref{sec:7.3.1.1}) is phonologically bound to its stem since the /h/ does not undergo word-final changes.

\ea \label{ex:2:66}
\gll [a-bax   jam]\\
      \oldstylenums{3}\textsc{s}-pour  water\\
\glt  ‘He poured water.'
\z

\ea \label{ex:2:67}
\gll [ɓax] {\ExampleSpace} {/a-ɓh  =aw/}  {→}  [aɓahaw]\\
     ‘sew’  {}  \oldstylenums{3}\textsc{s}-sew={\oneS}.{\IO}  {}   {‘He/she sews for me.’}\\
\z

\subsubsection{Word-final /n/ realised as [ŋ]}\label{sec:2.6.1.2}

Word-final changes where /n/ is realised as [ŋ] (\citealt{Bow1997c}) indicate a word break between the noun \textit{həlaŋ} ‘back’ and \textit{na} ‘\PSP' (\ref{ex:2:68}). Example \REF{ex:2:69} is more complicated. The initial consonant of the adverbiser [ŋa] (see \sectref{sec:3.5.2}) has assimilated to the final consonant of the noun, indicating that they are phonologically bound. However, the fact that the noun [dedeŋ] ‘truth’ exhibits  word-final changes indicates that [ŋa] cliticises after word-final changes in the noun have occurred. 

\ea \label{ex:2:68}
\gll [həlaŋ]  {\ExampleSpace}  /a  hlan     na/ \hspace{8pt} →\hspace{5pt} [ahəlaŋna]\\
      ‘back’  {}   to  back    {\PSP} {}  {}   ‘behind’ \\
\z

\ea \label{ex:2:69}
\gll [dedeŋ]  {\ExampleSpace}  {/dadan\textsuperscript{e}}   =Ca/  → \hspace{2pt}  [dɛdɛŋŋa]\\
      ‘truth’   {}   ‘truth’                     ={\ADJ}  {} {}  ‘truly’\\
\z

Word-final changes indicate a word break after the verb [ahaŋ] in \REF{ex:2:70}. In contrast, \REF{ex:2:71} demonstrates no word-final allophones indicating that the indirect object pronominal enclitic [=aw] is phonologically bound to the verb stem \mbox{/dz n –aj/}\footnote{The verb stems /h-j/ ‘greet’and /dz n -j/  ‘help’ both carry the /-j/ suffix. This suffix is deleted whenever an affix or extension is attached to the verb stem (\sectref{sec:6.3}).} (see \sectref{sec:7.3.1.1}).

\ea \label{ex:2:70}
\gll {[ahaj]            /a-h-j    =an     ma/    →       [ahaŋma]}\\
     {‘He/she speaks.’}  \oldstylenums{3}\textsc{s}-tell-{\CL}  =\oldstylenums{3}\textsc{s}.{\IO}  mouth     {}      {‘He/she greeted him/her.’}\\
\z

\ea \label{ex:2:71}
\gll   [adzənaj]         {/a-dz n-j}        =aw/   \hspace{30pt}     →   [ajənaw]\\
       {‘he/she helps’}  \oldstylenums{3}\textsc{s}-help-{\CL}  ={\oneS}.{\IO}  {} {}  {‘He/she helped me.’}\\
\z

\subsubsection{Prosodies do not cross word boundaries}\label{sec:2.6.1.3}

\citet{Bow1997c} showed that prosodies spread over a word but do not cross word boundaries. Nouns are illustrated in (\ref{ex:2:72}--\ref{ex:2:74}). The possessive pronouns in (\ref{ex:2:72}--\ref{ex:2:73}) are phonologically separate from the nouns that they modify since the prosodies do not spread leftwards over the nouns (labialisation in \ref{ex:2:72}, palatalisation in \ref{ex:2:73}). In contrast, \REF{ex:2:74} shows that the /a-/ prefix is part of the same phonological word as the noun root, since the prosody of the noun root spreads to the prefix.\footnote{Note that the labialisation prosody may not spread as far left as the prefix in some words (\sectref{sec:2.1}). The fact that it sometimes spreads indicates that the /a-/ is indeed phonologically bound. } 
 

\ea \label{ex:2:72}

\gll {/m za\textsuperscript{e}}  {s l m\textsuperscript{o}/}   →    [mɪʒɛsʊlɔm]\\
     person                       peace                          {}   {‘person characterised by peace’}\\
\z

\ea \label{ex:2:73}
\gll {/war}     {ala\textsuperscript{e }/} \hspace{12pt}  →   [warɛlɛ] \\
    child       eye                    {}      {}  {‘grain’ (lit. child eye)}\\
\z

\ea \label{ex:2:74}
\gll {/a-tama\textsuperscript{e}/} \hspace{18pt} → [ɛtɛmɛ]\\
     onion                     {}  {} ‘onion’\\
\z

Examples (\ref{ex:2:75}--\ref{ex:2:79}) illustrate verbs. The words [awij] and [nɛʃɛ] in \REF{ex:2:75} are shown to be separate words since the palatalisation prosody of the verb [nɛʃɛ] does not spread to [awij]. In contrast, the subject pronominal prefixes (shown in \ref{ex:2:76} and \ref{ex:2:78}) and suffixes (shown in \ref{ex:2:77} and \ref{ex:2:79}) are phonologically bound to the verb stem since prosodies will spread leftwards from verb stem to prefix and suffix to verb stem. In contrast, the subject morpheme is shown to be a prefix in \REF{ex:2:76} since it takes on the palatalisation prosody of the verb stem. Also, the pronominal morphemes shown in \REF{ex:2:77} and \REF{ex:2:79} are shown to be phonologically bound suffixes. Compare \REF{ex:2:76} with \REF{ex:2:77} and \REF{ex:2:78} with \REF{ex:2:79}. In the second example of each pair, the labialisation prosody of the subject pronominal morphemes /{}-am \textsuperscript{o}/ \REF{ex:2:77} and /{}-ak \textsuperscript{o}/ \REF{ex:2:79} spreads over the verb stems, even overcoming the underlying palatalisation prosody on the verb stem in \REF{ex:2:77}. 

\ea \label{ex:2:75}
\gll {/awj}                               n-   {s-j\textsuperscript{e}/} \hspace{9pt} →    [awijnɛʃɛ]\\
     {said}  {\oneS}- drink                  {}  {}   {‘He/she said, “I drink.”’}\\
\z

\ea \label{ex:2:76}
\gll /n-      {s-j\textsuperscript{e}/} \hspace{32pt} →  [nɛʃɛ]\\
    {\oneS}-  drink                {}    {}  {‘I drink.’}\\
\z

\ea \label{ex:2:77}
\gll /n- {s-j\textsuperscript{e}} {-am \textsuperscript{o}/} \hspace{1.5pt} →  [nɔsɔm]\\
     {\oneS}- drink -\oldstylenums{1}\textsc{Pex} {} {} {‘We drink.’}\\
\z

\ea \label{ex:2:78}
\gll /n- ɮar/ \hspace{37pt}  →  [naɮar] \\
{\oneS}- kick {} {} {‘I kick.’}\\
\z

\ea \label{ex:2:79}
\gll /m-                            ɮar    {-ak\textsuperscript{o}/}   →  [mɔɮʊrɔkʷ]\\
     \oldstylenums{1}\textsc{Pex}-  kick   -\oldstylenums{1}\textsc{Pex}  {} {‘We kick.’}\\
\z

\subsubsection{Deletion of the -\textit{aj} suffix in verbs}\label{sec:2.6.1.4}
The -\textit{aj} suffix in verbs drops off when suffixes or extensions are attached to the verb.  \REF{ex:2:80} and \REF{ex:2:81} show the verb /p -j/ ‘open.’ In the \oldstylenums{3}\textsc{s} form, the verb carries the \textit{{}-aj} suffix. The \oldstylenums{3}\textsc{s} direct object\textit{ na }is a separate word since the \textit{{}-aj} suffix remains on the stem \REF{ex:2:81}. The directional\is{Directionals} \textit{ala} is shown to be phonologically bound to the verb stem since when \textit{ala} is present \REF{ex:2:81} the \textit{{}-aj} suffix drops off.

\ea \label{ex:2:80}
\gll   [a-p-aj                                na]\\
      \oldstylenums{3}\textsc{s}-open-{\CL}  \oldstylenums{3}\textsc{s}.{\DO}\\
\glt  ‘He/she opens it.’
\z

\ea \label{ex:2:81}
\gll  [a-p=ala]\\
      \oldstylenums{3}\textsc{s}-open=towards\\
\glt  ‘It opens towards.’
\z

\subsubsection{Deletion of word-final /n/}\label{sec:2.6.1.5}

\citet{Bow1997c} showed that word-final /n/ is deleted before certain clitics (the possessive and plural in nouns, see Sections \ref{sec:3.1.2} and \ref{sec:4.2.2}, respectively) and before some verbal extensions (see \sectref{sec:7.5.1}). Word-final /n/ is not deleted in any other environment. \REF{ex:2:82} shows that word-final /n/ is deleted before the plural marker [=ahaj]. For comparison, \REF{ex:2:83} shows word-final changes between [ɛŋgɛrɛŋ] and [aɮa], necessitating [ŋ] the word-final allophone of /n/). Syllables are separated by a period in the phonetic form.

\ea \label{ex:2:82}
\gll   /ɓərɮan =ahj/  →  [ɓər.ɮa.haj]\\
       mountain =Pl   {} ‘mountains’\\
\z

\ea \label{ex:2:83}
\gll /angaran\textsuperscript{e}    aɮa/ \hspace{11pt} → [ɛ.ŋgɛ.rɛ.ŋa.ɮa]\\
      insect     now {} {} {‘insect now’}\\
\z

A similar phenomenon occurs in the verb complex (\ref{ex:2:84}--\ref{ex:2:85}). The adpositional \textit{=aka} (see \sectref{sec:7.5.1}) causes the deletion of word-final /n/ in a verb stem \REF{ex:2:84}.\footnote{The vowel is not deleted, resulting in a long vowel. } \REF{ex:2:85} shows the typical word-final allophone [ŋ] for comparison. 

\ea \label{ex:2:84}
\gll /a-mbəɗ                           =an                               =aka/   →   [a.mbə.ɗaa.ka]\\
     \oldstylenums{3}\textsc{s}-change =\oldstylenums{3}\textsc{s}.{\IO} =on     {}  {‘He/she replied.’}\\
\z

\ea \label{ex:2:85}
\gll /a-b=an        ana  mza\textsuperscript{e}/       →   [a.ba.ŋa.na.mɪ.ʒɛ]\\
     \oldstylenums{3}\textsc{s}-hit=\oldstylenums{3}\textsc{s}.{\IO}  to    person {} {‘He/she hit someone.’}\\
\z

\subsection{Affix, clitic, and extension}\label{sec:2.6.2}\is{Clitics!Criteria for|(}
\hypertarget{RefHeading1210781525720847}{}
Five criteria are used to categorise affixes, clitics, and extensions in Moloko. The first is whether the morpheme can occur in discourse without being bound to some other morpheme. Affixes, clitics, and extensions in Moloko are bound morphemes -- they cannot occur alone in discourse. The second criterion is whether prosodies will spread freely between the stem and morpheme in question. Prosodies will always spread between affix and stem, and sometimes between clitic or extension and stem, but prosodies never spread across word boundaries. The third criterion is whether word-final alternations are found in the final consonant of the stem when a morpheme is attached. Suffixes, clitics, and extensions will always block word-final changes in the stem. The fourth and fifth criteria are to distinguish clitics from affixes. Clitics can attach to words of different syntactic categories; whereas no separate word can be inserted between an affix and its stem. Finally, clitics function at the phrase or clause level with grammatical rather than lexical meaning.\footnote{\citet[22]{Payne1997}.} In contrast, affixes may have grammatical meaning but their meaning is applied to the word they modify. 

What we have classified as an affix in Moloko is tightly bound to the stem. No morpheme known to be a separate word can occur between the affix and its stem. Prosodies spread freely between affix and stem. There are no word-final alternations in the final consonant of the stem when a suffix is attached. Examples of affixes in this section include the /a-/ prefix in nouns and subject pronominal prefixes and suffixes in verbs. 

A clitic carries some of the characteristics of an affix and some of an independent word, and different clitics in Moloko fulfil the above criteria differently. A clitic is similar to an affix in that it is phonologically bound to the stem to which it is attached. However the nature of that phonological bondedness is different than for an affix and its stem. Grammatically, a clitic is different from an affix because a known separate word can occur in between the relevant stem and the clitic, and the clitic will then attach itself phonologically to the inserted word.  

The verbal extensions are a special class of clitics which are something between a prototypical affix and a prototypical clitic. They form a close phonological unit with the verb stem. The phonological structure of the verb word is more fully discussed with examples in \sectref{sec:7.1}, but a few summary statements are included here. When there is no suffix on the verb, extensions will cliticise to the verb stem. Prosodies on verb clitics always spread to the verb stem (see \sectref{sec:7.5}).  When there is a suffix on the verb, extensions form a \nohyphens{separate} phonological word and they cliticise to each other. In addition, the direct object pronominal extension is a separate word from the verb stem, but will be embedded amongst any other extensions that occur. In the presence of the direct object extension, the other extensions will cliticise to the direct object extension. The Perfect extension\is{Tense, mood, and aspect!Perfect} is a special enclitic in Moloko. It can occur at the end of the verb word or the end of the verb phrase (see \sectref{sec:7.5.3}).  The Perfect extension appears to have a stronger phonological connection with the verb stem than the other extensions because the neutral prosody of the extension will neutralise the prosody of the verb word even if the Perfect is phrase-final with intervening words (see \sectref{sec:7.5.3}).  

The adverbiser /Ca/ (see \sectref{sec:3.5.2}) is an interesting clitic in the way it is phonologically bound to its noun. The noun displays word-final changes, which would normally indicate a word break. However, initial consonant of the adverbiser enclitic is a reduplication of the final consonant of the noun (see \sectref{sec:2.6.1.2}) which indicates that the reduplication occurs after phonological word-final alterations are made to the noun. 

We consider both the plural marker (see \sectref{sec:4.2.2}) and possessive (see \sectref{sec:3.1.2}) to be clitics even though neither the plural nor the possessive will affect the prosody of the stem (see \sectref{sec:2.6.1}). However, there are no word-final changes that indicate a word break on the stem when the plural or possessive is added. Both plural marker and possessive are phonologically bound to a stem yet modify a larger structure (a noun phrase). They are clitics and not affixes since they bind to elements of different grammatical classes (noun or noun phrase in the case of the possessive; noun, noun phrase, numeral, or pronoun in the case of the plural).\is{Clitics!Criteria for|)} 