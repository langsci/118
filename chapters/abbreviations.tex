\addchap{Abbreviations}
% \addchap{Abbreviations and symbols}

\begin{multicols}{2} 
\setlength{\parindent}{0pt}
\begin{tabbing}
	\slash verb stem\slash \hspace{1em} \= exclusive (first person plural)\hspace{1em}\kill
	/ verb stem / \> underlying form\\
	\oldstylenums{1} \> 1\textsuperscript{st} person\\
	\oldstylenums{2} \> 2\textsuperscript{nd} person\\
	\oldstylenums{3} \> 3\textsuperscript{rd} person \\
	\ADJ \> adjectiviser\\
	adp. \> adposition \\
	\ADV \> adverbiser\\
	adv. \> adverb\\
	\CL \> verb class (/-j/ suffix)\\
	conj. \> conjunction\\
	\DAT \> dative preposition\\
	dem.\> demonstrative\\
	\DEM\> demonstrative\\
	\DEP \> dependent form of verb\\
	disc. \> discourse marker \\
	\DO \> direct object pronominal\\
	\EX \> exclusive (first person\\\> plural)\\
	\EXT \> existential\\
	\GEN \> genitive particle\\
    \textsc{hon} \> Honorific pronoun\\
	\HOR \> Hortative mood \\
	\ID \> ideophone\\
	\IFV \> Imperfective aspect\\
	\IMP \> imperative\\
	\IN \> inclusive (first person \\\>plural)\\
	interj. \> interjection \\
	\IO \> indirect object \\ \> pronominal \\
	\ITR \> habitual iterative aspect\\
	\LOC \> semantic location\\
	n. \> noun\\
	nclitic \> noun clitic\\
	\NEG \> negative \\
	n.pr. \> proper noun\\
	nsfx. \> noun suffix\\
	\NOM \> nominalised form of verb\\
	num. \> numeral\\
	\plural \> plural\\
	\PBL \> Possible mood\\
	\PFV \> Perfective aspect\\
	Pl \> plural noun clitic\\
	\PLU \> pluractional\\
	pn. \> pronoun\\
	\POSS \> possessive pronoun\\
	\POT \> Potential mood\\
	\PRF \> Perfect\\
	\PRG \> progressive aspect\\
	\PSP \> presupposition marker\\
	quant. \> quantifier \\
	\textsc{q} \>interrogative marker\\
   	\realis \> realis mood\\
	\singular \> singular\\
	\sennum \# \> sentence number from text\\
	spp. \> species\\
	v. \> verb\\
  	vclitic \> verb clitic\\
	vpfx. \> verb prefix\\
	vsfx. \> verb suffix\\
\end{tabbing}
 
\end{multicols} 
\setlength{\parindent}{10pt}