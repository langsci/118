\chapter[Noun phrase]{Noun phrase}\label{chap:5}
\hypertarget{RefHeading1211561525720847}{}
Moloko, an SVO language, has head initial noun phrases.  (\ref{ex:5:1}--\ref{ex:5:4}) show a few examples of noun phrases.  A noun (\textit{nafat}  ‘day’ and \textit{ləhe} ‘bush’ in \ref{ex:5:1}), multiple nouns (\textit{war elé háy} ‘millet grain’ in \ref{ex:5:3} and \textit{war dalay} ‘girl’ in \ref{ex:5:4}) or free pronoun (\textit{ne} {\oneS} \ref{ex:5:2}) is the head of the NP. In the examples in this chapter, the noun phrases are delimited by square brackets.\footnote{The first line in each example is the orthographic form. The second is the phonetic form (slow speech) with morpheme breaks.} 

\ea \label{ex:5:1}\relax 
 [Nafat  enen]  anday  atalay   a  [ləhe].\\
\gll      [nafat  ɛnɛŋ]  a-ndaj a-tal-aj a  [lɪhɛ]\\
      day  another  \SSS-{\PRG}    \SSS-walk-{\CL}  at  bush\\
\glt  ‘One day, he was walking in the bush.’
\z

\ea \label{ex:5:2}\relax 
 [Ne  ahan]  aməgəye.\\
\gll      [nɛ  =ahaŋ] amɪ-g-ijɛ\\
      {\oneS}  =\SSS.{\POSS}  {\DEP}-do-{\CL}\\
\glt  ‘It was me (emphatic) that did it.’ 
\z

\ea \label{ex:5:3} 
 Cəcəngehe  na,  [war  elé  háy  bəlen]   na,  ásak  asabay.\\
\gll      tʃɪtʃɪŋgɛhɛ  na [war   ɛlɛ  haj   bɪlɛŋ] na  á-sak  asa-baj\\
      now  {\PSP}    child  eye  millet  one  {\PSP}  \SSS+{\IFV}-multiply again-{\NEG}\\
\glt  ‘And now, one grain of millet, it doesn’t multiply anymore.’
\z

\ea \label{ex:5:4}\corpussource{Disobedient Girl, S. 38}\\ 
  Metesle   anga  [war  dalay  ngendəye].\\
\gll       mɛ-tɛɬ-ɛ aŋga  [war  dalaj  ŋgɛndijɛ]\\
      {\NOM}-curse-{\CL}  {\POSS}  child  girl  {\DEM}\\
\glt  ‘The curse belongs to that young woman.’
\z

In this chapter, noun phrase modifiers and the order of constituents are discussed (\sectref{sec:5.1}), using simple noun heads as examples. Then, noun heads are discussed (\sectref{sec:5.2}). Next, derived adjectives are discussed, which consist of a noun plus the adjectiviser (\sectref{sec:5.3}). After that, four kinds of noun plus noun constructions are discussed, the genitive construction (\sectref{sec:5.4.1}), the permanent attribution construction (\sectref{sec:5.4.2}), relative clauses (\sectref{sec:5.4.3}), and coordinated noun phrases (\sectref{sec:5.5}). Finally, adpositional phrases are treated in \sectref{sec:5.6}.

Some things one might expect to see in a noun phrase are not found in Moloko noun phrases, but are accomplished by other constructions. For example, some attributions are expressed at the clause level using an intransitive clause (see \sectref{sec:9.2.4.2}) or transitive verb with indirect object (see \sectref{sec:9.2.3}), and comparison is done through an oblique construction (see \sectref{sec:5.6.1}). 

\section{Noun phrase constituents}\label{sec:5.1}
\hypertarget{RefHeading1211581525720847}{}
A noun head can be modified syntactically by the addition of other full-word or clitic elements. In the examples which follow, the noun phrases are delimited by square brackets. Examples are given in pairs, where the noun phrase in the first of each pair is the direct object of the verb. In the second example of each pair, the noun phrase is the predicate in a predicate nominal construction (see \sectref{sec:10.1.2}). Note that most of the predicate nominal constructions require the presupposition marker \textit{na } (\chapref{chap:11}). The constituents being illustrated are bolded in each example. 

A noun modified by the plural marker (\ref{ex:5:5}--\ref{ex:5:6}) (see \sectref{sec:4.2.2}).

\ea \label{ex:5:5}
N\'{ə}mənjar  [awak \textbf{ahay}].\\
\gll  n\'{ə}-mənzar  [awak\textbf{=ahaj}]\\
      {\oneS}+{\IFV}-see  goat=Pl\\
\glt  ‘I see goats.’
\z

\ea \label{ex:5:6}
{}[Awak\textbf{  ahay}  na ],  [səlom  \textbf{ahay}  ga].\\
\gll  {}[awak\textbf{=ahaj}     na]   [sʊlɔm\textbf{=ahaj}   ga]\\
      goat=Pl  {\PSP}  good=Pl  {\ADJ}\\
\glt  ‘The goats [are] good.’
\z

A noun modified by a possessive pronoun (\ref{ex:5:7}--\ref{ex:5:8}) (see \sectref{sec:3.1.2}).

\ea \label{ex:5:7}
N\'{ə}mənjar  [awak \textbf{əwla}].   \\
\gll  n\'{ə}-mənzar [awak\textbf{=uwla}]\\
      {\oneS}+{\IFV}-see  goat={\oneS}.{\POSS}\\
\glt  ‘I see my goat.’
\z

\clearpage
\ea \label{ex:5:8}
{}[Awak \textbf{əwla}  na],  [səlom  ga].\\
\gll  [awak\textbf{=uwla}     na]   [sʊlɔm   ga]\\
      goat={\oneS}.{\POSS}  {\PSP}  good  {\ADJ}\\
\glt  ‘My goat [is] good.’
\z

A noun modified by an unspecified pronoun (\ref{ex:5:9}--\ref{ex:5:10}) (see \sectref{sec:3.1.5}).

\ea \label{ex:5:9}
N\'{ə}mənjar  [awak \textbf{enen}].\\
\gll  n\'{ə}-mənzar  [awak  \textbf{ɛnɛŋ}]\\
      {\oneS}+{\IFV}-see  goat  another\\
\glt  ‘I see another goat.’
\z

\ea \label{ex:5:10}
{}[Awak \textbf{enen}  ahay  na],  [səlom  ahay  ga].\\
\gll  {}[awak   \textbf{ɛnɛŋ}=ahaj   na]   [sʊlɔm=ahaj  ga]\\
      goat    other=Pl  {\PSP}  good=Pl  {\ADJ}\\
\glt  ‘Other goats [are] good.’  
\z

A noun modified by a numeral (\ref{ex:5:11}--\ref{ex:5:12}) (see \sectref{sec:3.3}).

\ea \label{ex:5:11}
N\'{ə}mənjar  [awak  əwla   ahay  \textbf{makar}].\\
\gll  n\'{ə}-mənzar  [awak=uwla=ahaj  \textbf{makar}]\\
      {\oneS}+{\IFV}-see  goat={\oneS}.{\POSS}=Pl  three\\
\glt  ‘I see my three goats.’
\z

\ea \label{ex:5:12}
{}[awak  əwla   ahay  \textbf{makar  ahay  }na],  [səlom  ahay  ga].\\
\gll  [awak=uwla=ahaj   \textbf{makar}\textbf{=ahaj}   na]   [sʊlɔm=ahaj   ga]\\
      goat={\oneS}.{\POSS}=Pl  three=Pl  {\PSP}  good=Pl  {\ADJ}\\
\glt  ‘My three goats [are] good.’
\z

A noun modified by a derived adjective (\ref{ex:5:13}--\ref{ex:5:14}) (see \sectref{sec:5.3}).

\ea \label{ex:5:13}
N\'{ə}mənjar  [awak  ahay  \textbf{malan  ahay  ga}].\\
\gll  n\'{ə}-mənzar  [awak=ahaj  \textbf{malaŋ}\textbf{=ahaj}  \textbf{ga}]\\
      {\oneS}+{\IFV}-see  goat=Pl  great=Pl  {\ADJ}\\
\glt  ‘I see the big goats.’
\z

\ea \label{ex:5:14}
{}[awak  ahay  \textbf{malan  ahay  ga}  na],  [səlom  ahay  ga].\\
\gll {}[awak=ahaj  \textbf{malaŋ}\textbf{=ahaj}  \textbf{ga}   na]   [sʊlɔm=ahaj   ga]\\
      goat=Pl  great=Pl  {\ADJ}  {\PSP}   good=Pl  {\ADJ}\\
\glt  ‘The big goats [are] good.’
\z

A noun modified by a demonstrative (\ref{ex:5:15}--\ref{ex:5:16}) (see \sectref{sec:3.2}).

\ea \label{ex:5:15}
N\'{ə}mənjar  [awak  ahay  makar \textbf{ngəndəye}].\\
\gll  n\'{ə}-mənzar  [awak=ahaj  makar  \textbf{ŋgɪndijɛ}]\\
      {\oneS}+{\IFV}-see  goat=Pl  three  {\DEM}\\
\glt  ‘I see those three goats.’
\z

\ea \label{ex:5:16}
{}[Awak  ahay  makar \textbf{ngəndəye} na],  [səlom  ahay  ga].\\
\gll  {}[awak=ahaj   makar   \textbf{ŋgɪndijɛ}   na]   [sʊlɔm=ahaj   ga]\\
      goat=Pl  three  {\DEM}  {\PSP}  good=Pl  {\ADJ}\\
\glt  ‘Those three goats [are] good.’
\z

A noun modified by a relative clause (\ref{ex:5:17}--\ref{ex:5:18}) (see \sectref{sec:5.4.3}).

\ea \label{ex:5:17}
N\'{ə}mənjar  [awak  əwla  ahay  makar  [\textbf{nok  aməvəlaw}].]\\
\gll  n\'{ə}-mənzar  [awak=uwla=ahaj  makar  [\textbf{nɔkʷ} \textbf{amə-vəl=aw}]]\\
      {\oneS}+{\IFV}-see  goat={\oneS}.{\POSS}=Pl  three  {\twoS}  {\DEP}-give={\oneS}.{\IO}\\
\glt  ‘I see my three goats that you gave to me.’
\z

\ea \label{ex:5:18}
{}[awak  əwla  ahay  makar  [\textbf{nok  aməvəlaw}]  na],  [səlom  ahay  ga].\\
\gll  {}[awak=uwla=ahaj   makar   [\textbf{nɔkʷ} \textbf{amə-vəl=aw}]  na]   [sʊlɔm=ahaj   ga]\\
      goat={\oneS}.{\POSS}=Pl  three  {\twoS}  {\DEP}-give={\oneS}.{\IO}  {\PSP}  good=Pl  {\ADJ}\\
\glt  ‘My three goats that you gave me [are] good.’
\z

A noun modified by a non-numeral quantifier (\ref{ex:5:19}--\ref{ex:5:20}) (see \sectref{sec:3.3.4}).

\ea \label{ex:5:19}
N\'{ə}mənjar  [awak  ahay  \textbf{gam}].\\
\gll  n\'{ə}-mənzar  [awak=ahaj  \textbf{gam}]\\
      {\oneS}+{\IFV}-see  goat=Pl  many\\
\glt  ‘I see many goats.’
\z

\ea \label{ex:5:20}
{}[Awak  ahay  \textbf{gam}  na],  [səlom  ahay  ga].\\
\gll  [awak=ahaj   \textbf{gam}   na]   [sʊlɔm=ahaj   ga]\\
      goat=Pl  many  {\PSP}  good=Pl  {\ADJ}\\
\glt  ‘Many goats [are] good.’
\z

A noun modified by a numeral and the adjectiviser \textit{ga} (\ref{ex:5:21}--\ref{ex:5:22}).

\ea \label{ex:5:21}
N\'{ə}mənjar  [awak  ahay  məfaɗ \textbf{ga}].\\
\gll  n\'{ə}-mənzar  [awak=ahaj  mʊfaɗ   \textbf{ga}]\\
      {\oneS}+{\IFV}-see  goat=Pl  four  {\ADJ}\\
\glt  ‘I see the four goats.’
\z

\ea \label{ex:5:22}
{}[Awak  ahay  məfaɗ \textbf{ga}],  [səlom  ahay  ga].\\
\gll  [awak=ahaj   mʊfaɗ   \textbf{ga}]   [sʊlɔm=ahaj   ga]\\
      goat=Pl  four  {\ADJ}  good=Pl  {\ADJ}\\
\glt  ‘The four goats [are] good.’
\z

The constituent order is shown in \figref{fig:8}, followed by illustrative examples (\ref{ex:5:23}--\ref{ex:5:30}). Not all constituents can co-occur in the same clause. There are restrictions on how complex a noun phrase can normally become. Restrictions include the fact that that quantifiers cannot co-occur in the same noun phrase as either derived adjectives or numerals. The order of relative clause and demonstrative does not appear to be strict. Note that nominal demonstratives are in a different position than local adverbial demonstratives.

\begin{figure}
\resizebox{\textwidth}{!}{\frame{\begin{tabular}{lllllllll}
\textbf{head} & possessive & plural & numeral & relative  & nominal  & quantifier & {\ADJ} & local adverbial \\
\textbf{noun} & 	   &	    &	      &  clause   & demonstrative &       &     & demonstrative\\
\end{tabular}
}}
\caption{Structure of the Moloko noun phrase}\label{fig:8}
\end{figure}

Modification by possessive pronoun and plural marker (\ref{ex:5:23}--\ref{ex:5:24}). 

\ea \label{ex:5:23}
N\'{ə}mənjar  [awak  əwla  ahay].\\
\gll  n\'{ə}-mənzar  [awak=uwla=ahaj]\\
      {\oneS}+{\IFV}-see  goat={\oneS}.{\POSS}=Pl\\
\glt  ‘I see my goats.’
\z

\ea \label{ex:5:24}
{}[Awak  əwla  ahay  na],  [səlom  ahay  ga].\\
\gll  [awak=uwla=ahaj     na]   [sʊlɔm=ahaj   ga]\\
      goat={\oneS}.{\POSS}=Pl  {\PSP}  good=Pl  {\ADJ}\\
\glt  ‘My goats [are] good.’
\z

Modification by nominal demonstrative, relative clause, and plural marker (\ref{ex:5:25}--\ref{ex:5:26}). 

\ea \label{ex:5:25}
N\'{ə}mənjar  [awak  ahay  ngəndəye  [nok  aməvəlaw]].\\
\gll  n\'{ə}-mənzar  [awak=ahaj  ŋgɪndijɛ  [nɔkʷ  amə-vəl=aw]]\\
      {\oneS}+{\IFV}-see  goat=Pl  {\DEM}  {\twoS}  {\DEP}-give={\oneS}.{\IO}\\
\glt  ‘I see those goats that you gave me.’
\z

\clearpage
\ea \label{ex:5:26}
{}[Awak  əwla  ahay  [nok aməvəlaw]  ngəndəye  na],  [səlom  ahay  ga].\\
\gll  {}[awak=uwla=ahaj   [nɔkʷ   amə-vəl=aw]   ŋgɪndijɛ   na]   [sʊlɔm=ahaj ga]\\ 
      goat={\oneS}.{\POSS}=Pl  {\twoS}  {\DEP}-give={\oneS}.{\IO}  {\DEM}  {\PSP}  good=Pl {\ADJ}\\  
\glt  ‘Those goats of mine that you gave me [are] good.’ 
\z

Modification by quantifier, relative clause, and plural marker (\ref{ex:5:27}--\ref{ex:5:28}). 

\ea \label{ex:5:27}
N\'{ə}mənjar  [awak  ahay  gam]  [nok  aməvəlaw  va  na].\\
\gll  n\'{ə}-mənzar  [awak=ahaj   gam]   [nɔkʷ  amə-vəl=aw   =va   na]\\
      {\oneS}+{\IFV}-see  goat=Pl  many  {\twoS}  {\DEP}-give={\oneS}.{\IO}  ={\PRF}  {\PSP}\\
\glt  ‘I see many goats, the ones that you gave me.’
\z

\ea \label{ex:5:28}
{}[Awak  əwla  ahay  [nok  aməvəlaw]  jəyga  na],  [səlom  ahay  ga].\\
\gll  {}[awak=uwla=ahaj   [nɔkʷ   amə-vəl=aw]   dʒijga   na]   [sʊlɔm=ahaj   ga]\\
      goat={\oneS}.{\POSS}=Pl  {\twoS}  {\DEP}-give={\oneS}.{\IO}  all  {\PSP}  good=Pl  {\ADJ}\\
\glt  ‘All of my goats that you gave to me [are] good.’
\z

Modification by quantifier, nominal demonstrative, and plural marker (\ref{ex:5:29}--\ref{ex:5:30}). 

\ea \label{ex:5:29}
N\'{ə}mənjar  [awak  ahay  ngəndəye  jəyga].\\
\gll  n\'{ə}-mənzar  [awak=ahaj  ŋgɪndijɛ   dʒijga]\\
      {\oneS}+{\IFV}-see  goat=Pl  {\DEM}  all\\
\glt  ‘I see all those goats.’
\z

\ea \label{ex:5:30}
{}[Awak  ahay  ngəndəye  jəyga  na],  [səlom  ahay  ga].\\
\gll  {}[awak=ahaj   ŋgɪndijɛ   dʒijga   na]   [sʊlɔm=ahaj   ga]\\
      goat=Pl  {\DEM}  all  {\PSP}  good=Pl  {\ADJ}\\
\glt  ‘All of those goats [are] good.’
\z

\section{Noun phrase heads}\label{sec:5.2}
\hypertarget{RefHeading1211601525720847}{}
Noun phrases can have a head that is either a simple noun \REF{ex:5:31}, nominalised verb (\ref{ex:5:32}, \sectref{sec:5.2.1}), or a pronoun (\ref{ex:5:33}, \sectref{sec:5.2.2}). In the examples, the noun phrases are delimited by square brackets and the head is bolded. 

\clearpage
\ea \label{ex:5:31}
{}[\textbf{Albaya}  ahay]  tánday  táwas.\\
\gll  {}[\textbf{albaja}=ahaj]    tá-ndaj    tá-was\\
      {young man}=Pl    \oldstylenums{3}\textsc{p}+{\IFV}-{\PROG}  \oldstylenums{3}\textsc{p}+{\IFV}-cultivate\\
\glt  ‘The young men are cultivating.’ 
\z

\ea \label{ex:5:32}
{}[\textbf{Məzəme}  əwla]  amanday  acəɓan  ana  Mana.\\
\gll  {}[\textbf{mɪ-ʒum-ɛ}=uwla]    ama-ndaj  a-tsəɓ=aŋ    ana   Mana\\
      {\NOM}{}-eat-{\CL}={\oneS}.{\POSS}  {\DEP}-{\PROG}  \oldstylenums{3}\textsc{s}-overwhelm=\oldstylenums{3}\textsc{s}.{\IO}  {\DAT} Mana\\
\glt  ‘[The act of] my eating is irritating Mana.’
\z

\ea \label{ex:5:33}
{}[\textbf{Ndahan} ga]  ánday  áwas.\\
\gll  {}[\textbf{ndahaŋ}  ga]   á-ndaj  á-was\\
      \oldstylenums{3}\textsc{s}    {\ADJ}  \oldstylenums{3}\textsc{s}+{\IFV}-{\PROG}  \oldstylenums{3}\textsc{s}+{\IFV}-cultivate\\
\glt  ‘He himself is cultivating.’
\z

\subsection{Noun phrases with nominalised verb heads}\label{sec:5.2.1}
\hypertarget{RefHeading1211621525720847}{}
When the head noun is a nominalised verb, the other elements in the noun phrase represent clausal arguments of the nominalised verb. The modifying noun represents the direct object Theme of the nominalised verb and the possessive pronoun or noun in a modifying genitive construction represents the subject of the verb. In \REF{ex:5:34}, the noun modifier \textit{ɗaf} ‘millet loaf’ represents the direct object of the nominalised verb \textit{məzəme} ‘eating’ and the \oldstylenums{3}\textsc{p} possessive pronoun \textit{ata} represents the subject of the nominalised verb, i.e., ‘they are eating millet loaf.’

\ea \label{ex:5:34}
A  [məzəme  ɗaf  ata]  ava  na,  tázlapay  bay.\\
\gll  a  [mɪ-ʒʊm-ɛ    ɗaf=atəta]    ava  na  tá-ɮap-aj  baj\\
      at  {\NOM}{}-eat-{\CL}    {millet loaf}=\oldstylenums{3}\textsc{p}.{\POSS}  in  {\PSP}  \oldstylenums{3}\textsc{p}+{\IFV}-talk-{\CL}  {\NEG}\\
\glt  ‘While eating (lit. in the eating of their millet loaf), they don’t talk to each other.’
\z

In \REF{ex:5:35}, \textit{məndəye  ango} literally ‘your lying down’ indicates that ‘you are lying.’ The possessive pronoun \textit{ango} is the subject of the nominalised verb \textit{məndəye}. In \REF{ex:5:36}, both subject and direct object of the nominalised verb are present. \textit{Mana}, the noun in the genitive construction (see \sectref{sec:5.4.1}) codes the subject of the nominalised verb and the ‘body-part’ verbal extension \textit{va} is the direct object, i.e., ‘Mana is resting his body.’ 

\clearpage
\ea \label{ex:5:35}\corpussource{Snake, S. 19}\\
Anjakay  nok ha  a  slam  [məndəye  ango]  ava.\\
\gll  à-nzak-aj        nɔkʷ ha   a   ɬam    [mɪ-nd-ijɛ=aŋgʷɔ]   ava\\
      \oldstylenums{3}\textsc{s}+{\PFV}-find{}-{\CL}  {\twoS}  until   at  place   {\NOM}{}-sleep{}-{\CL}={\twoS}.{\POSS}       in\\
\glt  ‘It found you even at the place you were sleeping.’ (lit. it found you until in your sleeping place)
\z

\ea \label{ex:5:36}
{} [membese va  a  Mana]\\
\gll  {}[mɛ-mbɛʃ-ɛ     va   a   Mana]\\
      \NOM-rest-{\CL}    body   {\GEN}  Mana\\
\glt  ‘Mana’s rest’ (lit. resting body of Mana)
\z

\subsection{Noun phrases with pronoun heads}\label{sec:5.2.2}
\hypertarget{RefHeading1211641525720847}{}
A free pronoun head is more limited in the number of modifiers that it can take than a lexical noun head. A pronoun head can only be modified by the adjectiviser (\ref{ex:5:37}--\ref{ex:5:38}) or possessive pronoun in emphatic situations (\ref{ex:5:39}--\ref{ex:5:40}) (see \sectref{sec:3.1.1.2}). Noun phrases with pronoun heads can not be modified by plural, number, demonstrative, adjective, or relative clause.\footnote{Pronouns can be the subject of a relative clause, see \REF{ex:5:17} and \sectref{sec:5.4.3}.} The pronoun heads are bolded in the examples. 

\ea \label{ex:5:37}
{}[\textbf{Ndahan  ga}]  [aməgəye].\\
\gll  {}[\textbf{ndahaŋ}  \textbf{ga}]  [amɪ-g-ijɛ]\\
      \oldstylenums{3}\textsc{s}    {\ADJ}  {\DEP}-do-{\CL}\\
\glt  ‘He is the one that did it.’ 
\z

\ea \label{ex:5:38}
{}[Amədəye  elele  nəndəye  na],  [\textbf{ne  ga}].\\
\gll  {}[amɪ-d-ijɛ    ɛlɛlɛ  nɪndijɛ  na]  [\textbf{nɛ}  \textbf{ga}]\\
      {\DEP}-prepare{}-{\CL}  sauce  {\DEM}  {\PSP}  {\oneS}  {\ADJ}\\
\glt  ‘The one that prepared the sauce there [was] me.’
\z

\ea \label{ex:5:39}
{}[\textbf{Ne  ahan}]  [aməgəye].\\
\gll  {}[\textbf{nɛ}\textbf{=ahaŋ}]    [amɪ-g-ijɛ]\\
      {\oneS}=\oldstylenums{3}\textsc{s}.{\POSS}  {\DEP}-do-{\CL}\\
\glt  ‘I myself [am] the one that did it.’ 
\z

\clearpage
\ea \label{ex:5:40}
{}[\textbf{Ne  ahan}]  nólo  a  kosoko  ava.\\
\gll  {}[\textbf{nɛ}\textbf{=ahaŋ}]    n\'{ɔ}-lɔ    a  kɔsɔkʷɔ  ava\\
      {\oneS}=\oldstylenums{3}\textsc{s}.{\POSS}  {\oneS}+{\IFV}-go  at  market  in\\
\glt  ‘I myself am going to the market.’
\z

\section{Derived adjectives}\label{sec:5.3}\is{Attribution!Derived adjectives|(} 
\hypertarget{RefHeading1211661525720847}{}
All adjectives in Moloko are derived from nouns -- there is no separate grammatical class of adjectives.\footnote{There are no comparative adjectives in Moloko -- comparison is done by means of a clause construction using a prepositional phrase described in \sectref{sec:5.6.1}.} Adjectives are derived from nouns\is{Derivational processes!Noun to adjective} by a very productive process in which the morpheme \textit{ga}  follows the noun.  \tabref{tab:35}. illustrates this process for simple nouns. 

\begin{table}
\begin{tabular}{l@{ }ll@{ }l}
\lsptoprule
\multicolumn{2}{l}{Noun} & \multicolumn{2}{l}{Derived Adjective}\\\midrule
\textit{səlom} & ‘goodness’  & \textit{səlom ga} & ‘good’ \\
\textit{gədan} & ‘force’ & \textit{gədan ga} & ‘strong’\\
\textit{deden} & ‘truth’ & \textit{deden ga} & ‘true’\\
\textit{gogwez} & ‘redness’ & \textit{gogwez ga} & ‘red’ \\
\textit{dalay} & ‘girl’ & \textit{dalay ga} & ‘feminine’\\
\textit{bərav} & ‘heart’ & \textit{bərav ga} & ‘with ability to support suffering’\footnote{An idiom.} \\
\textit{ɗaz ɗaz} & ‘redness’ & \textit{ɗaz ɗaz ga} & ‘red’ \\
\textit{kwəleɗeɗe} & ‘smoothness’ & \textit{kwəleɗeɗe ga} & ‘smooth’\\
\textit{pəyecece} & ‘coldness’ & \textit{pəyecece ga} & ‘cold’\\
\textit{malan} & ‘greatness’   & \textit{malan ga} & ‘great’ / ‘big’\\
\textit{hwəsese} & ‘smallness’ & \textit{hwəsese ga} & ‘small’\\
\lspbottomrule
\end{tabular}
\caption{Derived adjectives}\label{tab:35}
\end{table}

Nominalised verbs (see \sectref{sec:7.6}) can be further derived into adjectives by the adjectiviser. The process is illustrated in \tabref{tab:36}.

\begin{table}
\begin{tabular}{lll@{ }l}
\lsptoprule
{Verb} & {Nominalised verb} & \multicolumn{2}{l}{Derived adjective}\\\midrule
\textit{e{}-nj-e}   & \textit{mə-nj-əye}  & \textit{mə-nj-əye} & \textit{ga}  \\
\SSS-sit{}-\CL & \NOM-sit-\CL & \NOM-sit-\CL & \ADJ\\
 ‘He sat.’   &   ‘sitting’ (the event)   &  \multicolumn{2}{l}{‘seated’ (adjective)}\\
\midrule
a-\textit{dar-ay}   & \textit{me-der-e}   & \textit{me-der-e} & \textit{ga}  \\
\SSS-plant{}-\CL & {\NOM}-plant{}-\CL & {\NOM}-plant-\CL & \ADJ\\
  ‘He planted.’  &   ‘planting’ (the event)  & \multicolumn{2}{l}{‘planted’ (adjective)}\\
\lspbottomrule
\end{tabular}
\caption{Adjectives derived from nominalised verbs}\label{tab:36}
\end{table}

\subsection{Structure of noun phrase containing \textit{ga}}
\hypertarget{RefHeading1211681525720847}{}
\textit{Ga} is the final element of a noun phrase. Examples show the adjectivised nouns in complete clauses. In the examples in this section, the adjectiviser \textit{ga}  is bolded and the whole noun phrase construction including \textit{ga}  is delimited by square brackets.

\ea \label{ex:5:41}
Nazalay  [awak  gogwez  \textbf{ga}].\\
\gll  nà-z=alaj   [awak   gʷɔgʷeʒ  \textbf{ga}]\\
      {\oneS}+{\PFV}-take=away  goat  redness    {\ADJ}\\
\glt  ‘I took a red goat.’
\z

\ea \label{ex:5:42}\corpussource{Cicada, S. 5}\\
Tənjakay  [agwazla  malan  \textbf{ga}]  a  ləhe.\\
\gll  tə-nzak-aj  [agʷaɮa  malaŋ   \textbf{ga}]  a  lɪhɛ\\
      \oldstylenums{3}\textsc{p}-find-{\CL}    {spp. of tree}  bigness    {\ADJ}  at  bush\\
\glt  ‘They found a big tree (of a specific species) in the bush.’  
\z

\ea \label{ex:5:43}
{}[war  enen]  [cezlere  \textbf{ga}]\\
\gll  {}[war  ɛnɛŋ]   [tʃɛɮɛrɛ    \textbf{ga}]\\
      child    another    disobedient    {\ADJ}\\
\glt  ‘Another child [is] disobedient.’
\z

\largerpage
We consider that the adjectiviser is a separate phonological word with semantic scope over the preceding noun phrase.\footnote{\citet{Bow1997c} called this morpheme a noun affix. Also, for simple adjectivised noun constructions, speakers consider the adjectiviser to be part of the same word as the noun that is modified. However, in the absence of evidence for phonological bondedness, we consider \textit{ga} to be a separate phonological word.} The adjectiviser maintains its position at the right edge of a noun phrase regardless of the noun phrase components (\ref{ex:5:44}--\ref{ex:5:49}).  This fact indicates that it might be a clitic. However, we find no undisputable evidence that it is phonologically bound to the noun. Example \REF{ex:5:42} shows noun-final changes /n/ → [ŋ] before \textit{ga}. These changes might be due to assimilation of /n/ to point of articulation of /g/ within a word (see \sectref{sec:2.2}). However, the same change would occur at a word break, with word-final changes to /n/ (see \sectref{sec:2.2.4} and \sectref{sec:2.6.1.2}).\footnote{We have not no examples of word-final alterations of /h/ before \textit{ga.}} Also, the prosody of \textit{ga} does not neutralise any prosody on the word to which it is bound. 

\ea \label{ex:5:44}
Tákəwala  [kəra  mətece  elé  \textbf{ga.}]\\
\gll  tá-kuw=ala    [kəra  mɪ-tɛtʃ-ɛ    ɛlɛ  \textbf{ga}]\\
      \oldstylenums{3}\textsc{p}+{\IFV}-seek=to  dog  {\NOM}{}-close-{\CL}  eye  {\ADJ}\\
\glt  ‘They look for a puppy that hasn’t opened its eyes yet.’ (lit. a dog closing eyes)
\z

\ea \label{ex:5:45}\corpussource{Values, S. 47}\\
Ləme  Məloko  ahay  na,  nəmbəɗom  a  dəray  ava  na,\\
\gll  lɪmɛ    Mʊlɔkʷɔ=ahaj    na   n\`{ə}-mbʊɗ{}-ɔm        a  dəraj  ava  na \\
      \oldstylenums{1}\textsc{Pex}   Moloko=Pl  {\PSP}  {\oneS}+{\PFV}-change-\oldstylenums{1}\textsc{Pex}  at   head  in    {\PSP}\\
\glt ‘We the Moloko, we have become’ (lit. we the Moloko, we have changed in the head [to be])
      
\medskip
 ka [kərkaɗaw ahay nə hərgov ahay \textbf{ga}] a  ɓərzlan ava na.\\
\gll ka  [kərkaɗaw=ahaj    nə   hʊrgʷɔv=ahaj  \textbf{ga}]    a      ɓərɮaŋ     ava    na\\
      like  monkey=Pl        with    baboon=Pl  {\ADJ}  at    mountain    in    {\PSP}\\
\glt  ‘like the monkeys and baboons in the mountains’
\z

When the head noun in a phrase that contains the adjectiviser \textit{ga} is pluralised, both the head noun and the noun modifier are pluralised as well. Compare the singular noun phrase in \REF{ex:5:46} with the pluralised noun phrase in \REF{ex:5:47} where both the head noun and adjective are pluralised. %%\is{Plurality!Noun phrase plural} 
The same pattern of pluralisation is shown in (\ref{ex:5:48}--\ref{ex:5:49}). Note that the plural is not becoming individually ‘adjectivised.’ but rather the entire noun phrase is adjectivised. Note also that the adjectiviser always maintains its position at the right edge of the noun phrase. 
\clearpage
\ea \label{ex:5:46}
Naharalay  [awak  babəɗ \textbf{ga}]  a  mogom.\\
\gll  nà-har=alaj    [awak    babəɗ    \textbf{ga}]  a  mɔgʷɔm\\
      {\oneS}+{\PFV}-carry=away    goat    white    {\ADJ}  at  home\\
\glt  ‘I carried the white goat home.’
\z

\ea \label{ex:5:47}
Naharala  [awak  ahay  babəɗ  ahay  \textbf{ga}]  a  mogom.\\
\gll  nà-har=alaj    [awak=ahaj  babəɗ=ahaj  \textbf{ga}]  a  mɔgʷɔm\\
      {\oneS}+{\PFV}-carry=away    goat=Pl  white=Pl  {\ADJ}  at  home\\
\glt  ‘I carried the white goats home.’
\z

\ea \label{ex:5:48}
 [Məze  ahay  səlom  ahay  \textbf{ga}   na], tázala  təta  bay.\\
\gll  [mɪʒɛ=ahaj   sʊlɔm=ahaj   \textbf{ga}   na]   tá-z=ala  təta     baj\\
      person=Pl     good=Pl   {\ADJ}   {\PSP}  \oldstylenums{3}\textsc{p}+{\IFV}-take=to  ability  {\NEG}\\
\glt  ‘Good people (lit. people with the quality of goodness), they can’t bring [it].' 
\z


\ea \label{ex:5:49}\corpussource{Values, S. 49}\\
Nde  [məze  ahay  gogor  ahay  \textbf{ga}   na]  ngama.\\
\gll  ndɛ  [mɪʒɛ=ahaj  gʷɔgʷɔr=ahaj   \textbf{ga}  na]   ŋgama\\
      so    person=Pl  elder=Pl   {\ADJ}   {\PSP}  better\\
\glt  ‘So, our elders [have it] better.'
\z

Derived adjectives can be negated by following them with the negative \textit{bay}. 

\ea \label{ex:5:50}
{}[Agwəjer  mədere  \textbf{ga}  bay   na],  natoho.\\
\gll  {}[agʷødʒɛr  mɪ-dɛr-ɛ  \textbf{ga}  baj  na]  natɔhʷɔ\\
      grass  {\NOM}{}-braid-{\CL}  {\ADJ}  {\NEG}  {\PSP}  {over there}\\
\glt  ‘The grass [that is] not thatched [is] over there.’
\z

\ea \label{ex:5:51}
{}[Yam  pəyecece  \textbf{ga}  bay   na],  acar  bay.\\
\gll  {}[jam  pijɛtʃɛtʃɛ   \textbf{ga}  baj  na]   à-tsar    baj\\
      water  coldness  {\ADJ}  {\NEG}  {\PSP}  {\oldstylenums{3}\textsc{s}+{\PFV}-taste good} {\NEG}\\
\glt  ‘Lukewarm water doesn’t taste good.’ 
\z

\subsection{Functions of noun phrases containing \textit{ga}}
\hypertarget{RefHeading1211701525720847}{}
The morpheme \textit{ga} has two other functions besides adjectiviser.\footnote{These two functions for \textit{ga} do not indicate homophones. We interpret all cases of \textit{ga} as the same morpheme since all instances pattern in exactly the same way even when their function is different. We conclude that the same morpheme is functioning at the noun phrase level as an adjectiviser and at the discourse level in definiteness and emphasis.}  \textit{Ga} can also function as a discourse demonstrative\is{Deixis!Demonstrative function of \textit{ga}|(} to make the noun definite\is{Focus and prominence!Definiteness} and even sometimes emphatic.  Its function to render a pronoun emphatic is discussed in \sectref{sec:3.1.1.2}.  A set of examples from the Cicada story illustrates the discourse function. Examples (\ref{ex:5:52}--\ref{ex:5:54}) are from lines 5, 12 and 18 respectively  (the Cicada story is found in its entirety in \sectref{sec:1.6}). The first mention in the narrative of \textit{agwazla} ‘tree of a particular species’ is shown in \REF{ex:5:52}. The tree is introduced as \textit{agwazla} \textit{malan} \textit{ga}  ‘a large tree.’ Later on in the narrative, the particular tree that was found is mentioned again (\ref{ex:5:53} and \ref{ex:5:54}). In these occurrences however, the tree is not modified by an adjective, but the noun is simply marked by \textit{ga} (\textit{agwazla ga} ‘this tree of a particular species’ in \ref{ex:5:53} and \textit{memele ga} ‘the tree’ in \ref{ex:5:54}). In these last two examples, \textit{ga} indicates that ‘tree’ is referring to the particular tree previously mentioned in the discourse. 


\ea \label{ex:5:52}\corpussource{Cicada, S. 5}\\
Təlo  tənjakay  [agwazla  malan  \textbf{ga}]  a  ləhe.\\
\gll  t\`{ə}-lɔ  t\`{ə}-njak-aj        [agʷaɮa      malaŋ    \textbf{ga}]   a    lɪhɛ\\
      \oldstylenums{3}\textsc{p}+{\PFV}-go   \oldstylenums{3}\textsc{p}+{\PFV}-find-{\CL}   {spp. of tree}     largeness {\ADJ}   at   bush\\
\glt  ‘They went and found a large tree (a particular species) in the bush.’ 
\z


\ea \label{ex:5:53}\corpussource{Cicada, S. 14}\\
{}[Agwazla  \textbf{ga}]  səlom  ga  aɓəsay  ava  bay.\\
\gll  {}[agʷaɮa  \textbf{ga}]    sʊlɔm    ga     aɓəsaj     ava     baj\\
      {spp.of.tree} {\ADJ}   goodness  {\ADJ}     blemish   {\EXT}   {\NEG}\\
\glt  ‘This tree is good;  it has no faults.’
\z


\ea \label{ex:5:54}\corpussource{Cicada, S. 20}\\
Náamənjar  na  alay  [memele  \textbf{ga}  ndana]  əwɗɛ.\\
\gll  náá-mənzar     na=alaj   [mɛmɛlɛ  \textbf{ga}   ndana]  uwɗɛ\\
      {\oneS}+{\POT}-see     \oldstylenums{3}\textsc{s}.{\DO}=away   tree   {\ADJ}   {\DEM}   first\\
\glt  ‘First I want to see this tree that you spoke of.’
\z

In another story about a reconciliation ceremony between two warring parties (the Moloko and the Mbuko), the ceremony requires the cutting in two of a puppy. Which side received which part was a key element to the outcome of the ceremony. In the text, the first mention of \textit{dəray} ‘the head’ \REF{ex:5:55} is marked with \textit{ga} -- it is an expected part of the narrative frame.  When the outcome of the ceremony revealed that the Moloko got the head part (and so ‘won’ the contest) and the Mbuko received the hind parts, both are adjectivised:  \textit{dəray} \textit{ga} ‘the head’ and \textit{mətenen} \textit{ga} ‘the hindparts’ \REF{ex:5:56}. Note that \REF{ex:5:56} consists of two predicate possessive verbless clauses (see \sectref{sec:10.1.2}), each with a predicate that is an adjectivised noun. 

\ea \label{ex:5:55}
Asa  ləme  n\'{ə}gəsom  na  [dəray  \textbf{ga}]  na,  [səlom  ga].\\
\gll  asa  lɪmɛ  n\'{ə}-gʊs-ɔm    na      [dəraj    \textbf{ga}]    na  [sʊlɔm       ga]\\
      if    \oldstylenums{1}\textsc{Pex}  {\oneS}+{\IFV}-catch-\oldstylenums{1}\textsc{Pex}    \oldstylenums{3}\textsc{s}.{\DO}  head  {\ADJ}     {\PSP}  goodness     {\ADJ}\\
\glt  ‘If we got the head, [it would be] good.'
\z

\ea \label{ex:5:56}
{}[Dəray  \textbf{ga}]  anga  ləme  [mətenen  \textbf{ga}]  anga  Mboko  ahay.\\
\gll  {}[dəraj  \textbf{ga}]  aŋga  lɪmɛ    [mɪtɛnɛŋ     \textbf{ga}]    aŋga  mbɔkʷɔ=ahaj\\
      head  {\ADJ}  {\POSS}  \oldstylenums{1}\textsc{Pex}    hindparts  {\ADJ}    {\POSS}  Mbuko=Pl\\
\glt  ‘The head [is] ours; the hindparts [are] the Mbuko's.’ 
\z

Compare \REF{ex:5:57} and \REF{ex:5:58} (from lines 1 and 39, respectively of the Disobedient Girl story; shown in its entirety in \sectref{sec:1.5}). The noun \textit{bamba} ‘story,’ when first mentioned in the introduction of the story \REF{ex:5:57} is not adjectivised. When the same noun is mentioned again in the conclusion \REF{ex:5:58}, it is adjectivised \textit{ma bamba ga} ‘the story.’ 

\ea \label{ex:5:57}\corpussource{Disobedient Girl, S. 1}\\
{}[Bamba]  [bamba]  kəlo  dərgoɗ\\
\gll  {}[bamba]   [bamba]  kʊlɔ    dʊrgʷɔɗ\\
      story        story        under    silo\\
\glt  ‘Once upon a time…’ (lit. there’s a story under the silo)
\z

\ea \label{ex:5:58}\corpussource{Disobedient Girl, S. 39}\\
Ka  nehe  [ma  bamba \textbf{ga}]  andavalay.\\
\gll  ka  nɛhɛ  [ma    bamba \textbf{ga}]  à-ndava=alaj\\
      like  here  word   story     {\ADJ}     \oldstylenums{3}\textsc{s}+{\PFV}-finish=away\\
\glt  ‘It is like this the story ends.’  
\z

In the Cows in the Field story (not illustrated in this work) \textit{ga} is used to mark the five brothers (previously mentioned) whose field was damaged and who had to go to the police to resolve the problem (\ref{ex:5:59} and \ref{ex:5:60}), and the problem (\textit{ma ga} ‘that word’) that developed when they couldn’t find justice (\ref{ex:5:61} and \ref{ex:5:62}). 

\ea \label{ex:5:59}
{}[Məlama  ahay  məfaɗ  \textbf{ga}]  tanday  tágalay  ta  [sla  ahay  na]  a  Kədəmbor.\\
\gll  [məlama =ahaj   məfaɗ  \textbf{ga}]    ta-ndaj   tá-gal-aj          ta  [ɬa =ahaj   na]\\   
      brother     =Pl      four    {\ADJ}  \oldstylenums{3}\textsc{p}-{\PRG}  \oldstylenums{3}\textsc{p}+{\IFV}-drive-{\CL} \oldstylenums{3}\textsc{p}.{\DO}    cow =Pl       {\PSP}\\   
      
      \medskip
\gll a     Kʊdʊmbɔr\\
     to    Tokembere\\
\glt  ‘The four brothers, they were driving the cows to Tokembere.’ 
\z

\ea \label{ex:5:60}
Nəbohom  ta  alay  ləme  [zlom  \textbf{ga}].\\
\gll  n\`{ə}-bɔh-ɔm      ta=alaj  lɪmɛ  [ɮɔm   \textbf{ga}]\\
      {\oneS}\textsc{Pex}+{\PFV}-pour-\oldstylenums{1}\textsc{Pex}  \oldstylenums{3}\textsc{p}.{\DO}=away  \oldstylenums{1}\textsc{Pex}   five  {\ADJ}\\
\glt  ‘We gave them [our identity cards], we [were] the five [whose fields were damaged].’ 
\z

\ea \label{ex:5:61}
Sen  a slam  na  ava  nendəge  na,  nəmənjorom  [ma  \textbf{ga}].\\
\gll  ʃɛŋ     a   ɬam   na  ava   nɛndɪgɛ  na  n\`{ə}-mʊnzɔr-ɔm    [ma     \textbf{ga}]\\
 \textsc{id}walk  at  place   {\PSP}  in   {\DEM}   {\PSP}   {\oneS}+{\PFV}-see-\oldstylenums{1}\textsc{Pex}   word   {\ADJ}\\
\glt ‘Walking (later), at that place, we saw the problem.’
\z

\ea \label{ex:5:62}
Nəbohom  [ma  \textbf{ga}]  a  brəygad  ava.\\
\gll  n\`{ə}-bɔh-ɔm      [ma  \textbf{ga}]  a  brijgad  ava\\
      {\oneS}\textsc{Pex}+{\PFV}-pour-\oldstylenums{1}\textsc{Pex}    word  {\ADJ}   at   Brigade  in\\
\glt  ‘We took the problem to the Brigade.’
\z

The emphatic function of \textit{ga}\footnote{The emphatic function of \textit{ga} is discussed with respect to pronouns in \sectref{sec:3.1.1.2}.} mentioned above is even more obvious in the Values exhortation (see \sectref{sec:1.7}). Line 7 in the Values exhortation, shown in \REF{ex:5:63}, alludes to the commandments that \textit{Hərmbəlom  awacala  kə  okor  aka} ‘God wrote on the stone,’ and line 12 \REF{ex:5:64} exhorts the hearer \textit{kóogəsok ma  Hərmbəlom} ‘you should accept the word of God.’ Further in the text, the mention of \textit{anga} \textit{Hərmbəlom ga } ‘the very [word] of God himself’ (\ref{ex:5:65} from line 28) draws attention to the fact that the people don’t accept what God himself wrote on the stone tablets. This time, the marker \textit{ga}  has an emphatic function. 

\clearpage
\ea \label{ex:5:63}\corpussource{Values, S. 7}\\
Hərmbəlom  awacala  kə  okor  aka.\\
\gll  Hʊrmbʊlɔm à-wats=ala   kə   ɔkʷɔr   aka\\
      God    \oldstylenums{3}\textsc{s}+{\PFV}-write=to  on  stone  on\\
\glt  ‘God wrote them on the stone [tablet].’
\z

\ea \label{ex:5:64}\corpussource{Values, S. 12}\\
Yawa  nde  ele  nehe  ɗəw,  kóogəsok ma  Hərmbəlom.\\
\gll  jawa   ndɛ   ɛlɛ   nɛhɛ   ɗuw  k\'{ɔ}\'{ɔ}-gʷʊs-ɔkʷ     ma   Hʊrmbʊlɔm\\
       well    so  thing  {\DEM}  also  {\twoS}+{\POT}-catch-{\twoP}    word  God\\
\glt  ‘So, this thing here, you should accept the word of God.’ 
\z

\ea \label{ex:5:65}\corpussource{Values, S. 28}\\\relax
[Anga  Hərmbəlom \textbf{ga}]  kagas  asabay.\\
\gll  {}[aŋga   Hʊrmbʊlɔm   \textbf{ga}]     kà-gas     asa-baj\\
      {\POSS}  God    {\ADJ}    {\twoS}+{\PFV}-catch  again-{\NEG}\\
\glt  ‘The very [word] of God himself you no longer accept.’ 
\z

\is{Deixis!Demonstrative function of \textit{ga}|)}\is{Attribution!Derived adjectives|)}
\section{Nouns as modifiers}\label{sec:5.4}
\hypertarget{RefHeading1211721525720847}{}
There are three types of constructions where nouns figure in the modification of another head noun in Moloko. They are:

\begin{itemize}
\item Genitive construction. A head noun followed by a genitive noun phrase with the genitive particle \textit{a} (\ref{ex:5:66}) (see \sectref{sec:5.4.1}).
\item Permanent attribution construction. Two nouns are juxtaposed with no intervening particle (\ref{ex:5:67}) (see \sectref{sec:5.4.2}).
\item Relative clause (\ref{ex:5:68}) (see \sectref{sec:5.4.3}). 
\end{itemize}

\ea \label{ex:5:66}
{}[war  [a  bahay]]\\
\gll  {}[war  [a  bahaj]]\\
      child  {\GEN}  chief\\
\glt  ‘the chief’s child’
\z

\ea \label{ex:5:67}
{}[zar  Məloko]\\
\gll  {}[zar    mʊlɔkʷɔ]\\
      man    Moloko\\
\glt  ‘Moloko man’
\z

\ea \label{ex:5:68}
{}[war  [aməgəye  cəɗoy]  akaray  zana  aloko  apazan.\\
\gll  {}[war  [amɪ-g-ijɛ  tsʊɗoj]    à-kar-aj    zana=alɔkʷɔ  apazaŋ\\
      child  {\DEP}-do-{\CL}  wickedness  \oldstylenums{3}\textsc{s}+{\PFV}-steal{}-{\CL}  clothes=\oldstylenums{1}\textsc{Pin}  yesterday\\
\glt  ‘The child that did wickedness stole our clothes yesterday.’
\z

\subsection{Genitive construction}\label{sec:5.4.1}
\hypertarget{RefHeading1211741525720847}{}
The genitive construction follows the head noun in a noun phrase. The genitive noun phrase consists of the genitive particle \textit{a} plus a noun phrase expressing the possessor (\ref{ex:5:69} and \ref{ex:5:70}). 

\ea \label{ex:5:69}
{}[zar  [a  Hawa]]\\
\gll  {}[zar    [a  Hawa]]\\
      man    {\GEN}  Hawa\\
\glt  ‘Hawa’s husband’
\z

\ea \label{ex:5:70}
{}[hay  [a  baba  ango]]\\
\gll  {}[haj    [a  baba=aŋgʷɔ]]\\
      house  {\GEN}  father={\twoS}.{\POSS}\\
\glt  ‘your father’s house’
\z

\citet{Bow1997c} remarks that the particle \textit{a} appears to carry the tone HL, with a floating L.\footnote{Note that the genitive particle \textit{a} and the adposition \textit{a} (Sections \ref{sec:5.6.1} and \ref{sec:5.6.2}) are homophones.} She demonstrates in \REF{ex:5:71} that the floating low tone lowers the high tone of the noun (\textit{háy}) to become M. 

\ea \label{ex:5:71}
\textup{[ɗ\={a}f]  +  [á]  +  [háj]    $\rightarrow$   [ɗ\={ə}f á h\={a}j]}\\
\glt  ‘loaf’  \hspace{6pt}  {\GEN}  \hspace{1pt}  ‘millet’   \hspace{3pt}    ‘millet loaf’
\z

Also, the genitive particle will elide with any word-final vowel in a previous word; likewise it will elide with a vowel at the beginning of the following word. In any case, the tone effects remain.

\largerpage In a genitive construction, the relationship of the genitive noun phrase to the head  noun is a temporary attribute of or relationship to the head.\footnote{As compared with the permanent attribution construction \sectref{sec:5.4.2}.} The semantic relationship between head noun and genitive expresses the same range of semantic notions as the possessive pronoun (see \sectref{sec:3.1.2.1}). In the examples below, the genitive construction expresses ownership (both alienable and inalienable, \ref{ex:5:72}), kinship \REF{ex:5:73}, partitive \REF{ex:5:74}, and other looser associations (\ref{ex:5:75}--\ref{ex:5:77}). When applicable, a corresponding pronominal possessive construction is also given for comparison. 

\ea \label{ex:5:72}
{}[hay [a  Mana]  \hspace{43pt}  [hay əwla]\\  
\gll {}[haj    [a   Mana]  \hspace{30pt}     [haj=uwla]  \\
house    {\GEN}  Mana   \hspace{30pt}    house={\oneS}.{\POSS}\\
\glt ‘Mana’s house’  \hspace{42pt}       ‘the house that I live in’ (not the house I  made)\footnote{‘The house I made’ requires a relative clause: [\textit{hay} [\textit{əwla      amə-her-e  =va }]] ‘house mine to build.’}
\z


\ea \label{ex:5:73}
{}[hor  [a  Mana]]  \hspace{42pt}     [hor  ahan]\\
\gll  {}[hʷɔr   [a   Mana]]  \hspace{20pt}     [hʷɔr=ahaŋ]\\
      woman    {\GEN}    Mana   \hspace{20pt}    woman=\oldstylenums{3}\textsc{s}.{\POSS}\\
\glt  ‘Mana’s wife’   \hspace{50pt}      ‘his wife’
\z

\ea \label{ex:5:74}
{}[dəray  [a  Mana]]  \hspace{35pt}       [dəray  ahan]\\
\gll  {}[dəraj   [a   Mana]]    \hspace{25pt}   [dəraj=ahaŋ]\\
      head   {\GEN}  Mana   \hspace{30pt}    head=\oldstylenums{3}\textsc{s}.{\POSS}\\
\glt  ‘Mana’s head’   \hspace{50pt}      ‘his head’
\z

\ea \label{ex:5:75}
{}[slərele  [a  Mana]]   \hspace{32pt}      [slərele    ahan]\\
\gll  {}[ɬɪrɛlɛ   [a   Mana]]   \hspace{30pt}    [ɬɪrɛlɛ=ahaŋ]\\
      work   {\GEN}  Mana   \hspace{30pt}    work=\oldstylenums{3}\textsc{s}.{\POSS}\\
\glt  ‘Mana’s work’   \hspace{50pt}      ‘his work’
\z

\ea \label{ex:5:76}
{}[pəra  [a  Mala]]   \hspace{100pt}      [pəra  ahan]\\
\gll  {}[pəra   [a  Mala]]  \hspace{65pt}     [pəra=ahaŋ]\\
      spirit-place   {\GEN}   Mala   \hspace{65pt}              spirit-place=\oldstylenums{3}\textsc{s}.{\POSS}\\
\glt  ‘the spirit-place that Mala worships’  \hspace{5pt}   ‘his spirit-place’
\z

\ea \label{ex:5:77}
{}[zar  akar  [a  Mana]]  \hspace{62pt}     [zar  akar  ahan]\\
\gll  {}[zar   akar   [a   Mana]]  \hspace{50pt}   [zar   akar=ahaŋ]\\
      man   thief   {\GEN}   Mana   \hspace{50pt}       man  thief=\oldstylenums{3}\textsc{s}.{\POSS}\\
\glt  ‘the man who stole from Mana’ \hspace{10pt}    ‘the man who stole from him’
\z

There are several idioms or figurative expressions in Moloko which involve genitive constructions where the head noun in the noun phrase is a body part such as \textit{ma} ‘mouth’ (\ref{ex:5:78}--\ref{ex:5:80}) or \textit{hoɗ}  ‘stomach’ \REF{ex:5:81}. 

\ea \label{ex:5:78}
{}[ma  [a  gəver]]\\
\gll  {}[ma   [a  gɪvɛr]]\\
      mouth  {\GEN}  liver\\
\glt  ‘gall bladder’
\z

\ea \label{ex:5:79}
{}[ma  [a  gəlan]]\\
\gll  {}[ma   [a  gəlaŋ]]\\
      mouth  {\GEN}  kitchen\\
\glt  ‘door to the kitchen’
\z

\ea \label{ex:5:80}
{}[ma  [a  savah]]\\
\gll  {}[ma   [a  savax]]\\
      mouth  {\GEN}  {rainy season}\\
\glt  ‘beginning of rainy season’
\z

\ea \label{ex:5:81}
Ne  a  [hoɗ  [a  zazay]]  ava.\\
\gll  nɛ  a   [hʷɔɗ  [a   zazaj]]   ava\\
      {\oneS}  at  stomach  {\GEN}  peace  in\\
\glt  ‘I [am] very peaceful.’ (lit. I, in the centre of peace)
\z

All other modifiers in a genitive construction will modify the genitive noun and not the head noun. In \REF{ex:5:82}, the possessive modifies the genitive noun (my wife) and not the head noun (i.e., not ‘my bride price’). Likewise in \REF{ex:5:83}, the demonstrative modifies the genitive noun (‘this woman’) and not the head noun (i.e., not ‘this bride price’). In \REF{ex:5:84}, it is the genitive noun ‘animals’ that is pluralised and modified by ‘all’, not the head noun ‘chief.’

\ea \label{ex:5:82}
{}[Gembere  [a  hor  əwla]]  adal  anga  ango.\\
\gll  {}[gembɛrɛ  [a  hʷɔr=uwla]]    a-dal    aŋga=aŋgʷɔ\\
      {bride price}  {\GEN}  woman={\oneS}.{\POSS}  \oldstylenums{3}\textsc{s}-exceed  {\POSS}={\twoS}.{\POSS}\\
\glt  ‘The bride price of my wife exceeded [that] belonging to you.’
\z

\ea \label{ex:5:83}
{}[Gembere  [a  hor  nehe]  na],  acəɓava.\\
\gll  {}[gembɛrɛ  [a  hʷɔr  nɛhɛ]  na]  a-tsəɓ=ava\\
      {bride price}  {\GEN}  woman  {\DEM}  {\PSP}  \oldstylenums{3}\textsc{s}-overwhelm=in\\
\glt  ‘The bride price of this woman is exhorbitant.’ 
\z

\ea \label{ex:5:84}
Angala  [bahay  [a  gənaw  ahan  ahay  a  slala  ga  ava  jəyga]].\\
\gll  à-ŋgala     [bahaj  [a   gənaw=ahaŋ=ahaj  a  ɬala  ga  ava\\ 
      \oldstylenums{3}\textsc{s}+{\PFV}-return  chief  {\GEN}  animal=\oldstylenums{3}\textsc{s}.{\POSS}=Pl  at  village  {\ADJ}  in\\
      
      \medskip
\gll dzijga]]\\
     all\\
\glt  ‘He came back as the chief of all his animals in the village.’ 
\z

\subsection{Permanent attribution construction}\label{sec:5.4.2}\is{Attribution!Permanent attribution construction|(}
\hypertarget{RefHeading1211761525720847}{}
In a ‘permanent attribution construction,’ the noun phrase has a head composed of two (or even three) nouns, which acts as a unit within a larger noun phrase (\ref{ex:5:85}--\ref{ex:5:91}). The nouns in a permanent attribution construction do not comprise a compound made of phonologically bound words, but are separate words (prosodies do not spread from one noun to the other, \REF{ex:5:87}, \REF{ex:5:88}, \REF{ex:5:91}, and there are word-final changes in the first noun). Semantically, the second noun in the noun phrase indicates something about the identity of the first noun or gives a permanent attribute of the head noun.\footnote{As compared with the genitive construction which gives a more temporary attribute \sectref{sec:5.4.1}.} The glosses in each of the examples below confirm this observation. 

\ea \label{ex:5:85}
{}[zar  Ftak]\\
\gll  {}[zar Ftak]\\
      man    Ftak\\
\glt  ‘a man who was born in Ftak’
\z

\ea \label{ex:5:86}
{}[zar  akar]\\
\gll  {}[zar    akar]\\
      man   theft\\
\glt  ‘thief’ (someone who makes his living from stealing)
\z

\ea \label{ex:5:87}
{}[zar  jəgwer]\\
\gll  {}[zar   dʒɪgʷɛr]\\
      man    shepherd\\
\glt  ‘a shepherd’ (paid for his work)
\z

\ea \label{ex:5:88}
{}[zar  səlom]\\
\gll  {}[zar   sʊlɔm]\\
      man    goodness\\
\glt  ‘a man who is known for his goodness’
\z

\ea \label{ex:5:89}
{}[dalay  zazay]\\
\gll  {}[dalaj   zazaj]\\
      girl    peace\\
\glt  ‘girl of peace’ (peace identifies her)
\z

\ea \label{ex:5:90}
{}[zar  madan]\\
\gll  {}[zar   madaŋ]\\
      man    sorcery\\
\glt  ‘a known sorcerer’
\z

\ea \label{ex:5:91}
{}[zar  slərele]\\
\gll  {}[zar   ɬɪrɛlɛ]\\
      man    work\\
\glt  ‘a man who is known as someone who works hard’
\z

In a noun phrase with the permanent attribution construction as its head noun, other elements in the noun phrase modify the entire head (and not just one of the nouns in the construction, as is the case for the genitive construction, see \sectref{sec:5.4.1}). In \REF{ex:5:92}, the plural and the numeral modify the head noun \textit{ndam slərele} and the sense is ‘his three workmen,’ not ‘the man of his three works.’  In \REF{ex:5:93}, the noun phrase has a triple noun head, \textit{war elé  háy} ‘millet grain.’ In this noun phrase, the derived adjective \textit{bəlen ga} ‘one,’ the demonstrative \textit{nendəye} ‘that,’ and the relative clause \textit{nok ameze}  ‘the one that you brought’ all modify the triple noun head \textit{war} \textit{elé} \textit{háy}  ‘millet grain.’ They do not just modify the noun \textit{war} ‘child’ or \textit{háy} ‘millet.’ In the examples below, the noun phrase is delimited by square brackets and the permanent attribution construction is bolded. 

\ea \label{ex:5:92}
{}[\textbf{ndam  slərele} ahan  ahay  makar].\\
\gll  {}[\textbf{ndam}   \textbf{ɬɪrɛlɛ}=ahaŋ=ahaj   makar]\\
      people  work=\oldstylenums{3}\textsc{s}.{\POSS}=Pl  three\\
\glt  ‘his three workmen’
\z

\ea \label{ex:5:93}\corpussource{Disobedient Girl, 13}\\
{}[\textbf{War  elé  háy}  bəlen  ga  nendəye  nok  ameze   na],  \\
\gll  {}[\textbf{war} \textbf{ɛlɛ} \textbf{haj}     bɪlɛŋ   ga    nɛndijɛ    nɔkʷ     amɛ-ʒɛɗ-ɛ] na\\
      child    eye    millet   one   {\ADJ}    {\DEM}      {\twoS}         {\DEP}-take-{\CL}   {\PSP}\\
\glt ‘That one grain of millet that you took,’

\medskip
      káhaya  na  kə  ver  aka.\\
\gll  ká-h=aja na kə vɛr aka\\
      {\twoS}+{\IFV}-grind={\PLU}    \oldstylenums{3}\textsc{s}.{\DO}     on  {grinding stone}  on\\
\glt  ‘you should grind it on the grinding stone.’
\z

It is interesting that when dependent and nominalised clauses (see \sectref{sec:7.6} and \sectref{sec:7.7}) are within permanent attribution  and genitive constructions, the same modal differences seen in \sectref{sec:12.1.1} still apply. The nominalised form of the verb functions to give a particular situation a finished idea, with an event that has been accomplished before the point of reference, almost as a state. In contrast, the dependent form of the verb is employed in situations which have an incomplete idea, one that is not yet achieved. Compare \REF{ex:5:94} and \REF{ex:5:95}. Example \REF{ex:5:94} refers to someone whose identity is a shepherd\is{Tense, mood, and aspect!Mood in noun phrase} -- he is a man who makes his living caring for sheep or other animals. He probably is hired. This more permanent identity or state is expressed through the nominalised form of the verb in a permanent attribution construction. In contrast, \REF{ex:5:95} (a relative clause, see \sectref{sec:5.4.3}) reflects a man who cares for sheep but being a shepherd isn’t his identity -- he has sheep now but may not always have them. It is an incomplete or not completely realised situation expressed through the dependent form of the verb (a relative clause, but similar to the genitive). 

\ea \label{ex:5:94}
zar  məjəgwere\\
\gll  \ zar     mɪ-dʒɪgʷɛr-ɛ\\
      man    {\NOM}{}-shepherd-{\CL}\\
\glt  ‘a shepherd-man’ (lit. man shepherding)
\z

\largerpage
\ea \label{ex:5:95}
məze  aməjəgwere  təmak\\
\gll  mɪʒɛ   amɪ-dʒɪgʷɛr-ɛ     təmak\\
      person  {\DEP}-shepherd-{\CL}  sheep\\
\glt  ‘a person that cares for sheep’ (lit. person to care for sheep)
\z

Likewise, compare \REF{ex:5:96} and \REF{ex:5:97}. In \REF{ex:5:96}, the dependent verb form is used to give the idea that the person has stolen something from someone, perhaps only once in his life (a non-permanent attribution).  In contrast, the permanent attribution construction in \REF{ex:5:97}\footnote{\textit{Akar} is the irregular nominalised form of the verb \textit{karay} (see \sectref{sec:4.2}).} expresses that the man is a thief by identity or occupation -- he steals to make his living. Another nominalised form is shown in \REF{ex:5:98} and the form \textit{məze məkəre  ga} ‘person thefted' expresses a completed event. In this case, use of the adjectivised form indicates that the noun phrase head \textit{məze} ‘person'  is the person who experienced the theft.  

\ea \label{ex:5:96}
məze aməkəre  məze \\
\gll  mɪʒɛ amɪ-kɪr-ɛ mɪʒɛ \\
      person  {\DEP}-steal-{\CL}  person\\
\glt ‘the person that steals’ (lit. person to steal from person)
\z

\ea \label{ex:5:97}
zar  akar\\
\gll  zar    akar\\
      man    theft  \\
\glt ‘a thief’ (lit. man thief)
\z

\ea \label{ex:5:98} 
məze məkəre  ga \\
\gll mɪʒɛ mɪ-kɪr-ɛ ga \\
person  {\NOM}{}-steal-{\CL}  {\ADJ}\\
\glt ‘the person who was robbed’ 
\z\is{Attribution!Permanent attribution construction|)}

\subsection{Relative clauses}\label{sec:5.4.3}
\hypertarget{RefHeading1211781525720847}{}
Relative clauses are one of the final elements in a noun phrase. The structure of relative clauses in Moloko is shown in \figref{fig:9}. and consists of a pronoun (when necessary), a verb in dependent form (see \sectref{sec:7.7}) and a complement. A relative clause has no pronoun when the head of the relative clause  is the subject of the relative clause. If the head of the relative clause has a grammatical role other than subject, then a pronoun is used.

\begin{figure}
\frame{\centering\begin{tabular}{llll} (pronoun) & dependent verb & complement & (presupposition  marker)\end{tabular}}
\caption{\label{fig:9} Structure of relative clause}
\end{figure}

The head noun of the relative clause can be either the subject or  the direct object of the relative clause. When the head noun is the subject of the relative clause (\ref{ex:5:99}--\ref{ex:5:102}), there is a gap for subject in the relative clause (marked by Ø in the examples). For example, the understood subject of the relative clause in \REF{ex:5:99} is the same as \textit{war dalay} ‘the girl’ in the noun phrase. In the example, the Ø is a zero marking where the subject of the clause would otherwise be. There is a gap for subject because the subject of the relative clause is the same as the head of the noun phrase that is being modified. The relative clause is bolded and the noun phrase is delimited by square brackets in the examples in this section.

\ea \label{ex:5:99}\corpussource{Disobedient Girl, S. 38}\\
Metesle  anga [war  dalay  ngendəye  \textbf{amazata  aka  ala}  \\
\gll mɛtɛɬɛ aŋga [war dalaj ŋgɛndijɛ \textbf{Ø} \textbf{ama-z=ata}\textbf{=aka}\textbf{=ala} \\
curse  {\POSS}  child  girl  {\DEM}  { }    {\DEP}-bring=\oldstylenums{3}\textsc{p}.{\IO}=on=to\\
\glt ‘The curse [is] belonging to that girl, (the one) who had brought’\\

\clearpage
\medskip
\textbf{avəya  nengehe  ana  məze  ahay  na}].\\
\gll \textbf{avija} \textbf{nɛŋgɛhɛ} \textbf{ana} \textbf{mɪʒɛ}\textbf{=ahaj} \textbf{na}]\\
suffering    {\DEM}    {\DAT}    person=Pl          {\PSP}\\
\glt ‘this suffering to the people.’
\z

\ea \label{ex:5:100}
{}[Ləkwəye  hawər  ahay  na, \textbf{amanday  a  hay  a  zawər  ahay  ava}], \\
\gll  \ \ {}[lʊkʷøjɛ  hawər  =ahaj   na  \textbf{Ø}    \textbf{ama-ndaj} \textbf{a} \textbf{haj} \textbf{a} \textbf{zawər}\textbf{=ahaj} \textbf{ava}]\\
     \ \ {\twoP}    women  =Pl  {\PSP}    { }      {\DEP}-{\PROG}  at  house  {\GEN}  men=Pl  in\\
\glt  \ \ ‘You women, the ones that are living at your husband’s house,\\
\medskip
 \ \ səy  kogəsom  ma  a  zawər  aləkwəye  ahay.\\
\gll \ \ sij kɔ-gʊs-ɔm ma a zawər=alʊkʷøjɛ=ahaj\\
\ \ only    \oldstylenums{2}-catch-{\twoP}  mouth  {\GEN}  men={\twoP}.{\POSS}=Pl  \\
\glt \ \ ‘you must listen to your husbands.’
\z

\ea \label{ex:5:101}\corpussource{Disobedient Girl, S. 33}\\
Hərmbəlom  ága  ɓərav  va  kəwaya  \\
\gll  Hʊrmbʊlɔm  á-g-a         ɓərav   =va  kuwaja     \\
      God \oldstylenums{3}\textsc{s}+{\IFV}-do-{\CL}  heart   ={\PRF}  {because of}   \\
\glt ‘God had gotten angry because of ’\\

\medskip
[war  dalay  na  \textbf{amecen  sləmay  bay}  ngəndəye].  \\
\gll  [war      dalaj   na  \textbf{Ø}   \textbf{amɛ-tʃɛŋ}     \textbf{ɬəmaj}  \textbf{baj}      ŋgɪndijɛ]\\
      child   girl      {\PSP} { } {\DEP}-hear   ear       {\NEG}     {\DEM}\\
\glt  ‘that girl, that one that was disobedient.’\\
\z

\ea \label{ex:5:102}
Nde  [ləbara  əwla  ga \textbf{amətaraləkwəye}  \textbf{ma}]  nehe.\\
\gll  ndɛ  [ləbara  =uwla    ga  \textbf{Ø} \textbf{amə-tar}\textbf{=alʊkʷøjɛ}  \textbf{ma}]  nɛhɛ\\
      so  news  ={\oneS}.{\POSS}  {\ADJ}  { }    {\DEP}-call={\twoP}.{\IO}    mouth  {\DEM}\\
\glt  ‘So, this is my news that I have called you together (to hear).’ (lit. So, my news which called mouth to you [is] this here)
\z

When the head noun is the direct object of the relative clause, the relative clause must contain a subject pronoun. The pronoun  must be inserted before the verb in the relative clause (\ref{ex:5:103}--\ref{ex:5:105}). It is interesting that this subject pronoun of the relative clause is sometimes a free pronoun (\ref{ex:5:104}, \ref{ex:5:105}, \ref{ex:5:109}, see \sectref{sec:3.1.1}) but in other cases is a possessive pronoun (\ref{ex:5:103}, see \sectref{sec:3.1.2}).  Two examples from the same narrative\footnote{The entire narrative is not included in this work.} (\ref{ex:5:103} and \ref{ex:5:104}) use different pronouns for the subject of the relative clause. While \REF{ex:5:103} uses the \oldstylenums{3}\textsc{p} possessive pronoun \textit{ata}, \REF{ex:5:104} uses the free pronoun \textit{təta}. In some cases, the relative clause will contain the direct object pronominal \textit{na} following the dependent verb. The \DO pronominal represents the noun phrase head. In the examples below, the direct object pronominal \textit{na} is underlined. A gap for the direct object in the relative clause (\ref{ex:5:104} and \ref{ex:5:109}) is indicated by {Ø}.  

\ea \label{ex:5:103}
Tasan  oko  ana  [hay  \textbf{ata  aməgəye \underline{na} va}].\\
\gll  tà-s=aŋ            ɔkʷɔ   ana    [haj\textbf{=atəta}     \textbf{amɪ-g-ijɛ} \textbf{\underline{na}}=\textbf{va}]\\
      \oldstylenums{3}\textsc{p}+{\PFV}-cut=\oldstylenums{3}\textsc{s}.{\IO}  fire   {\DAT}   house=\oldstylenums{3}\textsc{p}.{\POSS}   {\DEP}-do-{\CL}      \oldstylenums{3}\textsc{s}.{\DO}={\PRF}\\
\glt  ‘They set fire to the house that the others had made.’
\z

\ea \label{ex:5:104}
A  slam  a  [hay  \textbf{təta aməgəye}  \textbf{a  dala  kosoko  ava na}],  tolo.\\
\gll  a  ɬam  a  [haj  \textbf{təta}  \textbf{amɪ-g-ijɛ}   \textbf{Ø}  \textbf{a}  \textbf{dala}  \textbf{kɔsɔkʷɔ}  \textbf{ava}   \textbf{na}]\\
      at   place  {\GEN}  house  \oldstylenums{3}\textsc{p}  {\DEP}-do-{\CL}  { }  {at}  money  market  in  {\PSP}\\
      
 \medskip
 \gll     t\`{ɔ}-lɔ\\
      \oldstylenums{3}\textsc{p}+{\PFV}-go\\
\glt  ‘To the place of the house that they made in the market, they went.’
\z


\ea \label{ex:5:105}
{}[War  háy  ngəndəye  \textbf{nok  ameze \underline{na} va}]  bəlen  ngəndəye  na,\\  
\gll  [war  haj      ŋgɪndijɛ   \textbf{nɔkʷ} \textbf{amɛ-ʒ-ɛ} \textbf{\underline{na}}\textbf{=va}]    bɪlɛŋ  ŋgɪndijɛ  na\\
      child    millet  {\DEM}  {\twoS}  {\DEP}-take-{\CL}  \oldstylenums{3}\textsc{s}.{\DO}={\PRF}  one   {\DEM}  {\PSP}\\
\glt ‘That grain that you have taken, that one [grain],’
      
\medskip
káahaya  kə  ver  aka.\\
\gll  káá-h=aja            kə  vɛr             aka\\
      {\twoS}+{\POT}-grind={\PLU}  on  {grinding stone}       on\\
\glt  ‘grind it on the grinding stone.’
\z

\REF{ex:5:106} is more complex since the subject of the relative clause includes the speaker along with the head of the noun phrase (\textit{məze  enen  ahay} ‘some other people’).  The relative clause begins with the \oldstylenums{1}\textsc{Pex} pronoun \textit{ləme}. The speaker brought food to those people who helped him to drive the cows. 

\largerpage
\ea \label{ex:5:106}
Dəyday  anga  fat  amədeɗe  va  n\'{ə}ngala  a  mogom\\  
\gll  dijdaj     aŋga   fat   amɪ-dɛɗ-ɛ  =va  n\'{ə}-ŋg=ala  \\
      \textsc{id}:approximately  {\POSS}  sun  {\DEP}-fall-{\CL}   ={\PRF}  {\oneS}+{\IFV}-return=to  \\
      
 \medskip
 \gll     a  mɔgʷɔm\\
      at  home\\
 \glt ‘At sunset, I went home’ (lit. [it was] approximately [time] belonging to the sun which already fell, I returned home)
      
\medskip
waya  amazata  ala  ɗaf  ana  \\
\gll waja ama-z=ata=ala ɗaf ana \\
      because  {\DEP}-take=\oldstylenums{3}\textsc{p}.{\IO}=to  {millet.loaf}  {\DAT} \\
\glt ‘to bring food for ’ (lit. because to bring food to)
     
\medskip
[məze  enen  ahay  \textbf{ləme} \textbf{aməngele  alay  sla  ahay  jəyga  na}].\\
 \gll [mɪʒɛ ɛnɛŋ=ahaj  \textbf{lɪmɛ} \textbf{amɪ{}-ŋgɛl-ɛ}\textbf{=alaj} \textbf{ɬa}\textbf{=ahaj} \textbf{dʒijga} \textbf{na}]\\
      person  another=Pl  \oldstylenums{1}\textsc{Pex}  {\DEP}-return-{\CL}=away  cow=Pl  all  {\PSP}\\
\glt  ‘all the people that drove the cows [to Tokembere].’ (lit. some other people we the ones returning all cows) 
\z

In all of the above examples, the head noun can be modified by other modifiers in addition to the relative clause. Sometimes, however, the relative clause itself is the entire noun phrase (\ref{ex:5:107}--\ref{ex:5:108}). These noun phrases that consist of relative clauses take no other noun phrase modifiers. Also, they are apparently limited in the type of clause construction in which they can occur. They can only be the predicate of a larger predicate nominal construction (see \sectref{sec:10.1.2}). Examples \REF{ex:5:107} and \REF{ex:5:108} are interrogative constructions with a predicate nominal structure (see \sectref{sec:10.3.1}). We found no natural examples where a headless relative clause served as a matrix component in a matrix verbal clause. Example \REF{ex:5:108} is an emphatic construction (see \sectref{sec:10.3.5}). 

\ea \label{ex:5:107}
{}[\textbf{Aməzəɗe  dəray  na}]  way?\\
\gll  {}[\textbf{Ø}    \textbf{amɪ-ʒɪɗ{}-ɛ}    \textbf{dəraj}  \textbf{na}]  waj\\
      { } {\DEP}-carry-{\CL}    head  {\PSP}  who\\
\glt  ‘Who will win?’ (lit. the one to carry the head, who?) 
\z


\ea \label{ex:5:108}\corpussource{Snake, S. 7}\\\relax
 {}Alma  [\textbf{amədəvala  okfom  nehe}]  may?\\
\gll  {}alma [\textbf{amə-dəv=ala} \textbf{ɔkʷfɔm} \textbf{nɛhɛ}]  maj\\
      {what } {\DEP}-fall=to     mouse    {\DEM}    what\\
\glt  ‘What made that mouse fall?’ (lit. what to fall this mouse, what?)
\z

Noun phrases with relative clauses can get quite complicated in Moloko even though they only occur in specific places in discourse. In \REF{ex:5:109}, there are two relative clauses together, both modifying the head noun \textit{ɛlɛ} ‘thing.’ In the first (\textit{ne amahan}  the thing ‘that I told her’) the head of the noun phrase corresponds to the direct object of the verb in the relative clause (marked as Ø in the example). In the second (\textit{aməjəye mege bay} the thing ‘that I said she should not do’) there is an embedded complement clause within the relative clause (delimited by lines). In this second relative clause, the element that corresponds to the head of the noun phrase is represented by Ø within the complement clause.


\ea \label{ex:5:109}\corpussource{Disobedient Girl, S. 29}\\
Agə na  va   \\
\gll  à-gə  na=va    \\
      \oldstylenums{3}\textsc{s}+{\PFV}-do     \oldstylenums{3}\textsc{s}.{\DO}={\PRF}   \\
\glt ‘She did it’ (lit. she did it, [the thing] that I told her;)\\
      
\medskip
[ele  \textbf{ne    amahan}  \textbf{aməjəye  {\textbar}mege  bay{\textbar}  na}]  esəmey.\\
\gll  [ɛlɛ     \textbf{nɛ} \textbf{ama-h=aŋ}   \textbf{Ø}  \textbf{amɪ-dʒ-ijɛ} {\textbar}\textbf{m\`{ɛ}-g-ɛ} \textbf{Ø} \textbf{baj}{\textbar} \textbf{na}] ɛʃɪmɛj\\
      thing   {\oneS}     {\DEP}-say=\oldstylenums{3}\textsc{s}.{\IO}  { }  {\DEP}-tell-{\CL}    \oldstylenums{3}\textsc{s}+{\HOR}-do-{\CL} { } {\NEG}    {\PSP}    {not so}\\
\glt  ‘the thing that I told her she should not do, not so?’ 
\z

Plural head nouns in noun phrases containing a relative clause have so far only been noted in elicited relative clauses and their interpretation is ambiguous. In these noun phrases, speakers insert the plural \textit{=ahay}  in one of two places: the plural \textit{=ahay} can occur immediately following the head noun, or in some instances it may follow the relative clause. The plural precedes the relative clause in (\ref{ex:5:110}--\ref{ex:5:111}). 

\ea \label{ex:5:110}
{}[Ele  ahay  \textbf{nok  aməzəɗe  na}],  anga  əwla  bay.\\
\gll  [ɛlɛ=ahaj  \textbf{nɔkʷ} \textbf{amɪ-ʒɪɗ-ɛ}   \textbf{na}]  aŋga=uwla    baj\\
      thing=Pl  {\twoS}  {\DEP}-take-{\CL}  {\PSP}  {\POSS}={\oneS}.{\POSS}  {\NEG}\\
\glt  ‘The things that you brought [are] not belonging to me.’
\z

\ea \label{ex:5:111}
{}[Məze  ahay  \textbf{aməzəɗe  dəray  na}],  tolo  a  mogom  nə  memle  ga.\\
\gll  {}[mɪʒɛ=ahaj  \textbf{amɪ-ʒɪɗ{}-ɛ} \textbf{dəraj}  \textbf{na}]  tɔ-lɔ  a  mɔgʷɔm  nə  mɛmlɛ  ga\\
      person=Pl  {\DEP}-take{}-{\CL}  head  {\PSP}  \oldstylenums{3}\textsc{p}-go  at  home  with  joy  {\ADJ}\\
\glt  ‘The people that won went home with joy.’
\z

When the plural \textit{=ahay}  occurs after the relative clause \REF{ex:5:113}, exactly what is pluralised is ambiguous. The relative clause follows a singular head noun in \REF{ex:5:112}. However, when the head noun is plural, the relative clause is sandwiched between the head noun and the plural marker \REF{ex:5:113}. In \REF{ex:5:113}, the possibilities are chief’s house/ chief’s houses / chiefs’ house / chiefs’ houses,’ depending on if \textit{ndam}, \textit{hay}, \textit{bahay}, or all three are pluralised. Thus, when plural forms are used in Moloko discourse, which possibility is correct must be already clear from the context. 

\ea \label{ex:5:112}
Dala  slərele  asan  \\
\gll  dala       ɬərɛlɛ      a-s=aŋ     \\
      money  work        \oldstylenums{3}\textsc{s}-please=\oldstylenums{3}\textsc{s}.{\IO}   \\
      
      \medskip
ana  [məze  \textbf{aməhere  hay  a  bahay}].\\
\gll   ana  [mɪʒɛ      \textbf{Ø}  \textbf{amɪ-hɛr-ɛ} \textbf{haj} \textbf{a}  \textbf{bahaj}]\\
{\DAT}   person   { } {\DEP}-build-{\CL}  house  {\GEN}      chief\\
\glt  ‘The person (the one) that built the chief’s house wants his wages (lit. work money pleases him).’
\z

\ea \label{ex:5:113}
Dala  slərele  asata \\ 
\gll  dala       ɬɪrɛlɛ    a-s=ata\\
      money    work  \oldstylenums{3}\textsc{s}-please=\oldstylenums{3}\textsc{p}.{\IO}\\
\glt  ‘Wages please’\\
\medskip
ana  [ndam  \textbf{aməhere  hay  a  bahay} ahay].\\
\gll ana [ndam \textbf{Ø} \textbf{amɪ-hɛr-ɛ} \textbf{haj} \textbf{a}  \textbf{bahaj}=ahaj] \\ 
{\DAT} people  { }  {\DEP}-build-{\CL}   house  {\GEN}  chief=Pl \\
\glt ‘the people that built the chief’s house/ chief’s houses / chiefs’ house / chiefs’ houses.’ 
\z

The end of the relative clause is sometimes delimited by the presupposition marker \textit{na} (see \chapref{chap:11}). \REF{ex:5:99} is repeated here as \REF{ex:5:114} (see also \ref{ex:5:104}, \ref{ex:5:106}, \ref{ex:5:107}). \textit{Na} indicates that the relative clause contains previously shared (or presupposed) information. \textit{Na} also physically delineates the end of the relative clause. In \REF{ex:5:114}, the presupposition marker \textit{na} is underlined. 


\ea \label{ex:5:114}\corpussource{Disobedient Girl, S. 38}\\
Metesle  anga  [war  dalay  ngəndəye  \textbf{amazata  aka  ala}  \\
\gll  Mɛtɛɬɛ  anga    [war    dalaj  ŋgəndəjɛ \textbf{Ø} \textbf{ama-z=ata}\textbf{=aka}\textbf{=ala}\\
      {\NOM}{}-curse   {\POSS}   child    girl       {\DEM}  { } {\DEP}{}-take=\oldstylenums{3}\textsc{p}.{\IO}=on=to \\
\glt ‘The curse belongs to that young woman that brought’ \\     
      
\medskip
\textbf{avəya  nengehe  ana  məze  ahay   \underline{na}}].\\
\gll   \textbf{avija}  \textbf{nɛŋgɛhɛ} \textbf{ana} \textbf{mɪʒɛ}\textbf{=ahaj}  \textbf{\underline{na}}]\\
            suffering     {\DEM}      {\DAT}  person=Pl   {\PSP}\\
\glt  ‘this suffering onto the people.’  
\z

Any information inside a relative clause must be known or presupposed information expected to be shared by the hearer. Relative clauses function in two ways. Firstly, relative clauses may specify the head noun among others. Secondly, in a narrative, relative clauses identify their content as carrying information concerning a key participant in the discourse and may allude to the moral of the story. 

Consider the Disobedient Girl text (see \sectref{sec:1.5} for the full narrative). The moral of the story is to instruct children (especially girls) to be obedient. There are relative clauses in S. 13 \REF{ex:5:115}, S. 29 \REF{ex:5:109}, S. 33 \REF{ex:5:101}, and S. 38 \REF{ex:5:114}. Note that all but one \REF{ex:5:115} of the relative clauses in this narrative concern the moral of the story. The Disobedient girl story involves suffering of a particular nature that was brought on by a particular girl who disobeyed specific instructions. The instructions that she disobeyed are in a relative clause within the husband’s lament when he finds her (\ref{ex:5:109}). The disobedient girl is the head of two relative clauses at the end of the story, one citing her as the reason that God got angry (\ref{ex:5:101}) and the other stating that she brought suffering to the Moloko people (\ref{ex:5:114}). The only relative clause that does not concern information relevant to the moral of the story \REF{ex:5:115} is from a section in the narrative where the man instructs his wife on how much millet to grind. The man tells her to take one grain of millet. Then he specifies with a relative clause ‘that one grain of millet you have taken.’ This relative clause specifies the one grain of millet (from the other grains in the sack) that will be multiplied for them. 

\ea \label{ex:5:115}\corpussource{Disobedient Girl, S. 13}\\
Asa  asok  aməhaya  na, \\  
\gll  asa  à-s=ɔk amə-h=aja  na\\
      if     \oldstylenums{3}\textsc{s}+{\PFV}-please={\twoS}.{\IO}  {\DEP}+{\PFV}-grind={\PLU}    {\PSP}\\
\glt  ‘If you want to grind,’\\
\medskip
kázaɗ  war  elé  háy  bəlen.\\
\gll  ká-zaɗ war ɛlɛ haj bɪlɛŋ\\
      {\twoS}+{\IFV}-take  child    eye  millet  one\\
\glt  ‘you take only one grain.’\\
\medskip
[War  elé  háy  bəlen  ga  nəndəye  \textbf{nok  amezəɗe na}],\\
\gll {}[war ɛlɛ haj bɪlɛŋ ga ŋɪndijɛ \textbf{nɔkʷ} \textbf{amɛ-ʒɪɗ-ɛ} \textbf{na}]\\
     child  eye       millet     {\DEM}  {\ADJ}  {\DEM}  {\twoS}   {\DEP}-take{}-{\CL}    {\PSP}\\
\glt ‘That one grain that you have taken,’\\

\largerpage
\medskip
Káhaya  na  kə  ver  aka. Anjaloko  de  pew.\\
\gll ká-h=aja na kə vɛr aka à-nz=alɔkʷɔ dɛ pɛw\\
     {\twoS}+{\IFV}-grind={\PLU}  \oldstylenums{3}\textsc{s}.{\DO}  on  stone on            \oldstylenums{3}\textsc{s}+{\PFV}-suffice=\oldstylenums{1}\textsc{Pin} enough  done\\
\glt ‘grind it on the grinding stone, and it will suffice for all of us.’\\
\z

Note that the relative clauses that contain information about the moral of the story are at the end of the narrative; there are no relative clauses related to the moral of the story at the beginning of the narrative -- the noun phrases in S.10--S.11 \REF{ex:5:116} that introduce her and identify her as disobedient contain no relative clause. 

\ea \label{ex:5:116}\corpussource Disobedient Girl, S. 10--11\\
Olo  azala  [dalay] azla  na [war  dalay   ndana] \\ 
\gll  à-lɔ           à-z=ala      [dalaj] aɮa  na  [war      dalaj    ndana] \\  
      \oldstylenums{3}\textsc{s}+{\PFV}-go   \oldstylenums{3}\textsc{s}+{\PFV}-take=to  girl  now   {\PSP}   child    girl    {\DEM}\\  
      
      \medskip
 [cezlere  ga].\\     
\gll [tʃɛɮɛrɛ       ga]\\
     disobedience {\ADJ}\\
\glt  ‘He went and took a wife, but that above-mentioned girl [was] disobedient.’
\z

In the Snake narrative (see \sectref{sec:1.4}), there is only one relative clause. This relative clause shows another function of relative clauses in discourse. The relative clause, \textit{amədəvala okfom nehe} ‘the thing that caused the mouse to fall’ in line 7 \REF{ex:5:108}, contains the first mention (albeit indirect) of the snake who is a central participant in the story and the reason that the story was told. 

\section{Coordinated noun phrases}\label{sec:5.5}
\hypertarget{RefHeading1211801525720847}{}
The basic way to coordinate two participants in Moloko is to join two noun phrases by the adposition \textit{nə} ‘with’ (see \sectref{sec:5.6.1}). Modifiers will have semantic scope over both of the coordinated elements. In \REF{ex:5:117}--\REF{ex:5:119}, the noun phrases are delimited by square brackets and the adpositions are bolded. 

\ea \label{ex:5:117}
Ləbara  anga  [[bahay  a  hay]  \textbf{nə}  [ndam  slərele  ahan  ahay  makar]].\\
\gll  ləbara  aŋga  [[bahaj   a   haj]  \textbf{nə}  [ndam   ɬɪrɛlɛ=ahaŋ=ahaj \\          
      news  {\POSS}  chief  {\GEN}  house  with  people  work=\oldstylenums{3}\textsc{s}.{\POSS}=Pl\\    
      
      \medskip
\gll makar]]\\
     three\\
\glt  ‘The story [is] belonging to the chief of the house with his three workmen.’
\z

\clearpage
\ea \label{ex:5:118}\corpussource Values, S. 47\\
Nəmbəɗom  a  dəray  ava  na, \\ 
\gll  n\`{ə}-mbʊɗ{}-ɔm      a  dəraj  ava  na\\
      {\oneS}+{\PFV}-change-\oldstylenums{1}\textsc{Pex}  at   head  in    {\PSP}\\
\glt  ‘We have become’ (lit. we changed in the head)\\
\medskip
ka [[[kərkaɗaw  ahay]  \textbf{nə}  [hərgov  ahay]  ga]  [a  ɓərzlan  ava   na]]\\
\gll  ka [[[kərkaɗaw=ahaj] \textbf{nə} [hʊrgʷɔv=ahaj] ga] [a  ɓərɮaŋ ava] na]\\
      like        monkey=Pl        with    baboon=Pl  {\ADJ}  at    mountain    in    {\PSP}\\
\glt  ‘like monkeys and baboons in the mountain.’
\z

\ea \label{ex:5:119}
{}[[Zar]  \textbf{nə}  [hor  ahan]]  tolo  a  mehele  ava.\\
\gll  {}[[zar]    \textbf{nə}  [hʷɔr=ahaŋ]]  tɔ-lɔ  a  mɛ-hɛl-ɛ    ava\\
      man    with  woman=\oldstylenums{3}\textsc{s}.{\POSS}  \oldstylenums{3}\textsc{p}-go  at  {\NOM}{}-unite{}-{\CL}  in\\
\glt  ‘A man and his wife went to the meeting.’ 
\z

\section{Adpositional phrase}\label{sec:5.6}
\hypertarget{RefHeading1211821525720847}{}
Adpositional phrases function to relate noun phrases to the clause, expressing physical, grammatical, or logical relationships.  \citet{FriesenMamalis2008} found two types of adpositional phrases in Moloko; simple and complex. Simple adpositional phrases (\sectref{sec:5.6.1}) consist of an adposition followed by the noun phrase. Complex adpositional phrases (\sectref{sec:5.6.2}) consist of a noun phrase framed by a preposition and a postposition. 

\subsection{Simple adpositional phrase}\label{sec:5.6.1}
\hypertarget{RefHeading1211841525720847}{}
There are seven adpositions in Moloko: \textit{a} ‘to,’ \textit{ana} ‘to’ \textit{nə} ‘with,’ \textit{aka} ‘on,’ \textit{a}\textit{ŋ}\textit{ga} ‘belonging to,’ \textit{afa} ‘at the house of,’ and \textit{ka} ‘like.’  

The preposition \textit{a} ‘at’\footnote{This particle is a homophone with the genitive particle (\sectref{sec:5.4.1}). } marks the relationship of location\is{Deixis!Locational} of the event (at, to, in; \ref{ex:5:120}, \ref{ex:5:121}).


\ea \label{ex:5:120}\corpussource Cicada, S. 4\\
T\'{ə}nday  t\'{ə}talay  \textbf{a} ləhe.\\
\gll  t\'{ə}-ndaj t\'{ə}-tal-aj    \textbf{a}  lɪhɛ\\
      \oldstylenums{3}\textsc{p}+{\IFV}-{\PRG}  \oldstylenums{3}\textsc{p}+{\IFV}-walk{}-{\CL}  at  bush\\
\glt  ‘They were walking in the bush.’ 
\z

\ea \label{ex:5:121}
Olo  \textbf{a}  Marva.\\
\gll  \`{ɔ}{}-lɔ    \textbf{a}  Marva\\
      \oldstylenums{3}\textsc{s}+{\PFV}-go  at  Maroua\\
\glt  ‘He/she went to Maroua.’ 
\z

The adposition \textit{ana} ‘to’ marks the indirect object which is the place where the action of the verb occurs; the recipient, benefactive, or malefactive (\ref{ex:5:122}, \ref{ex:5:123}, see \sectref{sec:9.2} for a discussion of semantic roles).

\ea \label{ex:5:122}
Tolo  na,  tasan  oko  \textbf{ana} hay  ata  aməgəye  na  va.\\
\gll  tə-lɔ    na  ta-s=aŋ    ɔkʷɔ  \textbf{ana} haj=atəta  amɪ-g-ijɛ   na=va\\
      \oldstylenums{3}\textsc{p}-go  {\PSP}  \oldstylenums{3}\textsc{p}-cut=\oldstylenums{3}\textsc{s}.{\DO}  fire  {\DAT} house=\oldstylenums{3}\textsc{p}.{\POSS}   {\DEP}-do-{\CL}  \oldstylenums{3}\textsc{s}.{\DO}={\PRF}\\
\glt  ‘They went and set fire to the house that they had built.’
\z

\ea \label{ex:5:123}
Adəkaka  alay  \textbf{ana}  Hərmbəlom.\\
\gll  a-dəkʷ=aka=alaj  \textbf{ana}  Hʊrmbʊlɔm\\
      \oldstylenums{3}\textsc{s}-arrive=on=away  {\DAT} God\\
\glt  ‘It reached God.’
\z

The adposition \textit{nə} ‘with’ marks the instrument \REF{ex:5:124} or comitative (accompaniment) relation (\ref{ex:5:125}, \ref{ex:5:126}; cf. \sectref{sec:5.5}). The adposition is also used to form the verb focus construction (\ref{ex:5:127}, see \sectref{sec:7.6.3}).

\ea \label{ex:5:124}
Naslay  sla  \textbf{nə} mekec.\\
\gll  na-ɬ{}-aj  ɬa  \textbf{nə} mɛkɛtʃ\\
      {\oneS}-slay{}-{\CL}  cow  with  knife\\
\glt  ‘I kill the cow with a knife.’
\z

\ea \label{ex:5:125}
Olo  \textbf{nə}  zar  ahan.\\
\gll  ɔ{}-lɔ     \textbf{nə}   zar=ahaŋ\\
      \oldstylenums{3}\textsc{s}-go  with  man=\oldstylenums{3}\textsc{s}.{\POSS}\\
\glt  ‘She went with her husband.’
\z

\ea \label{ex:5:126}
Zar  \textbf{nə}  hor  ahan  təta  a  mogom.\\
\gll  \ zar    \textbf{nə}  hʷɔr=ahaŋ    təta  a  mɔgʷɔm\\
      man    with  woman=\oldstylenums{3}\textsc{s}.{\POSS}  \oldstylenums{3}\textsc{p}  at  home\\
\glt  ‘The man and his wife [are] at home.’ 
\z

\clearpage
\ea \label{ex:5:127}
Nəskom  awak  \textbf{nə}  məskwəme.\\
\gll  n\`{ə}-sʊkʷɔm    awak  \textbf{nə}  mɪ-skʷøm-ɛ\\
      {\oneS}+{\PFV}-buy/sell  goat  with  {\NOM}{}-buy/sell-{\CL}\\
\glt  ‘I really bought the goat.’ (lit. I bought the goat with buying)
\z

The adposition \textit{nə} ‘with’ also participates in forming comparative constructions\is{Attribution!Comparative constructions} in Moloko. When one noun phrase is compared with another, it is done by means of a clause construction using the verb \textit{dal}, ‘overtake.’\footnote{The verb \textit{dal} ‘overtake’ takes subject prefixes and carries aspectual tone. Other constructions can be employed when comparing people \REF{ex:5:97} or ideas (line 49 in the Values exhortation).} The standard of comparison (\textit{baba =ahan} ‘his father’ in \ref{ex:5:128} and \ref{ex:5:129}, and \textit{mədəga =ahan} ‘his older sibling’ in \ref{ex:5:130}) is the direct object of the verb. The quality being compared (\textit{səber} ‘tallness’ in \ref{ex:5:128}, \textit{gədan} ‘strength’ in \ref{ex:5:129}, and \textit{məsəre ele} ‘knowledge’ in \ref{ex:5:130}) follows in an adpositional phrase.  

\ea \label{ex:5:128}
War  ahan  ádal  baba  ahan  nə  səber.\\
\gll  war=ahaŋ     á-dal       baba=ahaŋ     nə   ʃɪbɛr\\
      child=\oldstylenums{3}\textsc{s}.{\POSS}   \oldstylenums{3}\textsc{s}+{\IFV}-overtake   father=\oldstylenums{3}\textsc{s}.{\POSS}  with  tallness\\
\glt  ‘The child is taller than his father.’ (lit. his child surpasses his father with tallness)  
\z

\ea \label{ex:5:129}
War  ahan  ádal  baba  ahan  nə  gədan.\\
\gll  war=ahaŋ    á-dal     baba=ahaŋ     nə   gədaŋ\\
      child=\oldstylenums{3}\textsc{s}.{\POSS}   \oldstylenums{3}\textsc{s}+{\IFV}-overtake   father=\oldstylenums{3}\textsc{s}.{\POSS}  with  strength\\
\glt  ‘The child is stronger than his father.’ 
\z

\ea \label{ex:5:130}
War  na,  á-dal  mədəga  ahan  nə  məsəre  ele.\\
\gll  war   na   á-dal        mədəga=ahaŋ     nə   mɪ{}-ʃɪr-ɛ \\    
      child   {\PSP}   \oldstylenums{3}\textsc{s}+{\IFV}-overtake    {older sibling}=\oldstylenums{3}\textsc{s}.{\POSS}  with   {\NOM}{}-know-{\CL}\\  
      
      \medskip
\gll ɛlɛ\\
     thing\\
\glt  ‘The child is smarter than his older sibling.’ (lit. the child is greater than his older sibling with respect to knowledge)
\z

No ‘less than’ comparatives were found in the data. Superlative constructions are possible but are not used often in Moloko culture.  \REF{ex:5:131} illustrates what people say in an elicitation context.

\ea \label{ex:5:131}
Ádal  məze  ahay  jəyga  nə  məsəre  ele  a  lekwel  ava.\\
\gll  á-dal     mɪʒɛ=ahaj   ʣijga   nə  mɪ{}-ʃɪr-ɛ    ɛlɛ  a  lɛkʷɛl \\ 
      \oldstylenums{3}\textsc{s}+{\IFV}-overtake  person=Pl  all  with  {\NOM}{}-know-{\CL}  thing  at  school\\  
      
      \medskip
\gll ava\\
     in\\     
\glt  ‘He/she is the smartest child in his school.’
\z

The adposition \textit{aka}  ‘on’ is used with the verb \textit{lo} ‘go’ to mark the purpose of a trip \REF{ex:5:132}.

\ea \label{ex:5:132}
Aban  olo  \textbf{aka}  yam.\\
\gll  Abaŋ  ɔ{}-lɔ   \textbf{aka}   jam\\
      Aban  \oldstylenums{3}\textsc{s}-go  on  water\\
\glt  ‘Aban goes to get water.’ (lit. she goes on water)
\z

The adposition \textit{anga} indicates possession. The predicate possessive construction is discussed in \sectref{sec:10.1.2}. In the possessive construction, \textit{anga} indicates a possessive relationship between the noun in the adpositional phrase and the other noun phrase in the construction. In \REF{ex:5:133}, \textit{anga}  indicates that \textit{dəray} ‘head’ is possessed by \textit{ləme} ‘us.’

\ea \label{ex:5:133}
{}[Dəray  ga]  [\textbf{anga}  ləme.]\\
\gll  {}[dəraj  ga]    [\textbf{aŋga}  lɪmɛ]\\
      head  {\ADJ}    {\POSS}  \oldstylenums{1}\textsc{Pex}\\
\glt  ‘We got the head.’ (lit. the head, belonging to us)
\z

The adposition \textit{afa} ‘at the house of’ plus a noun phrase gives a location\is{Deixis!Locational} at the house of the referent specified in the noun phrase \REF{ex:5:134}. 

\ea \label{ex:5:134}
Nolo  afa  bahay.\\
\gll  nʊ{}-lɔ   afa    bahaj\\
      {\oneS}-go  {at.house.of}  chief\\
\glt  ‘I go to the chief’s house.’
\z

The adposition \textit{ka} ‘like’ introduces an adverbial complement that expresses manner. \textit{Ka} appears twice in \REF{ex:5:135}. In the second instance, \textit{ka} carries the directional extension \textit{ala} ‘towards.’

\clearpage
\ea \label{ex:5:135}\corpussource{Values, S. 47}\\
Nəmbəɗom  a  dəray  ava  na,   \\
\gll  n\`{ə}-mbʊɗ{}-ɔm    a  dəraj  ava  na\\
      {\oneS}+{\PFV}-change-\oldstylenums{1}\textsc{Pex}  at   head  in    {\PSP}\\
\glt ‘We have become’ (lit. changed in the head)\\
\medskip
{}[\textbf{ka} kərkaɗaw  ahay  nə  hərgov  ahay  ga  a  ɓərzlan  ava  na], \\
\gll  {}[\textbf{ka} kərkaɗaw=ahaj nə hʊrgʷɔv=ahaj ga a ɓərɮaŋ ava na]\\
      like  monkey=Pl  with    baboon=Pl  {\ADJ}  at  mountain    in    {\PSP}\\
\glt ‘like monkeys and baboons on the mountains,’ \\     
\medskip
{}[\textbf{ka}  ala  kəra    na],  nəsərom  dəray  bay  pat.\\
\gll  {}[\textbf{ka}=ala kəra na]  n\`{ə}-sʊr-ɔm dəraj baj pat\\
      like=to  dog      {\PSP}  \oldstylenums{1}+{\PFV}-know-\oldstylenums{1}\textsc{Pex}  head  {\NEG}  all\\
\glt  ‘[and] like dogs, we don’t know anything!’
\z

\subsection{Complex adpositional phrase}\label{sec:5.6.2}
\hypertarget{RefHeading1211861525720847}{}
There are two complex adpositional phrases, each composed of the combination of a preposition and a postposition that surround the noun phrase. The adpositions give locational\is{Deixis!Locational} information. The first, \textit{kə…aka} ‘on’ marks the noun phrase as being a location to which the event expressed by the verb is directed. It can be employed in a physical sense (\ref{ex:5:136}--\ref{ex:5:138}) or a figurative sense \REF{ex:5:139}.  


\ea \label{ex:5:136}\corpussource{Cicada, S. 9}\\
Káafəɗom  anaw  \textbf{kə} mahay  əwla \textbf{aka}.\\
\gll  káá-fʊɗ-ɔm    an=aw    \textbf{kə} mahaj=uwla \textbf{aka}\\
      \oldstylenums{2}+{\POT}-place-{\twoP}  {\DAT}={\oneS}.{\IO}  on  door={\oneS}.{\POSS}  on\\
\glt  ‘You should place  [the tree] at my door.’  
\z

\ea \label{ex:5:137}
Enjé  \textbf{kə} delmete  \textbf{aka} a  slam  enen.\\
\gll  ɛ{}-ndʒ{}-ɛ    \textbf{kə} dɛlmɛtɛ \textbf{aka} a  ɬam  ɛnɛŋ\\
      \oldstylenums{3}\textsc{s}-leave-{\CL}    on  neighbor    on  at  place  another\\
\glt  ‘He left to go to his neighbor at some other place.’  
\z

\ea \label{ex:5:138}
Azaɗ  oloko  \textbf{kə}  dəray  a  məwta  \textbf{aka}.\\
\gll  à-zaɗ    ɔlɔkʷɔ  \textbf{kə}  dəraj  a  muwta  \textbf{aka}\\
      \oldstylenums{3}\textsc{s}+{\PFV}-carry  wood  on  head  {\GEN}  truck  on\\
\glt  ‘He/she carried the wood on top of the truck.’ (lit. on the head of the truck)
\z

\ea \label{ex:5:139}
Hərmbəlom  agə  ɓərav  va  \textbf{ka} war  anga  məze  dedelen  ga \textbf{aka}.\\
\gll  Hʊrmbʊlɔm  a-gə  ɓərav  =va  \textbf{ka} war  aŋga  mɪʒɛ  dɛdɛlɛŋ  ga \textbf{aka}\\
      God    \oldstylenums{3}\textsc{s}-do  heart  ={\PRF}  on  child  {\POSS}  person  black  {\ADJ}  on\\
\glt  ‘God was angry with the black man’s child.’  (lit. God did heart on the child that belongs to the black person)
\z

The second complex adpositional phrase, \textit{a…ava} ‘in,’ the preposition and postposition surround a noun phrase to mark that noun phrase as being a physical location in which the action of the verb is directed (\ref{ex:5:140} and \ref{ex:5:141}).

\ea \label{ex:5:140}
Olo  \textbf{a}  kosoko  \textbf{ava}.\\
\gll  ɔ{}-lɔ    \textbf{a}  kɔsɔkʷɔ  \textbf{ava}\\
      \oldstylenums{3}\textsc{s}-go  at  market  in\\
\glt  ‘He/she goes to market.’
\z

\ea \label{ex:5:141}
Afaɗ  dala  \textbf{a}  ombolo  \textbf{ava}.\\
\gll  a-faɗ  dala  \textbf{a}  ambɔlɔ  \textbf{ava}\\
      \oldstylenums{3}\textsc{s}-put  money  at  sack  in\\
\glt  ‘He/she put the money into [his] sack.’
\z

The postpositions \textit{aka} ‘on’ and \textit{ava} ‘in’ have the same forms as the verb adpositional\is{Adpositionals} extensions =\textit{aka} ‘on’ and =\textit{ava}  ‘in’ (see \sectref{sec:7.5.1}). The extensions permit the presence of the complex adpositional phrase which gives further precision concerning the location of the event (\ref{ex:5:142} and \ref{ex:5:143}\footnote{Even though the verb in this example has verbal extensions, it is not conjugated for subject since it is a climactic point in the story where nominalised forms are often found.  This is discussed further in Sections \ref{sec:7.6} and \ref{sec:8.2.3}.}). In the examples, the postpositions and verbal extensions are both bolded. 

\ea \label{ex:5:142}
Afəɗ\textbf{aka}  war  elé  háy  na  \textbf{kə}  ver  \textbf{aka}.\\
\gll  a-fəɗ=\textbf{aka}  war  ɛlɛ  haj  na  \textbf{kə}  vɛr  \textbf{aka}\\
      \oldstylenums{3}\textsc{s}-place=on  child  eye  millet  {\PSP}  on  stone  on\\
\glt  ‘She put the grain of millet on the grinding stone.’
\z


\ea \label{ex:5:143}
Məmət\textbf{ava}  alay  \textbf{a}  ver  \textbf{ava}.\\
\gll  mə-mət=\textbf{ava}=alaj  \textbf{a}   vɛr   \textbf{ava}\\
      {\NOM}{}-die=in=away  at  room  in\\
\glt  ‘She died in the room.’  
\z