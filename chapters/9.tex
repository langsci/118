\chapter[Verb types and transitivity]{Verb types and transitivity}\label{chap:9}
\hypertarget{RefHeading1212621525720847}{}
The way  Moloko expresses transitivity is one of its remarkable features. \citet{FriesenMamalis2008} reported that Moloko verb lexemes are underspecified with respect to transitivity\is{Transitivity|(}.  This chapter extends and deepens their work. Almost every Moloko verb can occur in clauses which are intransitive, transitive, or bitransitive and therefore cannot be classed as belonging to any one transitivity type.  Even clauses with no grammatical arguments exist -- a transitivity of zero\is{Transitivity!Clauses with zero transitivity}. The unique way that the semantics of the verb are realised by the affixes and extensions is one of the things that shows the genius of the language. 

It is important to understand four important features of Moloko verbs with respect to transitivity. The first is that there are two kinds of transitive constructions in Moloko and an Agent-Theme-Location semantic analysis is necessary to interpret these two constructions (\sectref{sec:9.1}). For transitive clauses, the grammatical relations of Moloko verbs directly and uniformly reflect the semantic picture. Subject expresses Agent. Direct object expresses semantic Theme, the core participant that literally or metaphorically changes state or position. Indirect object expresses semantic Location (\LOC) which can be  (depending on the verb type) either a literal or a metaphorical \LOC (recipient or beneficiary).\footnote{This semantic picture holds for bitransitive clauses (Sections~\ref{sec:9.2.4} and \ref{sec:9.2.5}). For intransitive clauses, the subject can correspond to a range of semantic roles; it can be any one of Agent, Theme, or \LOC (Sections~\ref{sec:9.2.4.2} and \ref{sec:9.2.5}).}

The second feature is that most Moloko verbs are ambitransitive -- the same verb with the same morphology may occur in clauses that are bitransitive, transitive, or intransitive. Moloko verbs are divided into classes based on the type of transitive and ditransitive construction(s) that the verb has (\sectref{sec:9.2}).  The third feature of Moloko verbs with respect to transitivity is that some verbs exhibit noun incorporation (\sectref{sec:9.3}). The final feature of Moloko verbs is that there are clauses with zero transitivity (\sectref{sec:9.4}).\is{Transitivity!Clauses with zero transitivity|)}

\largerpage With the exception of the reciprocal (see \sectref{sec:9.2.5}), there are no affixes, extensions, or particles that express changes in transitivity as might be expected in a Chadic language.\footnote{Causative verbal extensions, for example, are widespread in Chadic languages \citep[276]{Newman1977}.} In Moloko, it is the number and type of grammatical relations that a verb has that reflects the semantics of the construction.

\section{Two kinds of transitive clauses}\label{sec:9.1}
\hypertarget{RefHeading1212641525720847}{}
Moloko has two kinds of transitive clauses -- transitive clauses with subject and direct object (\ref{ex:9:1}--\ref{ex:9:2}) and transitive clauses with subject and indirect object (\ref{ex:9:3}--\ref{ex:9:4}). These two grammatically different transitive clauses illustrate that the semantics of Moloko verbs allows three core participants (represented by subject, direct object, and indirect object). Moloko verbs do not have just Agent-Patient semantic frames for events. In this work we follow an Agent-Theme-Location analysis, as developed by \citet{DeLancey1991}, in which ‘Location’ (\LOC) has a particular definition. Indirect object always expresses semantic {\LOC} -- the participant that represents the place where the Theme is directed to. As such the indirect object can express (depending on the verb type, see Sections~\ref{sec:9.2.3}--\ref{sec:9.2.5}) the recipient or beneficiary of the event. Direct object always expresses semantic Theme, the core participant that changes position or state because of the event. Subject in transitive clauses expresses the Agent.

It is the verbal pronominals that best illustrate the grammar of the two types of transitive clauses because the grammatical distinction between direct and indirect object is expressed by a core pronominal (the direct object pronominal and the indirect object pronominal enclitic). Note that when the indirect object is a noun phrase, it is inside a prepositional phrase. The indirect object prepositional phrase in Moloko is not a syntactic oblique, however, because the pronominals indicate that it represents a core participant of the event. For this reason, most of the examples are given in pairs in this chapter. The first example in each pair shows full noun phrase arguments for each core participant. The second example in each pair shows the same clause with all core participants represented by verbal pronominals. Pronominals are bolded in the second example in each pair. 

Examples \REF{ex:9:1} and \REF{ex:9:2} show a transitive clause with subject (\textit{Mana}) and direct object (\textit{awak} ‘goat’ in \ref{ex:9:1}, \textit{na} \oldstylenums{3}\textsc{s} direct object pronominal in \ref{ex:9:2}).

\ea \label{ex:9:1}
Mana  aslay  awak.\\
\gll  Mana a{}-ɬ{}-aj    awak\\
      Mana  \oldstylenums{3}\textsc{s}-slay{}-{\CL}  goat\\
\glt  ‘Mana slays a goat.’ 
\z

\ea \label{ex:9:2}
\textbf{A}slay  \textbf{na}.\\
\gll  \textbf{a-}ɬ{}-aj      \textbf{na}\\
      \oldstylenums{3}\textsc{s}-slay{}-{\CL}      \oldstylenums{3}\textsc{s}.{\DO}\\
\glt  ‘He slays it.’ 
\z

Examples \REF{ex:9:3} and \REF{ex:9:4} show a transitive clause with subject (\textit{Mana}) and indirect object (\textit{ana kəra} ‘to dog’ in \ref{ex:9:3}, =\textit{an} ‘to him’ in \ref{ex:9:4}). 

\ea \label{ex:9:3}
Mana  aɓan  ana  kəra. \\
\gll  Mana   a-ɓ=aŋ    ana   kəra \\
      Mana  \oldstylenums{3}\textsc{s}-hit=\oldstylenums{3}\textsc{s}.{\IO}  {\DAT} dog\\
\glt  ‘Mana hits a dog.’ (lit. he hits to him to dog)
\z

\ea \label{ex:9:4}
Aɓan. \\
\gll  a-ɓ=aŋ \\
      \oldstylenums{3}\textsc{s}-hit=\oldstylenums{3}\textsc{s}.{\IO}\\
\glt  ‘He hits him.’ (lit. he hits to him)
\z

Crosslinguistic studies might lead one to expect a verb like ‘hit’ to take a direct object; however verbs in Moloko require an Agent-Theme-\LOC semantic model to explain their behaviour. The indirect object \textit{kəra} ‘dog’ is the semantic \LOC \ -- here the recipient of the action -- the participant that represents the place where the Theme (the hit) is directed to. The participant that changes position or state in this event (the hit) is implicit in verbs of this type (see \sectref{sec:9.2.3}). 

Returning to the transitive clause with subject and direct object (\ref{ex:9:1} and \ref{ex:9:2}), the direct object \textit{awak} ‘goat’ is the Theme -- the participant that changes position or state because of an event (it is slain). 

\section{Verb types}\label{sec:9.2}
\hypertarget{RefHeading1212661525720847}{}
Most Moloko verbs are ambitransitive (i.e., labile) in that they can occur in intransitive, transitive, and sometimes bitransitive clauses with no morphological change in the verb complex (except of course the addition of the appropriate pronominals, \sectref{sec:7.3}).\footnote{Some verbs in related Chadic languages can also be ambitransitive. These include Cuvok\il{Cuvok} \citep{Ndokobai2006},  Buwal\il{Buwal} \citep{Viljoen2013}, and Vame\il{Vame} \citep{Kinnaird2006}.} Nevertheless, they can be divided into classes that exhibit different morphological and syntactic patterns relating/with respect to transitivity. Verbs are classified here as to the maximum number of grammatical relations that the verb can take as well as the type of grammatical relations: %%\is{Verb classification!Transitivity}

\largerpage
\begin{itemize}
\item Group 1: Verbs that can only be intransitive (\sectref{sec:9.2.1})
\item Group 2: Verbs that can be transitive with direct object (\sectref{sec:9.2.2})
\item Group 3: Verbs that can be transitive with indirect object (\sectref{sec:9.2.3})
\item Group 4: Verbs that can be bitransitive (\sectref{sec:9.2.4})
\item Group 5: Transfer verbs (\sectref{sec:9.2.5})
\end{itemize}

Examples are given in pairs in this chapter, first with full noun phrase arguments and then the same clause is given with the noun phrases replaced by pronominals. Examples with pronominals are necessary because the centrality of the distinction of verb types in Moloko is more apparent from the pronominals, especially for the indirect object. The indirect object can be expressed with a core pronominal within the verb complex, or a full noun phrase within an adpositional phrase. 

\subsection{Group 1: Verbs that can only be intransitive}\label{sec:9.2.1}\is{Tense, mood, and aspect!Progressive|(}
\hypertarget{RefHeading1212681525720847}{}
Only one verb in Moloko can never take an object (neither direct nor indirect). The locational clause contains the verb \textit{nday} and states that the subject is presently located somewhere (\ref{ex:9:5}--\ref{ex:9:6}). An explicit free noun phrase subject is not required when this verb is the main predicate since the subject is indicated in the verb prefix; however an adpositional phrase giving the location is required and follows the verb. This same verb functions as a progressive aspect auxiliary (see \sectref{sec:8.2.1}).\footnote{It is interesting that the locational extension =\textit{aka} is also used to express progressive aspect \sectref{sec:7.5.1}.}

\ea \label{ex:9:5}
Hawa  \textbf{anday} a  mogom.\\
\gll  Hawa \textbf{a-ndaj}    a  mɔgʷɔm\\
      Hawa  {\oldstylenums{3}\textsc{s}-be.located}  at  home\\
\glt  ‘Hawa is at home.’ 
\z

\ea \label{ex:9:6}
\textbf{Anday}  a  Marva.\\
\gll  \textbf{a-ndaj}   a   marva\\
      \oldstylenums{3}\textsc{s}-be  at  Maroua\\
\glt  ‘She is in Maroua.’
\z
\is{Tense, mood, and aspect!Progressive|)}
\subsection{Group 2: Verbs that can be transitive with direct object}\label{sec:9.2.2}

Clauses with reflexive-causative verbs can have either one core argument (subject) or two core arguments (subject and direct object). We have never found these verbs in a context where they take an indirect object as third core argument. 

Verbs from this class express reflexive actions when in an intransitive clause (action is to self; \ref{ex:9:7}) and causative actions when in a transitive clause with a direct object (action is to direct object; \ref{ex:9:8}). 

\ea \label{ex:9:7}
Mana  enjé  a  mogom. \\
\gll  Mana   \`{ɛ}-nʒ-\'{ɛ}     a   mɔgʷɔm \\
      Mana  \oldstylenums{3}\textsc{s}+{\PFV}-leave-{\CL}  at  home\\
\glt  ‘Mana went home.’ (lit. Mana left to home)  
\z

\ea \label{ex:9:8}
Mana  enjé  awak  a  mogom. \\
\gll  Mana   \`{ɛ}-nʒ-\'{ɛ}     awak   a   mɔgʷɔm\\
      Mana  \oldstylenums{3}\textsc{s}+{\PFV}-leave-{\CL}  goat  at  home\\
\glt  ‘Mana took the goat home.’ (lit. Mana left goat to home) 
\z

\tabref{tab:70} presents the morphology and clause structures for sample verbs in this category, across both intransitive and transitive clause constructions.

\begin{table}
\begin{tabularx}{\textwidth}{XX}
\lsptoprule

{Intransitive} & {Transitive}\\\midrule
\textit{Hawa}  \textit{e-nj-\'e} & \textit{Hawa}  \textit{e-nj-\'e}    \textit{awak}  \textit{a}  \textit{mogom}\\
Hawa    \oldstylenums{3}\textsc{s}+{\PFV}-leave-{\CL} & Hawa    \oldstylenums{3}\textsc{s}+{\PFV}-leave-{\CL}  goat    at    home\\
‘Hawa is gone.’ (lit. Hawa left) & ‘Hawa took the goat home.’ \\
\\
\textit{e-nj-\'e} & \textit{e-nj-\'e}    \textit{na}  \textit{a}  \textit{mogom}\\
\oldstylenums{3}\textsc{s}+{\PFV}-leave-{\CL} & \oldstylenums{3}\textsc{s}+{\PFV}-leave-{\CL}  \oldstylenums{3}\textsc{s}.{\DO}   at   home\\
‘She left.’ & ‘She took it home.’ \\
\midrule
\textit{Hawa }  \textit{a-həɓ-ay} & \textit{məwta}  \textit{a-həɓ-ay}    \textit{məze}\\
Hawa   \oldstylenums{3}\textsc{s}+{\PFV}-dance-{\CL} & car   \oldstylenums{3}\textsc{s}+{\PFV}-dance-{\CL}   person\\
‘Hawa danced.’ &  ‘The car shook people up.’ \\
& (lit. the car danced people)\\
\\
\textit{a-həɓ-ay} & \textit{a-həɓ-ay}    \textit{na}\\
\oldstylenums{3}\textsc{s}+{\PFV}-dance-{\CL} & \oldstylenums{3}\textsc{s}+{\PFV}-dance-{\CL}   \oldstylenums{3}\textsc{s}.{\DO}\\
 ‘She danced.’ & ‘It shook him.’\\
\midrule
\textit{Hawa }  \textit{e-cək-e} & \textit{Hawa}  \textit{e-cək-e}    \textit{zar} \\
Hawa    \oldstylenums{3}\textsc{s}+{\PFV}-stand-{\CL} & Hawa    \oldstylenums{3}\textsc{s}+{\PFV}-stand-{\CL}   man\\
‘Hawa stood up.’ & ‘Hawa helped the man to stand up.’ \\
& (lit. Hawa stood man)\\
\\
\textit{e-cək-e} & \textit{e-cək-e}    \textit{na}\\
\oldstylenums{3}\textsc{s}+{\PFV}-stand-{\CL} & \oldstylenums{3}\textsc{s}+{\PFV}-stand-{\CL}   \oldstylenums{3}\textsc{s}.{\DO} \\
‘She stood up.’ & ‘She stood him up.’ \\
\midrule
\textit{Hawa}  \textit{a-yəɗ-ə=va} & \textit{slərele}  \textit{a-yəɗ{}-ay}    \textit{Hawa}\\ 
Hawa    \oldstylenums{3}\textsc{s}+{\PFV}-tire-{\CL}  {=\PRF} & work    \oldstylenums{3}\textsc{s}+{\PFV}-tire-{\CL}   Hawa\\
‘Hawa is tired.’ & ‘Work tired Hawa out.’\\
\\
\textit{Hawa} \textit{á-yəɗ-ay}\\
Hawa  \oldstylenums{3}\textsc{s}+{\IFV}-tire-{\CL}\\
‘Hawa can/might get tired.’ (lit. Hawa tires) \\
\lspbottomrule
\end{tabularx}
\caption{\label{tab:70}Group 2 verbs}\end{table}

\subsection{Group 3: Verbs that can be transitive with indirect object}\label{sec:9.2.3}\is{Attribution!Expressed using verb}
\hypertarget{RefHeading1212701525720847}{}
Some transitive verbs in Moloko never take a direct object but rather have only what we have been referring to as an indirect object in this work. These verbs express experience, feeling, or emotion. The indirect object expresses the semantic \LOC (recipient, beneficiary, experiencer) of the event. A semantic core participant that moves or undergoes a change of state or is in a state (Theme) may be implicit or be lexicalised into the verb. 

The verb \textit{rəɓ-ay} ‘to be beautiful’ involves a thing and its quality (\ref{ex:9:9}--\ref{ex:9:10}), and the person whose opinion or perception is being cited is coded as the indirect object.  In an intransitive clause, the subject (\textit{dalay}  ‘girl’) is at the state of being beautiful. In a transitive clause (with an indirect object), the subject (\textit{dalay} ‘girl’) is felt to be beautiful by the indirect object (\textit{=aw} ‘to me’). 


\ea \label{ex:9:9}
Dalay  arəɓay.\\
\gll  dalaj  a-rəɓ-aj\\
      girl    \oldstylenums{3}\textsc{s}-{be beautiful}-{\CL}\\
\glt  ‘The girl is beautiful.’
\z

\noindent\parbox{\textwidth}{\ea \label{ex:9:10}
Dalay  arəɓaw.\\
\gll  dalaj  a-rəɓ=aw\\
      girl    {\oldstylenums{3}\textsc{s}-be beautiful={\oneS}.{\IO}}\\
\glt  ‘The girl is beautiful to me.’ 
\z}

The experience verb  /ts r/ ‘taste good’ is grammatically expressed in \REF{ex:9:11} as the subject \textit{ɗaf}  ‘millet loaf' tastes good to the semantic \LOC expressed by the indirect object (the pronominal enclitic \textit{=aw} ‘to me’).

\ea \label{ex:9:11}
Ɗaf  acaraw. \\
\gll  ɗaf     à-tsar=aw \\
      {millet loaf}  {\oldstylenums{3}\textsc{s}+{\PFV}-taste good}={\oneS}.{\IO}\\
\glt  ‘Millet loaf tasted good to me.’ 
\z

Likewise with the verb /g r -j/  ‘fear’ \REF{ex:9:12},  the elephant causes fear at the \LOC ‘the children.’

\ea \label{ex:9:12}
Mbelele  agarata  ana  babəza  ahay. \\
\gll  mbɛlɛlɛ   à-gar=ata  ana  babəza=ahaj\\
      elephant  \oldstylenums{3}\textsc{s}+{\PFV}-fear=\oldstylenums{3}\textsc{p}.{\IO}  {\DAT} children=PL\\
\glt  ‘The children are afraid of the elephant.’
\z

The verbs /dz n-j/ ‘help,’ /ɓ-j/ ‘hit,’ and /s/ ‘please' are also in this group of verbs. The receiver of the help or hit is expressed by the indirect object which is affected positively (in the case of help) or negatively (in the case of hit) by the event. For these verbs, the semantic Theme (the hit or the help) never appears as a direct object since it is part of the meaning of these verbs. \tabref{tab:71} presents examples of verbs of this type.

% \largerpage[3]
\begin{table}[p]
\caption{Group 3 verbs\label{tab:71}}
\footnotesize\begin{tabular}{ll}
\lsptoprule
{Intransitive} & {Transitive}\\\midrule
\textit{Hawa à-rəɓ{}-aj} & \textit{hʷɔr}  \textit{à-rəɓ=an}    \textit{ana}  \textit{Mana}\\*
Hawa    \oldstylenums{3}\textsc{s}+{\PFV}-{be beautiful-\CL}  &  Hawa  \oldstylenums{3}\textsc{s}+{\PFV}-{be beautiful}  =\oldstylenums{3}\textsc{s}.{\IO}   {\DAT} Mana\\*
‘Hawa was beautiful.’  &  ‘The woman was beautiful to Mana.’ \\
\\
\textit{a-rəɓ{}-ay}  &  \textit{a-rəɓ=an}\\*
\oldstylenums{3}\textsc{s}+{\PFV}-{be beautiful-\CL}  &  Hawa  \oldstylenums{3}\textsc{s}+{\PFV}-{be beautiful}  =\oldstylenums{3}\textsc{s}.{\IO}\\*
‘She was beautiful.’   &  ‘She was beautiful to him.’\\
\midrule
\textit{ɗaf }    \textit{a-car}   &  \textit{ɗaf }     \textit{a-car=an }      \textit{ana }  \textit{Mana}\\*
{millet loaf}    \oldstylenums{3}\textsc{s}+{\PFV}-{taste good}  &  {millet loaf}  \oldstylenums{3}\textsc{s}+{\PFV}-{taste good}=\oldstylenums{3}\textsc{s}.{\IO}    {\DAT}   Mana\\*
‘Millet loaf tasted good.’  &  ‘Millet loaf tasted good to Mana.’ \\
\\
\textit{a-car}  &  \textit{a-car=an}\\*
\oldstylenums{3}\textsc{s}+{\PFV}-{taste good}  &  \oldstylenums{3}\textsc{s}+{\PFV}-{taste good}=\oldstylenums{3}\textsc{s}.{\IO}\\*
‘It tasted good.’  &  ‘It tasted good to him.’\\
\midrule
\textit{Mana }  \textit{a-gar-ay}   & \textit{mbelele }  \textit{a-gar=an }    \textit{ana }  \textit{Mana}\\*
Mana    \oldstylenums{3}\textsc{s}+{\PFV}-fear-{\CL}  &  elephant  \oldstylenums{3}\textsc{s}+{\PFV}-fear=\oldstylenums{3}\textsc{s}.{\IO}    {\DAT}  Mana\\*
‘Mana was afraid.’  &  ‘An elephant made Mana afraid.’ \\
\\
\textit{a-gar-ay}  &   \textit{a-gar=an}\\*
\oldstylenums{3}\textsc{s}+{\PFV}-fear-{\CL}  &  \oldstylenums{3}\textsc{s}+{\PFV}-fear-{\CL}=\oldstylenums{3}\textsc{s}.{\IO}\\*
‘He was afraid.’  &  ‘It made him afraid.’\\
\midrule

\textit{fat} \textit{á-war}  & \textit{fat}  \textit{á-wal=an}  \textit{ana}  \textit{Mana}\\*
sun    \oldstylenums{3}\textsc{s}+{\IFV}-hurt  &  sun  \oldstylenums{3}\textsc{s}+{\IFV}-hurt=\oldstylenums{3}\textsc{s}.{\IO}   {\DAT}    Mana\\*
‘The sun hurts.’  &  ‘The sun hurts Mana.’ (lit. The sun hurts to Mana) \\
\\
\textit{á-war}  &  \textit{á-wal=an}\\*
\oldstylenums{3}\textsc{s}+{\IFV}-hurt  &  \oldstylenums{3}\textsc{s}+{\IFV}-hurt=\oldstylenums{3}\textsc{s}.{\IO}\\*
‘It hurts.’  &  ‘It hurts him.’\\
\midrule
\textit{Mana }  \textit{á-ɗas}  & \textit{Mana}  \textit{á-ɗəs=an}          \textit{ana}   \textit{Hərmbəlom}\\* 
Mana   \oldstylenums{3}\textsc{s}+{\IFV}-{be heavy}  &  Mana   \oldstylenums{3}\textsc{s}+{\IFV}-{be heavy}=\oldstylenums{3}\textsc{s}.{\IO}  {\DAT} God\\*
‘Mana is honourable.’ (lit. Mana is heavy).  &  ‘Hawa honours God.’ (lit. Hawa honours to God) \\
\\
\textit{á-ɗas}  &  \textit{{}á-ɗəs=an}\\*
\oldstylenums{3}\textsc{s}+{\IFV}-{be heavy}  &  \oldstylenums{3}\textsc{s}+{\IFV}-{be heavy}=\oldstylenums{3}\textsc{s}.{\IO}\\*
‘He is honourable.’   &  ‘He honours him.’ \\
\midrule
\textit{Hawa }  \textit{á-jən-ay}  & \textit{Hawa }  \textit{á-jən=an }    \textit{ana }  \textit{Mana}\\*
Hawa    \oldstylenums{3}\textsc{s}+{\IFV}-help-{\CL}  &  Hawa    \oldstylenums{3}\textsc{s}+{\IFV}-help =\oldstylenums{3}\textsc{s}.{\IO}  {\DAT}   Mana\\*
‘Hawa helps (Hawa is a helpful person.'   &  ‘Hawa helps Mana.’ \\*

\\
\textit{á-jən-ay}  &  \textit{á-jən=an }\\*
\oldstylenums{3}\textsc{s}+{\IFV}-help-{\CL}  &  \oldstylenums{3}\textsc{s}+{\IFV}-help =\oldstylenums{3}\textsc{s}.{\IO}\\*
‘She is a helpful person.’ &  ‘She helps him.’\\\lspbottomrule
\end{tabular}
\end{table}

Note that an intransitive clause appears to be ungrammatical for the verbs \newline/ɓ-j/ ‘hit’ and /s/ ‘please’ (\ref{ex:9:13}--\ref{ex:9:14}).

\ea \label{ex:9:13}
\ea {Hawa  áɓan  ana  kəra. \\
\gll  Hawa  á-ɓ=aŋ  ana  kəra\\
      Hawa    \oldstylenums{3}\textsc{s}+{\IFV}-hit=\oldstylenums{3}\textsc{s}.{\IO}  {\DAT}   dog\\
\glt  ‘Hawa hits the dog.’}

\medskip
\ex
{áɓan. \\
\gll  á-ɓ=aŋ\\
      \oldstylenums{3}\textsc{s}+{\IFV}-hit=\oldstylenums{3}\textsc{s}.{\IO}\\
\glt  ‘She hits it.’}
\z\z

\ea \label{ex:9:14}
\ea Sese ásan ana Mana. \\
\gll  ʃɛʃɛ  á-s=aŋ         ana   Mana\\
      meat    \oldstylenums{3}\textsc{s}+{\IFV}-please=\oldstylenums{3}\textsc{s}.{\IO}  {\DAT}  Mana\\
\glt  ‘Meat is pleasing to Mana.’

\medskip
\ex
{ásan. \\
\gll  á-s=aŋ\\
      \oldstylenums{3}\textsc{s}+{\IFV}-please =\oldstylenums{3}\textsc{s}.{\IO}\\
\glt  ‘It pleases him.’}
\z\z


\subsection{Group 4: Verbs that can be bitransitive}\label{sec:9.2.4}
\hypertarget{RefHeading1212721525720847}{}
Verbs that can occur in bitransitive clauses with subject, direct object, and indirect object can also occur in intransitive clauses (subject only) and transitive clauses (subject and direct object). When present, the indirect object always expresses the benefactive or malefactive. 

The semantics of transitive and bitransitive clauses is uniform for these verbs – subject always expresses semantic Agent, direct object always expresses semantic Theme, and indirect object always expresses semantic \LOC (typically Beneficiary or Maleficiary). Intransitive clauses are more flexible in that the subject can express either Agent or Theme for some verbs. Transitive and bitransitive clauses are discussed for these verbs in \sectref{sec:9.2.4.1} and intransitive clauses are discussed in \sectref{sec:9.2.4.2}.

\subsubsection{Group 4 verbs in transitive and bitransitive clauses}\label{sec:9.2.4.1}

The verb \textit{p-ay} ‘open’ illustrates this verb type. In a transitive clause \REF{ex:9:15}, the subject (\textit{Mana}) performs the action on the direct object (\textit{mahay} ‘door’). 


\ea \label{ex:9:15}
Mana  apay  mahay.\\
\gll  Mana   à-p-aj    mahaj\\
      Mana  \oldstylenums{3}\textsc{s}+{\PFV}-open{}-{\CL}  door\\
\glt  ‘He/she opened the door.’ 
\z

In a bitransitive clause \REF{ex:9:16}, the action done to the direct object is for the benefit of the indirect object.

\ea \label{ex:9:16}
Mana  apan  mahay  ana  Hawa.\\
\gll  Mana   à-p=aŋ      mahaj  ana  Hawa\\
      Mana  \oldstylenums{3}\textsc{s}+{\PFV}-open=\oldstylenums{3}\textsc{s}.{\IO}  door  {\DAT} Hawa\\
\glt  ‘Mana opened the door for Hawa.’ 
\z

The verb \textit{mənjar} ‘see’ occurs in intransitive, transitive, and bitransitive clauses. In a transitive clause \REF{ex:9:17}, the subject (\textit{Mala}) sees the direct object (\textit{awak} ‘goat’).\footnote{The indirect object ‘goat’ undergoes a change of state from being unseen to being seen at a particular \LOC.} 

\ea \label{ex:9:17}
Mala  ámənjar  awak.\\
\gll  Mala á-mənzar   awak\\
      Mala  \oldstylenums{3}\textsc{s}+{\IFV}-see  goat\\
\glt  ‘Mala sees a goat.’{ }
\z

In a bitransitive clause \REF{ex:9:18}, the subject (\textit{Mala}) sees the direct object (\textit{awak} ‘goat’) on behalf of the indirect object beneficiary (\textit{bahay} ‘chief’). The chief is the metaphorical \LOC to which the action is directed.

\ea \label{ex:9:18}
Mala  olo  amənjaran  awak  ana  bahay.\\
\gll  Mala  ɔ{}-lɔ    a-mənzar=aŋ  awak  ana  bahaj\\
      Mala  \oldstylenums{3}\textsc{s}-go  \oldstylenums{3}\textsc{s}-see=\oldstylenums{3}\textsc{s}.{\IO}  goat  {\DAT} chief\\
\glt  ‘Mala went to see a person’s goat in the chief’s place.’{ }
\z

For the verb /h/ ‘say’ \REF{ex:9:19}, the subject \textit{Mana} says the utterance (expressed by the direct object pronominal \textit{na}) to \textit{Hawa}. 

\ea \label{ex:9:19}
Mana  ahan  na  ana  Hawa.\\
\gll  Mana   à-h=aŋ   na  ana   Hawa\\
      Mana  \oldstylenums{3}\textsc{s}+{\PFV}-say=\oldstylenums{3}\textsc{s}.{\IO}  \oldstylenums{3}\textsc{s}.{\DO}  {\DAT} Hawa\\
\glt  ‘Mana told it to Hawa.’ 
\z

\tabref{tab:72} presents examples of this verb type with benefactive indirect object. 

% \largerpage[2]
\begin{table}[p]
\caption{Group 4 verbs where {\IO} expresses benefactive \label{tab:72}}
\footnotesize
\begin{tabular}{ll}
\lsptoprule
{Intransitive} & {Transitive}\\\midrule
\textit{Hawa}  \textit{a-bah} \textit{yam} & \textit{Hawa} \textit{a-bah=an} \textit{yam} \textit{ana} \textit{Mana}\\
Hawa \oldstylenums{3}\textsc{s}+{\PFV}-pour  water  &  Hawa \oldstylenums{3}\textsc{s}+{\PFV}-pour=\oldstylenums{3}\textsc{s}.{\IO}  water  {\DAT} Mana\\
‘Hawa poured water.’  &  ‘Hawa poured water for Mana.’\\
% \\
\textit{a-bah} \textit{na}  &  \textit{a-bah=an}  \textit{na}\\
\oldstylenums{3}\textsc{s}+{\PFV}-pour  \oldstylenums{3}\textsc{s}.{\DO}  &  \oldstylenums{3}\textsc{s}+{\PFV}-pour=\oldstylenums{3}\textsc{s}.{\IO}   \oldstylenums{3}\textsc{s}.{\DO}\\
‘She poured it.’  &  ‘She poured it for him.’   \\\midrule
\textit{Mana}  \textit{a-sl-ay} \textit{awak}  &   \textit{Mana}  \textit{a-sl=an} \textit{awak} \textit{ana}  \textit{bahay}\\
Mana    \oldstylenums{3}\textsc{s}+{\PFV}-slay-{\CL}  goat  &  Mana    \oldstylenums{3}\textsc{s}+{\PFV}-slay=\oldstylenums{3}\textsc{s}.{\IO}      goat       {\DAT}   chief\\
‘Mana slaughtered a goat.’  &  ‘Mana slaughtered the goat for the chief.’ \\
% \\
\textit{a-sl-ay}    \textit{na}  &  \textit{a-sl=an}  \textit{na}\\
\oldstylenums{3}\textsc{s}+{\PFV}-slay-{\CL}    \oldstylenums{3}\textsc{s}.{\DO}  &  \oldstylenums{3}\textsc{s}+{\PFV}-slay=\oldstylenums{3}\textsc{s}.{\IO}  \oldstylenums{3}\textsc{s}.{\DO}\\
‘He slaughtered it.’   &  ‘He slaughtered it for him.’\\\midrule
\textit{Hawa}  \textit{e{}-d-e}    \textit{ɗaf}  &  \textit{Hawa}  \textit{a-d=an}  \textit{ɗaf} \textit{ana}  \textit{Mana}\\
Hawa   \oldstylenums{3}\textsc{s}+{\PFV}-make-{\CL}    millet loaf  &  Hawa    \oldstylenums{3}\textsc{s}+{\PFV}-make=\oldstylenums{3}\textsc{s}.{\IO}  millet loaf  {\DAT}  Mana\\
‘Hawa made millet loaf.’  &  ‘Hawa made millet loaf for Mana.’\\
% \\
\textit{e-d-e }    \textit{na}  &  \textit{a-d=an }        \textit{na}\\
\oldstylenums{3}\textsc{s}+{\PFV}-make-{\CL}   \oldstylenums{3}\textsc{s}.{\DO}  &  \oldstylenums{3}\textsc{s}+{\PFV}-make=\oldstylenums{3}\textsc{s}.{\IO}  \oldstylenums{3}\textsc{s}.{\DO}\\
‘She made it.’   &  ‘She made it for him.’\\\midrule
\textit{Hawa}  \textit{a-bal-ay}  \textit{zana}  & \textit{Hawa} \textit{a-bal=an} \textit{zana} \textit{ana}  \textit{Mana}\\
Hawa    \oldstylenums{3}\textsc{s}+{\PFV}-wash-{\CL}    clothes  &  Hawa    \oldstylenums{3}\textsc{s}+{\PFV}-wash=\oldstylenums{3}\textsc{s}.{\IO}     clothes  {\DAT}  Mana\\
‘Hawa washed clothes.’   &  ‘Hawa washed clothes for Mana.’ \\
% \\
\textit{a-bal-ay}    \textit{na}  &  \textit{a-bal=an}  \textit{na}\\
\oldstylenums{3}\textsc{s}+{\PFV}-wash-{\CL}    \oldstylenums{3}\textsc{s}.{\DO}  &  \oldstylenums{3}\textsc{s}+{\PFV}-wash=\oldstylenums{3}\textsc{s}.{\IO}   \oldstylenums{3}\textsc{s}.{\DO}\\
‘She washed it.’   &  ‘She washed it for him.’ \\\midrule
\textit{Hawa }  \textit{a-rah }    \textit{cafəgal}   & \textit{Hawa }  \textit{a-rah=an} \textit{cafəgal}  \textit{ana }  \textit{Mana}\\
Hawa    \oldstylenums{3}\textsc{s}+{\PFV}-fill    bucket  &  Hawa  \oldstylenums{3}\textsc{s}+{\PFV}-slay=\oldstylenums{3}\textsc{s}.{\IO}      bucket    {\DAT}  Mana\\
‘Hawa poured the bucket.’  &  ‘Hawa poured the bucket for Mana.’\\
% \\
\textit{a-rah }    \textit{na}  &  \textit{a-rah=an}  \textit{na}\\
\oldstylenums{3}\textsc{s}+{\PFV}-fill    \oldstylenums{3}\textsc{s}.{\DO}  &  \oldstylenums{3}\textsc{s}+{\PFV}-fill=\oldstylenums{3}\textsc{s}.{\IO}    \oldstylenums{3}\textsc{s}.{\DO}\\
‘She filled it.’  &  ‘She filled it for him.’\\\midrule
\textit{Mala   á-mənjar   awak}   &   \textit{Mala   a-mənjar=an   awak   ana   bahay}\\
Mala   \oldstylenums{3}\textsc{s}+{\IFV}-see   goat  &  Mala   \oldstylenums{3}\textsc{s}-see=\oldstylenums{3}\textsc{s}.{\IO}  goat   {\DAT} chief\\
‘Mala sees a goat.’  &  ‘Mala saw someone’s goat for the chief.’\\
% \\
\textit{á-mənjar  }  \textit{na}  &   \textit{a-mənjar=an   na }\\
\oldstylenums{3}\textsc{s}+{\IFV}-see    \oldstylenums{3}\textsc{s}.{\DO}  & \oldstylenums{3}\textsc{s}-see=\oldstylenums{3}\textsc{s}.{\IO}    \oldstylenums{3}\textsc{s}.{\DO} \\
‘He sees it.’  &  ‘He saw it for him.’\\
\lspbottomrule
\end{tabular}
\end{table}

For some transitive verbs of this type, the indirect object (when present) marks the malefactive of the event. The indirect object will be negatively affected by the event. For the verb \textit{paɗ-ay} ‘eat’ in \REF{ex:9:20} the subject (\textit{awak} ‘goat’) ate the direct object (\textit{háy} ‘millet’), incurring a negative effect on the indirect object (\textit{=aw} ‘to me’).\footnote{This phenomenon is also known as possessor raising or external possession. We consider that the semantics for this construction in Moloko are malefactive rather than possessive because a possessive construction can also be employed (without an indirect object): \textit{awak a-paɗ-ay na háy əwla =va,} ‘the goat ate my millet’. The construction with an indirect object connects the millet to its owner with less precision than the possessive construction, and concentrates on the loss that the owner incurred (due to the damages done to his millet field) rather than the fact that he owned the field.}  

\ea \label{ex:9:20}
Awak  apaɗ\textbf{aw}   na  háy  va.\\
\gll  awak a-paɗ\textbf{=aw}    na   haj=va\\
      goat    \oldstylenums{3}\textsc{s}-crunch={\oneS}.{\IO}   \oldstylenums{3}\textsc{s}.{\DO}   millet={\PRF}\\
\glt  ‘The goat has eaten my millet.’ (lit. the goat has eaten to me the millet) 
\z

The indirect object also expresses the malefactive with the verbs \textit{mbəzen} ‘ruin’ \REF{ex:9:21} and \textit{cen} ‘understand’ \REF{ex:9:22}. In \REF{ex:9:21} the subject (\textit{sla=ahay }‘the cows’) have ruined the direct object (\textit{gəvah} ‘the field’) with a negative effect on the indirect object (\textit{=aloko} ‘to us’).

\ea \label{ex:9:21}
Sla  ahaj  təmbəzaloko  na  gəvah  va.\\
\gll  ɬa=ahaj   t\'{ə}-mbəz=alɔkʷɔ        na     gəvah=va\\
      cow=Pl       \oldstylenums{3}\textsc{p}+{\PFV}-ruin=\oldstylenums{1}\textsc{Pin}.{\IO}  \oldstylenums{3}\textsc{s}.{\DO}  \textit{   }field={\PRF}\\
\glt  ‘The cows have ruined our field.’ (lit. The cows have ruined to us the field)
\z

Example \REF{ex:9:22} shows a bitransitive clause with the verb \textit{cen} ‘hear’/'understand.' The subject (\textit{a-} \oldstylenums{3}\textsc{s} subject pronominal) didn’t understand the direct object (\textit{ma =əwla} ‘my words’) with a negative effect on the indirect object (\textit{=aw} ‘to me’).\footnote{Note that phonetically the word-final /n/ drops off when the indirect object clitic attaches.} 

\ea \label{ex:9:22}
Acaw  aka  va  ma  əwla  bay.\\
\gll  à-ts=aw     =aka=va     ma=uwla      baj\\
      \oldstylenums{3}\textsc{s}+{\PFV}-understand={\oneS}.{\IO}  =on={\PRF}  word={\oneS}.{\POSS}  {\NEG}\\
\glt  ‘He/she didn’t understand my words.’ (lit. he had understood on my words not)
\z

\tabref{tab:73} provides examples of group 4 verbs where the indirect object expresses the malefactive.

\begin{table}
\begin{tabular}{ll}
\lsptoprule
{Transitive} & {Bitransitive}\\\midrule
\textit{awak   a-paɗ-ay        háy} & \textit{awak   a-paɗ\textbf{=aw}   na   háy=va}\\
goat   \oldstylenums{3}\textsc{s}+\textsc{{\PFV}}{}-crunch{}-{\CL}    millet & goat   \oldstylenums{3}\textsc{s}-crunch={\oneS}.{\IO}  \oldstylenums{3}\textsc{s}.{\DO}   millet={\PRF}\\
‘The goat ate the millet.’ & ‘The goat has eaten my millet.’ \\
\\
\textit{a-paɗ-ay        na }& \textit{a-paɗ\textbf{=aw}   na=va}\\
\oldstylenums{3}\textsc{s}+\textsc{{\PFV}}{}-crunch{}-{\CL}    \oldstylenums{3}\textsc{s}.{\DO} & \oldstylenums{3}\textsc{s}-crunch={\oneS}.{\IO}    \oldstylenums{3}\textsc{s}.{\DO}={\PRF}\\
‘He ate it.’ & ‘The goat has eaten it to me.’ \\
&  (the goat has eaten it and I am affected)\\\midrule
\textit{avar }  \textit{e-mbezen }  \textit{háy} & \textit{sla }  \textit{a-mbəz=aloko }    \textit{na }  \textit{gəvah}\textit{=va}\\
rain    \oldstylenums{3}\textsc{s}-ruin      millet & cow    \oldstylenums{3}\textsc{s}+{\PFV}-ruin=1\textsc{Pin}.{\IO}    \oldstylenums{3}\textsc{s}.{\DO}    field{=\PRF}\\
‘The rain ruined the millet.’ & ‘The cow has ruined our field.’\\\\
 \textit{e-mbezen }  \textit{na} & \textit{a-mbəz=aloko }    \textit{na}\textit{=va}\\
\oldstylenums{3}\textsc{s}-ruin       \oldstylenums{3}\textsc{s}.{\DO} & \oldstylenums{3}\textsc{s}+{\PFV}-ruin=1\textsc{Pin}.{\IO}    \oldstylenums{3}\textsc{s}.{\DO}{=\PRF}\\
‘It ruined it.’ & ‘It has ruined it for us.’\\\midrule
\textit{awak}  \textit{a-zom} \textit{háy} & \textit{awak} \textit{a-zom=an} \textit{háy}  \textit{a}  \textit{Mana}\\
goat  \oldstylenums{3}\textsc{s}+{\PFV}-eat    millet & goat    \oldstylenums{3}\textsc{s}+{\PFV}-eat=\oldstylenums{3}\textsc{s}.{\IO}    millet    {\GEN}  Mana\\
‘The goat ate millet.’ & ‘The goat ate Mana’s millet.’ \\
&  (lit. the goat ate to him millet of Mana).’\\
\\
\textit{a-zom}    \textit{na} & \textit{a-zom=an}    \textit{na}\\
\oldstylenums{3}\textsc{s}+{\PFV}-eat  \oldstylenums{3}\textsc{s}.{\DO} & \oldstylenums{3}\textsc{s}+{\PFV}-eat=\oldstylenums{3}\textsc{s}.{\IO}    \oldstylenums{3}\textsc{s}.{\DO}\\
‘He ate it.’ & ‘It ate it, affecting him.’\\
\lspbottomrule
\end{tabular}
\caption{Group 4 verbs where {\IO} expresses malefactive \label{tab:73}}
\end{table}


Moloko uses a transitive clause with a third person plural subject pronominal when the identity of the Agent is unimportant or unknown in the discourse. The literal meaning of \REF{ex:9:23} is ‘They are greeting you,’ but this construction is used even when the person greeting is singular and the speaker knows who it is but doesn’t want to say.\footnote{The verb /h-j/ ‘say’ shows incorporation of the ‘body-part’ noun \textit{ma} ‘word/mouth’ (\sectref{sec:9.3}).} Example \REF{ex:9:24} is from the Disobedient Girl text (see \sectref{sec:1.5}). The example literally means ‘they brought her out’ but the identity of those who carried her is unimportant in the story.
\clearpage
\ea \label{ex:9:23}
Tahok  ma.\\
\gll  ta-h=ɔkʷ  ma\\
      \oldstylenums{3}\textsc{p}-tell={\twoS}.{\IO}    mouth\\
\glt  ‘You are being greeted.’ (lit. they are telling word to you) 
\z

\ea \label{ex:9:24}
\corpussource{Disobedient Girl, S. 30}\\
Tazlərav  na  ala.\\
\gll  tà-ɮərav    na=ala \\
      \oldstylenums{3}\textsc{p}+{\PFV}-exit \oldstylenums{3}\textsc{s}.{\DO}=to \\
\glt  ‘She was brought out [of the house].’ (lit. they brought her out)
\z

\subsubsection[Group 4 verbs in intransitive clauses]{Group 4 verbs in intransitive clauses}\label{sec:9.2.4.2}

There are two semantic possibilities for intransitive clauses of Group 4 verbs in Perfective aspect\is{Tense, mood, and aspect!Perfective aspect}\is{Tense, mood, and aspect!Aspect in intransitive clauses}. Subject can be the semantic Agent or the semantic Theme. Some verbs have both possibilities, but for other verbs, subject can express only Agent or only Theme. For the verb \textit{d-e} ‘prepare,’ the subject of an intransitive clause is the semantic Agent \REF{ex:9:25} and the semantic Theme is unspecified. 

\ea \label{ex:9:25}
Hawa  ede. \\
\gll  Hawa   \`ɛ{}-d-ɛ \\
      Hawa  \oldstylenums{3}\textsc{s}+{\PFV}-prepare-{\CL}\\
\glt  ‘Hawa made [something].’ 
\z

With \textit{cen} ‘hear,’ an intransitive clause in Perfective aspect\is{Tense, mood, and aspect!Perfective aspect} \REF{ex:9:26} expresses an event where the subject hears and understands (what they hear/understand may not be explicit in the clause). 

\ea \label{ex:9:26}
Mana  ecen. \\
\gll  Mana   \`ɛ{}-tʃɛŋ \\
      Mana  \oldstylenums{3}\textsc{s}+{\PFV}-understand\\
\glt  ‘Mana heard/understood (something).’
\z

In contrast, for the verb \textit{p-ay} ‘open,’ the subject of an intransitive clause is the semantic Theme which is affected by the action \REF{ex:9:27}. More examples are shown in \tabref{tab:74}.

\clearpage
\ea \label{ex:9:27}
Mahay  apay.\\
\gll  mahaj   à-p-aj\\
      door  \oldstylenums{3}\textsc{s}+{\PFV}-open{}-{\CL}\\
\glt  ‘The door opened.’ 
\z

There is also a difference between the Imperfective\is{Tense, mood, and aspect!Imperfective aspect|(}, Perfective, and Perfect in an intransitive clause\is{Tense, mood, and aspect!Perfect|(} that doesn’t hold for transitive and bitransitive clauses.\is{Tense, mood, and aspect!Perfective aspect}\footnote{Intransitive clauses with transfer verbs \sectref{sec:9.2.5} also show this semantic picture.} In intransitive clauses for these verbs, Imperfective aspect indicates that the subject is at the state of being potentially able to do or submit to the action (more of an irrealis idea) while Perfect is a resultative state. In contrast, for transitive and bitransitive clauses, Imperfective aspect expresses an incomplete event  (see \sectref{sec:7.4.2}) and the Perfect expresses that the event was completed prior to a point of reference (see \sectref{sec:7.5.3}).  For example, an intransitive clause with the verb /p -j/ ‘open’ expresses an event with an unspecified Agent when the verb is Perfective: ‘the door opened’ \REF{ex:9:28}. 

\ea \label{ex:9:28}
Mahay  apay.\\
\gll  mahaj à-p-aj\\
      door  \oldstylenums{3}\textsc{s}+{\PFV}-open{}-{\CL}\\
\glt  ‘The door opened.’
\z

Likewise with the verb /b h/ ‘pour,’ water ‘is poured’ \REF{ex:9:29}.  

\ea \label{ex:9:29}
Yam  abah.\\
\gll  jam     à-bax\\
      water  \oldstylenums{3}\textsc{s}+{\PFV}-pour\\
\glt  ‘Water poured.’
\z

If the verb is Imperfective, the clause means that the door is able to be opened, i.e., it is not locked \REF{ex:9:30}.

\ea \label{ex:9:30}
Mahay  ápay.\\
\gll  mahaj   á-p-aj\\
      door  \oldstylenums{3}\textsc{s}+{\IFV}-open{}-{\CL}\\
\glt  ‘The door opens.’
\z

In the Perfect, the clause means that the door is open (i.e., someone has already opened it, \ref{ex:9:31}).

\begin{footnotesize}
\begin{landscape}
\begin{longtable}{lll}%\footnotesize
% \resizebox{\textwidth}{!}{\begin{tabular}{lll}
\caption{Group 4 Intransitive clauses\label{tab:74}}\\\lsptoprule
%resizetable
{Perfective} & {Imperfective\is{Tense, mood, and aspect!Imperfective aspect|)}} & {Perfect\is{Tense, mood, and aspect!Perfect|)}}\\\midrule\endfirsthead
\midrule {Perfective} & {Imperfective} & {Perfect}\\\midrule\endhead
\lspbottomrule\endlastfoot	 
\multicolumn{3}{c}{\textit{zom} ‘eat’}\\\midrule
\textit{Mana }  \textit{a-zom} & \textit{Mana}  \textit{á-zom} & \textit{Mana}  \textit{a-zəm=va}\\
Mana    \oldstylenums{3}\textsc{s}+{\PFV}-eat & Mana \oldstylenums{3}\textsc{s}+{\IFV}-eat & Mana    \oldstylenums{3}\textsc{s}+{\PFV}-eat={\PRF} \\
‘Mana ate [something].’ & ‘Mana is about to eat [something].’ & ‘Mana ate [something] already.’\\\cmidrule(lr){1-3}
& \textit{háy}  \textit{á-zom} & \textit{háy}  \textit{á-zəm=va} \\
& millet   \oldstylenums{3}\textsc{s}+{\IFV}-eat & millet    \oldstylenums{3}\textsc{s}+{\IFV}-eat={\PRF}\\
& ‘There are insects in the millet.’ & ‘The millet has been eaten.’ \\
& (lit. millet eats) &\\\midrule
\multicolumn{3}{c}{\textit{sl-ay} ‘slaughter’}\\\midrule
\textit{Mana} \textit{a-sl-ay} & \textit{Mana}  \textit{á-sl-ay} & \textit{Mana}  \textit{a-sla}  \textit{ =va}\\
 Mana    \oldstylenums{3}\textsc{s}+{\PFV}-slay-{\CL} & Mana   \oldstylenums{3}\textsc{s}+{\IFV}-slay-{\CL} & Mana   \oldstylenums{3}\textsc{s}+{\PFV}-slay={\PRF}\\
‘Mana slaughtered [something].’ & ‘Mana is about to slaughter [something].’ & ‘Mana has slaughtered [something].’\\\cmidrule(lr){1-3}
& \textit{awak}  \textit{á-sl-ay } & \textit{awak}  \textit{a-sla}\textit{=va}\\
 & goat   \oldstylenums{3}\textsc{s}+{\IFV}-slay-{\CL} & goat    \oldstylenums{3}\textsc{s}+{\PFV}-slay={\PRF} \\
 & ‘The goat is good for slaughtering.’ & ‘The goat has been slaughtered.’ \\\midrule
\multicolumn{3}{c}{\textit{s{}-e} ‘drink’}\\\midrule
 Mana   \textit{e-s{}-e} & Mana   \textit{\'e{}-s{}-e} \\
 Mana    \oldstylenums{3}\textsc{s}+{\PFV}-drink-{\CL} & Mana    \oldstylenums{3}\textsc{s}+{\IFV}-drink-{\CL} \\
 ‘Mana drank [something].’ & ‘Mana is about to drink [something].’ \\\cmidrule(lr){1-3}\newpage
& \textit{yam}  \textit{\'e{}-s{}-e } & \textit{yam }  \textit{a-sə=va} \\
& water    \oldstylenums{3}\textsc{s}+{\IFV}-drink-{\CL} & water   \oldstylenums{3}\textsc{s}+{\PFV}-drink={\PRF}\\
& ‘The water is drinkable.’ (lit. water drinks).’ & ‘The water has been drunk.’\\\midrule
\multicolumn{3}{c}{\textit{bal-aj} ‘wash’}\\\midrule
\textit{Hawa}  \textit{a-bal-ay} & \textit{Hawa}  \textit{á-bal-ay} & \textit{Hawa}  \textit{a-bal  =va}\\
Hawa  \oldstylenums{3}\textsc{s}+{\PFV}-wash-{\CL} & Hawa  \oldstylenums{3}\textsc{s}+{\IFV}-wash-{\CL} & Hawa    \oldstylenums{3}\textsc{s}+{\PFV}-wash  ={\PRF} \\
‘Hawa washed [herself].’ & ‘Hawa washes [herself].’ & ‘Hawa is washed.’\\\cmidrule(lr){1-3}
& \textit{zana}  \textit{á-bal-ay} & \textit{zana}  \textit{a-bal=va}\\
& cloth \oldstylenums{3}\textsc{s}+{\IFV}-wash-{\CL} & cloth \oldstylenums{3}\textsc{s}+{\PFV}-wash={\PRF}\\
& ‘The cloth can be washed.’  & ‘The cloth is clean.’ (washed) \\
& (lit. the cloth washes) & \\\midrule
\multicolumn{3}{c}{\textit{p-ay} ‘open’}\\\midrule
\textit{mahaj}  \textit{a-p-ay} & \textit{mahay }  \textit{á-p-ay} & \textit{mahay   a-p=va}\\
door    \oldstylenums{3}\textsc{s}+{\PFV}-open-{\CL} & door    \oldstylenums{3}\textsc{s}+{\IFV}-open-{\CL} & door  \oldstylenums{3}\textsc{s}-open={\PRF} \\
‘The door opened.’ & ‘The door opens.’ (is able to open) & ‘The door is open.’\\\midrule

\multicolumn{3}{c}{\textit{bax} ‘pour’}\\\midrule
\textit{yam}  \textit{a-bah} & \textit{yam}  \textit{á-bah} & \textit{yam }  \textit{a-bah=va}\\
 water    \oldstylenums{3}\textsc{s}+{\PFV}-pour & water    \oldstylenums{3}\textsc{s}+{\IFV}-pour & water   \oldstylenums{3}\textsc{s}-pour={\PRF}\\
‘Water poured.’ & ‘Water is able to be poured.’ & ‘Water is poured out.’\\
                 &  (lit. water pours)           & \\\midrule
\multicolumn{3}{c}{\textit{mbɪʒɛŋ} ‘ruin’ }\\\midrule
\textit{háy}  \textit{e-mbəzen} & \textit{háy}  \textit{á-mbəzen} & \textit{háy}  \textit{á-mbəzə=va}\\
millet   \oldstylenums{3}\textsc{s}+{\PFV}-ruin & millet    \oldstylenums{3}\textsc{s}+{\IFV}-ruin & millet    \oldstylenums{3}\textsc{s}+{\IFV}-ruin={\PRF}\\
‘The millet ruined.’ & ‘The millet is ruining.’ & ‘The millet has ruined.’\\
% \end{tabular}}
\end{longtable}
\end{landscape}
\end{footnotesize}


\ea \label{ex:9:31}
Mahay  apava.\\
\gll  mahaj   a-pa=va\\
      door  \oldstylenums{3}\textsc{s}-open={\PRF}\\
\glt  ‘The door is open.’
\z

Imperfective %%\is{Tense, mood, and aspect!Imperfective aspect} 
aspect in an intransitive clause presents a situation where a state or capability is expressed. For the verb \textit{mənjar} ‘see,’ an intransitive clause in Imperfective aspect \REF{ex:9:32} can have an abilitative sense in that the subject \textit{Mala} is able to see. It can also mean that the subject is visible (subject expresses semantic Theme).

\ea \label{ex:9:32}
Mala  ámənjar. \\
\gll  Mala  á-mənzar\\
      Mala  \oldstylenums{3}\textsc{s}+{\IFV}-see\\
\glt  ‘Mala sees.’ (i.e. he is not blind) / ‘Mala can be seen.’ 
\z

\largerpage
\tabref{tab:74} presents examples of Group 4 verbs in intransitive clauses.\is{Attribution!Expressed using verb} The corresponding transitive forms for most of these verbs are discussed in \sectref{sec:9.2.4.1}. The three columns show Perfective, Imperfective, and Perfect forms of the verbs. Perfective aspect\is{Tense, mood, and aspect!Perfect} (column 1) expresses either an\is{Tense, mood, and aspect!Perfective aspect} action that the Agent did (with an unexpressed Theme) or an event that happened to the Theme (with an unexpressed Agent). Imperfective aspect (column 2) indicates readiness of the Agent to do the action or expresses ability of the Theme to submit to the action. The Perfect (column 3) expresses a resultative -- a finished action or the state resulting from the event. For some verbs, the subject can express either Agent or Theme (\textit{zom, slay, se, balay, pay}). For others, the subject of an intransitive clause can only express Theme (\textit{bah, mbəzen}).


\subsection{Group 5: Transfer verbs}\label{sec:9.2.5}
\hypertarget{RefHeading1212741525720847}{}
Three transfer verbs in Moloko are notable. They are \textit{dəbənay} ‘learn/teach,’ \textit{skom} ‘buy/sell,’ and \textit{vəl}  ‘give.’ These verbs are especially labile in terms of their semantic expression in that a transitive clause can have \textit{either} a direct or an indirect object. 

The verb \textit{vəl}  ‘give’ is shown in a bitransitive clause in \REF{ex:9:33}. The subject (\textit{bahay} ‘chief’) transfers the direct object (\textit{dalay}  ‘girl’) to the indirect object (\textit{Mana}).

\ea \label{ex:9:33}
Bahay  avəlan  dalay  ana  Mana.\\
\gll  bahaj   à-vəl=aŋ     dalaj   ana   Mana\\
      chief  \oldstylenums{3}\textsc{s}+{\PFV}-give=\oldstylenums{3}\textsc{s}.{\IO}  girl  {\DAT} Mana\\
\glt  ‘The chief gave the girl to Mana (in marriage).’ 
\z

When \textit{vəl}  ‘give’ occurs in a transitive clause, the second core argument can be either a direct object \REF{ex:9:34} or an indirect object \REF{ex:9:35}.  In \REF{ex:9:34}, the chief is marrying off his daughter to an unspecified suitor. The subject (\textit{bahay}  ‘chief’) transfers the direct object (\textit{dalay}  ‘girl’) to someone who is unspecified in the clause. 

\ea \label{ex:9:34}
Bahay  ávar  dalay.\\
\gll  bahaj    á-var      dalaj\\
      chief  \oldstylenums{3}\textsc{s}+{\IFV}-give  girl\\
\glt   ‘The chief is marrying off his daughter [to someone].’ (lit. chief gives girl) 
\z

In \REF{ex:9:35}, the subject (\textit{bahay}  ‘chief’) transfers something or someone to the indirect object (\textit{Mana}). What he gave would probably be specified in the immediate context, but is out of sight in this clause.

\ea \label{ex:9:35}
Bahay  avəlan  ana  Mana.\\
\gll  bahaj   à-vəl=aŋ   ana   Mana\\
      chief  \oldstylenums{3}\textsc{s}+{\PFV}-give=\oldstylenums{3}\textsc{s}.{\IO}  {\DAT} Mana\\
\glt  ‘The chief gave [something] to Mana.’
\z

When the verb \textit{vəl}  ‘give’ occurs in an intransitive negative clause (Imperfective\is{Tense, mood, and aspect!Imperfective aspect}, \ref{ex:9:36}), it expresses that the subject is in the state of not giving anything to anyone, or not being the giving kind.\footnote{Note the phonological change of the final consonant (\textit{r} becomes \textit{l} when there is a suffix, see \sectref{sec:6.2}). } Without the negative marker, the meaning would probably be ‘the chief is the giving kind.’\footnote{This is a specific example from a text. We have not seen one-participant clauses for this verb type in Perfective aspect. %%\is{Tense, mood, and aspect!Perfective aspect}
The semantics of one-participant clauses for group four verbs is discussed in \sectref{sec:9.2.4.2}.}

\ea \label{ex:9:36}
Bahay  ávar  bay.\\
\gll  bahaj  á-var       baj\\
      chief  \oldstylenums{3}\textsc{s}+{\IFV}-give  {\NEG}\\
\glt  ‘The chief is not the giving kind.’ (lit. chief doesn’t give) 
\z

The verb \textit{dəbənay} ‘learn’/‘teach’ occurs in transitive and bitransitive clauses.\footnote{We found no clauses with one core participant for this verb. } In bitransitive clauses illustrated by \REF{ex:9:37}, the subject (\textit{bahay}  ‘chief’) transfers the direct object (\textit{Məloko} ‘Moloko language’) to the indirect object (\textit{ana babəza ahay} ‘to the children’).\footnote{The indirect object is expressed in an adpositional phrase as well as the verbal pronominal extension \textit{=ata} ‘to them.’ The indirect object expresses the recipient or beneficiary of the event.} 

\ea \label{ex:9:37}
Bahay  adəbənata  Məloko  ana  babəza  ahay. \\
\gll  bahaj a-dəbən=ata   Mʊlɔkʷɔ    ana   babəza=ahaj \\
      chief  \oldstylenums{3}\textsc{s}-learn=\oldstylenums{3}\textsc{p}.{\IO}  Moloko    {\DAT} children=Pl\\
\glt  ‘The chief teaches Moloko to the children.’ 
\z

In transitive clauses with subject and direct object \REF{ex:9:38}, the subject (\textit{babəza ahay} ‘children’) transfers the direct object (\textit{Məloko} ‘Moloko language’) to self. 

\ea \label{ex:9:38}
Babəza  ahay  tədəbənay  Məloko.\\
\gll  babəza=ahaj   tə-dəbən-aj   Mʊlɔkʷɔ\\
      children=Pl  \oldstylenums{3}\textsc{p}-learn{}-{\CL}  Moloko \\
\glt  ‘The children learn Moloko.’ 
\z

\REF{ex:9:39} illustrates a transitive clause with subject and indirect object.  The subject (\textit{Məloko} ‘Moloko language;’ the semantic Theme) is transferred to the indirect object (\textit{=ok}  ‘to you’).  

\ea \label{ex:9:39}
Məloko  adəbənok  na  jajak.\\
\gll  Mʊlɔkʷɔ  a-dəbən=ɔkʷ  na  dzadzak\\
      Moloko  \oldstylenums{3}\textsc{s}-learn={\twoS}.{\IO}  \textsc{{\PSP}}  fast\\
\glt  ‘Moloko is easy for you to learn.’ (lit. Moloko learns to you quickly)
\z

The verb \textit{skom} ‘buy’/‘sell’ is also a transfer verb with two semantic {\scshape loc}s. The event of buy/sell is accomplished through transfer of the Theme from one \LOC to another. In a bitransitive clause \REF{ex:9:40}, the subject (\textit{nə-}  ‘I’) causes the direct object (\textit{awak} ‘goat’) to go to the indirect object (\textit{ana Mana} ‘to Mana’). 

\ea \label{ex:9:40}
Nəskoman  awak  ana  Mana.\\
\gll  nə-sʊkʷɔm=aŋ     awak   ana   Mana\\
      {\oneS}-buy/sell=\oldstylenums{3}\textsc{s}.{\IO}    goat  {\DAT} Mana\\
\glt   ‘I sell a goat to Mana.’ 
\z

In a transitive clause with direct object \REF{ex:9:41}, the subject (\textit{nə-} ‘I’) transfers the direct object (\textit{awak} ‘goat’) to self. We found no intransitive clauses for this verb.

\ea \label{ex:9:41}
Nəskomala  awak.\\
\gll  nə-sʊkʷɔm=ala   awak\\
      {\oneS}-buy/sell=to  goat\\
\glt  ‘I bought a goat.’ 
\z

The verb \textit{hay} ‘speak’ also appears to be in this class, but we have not found this verb in all contexts. In \REF{ex:9:42}, Mana caused what he said (\textit{na} ‘it’) to go to the men.  

\ea \label{ex:9:42}
Mana  ahata  na  va  ana  zawər  ahay.\\
\gll  Mana   à-h=ata   na=va   ana   zawər=ahaj\\
      Mana  \oldstylenums{3}\textsc{s}+{\PFV}-speak=\oldstylenums{3}\textsc{p}.{\IO}  \oldstylenums{3}\textsc{s}.{\DO}={\PRF}  {\DAT} men=Pl\\
\glt  ‘Mana has already told it to the men.’
\z

\tabref{tab:75} presents examples of these transfer verbs in intransitive, transitive, and bitransitive clauses.

\begin{sidewaystable}\scriptsize
\begin{tabular}{llll}
\lsptoprule
%sideways table?
{Intransitive} & {Transitive with direct object} & {Transitive with indirect object} & {Bitransitive}\\\midrule
\textit{Hawa  á-var}  \textit{bay} & \textit{Hawa}  \textit{á-var}     \textit{yam} & \textit{Hawa à-vəl=an ana Mana} & \textit{Hawa à-vəl=an              yam    ana   Mana}\\
Hawa  \oldstylenums{3}\textsc{s}+{\IFV}-give  \NEG & Hawa   \oldstylenums{3}\textsc{s}+{\IFV}-give  water & Hawa \oldstylenums{3}\textsc{s}+{\PFV}-give=\oldstylenums{3}\textsc{s}.{\IO} {\DAT} Mana & Hawa \oldstylenums{3}\textsc{s}+{\PFV}-give=\oldstylenums{3}\textsc{s}.{\IO} water {\DAT} Mana\\
‘Hawa is not the giving kind.’  & ‘Hawa gives water [to someone].’ & ‘Hawa gave [something] to Mana.’ & ‘Hawa gave water to Mana.’\\                         
(lit. Hawa doesn’t give) & & &\\
\\
\textit{á-var} \textit{bay} & \textit{á-var} \textit{na} & \textit{à-vəl=an} & \textit{à-vəl=an} \textit{na}\\
\oldstylenums{3}\textsc{s}+{\IFV}-give   \NEG & \oldstylenums{3}\textsc{s}+{\IFV}-give  \oldstylenums{3}\textsc{s}.{\DO} & \oldstylenums{3}\textsc{s}+{\PFV}-give=\oldstylenums{3}\textsc{s}.{\IO} & \oldstylenums{3}\textsc{s}+{\PFV}-give=\oldstylenums{3}\textsc{s}.{\IO}    \oldstylenums{3}\textsc{s}.{\DO}\\
‘She is not the giving kind.’ & ‘She gives it [to someone].’ & ‘She gave [something] to him.’ & ‘She gave it to him.’ \\\midrule
& \textit{babəza=ahay tə-dəbən-ay Məloko} & \textit{Məloko a-dəbən=ok} \textit{na   jajak}  & \textit{bahay    a-dəbən=ata}  \\
& children=Pl \oldstylenums{3}\textsc{p}-learn{}-{\CL} Moloko & Moloko  \oldstylenums{3}\textsc{s}-learn={\twoS}.{\IO} \textsc{{\PSP} } fast & chief   \oldstylenums{3}\textsc{s}-learn =\oldstylenums{3}\textsc{p}.{\IO}  Moloko \\
& ‘The children learn Moloko.’ & ‘Moloko is easy for you to learn.’\\    
& & (lit.Moloko learns to you quickly) & \textit{ana }  \textit{babəza=ahaj}\\
& &  & {\DAT}  children =Pl\\
& & & ‘The chief teaches Moloko to the children.’\\\midrule
& \textit{nə-skom=ala    awak} & & \textit{nə-skom=an        awak ana   Mana}\\
& {\oneS}-buy/sell=to    goat &  & {\oneS}-buy/sell=\oldstylenums{3}\textsc{s}.{\IO}  goat   {\DAT} Mana\\
& ‘I bought a goat.’  & &  ‘I sell a goat to Mana.’\\\midrule
\textit{Mana   a-h-ay        bay} &  &  & \textit{Hawa }    \textit{a-h=an }      \textit{ma     ana   Mana}\\
Mana   \oldstylenums{3}\textsc{s}-tell-{\CL}  \NEG & &  & Hawa      \oldstylenums{3}\textsc{s}-tell=\oldstylenums{3}\textsc{s}.{\IO} mouth {\DAT} Mana\\
‘Mana doesn’t say.’ & & & ‘Hawa greets Mana.’\\
\lspbottomrule
\end{tabular}
\caption{Group 5 verbs\label{tab:75}}
\end{sidewaystable}

A fourth participant is possible for the verb \textit{vəl}  ‘give’ and appears as an oblique adjunct. When there is both a Beneficiary and a Recipient (which is the core \LOC), a preposition (\textit{kəla}) plus one of the possessive pronouns (see \sectref{sec:3.1.2}) mark the benefactive. In \REF{ex:9:43} the subject (‘you,’ {\twoS} imperative verb) transfers the direct object (\textit{dala} ‘money’) to the indirect object (\textit{=an} ‘to him' and \textit{ana Mana} ‘to Mana’) for the benefit of the person expressed by a possessive pronoun in the oblique prepositional phrase (\textit{kəla} \textit{əwla} ‘my benefit,’ bolded in the examples).  


\ea \label{ex:9:43}
Vəl\textbf{an}  dala  \textbf{kəla} \textbf{əwla}  ana  Mala.\\
\gll  vəl\textbf{=aŋ}  dala  \textbf{kəla}\textbf{=uwla}    ana  Mala\\
      give=\oldstylenums{3}\textsc{s}.{\IO}  money  {for (benefactive)}={\oneS}.{\POSS}  {\DAT} Mala\\
\glt  ‘Give Mala the money for me (lit. my benefit).’
\z

In \REF{ex:9:44} the subject pronominal (\textit{a-} ‘\oldstylenums{3}\textsc{s}’) transfers the direct object (\textit{awak} ‘goat’) to the indirect object (pronominal enclitic \textit{=ok} ‘to you’) for the benefit of the pronoun in the oblique (\textit{kəla =əwla} ‘my benefit’).

\ea \label{ex:9:44}
Avəl\textbf{ok}  awak  \textbf{kəla} \textbf{əwla}.\\
\gll  a-vəl\textbf{=ɔkʷ}    awak  \textbf{kəla}\textbf{=uwla}\\
      \oldstylenums{3}\textsc{s}-give={\twoS}.{\IO}  goat  {for (benefactive)}={\oneS}.{\POSS} \\
\glt  ‘He/she gave you the goat on my behalf (lit. my benefit).’
\z

\section{“Body-part” verbs (noun incorporation)}\label{sec:9.3}\is{Noun incorporation|(}
\hypertarget{RefHeading1212761525720847}{}
\citet{FriesenMamalis2008} identified a unique group of verb constructions in Moloko. In these constructions, a special, sometimes phonologically reduced noun form that represents a part of the body is incorporated into the verb phrase.  This is a case of noun incorporation where these \textit{body-part} nouns are closely associated with the verb complex and their incorporation changes the lexical characteristics of the verb. These body-part nouns include \textit{ma} ‘mouth,’ (\ref{ex:9:45}, \sectref{sec:9.3.1.3}), \textit{elé} ‘eye,’ (\ref{ex:9:46}, \sectref{sec:9.3.1.1}), \textit{sləmay} ‘ear,’ (\ref{ex:9:47}, \sectref{sec:9.3.1.2}), and \textit{va} or \textit{har} ‘body,’ (\ref{ex:9:48}, \ref{ex:9:49}, Sections~\ref{sec:9.3.1.4} and \ref{sec:9.3.1.5}, respectively). These nouns can be incorporated into transitive or bitransitive verbs from the types in Sections~\ref{sec:9.2.2} and \ref{sec:9.2.3}.

\ea \label{ex:9:45}
Ataraŋ  aka  \textbf{ma}  ana  war  ese.\\
\gll  a-tar=aŋ     =aka   \textbf{ma}  ana  war  ɛʃɛ\\
      \oldstylenums{3}\textsc{s}-call=\oldstylenums{3}\textsc{s}.{\IO}  =on  mouth  {\DAT} child  again\\
\glt  ‘He/she calls the child again.’ (lit. he calls mouth to him to the child again)
\z

\ea \label{ex:9:46}
Mala  amənjar  \textbf{elé.}\\
\gll  Mala   a-mənzar   \textbf{ɛlɛ}\\
      Mala   \oldstylenums{3}\textsc{s}-see    eye\\
\glt  ‘Mala looks around attentively.’ 
\z

\ea \label{ex:9:47}
Acaka  va  \textbf{sləmay}  ana  mama  ahan  bay.\\
\gll  a-ts=aka=va  \textbf{ɬəmaj}   ana  mama=ahaŋ    baj\\
      \oldstylenums{3}\textsc{s}-hear=on={\PRF}  ear  {\DAT} mother=\oldstylenums{3}\textsc{s}.{\POSS}  {\NEG}\\
\glt  ‘He/she is disobedient to his mother.’ (he disobeys his mother)\footnote{Note that the word-final /n/ is deleted on the root /ts n\textsuperscript{e}/when the verbal extension is attached \sectref{sec:2.6.1}.} 
\z

\ea \label{ex:9:48}
Tandalay  talala  təzləge  \textbf{va} ana  Məloko  ahay.\\
\gll  ta-ndalaj ta-l=ala  tɪ-ɮɪg-ɛ   \textbf{va}  ana  Mʊlɔkʷɔ=ahaj\\
      \oldstylenums{3}\textsc{p}-{\PRG}   \oldstylenums{3}\textsc{p}-go=to  \oldstylenums{3}\textsc{p}-throw-{\CL} body  {\DAT} Moloko=Pl\\
\glt  ‘They were coming and fighting with the Molokos.’ (lit. they were coming they threw body to Molokos)
\z

\ea \label{ex:9:49}
Ma  ango  agəsaw  \textbf{har}.\\
\gll  ma=aŋgʷɔ    a-gəs=aw  \textbf{har}\\
      word={\twoS}.{\POSS}  \oldstylenums{3}\textsc{s}-catch={\oneS}.{\IO}  body\\
\glt  ‘It pleases me.’ (lit. it catches body to me)
\z

The body-part noun follows directly after all other elements in the verb complex. It appears to be in the same position as any other noun phrase direct object in the verb phrase (see \chapref{chap:8}); however it is in more tightly bound to the verb complex than a noun phrase. The body-part noun does not fill the \DO pronominal slot, because verbal extensions that follow the \DO pronominal in the Moloko verb complex precede the body-part (see \ref{ex:9:45} and \ref{ex:9:47} which each have an adpositional extension, see \sectref{sec:7.5.1}). It is not phonologically bound to the verb since, unlike the Perfect verbal extension \textit{=va} which is part of the verb complex, the body-part \textit{va} does not neutralise the prosody on the verb stem \REF{ex:9:48}. However, the incorporated noun is grammatically closer to the verb complex than a noun phrase direct object would be because the body-part can never be separated from the verb complex. The body-part can never be fronted in the clause (see \sectref{sec:8.1}). Nor can the body-part be separated from the verb complex by the presupposition marker. Both of these situations can occur for noun phrase direct objects and are illustrated in \sectref{sec:11.2} (\ref{ex:9:29} and \ref{ex:9:30}). 

Incorporation of the body-part noun never co-occurs with another direct object or with the \DO pronominal \textit{na}. A transitive clause with subject, indirect object and incorporated body-part noun can occur where the indirect object expresses semantic \LOC (sometimes metaphorical).

This section is organised by body-part plus verb collocations:

\begin{itemize}
\item \textit{elé} ‘eye’ (\sectref{sec:9.3.1}). Used with verbs of seeing. 
\item \textit{sləmay} ‘ear’ (\sectref{sec:9.3.2}). Collocates with verbs of cognition. 
\item \textit{ma} ‘mouth’ (\sectref{sec:9.3.3}). \textit{Ma} also can mean ‘word’ or ‘language.’ Used with verbs of speaking. 
\item \textit{va} ‘body’ (\sectref{sec:9.3.4}). \textit{Va} is phonologically reduced from \textit{hərva} ‘body.’ Used to form reciprocal actions. 
\item \textit{har} ‘body’ (\sectref{sec:9.3.5}). \textit{Har} is also phonologically reduced from \textit{hərva} ‘body.’ 
\end{itemize}

Note that there are Moloko idioms that employ body parts with the verb \textit{g-e} ‘do.' To get angry is to ‘do heart’ \REF{ex:9:50}. 

\ea \label{ex:9:50}
Ege  ɓərav.\\
\gll  ɛ{}-g-ɛ   ɓərav\\
      \oldstylenums{3}\textsc{s}-do-{\CL}  heart\\
\glt  ‘He/she is angry.’ (lit. he/she does heart)
\z

The idiom for ‘think’ is literally ‘do brain’ \REF{ex:9:51}. 

\ea \label{ex:9:51}
Ge  endeɓ!\\
\gll  g-ɛ       ɛndɛɓ\\
      do[{\twoS}.{\IMP}]-{\CL}  brain\\
\glt  ‘Think!’ (lit. do brain)
\z

\subsection{\textit{elé}  ‘eye’}\label{sec:9.3.1}\label{sec:9.3.1.1}

The body-part noun \textit{elé}  ‘eye’ collocates with some verbs to lexicalise the engagement of the eyes and reduce the focus on what is seen.  This body-part word is used in its full form. For example, the verb \textit{mənjar} normally means ‘see’ (see \tabref{tab:76}). With the incorporation of \textit{elé} (\ref{ex:9:52}-- \ref{ex:9:53}), the verb plus body-part construction has a more active experiential meaning in that the subject of the clause (\textit{Mala}) is looking around attentively. Since there can be no direct object, there is no explicit referential object as stimulus -- the speaker is vague about what exactly Mala will look at. 

\ea \label{ex:9:52}
Mala  amənjar  \textbf{elé.}\\
\gll  Mala   a-mənzar   \textbf{ɛlɛ}\\
      Mala   \oldstylenums{3}\textsc{s}-see    eye\\
\glt  ‘Mala looks around attentively.’ 
\z

\ea \label{ex:9:53}
Mala  olo  aməmənzəre  \textbf{elé}  a  ləhe.\\
\gll  Mala  ɔ{}-lɔ  amɪ-mɪnʒɪrɛ  \textbf{ɛlɛ}  a  lɪhɛ\\
      Mala  \oldstylenums{3}\textsc{s}-go  {\DEP}-see    eye  at  bush\\
\glt  ‘Mala went to see his fields.’ (lit. Mala went to see in the bush)
\z

With the verb \textit{har} ‘carry’ \REF{ex:9:54}, the addition of \textit{elé} also gives an entirely new lexical item -- expressing the idea of looking around intensively or studying every square inch (see \tabref{tab:76}.). 

\ea \label{ex:9:54}
Nolo  nahar  \textbf{elé}  a  gəvah  əwla  ava  jəyga.\\
\gll  nɔ-lɔ  na-har     \textbf{ɛlɛ}   a   gəvax=uwla    ava  dzijga\\
      {\oneS}-go  {\oneS}-carry    eye  at  field={\oneS}.{\POSS}  in  all\\
\glt  ‘I go [and] look around my whole field.’ (lit. I carry eye in my field  all)
\z

\tabref{tab:76} compares examples with and without the body-part.

\begin{table}
\begin{tabular}{ll}
\lsptoprule
{Clause without body-part} & {Clause with body-part}\\\midrule
\textit{Mana }  \textit{a-mənjar  }  \textit{war}  & \textit{a-mənjar  }  \textbf{\textit{elé}}\\
Mana    \oldstylenums{3}\textsc{s}-see      child & \oldstylenums{3}\textsc{s}-see      eye \\
‘Mana sees the child.’  & ‘He/she looks around intently.’\\\midrule
\textit{Mana }  \textit{a-har }   \textit{eteme }  \textit{a }  \textit{dəray }  \textit{ava} & \textit{ka-har=aka }  \textbf{\textit{elé}}\textit{  a   }\textit{gəvah=ango}     \textit{ava }\textit{jəyga}\\
Mana   \oldstylenums{3}\textsc{s}-carry  onion    in    head    in & {\twoS}-carry=on    eye   at field={\twoS}.{\POSS}  in   all\\
‘Mana carries onions on [his] head.’ & ‘You look around your whole field.’ \\
\lspbottomrule
\end{tabular}
\caption{Selected verbs with and without the incorporation of elé ‘eye’\label{tab:76}}
\end{table}

\subsection{\textit{sləmay} ‘ear’}\label{sec:9.3.2}\label{sec:9.3.1.2}

A second body-part noun is \textit{sləmay} ‘ear’ which collocates with some cognition verbs.  This body-part noun is used in its full form. Like \textit{elé}  ‘eye,’ it adds a new, more active lexical meaning to the verb with which it collocates. 

For example, the normal lexical meaning of the verb \textit{cen } is ‘hear’ or ‘understand’ \REF{ex:9:55} and the verb is bitransitive (see \sectref{sec:9.2.4}). The incorporation of the body-part \textit{sləmay} ‘ear’ gives a much more active or intensive idea -- not just hear and understand someone, but also listen to them or obey them \REF{ex:9:56}. The focus is on the fact that the person is benefitting from using his ears to intently listen, rather than on the person speaking or the content of their message. 

\ea \label{ex:9:55}
Mana  écen  bay.\\
\gll  Mana   \'{ɛ}-tʃɛŋ     baj\\
      Mana  \oldstylenums{3}\textsc{s}+{\IFV}-hear  {\NEG}\\
\glt  ‘Mana is deaf/doesn’t understand.’ 
\z

\ea \label{ex:9:56}
Mana  écen  \textbf{sləmay}  bay.\\
\gll  Mana  \'{ɛ}-tʃɛŋ     \textbf{ɬəmaj}   baj\\
      Mana  \oldstylenums{3}\textsc{s}+{\IFV}-hear  ear  {\NEG}\\
\glt  ‘Mana is deaf/disobedient.’
\z

Examples are in \tabref{tab:77}.

\begin{table}
\resizebox{\textwidth}{!}{\begin{tabular}{ll}
\lsptoprule
{Clause without ‘body-part’} & {Clause with body-part}\\\midrule
\textit{Mana  a-c=aw =aka ma=əwla bay} & \textit{Mana a-c=aka=va} \textbf{\textit{sləmay}} \textit{ana    mama=ahan       bay}\\
Mana   \oldstylenums{3}\textsc{s}-hear={\oneS}.{\IO} =on  word/mouth={\oneS}.{\POSS}  \NEG & Mana \oldstylenums{3}\textsc{s}-hear=on{=\PRF} ear {\DAT} mother=\oldstylenums{3}\textsc{s}.{\POSS} \NEG\\
‘Mana didn’t understand my words.’ & ‘Mana is disobedient to his mother.’ \\
     & (lit. Mana doesn’t hear ear to his mother)\\
\lspbottomrule
\end{tabular}}
\caption{Selected verbs of cognition with and without  incorporation of sləmay ‘ear’ \label{tab:77}}
\end{table}

\subsection{\textit{ma} ‘mouth’}\label{sec:9.3.3}\label{sec:9.3.1.3}

The ‘body-part’ noun \textit{ma} ‘mouth’ (which also means ‘word’ and ‘language’) collocates with some speech verbs. It is found in its full form in the verb plus body-part constructions. Example \REF{ex:9:57} shows the verb \textit{hay} ‘say’ with the body-part noun \textit{ma} ‘mouth.’ 

\ea \label{ex:9:57}
Tahok  ma.\\
\gll  ta-h=ɔkʷ    ma\\
      \oldstylenums{3}\textsc{p}-tell={\twoS}.{\IO}    mouth\\
\glt  ‘You are being greeted.’ (lit. they are telling word to you) 
\z

The example pairs shown in \tabref{tab:78} illustrate its use with three speaking verbs; \textit{taray} ‘call,’ \textit{hay} ‘say’ and \textit{jay} ‘speak.’ Examples are shown with the direct object pronominal \textit{na} (column 1) and with \textit{ma} ‘mouth’ (column 2). With the body-part incorporation, there can be no other direct object. 

\begin{table}
\begin{tabular}{ll}
\lsptoprule
{Transitive clause} & {Clause with ‘body-part’ incorporation}\\\midrule
\textit{Mana }  \textit{a-tar-ay} & \textit{Mana }  \textit{a-tar=an }  \textbf{\textit{ma}} \textit{ana} Hawa \\
Mana   \oldstylenums{3}\textsc{s}-call-{\CL} & Mana    \oldstylenums{3}\textsc{s}-call=\oldstylenums{3}\textsc{s}.{\IO}   mouth/word   {\DAT}  Hawa \\
‘Mana calls out.’ & ‘Mana calls to Hawa.’\\
\\
\textit{a-tar-ay} & \textit{a-tar=an ma} \\
\oldstylenums{3}\textsc{s}-call-{\CL} & \oldstylenums{3}\textsc{s}-call=\oldstylenums{3}\textsc{s}.{\IO} mouth/word \\
‘He calls out.’  & ‘He calls to her.’\\\midrule
\textit{Mana }  \textit{a-h-ay }    \textit{bay} & Mana   \textit{a-h=an }    \textbf{\textit{ma}}  \textit{ana} Hawa \\
Mana   \oldstylenums{3}\textsc{s}-tell-{\CL}   {\NEG} & Mana  \oldstylenums{3}\textsc{s}-tell=\oldstylenums{3}\textsc{s}.{\IO}   mouth/word  {\DAT} Hawa \\
‘Mana doesn’t say.’ & ‘Mana  greets Hawa.’\\
\\
\textit{a-h-ay }    \textit{bay} & \textit{a-h=an }    \textbf{\textit{ma}}\\
\oldstylenums{3}\textsc{s}-tell-{\CL}   {\NEG} & \oldstylenums{3}\textsc{s}-tell=\oldstylenums{3}\textsc{s}.{\IO}   mouth/word \\
‘He doesn’t say.’  & ‘He greets her.’\\\midrule
\textit{Mana }  \textit{a-j-ay}  & Mana    \textit{a-j-ay }      \textbf{\textit{ma}}\\
Mana   \oldstylenums{3}\textsc{s}+{\PFV}-speak-{\CL} & Mana    \oldstylenums{3}\textsc{s}+{\PFV}-speak-{\CL}    mouth/word\\
‘Mana speaks!’ & ‘Mana  greets.’\\
\\
\textit{a-j-ay} & \textit{a-j-ay }      \textbf{\textit{ma}}\\
\oldstylenums{3}\textsc{s}+{\PFV}-speak-{\CL} & \oldstylenums{3}\textsc{s}+{\PFV}-speak-{\CL}    mouth/word\\
‘He speaks!’ & ‘He greets.’\\
\lspbottomrule
\end{tabular}
\caption{Selected speech verbs with and without ma ‘mouth’ as direct object\label{tab:78}}
\end{table}

A similar creation of new lexical meaning occurs with verbs that are normally not speech verbs but that become speech verbs when they collocate with \textit{ma}. The verbs \textit{sok{}-oy} ‘point,’ \textit{zom} ‘eat,’ and \textit{njakay}  ‘find’ are shown in \tabref{tab:79}. The incorporation of \textit{ma} with \textit{sok-oy} ‘point’ gives a particular manner of communication: \textit{sok{}oy} \textit{ma} ‘whisper.’ Incorporation of \textit{ma} with the verb \textit{zom} ‘eat’ gives the idea of helping someone else to eat. Incorporation of \textit{ma} with \textit{njakay}  ‘find’ yields an expression 'to find trouble.' 


\clearpage 
\begin{table}[t]
\begin{tabular}{ll}
\lsptoprule
{Transitive clause} & {Clause with body part incorporation}\\\midrule
\textit{Hawa }  \textit{a-sok{}-oy }  \textit{ahar} & \textit{Hawa a-sok{}-oy }  \textbf{\textit{ma}}\\
Hawa  \oldstylenums{3}\textsc{s}-point-{\CL}  hand & Hawa  \oldstylenums{3}\textsc{s}-point-{\CL}  mouth/word\\
‘Hawa points.’\footnote{Perhaps \textit{ahar} ‘hand’ is another body-part direct object that acts as semantic Theme. We found no other verbs that collocate with \textit{ahar}. } & ‘Hawa whispers.’  \\\midrule
\textit{Hawa }  \textit{o{}-zom }  \textit{ɗaf} & \textit{Hawa }  \textit{a-zəm=an }  \textbf{\textit{ma}}  \textit{ana }    \textit{bahay}\\ 
Hawa   \oldstylenums{3}\textsc{s}-eat    {millet loaf} & Hawa   \oldstylenums{3}\textsc{s}-eat=\oldstylenums{3}\textsc{s}.{\IO}   mouth/word  {\DAT}    chief\\
‘Hawa eats millet loaf.’ & ‘Hawa fed the chief.’ (made him eat)\\
\\
\textit{o{}-zom }  \textit{na} & \textit{a-zəm=an }  \textbf{\textit{ma}}\\
\oldstylenums{3}\textsc{s}-eat    \oldstylenums{3}\textsc{s}.{\DO} & \oldstylenums{3}\textsc{s}-eat=\oldstylenums{3}\textsc{s}.{\IO}   mouth/word\\
‘She eats it.’ & ‘She fed him.’\\\midrule
\textit{Hawa }  \textit{a-njak-ay }  \textit{asak }\textit{=ahan} & \textit{Hawa }  \textit{a-njak-ay }  \textbf{\textit{ma}}\\
Hawa   \oldstylenums{3}\textsc{s}-find-{\CL}    foot=\oldstylenums{3}\textsc{s}.{\POSS} & Hawa   \oldstylenums{3}\textsc{s}-find-{\CL}    mouth/word\\
‘Hawa gives birth.’  & ‘Hawa is in trouble.’ \\
(lit. Hawa finds her feet)\footnote{Although \textit{asak} ‘foot’ is another body part, this is not a case of noun incorporation since \textit{asak} is a noun (in a possession construction with\textit{ =ahan}) and not within the verb complex as is \textit{ma} ‘mouth.’} & (lit. she finds mouth/word)\\
\\
\textit{a-njak-ay na} & \textit{a-njak-ay }  \textbf{\textit{ma}}\\
\oldstylenums{3}\textsc{s}-find-{\CL}    \oldstylenums{3}\textsc{s}.{\DO} & \oldstylenums{3}\textsc{s}-find-{\CL}    mouth/word\\
‘She finds it.’ & ‘Here comes trouble.’ \\
\lspbottomrule
\end{tabular}
\caption{Selected non-speech verbs that collocate with ma.\label{tab:79}}
\end{table}

\subsection{\textit{va} ‘body’}\label{sec:9.3.4}\label{sec:9.3.1.4}

There are two different phonologically reduced forms of the word \textit{hərva} ‘body’ -- \textit{va} and \textit{har}. When collocated with certain verbs, the verb plus incorporated body-part takes on a new lexical meaning. This is a non-productive process found with only a few verbs.   

\newpage 
The first reduced form of \textit{hərva} ‘body’ is \textit{va.}\footnote{Note that there are three homophones of \textit{va} which one must take care to distinguish:  [\textit{=va}] ‘perfect,’ [\textit{va}] ‘body,’ and [\textit{ava}] ‘in’.  They all can occur immediately following the verb stem.} This body-part is used for forming reciprocals with plural subjects of a few verbs in a context of killing and loving (\textit{zləge} ‘throw’ \ref{ex:9:58}--\ref{ex:9:59}, \textit{kaɗ}  ‘kill by clubbing’ \ref{ex:9:60}, and \textit{ndaɗay}  ‘need,’ \ref{ex:9:61}). The body-part \textit{va}  indicates that the plural subjects are performing the actions against one another. 

 
\ea \label{ex:9:58}
Tandalay  talala  təzləgə  \textbf{va}  ana  Məloko  ahay.\\
\gll  ta-nd=alaj ta-l =ala  tɪ-ɮɪg-ɛ    \textbf{va}  ana  Mʊlɔkʷɔ=ahaj\\
      \oldstylenums{3}\textsc{p}-{\PRG}=away   \oldstylenums{3}\textsc{p}-go =to  \oldstylenums{3}\textsc{p}-throw-{\CL}   body  {\DAT} Moloko=Pl\\
\glt  ‘They were coming and fighting with the Molokos.’ (lit. they were coming they threw body to Molokos)
\z

\ea \label{ex:9:59}
Kafta  məze  ahay  təzləgə  \textbf{va}  va  na,  nəwəɗokom  ala  dəray.\\
\gll  kafta  mɪʒɛ  =ahaj  tɪ-ɮɪgɪ    \textbf{va}  =va  na  nu-wuɗɔkʷ{}-ɔm     =ala\\  
      day       person  =Pl  \oldstylenums{3}\textsc{p}-throw      body  ={\PRF}  {\PSP}  {\oneS}-separate-\oldstylenums{1}\textsc{Pex} =to \\ 
      
      \medskip
\gll dəraj\\
     head\\
\glt  ‘On the day that they had finished fighting each other, we separated as equals.’
\z

\ea \label{ex:9:60}
Takaɗ  \textbf{va}.\\
\gll  ta-kaɗ   \textbf{va}\\
      \oldstylenums{3}\textsc{p}-kill  body\\
\glt  ‘They kill each other.’ (lit. they kill.by.clubbing body)
\z

The body-part \textit{va} ‘body’ occurs twice in the clause expressing the reciprocal idea of loving one another in \REF{ex:9:61} -- as incorporated noun and also as the noun phrase within an adpositional phrase (\textit{va} is bolded in the example).

\ea \label{ex:9:61}
Kondoɗom  \textbf{va}  a  \textbf{va}  ava.\\
\gll  kɔ-ndɔɗ-ɔm    \textbf{va}  a  \textbf{va}  ava\\
      {\twoP}-need-{\twoP}    body  at  body  in\\
\glt  ‘Love one another.’ (lit. need body in the body)
\z

\tabref{tab:80} compares transitive clauses with a direct object and clauses with the same verbs collocated with the body-part. To facilitate comparison between the incorporated body-part \textit{va}  and the direct object pronominal extension \textit{na}, the examples in the table are given in pairs. The first example in each pair shows the full noun phrase, and the second example in the pair shows the same clause with only pronominal affixes and extensions. The body-part \textit{va} is bolded.

\begin{table}
\resizebox{\textwidth}{!}{\begin{tabular}{ll}
\lsptoprule
{Transitive clause} & {Clause with body-part incorporation}\\\midrule
\textit{Məloko}   \textit{=ahay}  \textit{tə-zləg-e}  \textit{hay} & \textit{kəra=ahay tə-zləg-e} \textbf{\textit{va}}\\
Moloko     =Pl        \oldstylenums{3}\textsc{p}-sow-{\CL}    millet & dog=Pl    \oldstylenums{3}\textsc{p}-sow-{\CL}  body\\
‘Moloko people sow/throw millet.’ & ‘Dogs fight each other.’\\
\\
\textit{tə-zləg-e}  \textit{na} & \textit{tə-zləg-e} \textbf{\textit{va}}\\
\oldstylenums{3}\textsc{p}-sow-{\CL}    \oldstylenums{3}\textsc{s}.{\DO} & \oldstylenums{3}\textsc{p}-sow-{\CL}  body\\
‘They sow/throw it.’ & ‘They fight each other.’\\\midrule
\textit{babəza}\textit{=ahay}  \textit{ta-kaɗ} \textit{kəra} & \textit{məze=ahay  ta-kaɗ} \textbf{\textit{va}}\\
children=Pl    \oldstylenums{3}\textsc{p}-club    dog & person=Pl   \oldstylenums{3}\textsc{p}-club  body\\
‘The children kill a dog.’ & ‘The people kill each other.’\\
\\
\textit{ta-kaɗ} \textit{na} & \textit{ta-kaɗ} \textbf{\textit{va}}\\
\oldstylenums{3}\textsc{p}-club    \oldstylenums{3}\textsc{s}.{\DO} & \oldstylenums{3}\textsc{p}-club   body\\
‘They kill it.’ & ‘They kill each other.’\\\midrule
\textit{loko}  \textit{na  ko-ndoɗ-om}  \textit{baba=aloko} & \textit{loko} \textit{na }  \textit{ko-ndoɗ-om}  \textbf{\textit{va}}\\
{\onePin}   {\PSP}  \onePin-love-\onePin  father=\onePin.{\POSS} & {\onePin}   {\PSP}    \onePin–love-\onePin     body\\
‘We (for our part) love our father.’ & ‘We (for our part) love one another.’\\
\\
\textit{ko-ndoɗ-om}    \textit{na} & \textit{ko-ndoɗ-om}          \textbf{\textit{va}}\\
\onePin–love-\onePin    \oldstylenums{3}\textsc{s}.{\DO} & \onePin–love-\onePin    body\\
‘We love him.’ & ‘We love one another.’\\
\lspbottomrule
\end{tabular}}
\caption{Selected verbs with and without the body-part va ‘body’\label{tab:80}}
\end{table}

The verb \textit{zaɗ}  ‘take’ also can incorporate the body-part \textit{va}  ‘body.’ The normal lexical meaning of the verb \textit{zaɗ}  is ‘take’ but the combination \textit{zaɗ} \textit{va} (\ref{ex:9:62} and \ref{ex:9:63}) carries the idea of ‘resemble’ or ‘look like’ and occurs with singular as well as plural subjects. With a plural subject \REF{ex:9:63}, the clause has a reciprocal idea -- the subjects resemble each other. 

\newpage 
\ea \label{ex:9:62}
Məlama  ango  azaɗ  \textbf{va}  nə  nok.\\
\gll  məlama=aŋgʷɔ     a-zaɗ   \textbf{va}  nə  nɔkʷ\\
      sibling={\twoS}.{\POSS}  \oldstylenums{3}\textsc{s}-take  body  with  {\twoS}\\
\glt  ‘Your sibling resembles you.’ (lit. your sibling takes body with you)
\z

\ea \label{ex:9:63}
Məlama  ango ahay  jəyga  tazaɗ  \textbf{va}.\\
\gll  məlama=aŋgʷɔ=ahaj   dʒijga   ta-zaɗ   \textbf{va}\\
      sibling={\twoS}.{\POSS}=Pl  all  \oldstylenums{3}\textsc{p}-take  body\\
\glt  ‘All your siblings look alike.’ (lit. siblings take [each other’s] body)
\z

The body part \textit{va} can also collocate with other verbs. For example \textit{e{}mbesen} means ‘he/she breathes,’ but \textit{e{}mbesen} \textit{va} means ‘he/she is resting’ \REF{ex:9:64}.

\ea \label{ex:9:64}
Embesen  va  kə  cəveɗ  aka.\\
\gll  ɛ{}-mbɛʃɛŋ  va  kə  tʃɪvɛɗ  aka\\
      \oldstylenums{3}\textsc{s}-breathe  body  on  road  on\\
\glt  ‘He rests enroute [to somewhere].’
\z

\subsection{\textit{har} ‘body’}\label{sec:9.3.5}\label{sec:9.3.1.5}

A second reduced form of \textit{hərva}, \textit{har} ‘body,’ demonstrates another non-productive collocation with some verbs. With the verb \textit{wəɗakay}, which normally means ‘divide,’ the incorporation of \textit{har} gives a new lexical meaning containing the idea of the participants dispersing (lit. a reflexive idea of ‘dividing themselves up’ \ref{ex:9:65}). 

\ea \label{ex:9:65}
\corpussource{Values, S. 16}\\
T\'{ə}lala,  a  həlan  ga  ava  ese,  təwəɗakala  \textbf{har}  a  məsəyon  ava.\\
\gll  t\'{ə}-l=ala        a   həlaŋ  ga   ava   ɛʃɛ   t\'{u}-wuɗak=ala   \textbf{har}   a   mʊsijɔŋ\\   
      \oldstylenums{3}\textsc{p}-go+{\IFV}=to    at  back  {\ADJ}  in  again  \oldstylenums{3}\textsc{p}-divide+{\IFV}=to     body  at  mission\\ 

\medskip
ava\\
in\\
\glt  ‘They come [home] again, they disperse after church.’
\z

With the verb \textit{gas} which normally means ‘catch,’ \textit{har}  gives the lexical idea of pleasing, which is located at the indirect object \REF{ex:9:66}.

\newpage
\ea \label{ex:9:66}
Membese  va  nə  nok  egəne  na,  agəsaw  \textbf{har}  ava  gam.\\
\gll  mɛ-mbɛʃ-ɛ     va  nə  nɔkʷ  ɛgɪnɛ  na  a-gəs=aw  \textbf{har}=va  gam\\
      {\NOM}{}-breathe-{\CL}  body  with  {\twoS}  today  {\PSP}  \oldstylenums{3}\textsc{s}-catch={\oneS}.{\IO}  body{=\PRF}  {a lot}\\
\glt  ‘Spending time with you today pleased me a lot.’ (lit. it catches body to me) 
\z
\is{Noun incorporation|)}
\largerpage
\section{Clauses with zero grammatical arguments}\label{sec:9.4}\is{Transitivity!Clauses with zero transitivity|(}
\hypertarget{RefHeading1212781525720847}{}
There are clauses in Moloko with no grammatically explicit arguments - these clauses have a transitivity of zero.\footnote{The ideophone clause can also have zero transitivity %%\is{Transitivity!Clauses with zero transitivity}
(\sectref{sec:3.6.3}). See also zero transitivity in nominalised forms, \sectref{sec:8.2.3}.} Nominalised and dependent verb forms are not inflected for subject (see Sections \ref{sec:7.6} and \ref{sec:7.7}, respectively). When they also carry no \DO or \IO pronominal, the clause has zero transitivity. The use of verb forms with no grammatical relations has a discourse function to temporarily take participants out of sight. In the Disobedient Girl story peak episode S. 22 \REF{ex:9:67}, the dependent verb \textit{aməhaya} ‘grinding,’ is unconjugated for subject, direct object, and indirect object. The effect is to keep the participants out of sight as the events unfold and increase vividness as the audience is drawn into the story. All the audience hears is the sound of grinding. The millet is expanding, filling the room and the disobedient girl is lost inside it as she is being suffocated by the millet.  

\ea \label{ex:9:67}
\corpussource{Disobedient Girl, S. 22}\\
Njəw  njəw  njəw  aməhaya  azla.\\
\gll  {nzuw  nzuw  nzuw}    amə-h=aja        aɮa\\
      \textsc{id}:grind  {\DEP}-grind={\PLU} now\\
\glt  ‘\textit{Njəw  njəw  njəw} [she] ground [the millet] now.’  
\z

Likewise in line S. 15 of the Snake story \REF{ex:9:68}, the nominalised form of the verb ‘to penetrate’ occurs with neither \DO nor indirect object pronominals. The climactic moment when the storyteller spears the snake is in a clause with zero transitivity.\is{Focus and prominence!Discourse peak} Participants are out of sight in the discourse. 

\ea \label{ex:9:68}
\corpussource{Snake story, S. 15}\\
Mecesle  mbəraɓ!\\
\gll  mɛ-tʃɛɬ-ɛ  mbəraɓ\\
      {\NOM}{}-penetrate{}-{\CL}      \textsc{id}:penetrate \\
\glt  ‘It penetrated, \textit{mbəraɓ}!’
\z
\is{Transitivity|)}\is{Transitivity!Clauses with zero transitivity|)}
