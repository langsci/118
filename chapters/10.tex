\chapter[Clause]{Clause}\label{chap:10}
\hypertarget{RefHeading1212801525720847}{}
Moloko is an SVO language, which means that the default order of clausal constituents in a simple clause is subject, followed by verb (or predicate), and finally object.\footnote{Elements can be fronted only in a special \textit{na} construction described in \chapref{chap:11}.}  Clause types in Moloko are closely related to the verb type and transitivity of the clause (see \chapref{chap:9}). In this chapter the basic structure of declarative clauses for all verb types is discussed (\sectref{sec:10.1}). The \textit{na} construction can be superimposed upon the basic clause structure, changing the word order. Since the \textit{na} construction is more complex and can involve more than one clause, \textit{na} constructions are discussed in a separate chapter (\chapref{chap:11}). Negation, interrogative, command, and exclamatory clause structures can be further superimposed on a simple or \textit{na}-marked clause to add a functional element (Sections~\ref{sec:10.2}--\ref{sec:10.5}). Clause combining is discussed in \chapref{chap:12}.

\section{Declarative clauses}\label{sec:10.1}
\hypertarget{RefHeading1212841525720847}{}
Moloko has two basic types of declarative clauses, depending on whether the clause contains a verb or not. The verbal clause is described in \sectref{sec:10.1.1}. Clauses where an existential or an ideophone\is{Ideophone} is the central element are a subtype of verbal clauses.  The special features of the structure of existential and ideophone clauses are discussed in \sectref{sec:3.4} and \sectref{sec:3.6}, respectively. Non-verbal clauses are described in \sectref{sec:10.1.2}. These include predicate nominal, predicate adjective, and predicate possessive clauses.  

There is not a lot of variation in the word order of the elements of the basic clause, but the number of grammatically explicit core participants controls the semantic roles assigned to the subject, direct object, and indirect object (see \chapref{chap:9}).  

\largerpage
\subsection{Verbal clause}\label{sec:10.1.1}
\hypertarget{RefHeading1212861525720847}{}
The basic structure of Moloko verbal clauses includes the illustrated elements in the order shown in \figref{fig:16}. Elements whose inclusion in the clause is optional are in parentheses. The order of clause constituents for all clause types is always SVO (with V and O being within the verb phrase). The verb phrase (\chapref{chap:8}) is the centre of the clause (and also its final element) and can contain information concerning the subject, direct object, indirect object, aspect, mood, direction, location, repetition, and discourse-importance of the event or state expressed by the verb (see Sections~\ref{sec:7.3}--\ref{sec:7.5}). All other elements are optional. When present, the temporal adverb gives locational information concerning the event. If a full subject noun phrase is present, it precedes the verb phrase, and any other core clause constituents follow the verb in the verb phrase (direct object, indirect object, obliques). The subject controls the subject inflections on the verb word. 
  
\begin{figure}
\frame{\begin{tabular}{lll}
(temporal noun phrase) &   (subject noun phrase)  & \textbf{Verb phrase}\\
\end{tabular}}
\caption{Order of constituents for verbal clause\label{fig:16}}
\end{figure}

The first element in the clause can be a temporal noun phrase (\ref{ex:10:1}).

\ea \label{ex:10:1}
\textbf{Apazan} albaya  ahay  tolo  a  ləhe.\\
\gll  \textbf{apazaŋ}  albaja=ahaj    tɔ-lɔ    a  lɪhɛ\\
      yesterday  youth=Pl      \oldstylenums{3}\textsc{p}+{\PFV}-go  at  bush\\
\glt  ‘Yesterday the youths went to the bush.’ 
\z

The subject is expressed by the subject pronominal on the verb (\sectref{sec:7.3.1}). A coreferential noun phrase can be present for discourse functions (\ref{ex:10:2} and \ref{ex:10:3}). The coreferential noun phrase precedes the verb.  

\ea \label{ex:10:2}
\textbf{Hawa}  ahəmay.\\
\gll  \textbf{Hawa}   à-həm-aj\\
      Hawa  \oldstylenums{3}\textsc{s}+{\PFV}-run{}-{\CL}\\
\glt  ‘Hawa ran.’
\z

\ea \label{ex:10:3}
\textbf{Ne  ahan}  nozom  na.\\
\gll  \textbf{nɛ}\textbf{=ahaŋ}     n\'{ɔ}-zɔm    na\\
      {\oneS}=\oldstylenums{3}\textsc{s}.{\POSS}  \oldstylenums{3}\textsc{s}+{\PFV}-eat  \oldstylenums{3}\textsc{s}.{\DO}\\
\glt  ‘I myself ate it.’
\z

The simplest form of the verbal clause type consists of a verb complex only. A verb complex can stand alone as a clause because, in addition to the verb stem, it contains information on grammatical relations (subject in the subject prefix, direct object and indirect object in a verb extension or suffix). The verb complex also includes directional and (non-core) locational information and indicates aspect and mood. It is interesting that the SVO order is maintained in the affixes (s-v-o), as seen in \figref{fig:7.2}. (from \sectref{sec:7.1}).

The examples below are clauses consisting of just a verb complex. They all have information on the subject (from subject inflections, \ref{ex:10:4}, \ref{ex:10:6}, \ref{ex:10:7}, \ref{ex:10:8}) or the form of the imperative (\ref{ex:10:5} and \ref{ex:10:9}). Some have information on the direct object (\ref{ex:10:6}--\ref{ex:10:9}), indirect object (\ref{ex:10:8} and \ref{ex:10:9}), direction of the action (\ref{ex:10:5}, \ref{ex:10:7}, \ref{ex:10:9}), and discourse information \REF{ex:10:5}.  

\ea \label{ex:10:4}
Nəhəmay.\\
\gll  n\`{ə}-həm-aj\\
      \oldstylenums{1}\textsc{s}+{\PFV}-run-\CL\\
\glt  ‘I ran.’
\z

\ea \label{ex:10:5}
Dəraka  alay!\\
\gll  dər=aka=alaj\\
      move=on=away\\
\glt  ‘Move further over!’
\z

\ea \label{ex:10:6}
\glll Nozom  na.\\
      n\'{ɔ}-zɔm    na\\
      \oldstylenums{1}\textsc{s}+{\PFV}-eat    \SSS.\DO\\
\glt  ‘I ate it.’
\z

\ea \label{ex:10:7}
Nabah  na  alay. \\
\gll  nà-bax     na=alaj \\
      {\oneS}+{\PFV}-pour {\DO}=away\\
\glt  ‘I poured it away from myself.’
\z

\ea \label{ex:10:8}
Nəvəlan  na.\\
\gll  n\`{ə}-vəl=aŋ    na\\
      {\oneS}+{\PFV}-give=\oldstylenums{3}\textsc{s}.{\IO}  \oldstylenums{3}\textsc{s}.{\DO}\\
\glt  ‘I gave it to him.’
\z

\ea \label{ex:10:9}
Zaw  na  ala!\\
\gll  z=aw   na=ala\\
      carry[{\twoS}.{\IMP}]={\oneS}.{\IO}  \oldstylenums{3}\textsc{s}.{\DO}=to\\
\glt  ‘Bring it to me!’
\z

\subsection{Predicate nominal, predicate adjective, and predicate possessive clauses}\label{sec:10.1.2}
\hypertarget{RefHeading1212881525720847}{}
Predicate nominal (\ref{ex:10:10}--\ref{ex:10:12}), predicate adjective \REF{ex:10:13}, and predicate possessive (\ref{ex:10:14} and \ref{ex:10:15}) clauses lack any verb and consist of a juxtaposition of two noun phrases, in an order shown in \figref{fig:17}.

\begin{figure}
\frame{\begin{tabular}{ll} Subject noun phrase   &  Predicate noun phrase\end{tabular}}
\caption{Constituent order of predicate nominal\slash adjective\slash possessive clauses}\label{fig:17}
\end{figure}

Predicate nominal clauses typically express the notions of proper inclusion (i.e., the clause indicates that the subject is a member of the particular class of items indicated by the predicate, \ref{ex:10:10}) or equation (i.e., the clause indicates that the subject is identical to the predicate, \ref{ex:10:11} and \ref{ex:10:12}). In the following examples, each noun phrase is delimited by square brackets.

\ea \label{ex:10:10}
{} [Mana ]  [zar  mehere].\\
\gll  {}[Mana]    [zar  mɛ-hɛr-ɛ]\\
      Mana    man  {\NOM}{}-build-{\CL}\\
\glt  ‘Mana [is] a builder.’ (lit. Mana, building-man)
\z

\ea \label{ex:10:11}
{}[Sləmay  əwla]  [Abangay].\\
\gll  {}[ɬəmaj=uwla]  [Abaŋgaj]\\
      name={\oneS}.{\POSS}  Abangay\\
\glt  ‘My name [is] Abangay.’
\z

\ea \label{ex:10:12}
{}[Zar  nehe]  [baba  əwla].\\
\gll  {}[zar   nɛhɛ]    [baba=uwla]\\
      man    {\DEM}    father={\oneS}.{\POSS}\\
\glt  ‘The man [is] my father.’ 
\z

Predicate adjective clauses consist of a subject noun phrase and a derived adjective %%\is{Attribution!Derived adjectives}
(\sectref{sec:5.3}) as the predicate noun phrase. These clauses express an attribute of the subject \REF{ex:10:13}.

\ea \label{ex:10:13}
{}[Ndahan]  [malan  ga].\\
\gll  {}[ndahaŋ]  [malaŋ ga]\\
      \oldstylenums{3}\textsc{s}    largeness  {\ADJ}\\
\glt  ‘He/she [is] big.’
\z

Predicate possessive clauses have a subject noun phrase and a possessive prepositional phrase (see \sectref{sec:5.6.1}) as the predicate phrase. The participant named in the possessive phrase is expressed via a full noun phrase. These clauses express that the subject noun phrase is associated with the participant named in the possessive phrase. The semantic range for the predicate possessive clauses is the same as that of any possessive or genitive construction (see Sections \ref{sec:3.1.2.1} and \ref{sec:5.4.1}).

\ea \label{ex:10:14}
{} [Babəza  ahay  nəndəye]  [\textbf{anga}  bahay].\\
\gll  {}[babəza=ahaj  nɪndijɛ]    [\textbf{aŋga}  bahaj]\\
      children=Pl  {\DEM}    {\POSS}  chief\\
\glt  ‘The children here belong to the chief.’ / ‘The children here[are] belonging to the chief.’ 
\z

\ea \label{ex:10:15}
{}[Dəray  ga]  [\textbf{anga}  ləme].\\
\gll  {}[dəraj  ga]    [\textbf{aŋga}  lɪmɛ]\\
      head  {\ADJ}    {\POSS}  \oldstylenums{1}\textsc{Pex}\\
\glt  ‘The head belonged to us.’/ ‘The head [was] belonging to us.’ 
\z

For all three of these clause types, the subject may be marked as presupposed (see \sectref{sec:11.2}). For a predicate nominal construction, fronting and marking the predicate with \textit{na} expresses equation in (\ref{ex:10:16}--\ref{ex:10:18}).

\ea \label{ex:10:16}
{}[Zar  mehere   na],  [Mana].\\
\gll  {}[zar    mɛ-hɛr-ɛ]    na  [Mana]\\
      man   {\NOM}{}-build-{\CL}   {\PSP}  Mana\\
\glt  ‘The builder [is] Mana.’ 
\z

\ea \label{ex:10:17}
{}[Bahay  a  Laway  na],  [Ajəva].\\
\gll  {}[bahaj   a   Lawaj     na]  [Adzəva]\\
      chief  {\GEN}  Lalaway    {\PSP}  Adzava\\
\glt  ‘The chief of Lalaway [is] Adzava.’
\z

\ea \label{ex:10:18}
{}[Malan  ga   na],  [ndahan].\\
\gll  [malaŋ   ga]    na  [ndahaŋ]\\
      largeness  {\ADJ}    {\PSP}  \oldstylenums{3}\textsc{s}\\
\glt  ‘The biggest one [is] him.’ (lit. big, him)
\z

\section{Negation constructions}\label{sec:10.2}
\hypertarget{RefHeading1212901525720847}{}
Negation constructions are specific constructions superimposed on a clause to create negation of the entire proposition (\sectref{sec:10.2.2}) or negation of one element of the clause (\sectref{sec:10.2.3}). For both, Moloko uses a negative particle \textit{baj} or compound at the end of the clause or noun phrase (\sectref{sec:10.2.1}).

\subsection{Negative particles}\label{sec:10.2.1}
\hypertarget{RefHeading1212921525720847}{}
The all-purpose negative is the particle \textit{bay,} which follows the verb phrase and occurs (\ref{ex:10:19}--\ref{ex:10:21}) before any interrogative word (see \sectref{sec:10.3}). In the examples in this section, the negative is bolded and the negation construction is in square brackets. 

\ea \label{ex:10:19}
{}[Alala  \textbf{bay}].\\
\gll  {}[à-l=ala    \textbf{baj}]\\
      \oldstylenums{3}\textsc{s}+{\PFV}-go=to    {\NEG}\\
\glt  ‘He/she didn’t come.’
\z

\ea \label{ex:10:20}
{}[War  ga  ecen  sləmay  \textbf{bay}].\\
\gll  {}[war  ga  ɛ{}-tʃɛŋ  ɬəmaj  \textbf{baj}]\\
      child  {\ADJ}  \oldstylenums{3}\textsc{s}-hear  ear  {\NEG}\\
\glt  ‘That child did not obey.’ (lit. that child, he hears ear not)
\z

\ea \label{ex:10:21}
[Táazləgalay  avəlo  \textbf{bay}].\\
\gll  [táá-ɮəg=alaj    avʊlɔ  \textbf{baj}]\\
      \oldstylenums{3}\textsc{p}+{\POT}-throw=away    above  {\NEG}\\
\glt  ‘They should not throw it too high.’
\z

In (\ref{ex:10:22}--\ref{ex:10:24}) the negative is clause-final and may have sematic scope over the entire proposition (c.f. constituent negation, \sectref{sec:10.2.3}). See especially \REF{ex:10:23} where it is clear that the entire proposition is being negated, and not just the information within the constituent closest to the negative. The meaning is ‘don’t insult a small person.’ If the information in only one constituent was being negated, the meaning would have been ‘insult a person who is not small.’

\ea \label{ex:10:22}
[Tagaw  ele  lala  \textbf{bay}].\\
\gll  [ta-g=aw    ɛlɛ   lala  \textbf{baj}]\\
      \oldstylenums{3}\textsc{p}-do={\oneS}.{\IO}   thing  good  {\NEG}\\
\glt  ‘They do bad things to me.’ / ‘They don’t do good things to me.’ 
\z

\ea \label{ex:10:23}
{}[Kárasay  məze  cəɗew  ga  \textbf{bay}].\\
\gll  {}[ká-ras-aj     mɪʒɛ   tʃɪɗɛw     ga   \textbf{baj}]\\
      {\twoS}+{\IFV}-minimise{}-{\CL}  person  smallness  {\ADJ}  {\NEG}\\
\glt  ‘Don’t insult one of the little people.’ 
\z

\ea \label{ex:10:24}
{}[Anday  dəren \textbf{bay}].\\
\gll  {}[à-ndaj    dɪrɛŋ  \textbf{baj}]\\
      \oldstylenums{3}\textsc{s}+{\PFV}-{\PRG}  far  {\NEG}\\
\glt  ‘He/she was not far.’  
\z

In \REF{ex:10:25}, \textit{bay} is not clause final but is the final element in a noun phrase within the clause. In this case, the information expressed within the noun phrase itself is negated; \textit{ele lala} \textbf{\textit{bay}} ‘a bad thing.’ 

\ea \label{ex:10:25}
Nde,  [ele  lala  \textbf{bay}]  kə  təta  aka.\\
\gll  ndɛ     [ɛlɛ   lala    \textbf{baj}]   kə   təta   aka\\
      so    thing  {well done}  {\NEG}  on  them  on\\
\glt  ‘So, a bad thing [was] upon them.’ 
\z

When relative clauses are negated, the negative may have semantic scope over the entire relative clause (\ref{ex:10:26}, \ref{ex:10:27}). 

\ea \label{ex:10:26}\corpussource{Values, S. 6}\\
Ele  ahay  [aməgəye  \textbf{bay}]  nəngehe  pat  tahata  na  va.\\
\gll  ɛlɛ=ahaj  [amɪ-g-ijɛ   \textbf{baj}]  nɪŋgɛhɛ  pat  ta-h=ata  na=va\\
      thing=Pl  {\DEP}-do-{\CL}  {\NEG}  {\DEM}  all  \oldstylenums{3}\textsc{p}-tell=\oldstylenums{3}\textsc{p}.{\IO}  \oldstylenums{3}\textsc{s}.{\DO}={\PRF}\\
\glt  ‘All these things that [we] are not supposed to do, they have already told them.’  
\z

\ea \label{ex:10:27}
Kəra  [aməmənjere  elé  \textbf{bay}]  táslay  na  gəraw.\\
\gll  kəra    [amɪ-mɪndʒɛr-ɛ  ɛlɛ  \textbf{baj}]  tá-ɬ{}-aj    na  gəraw\\
      dog    {\DEP}-see-{\CL}  eye  {\NEG}  \oldstylenums{3}\textsc{p}+{\IFV}-slay{}-{\CL}    \oldstylenums{3}\textsc{s}.{\DO}  {\textsc{id}:cut through middle}\\
\glt  ‘The dog that couldn’t see they slew it through the middle.’
\z

The negative can form a compound with some adverbs. Negated and non-negated clauses with four adverbs are shown in \tabref{tab:81}. The negative \textit{asabay} ‘never again’ is a compound of  the adverb \textit{ese} ‘again’ and \textit{bay}. The evidence of phonological binding is that the adverb \textit{ese}  loses its palatalisation when it compounds with \textit{bay} (line 1 in \tabref{tab:81}). Likewise, \textit{fabay} (line 2 in \tabref{tab:81}) is considered phonologically bound since the word-final /n/ in the adverb \textit{fan} ‘already’ is deleted when the negative is added. These changes occur with some clitics (see \sectref{sec:2.6.1.5}). The other adverbs are considered to be separate phonological words since there are no other indications that the negative is phonologically bound to the adverb since the prosody of other adverbs is not affected (e.g., \textit{kəlo} ‘before,’ line 3 in \tabref{tab:81}).  

\begin{table}
\begin{tabular}{lll}
\lsptoprule
{Line} & {Non-negated clause with adverb} & {Negated clause}\\\midrule
1 & \textit{n\'{o}{o}-lo} \textbf{\textit{ese}} & \textit{n\'{o}{o}-lo} \textbf{\textit{asabay}}\\
& {\oneS}+{\POT}-go  again & {\oneS}+{\POT}-go  again+{\NEG} \\
& ‘I will go again.’ & ‘I will not go again.’\\\midrule
2 & \textit{n\'{e}-g-e  na} \textbf{\textit{fan}} & \textit{n\'{e}-g-e  na} \textbf{\textit{fabay}}\\
& {\oneS}+{\IFV}-do-{\CL}  \oldstylenums{3}\textsc{s}.{\DO}  already & {\oneS}+{\IFV}-do-{\CL}  \oldstylenums{3}\textsc{s}.{\DO}  already+{\NEG}\\
& ‘I have done it already.’ & ‘I haven’t done it yet.’\\\midrule
3 & \textit{nə-mənjar  ndahan} \textbf{\textit{kəlo}} & \textit{nə-mənjar  ndahan} \textbf{\textit{kəlo}} \textbf{\textit{bay}}\\
& {\oneS}-see  \oldstylenums{3}\textsc{s}  before & {\oneS}-see  \oldstylenums{3}\textsc{s}  before {\NEG}\\
& ‘I have seen her before.’ & ‘I have never seen her before.’  \\\midrule
4 & \textit{káa-z=ala} \textbf{\textit{təta}} & \textit{káa-z=ala} \textbf{\textit{təta}} \textbf{\textit{bay}}\\
& {\twoS}+{\POT}-take=to  ability & {\twoS}+{\POT}-take=to  ability {\NEG}\\
& ‘You can bring [it].’ & ‘You can’t bring [it].’  \\
\lspbottomrule
\end{tabular}
\caption{Negation of clauses with adverbs\label{tab:81}}
\end{table}

\largerpage
\subsection{Clausal negation construction}\label{sec:10.2.2}
\hypertarget{RefHeading1212941525720847}{}
For clausal negation, there is no change in word order and no change in clause constituents apart from the addition of the clause final negative particle. A negative clause asserts that some event or state does not hold. Various types of clausal negation in Moloko are illustrated in (\ref{ex:10:28}--\ref{ex:10:43}). Each pair of examples represents a positive and a negative assertion for comparison. 

The negation of an intransitive clause is illustrated in \REF{ex:10:28} and \REF{ex:10:29}. 

\ea \label{ex:10:28}
Ahəmay.\\
\gll  a-həm-aj\\
      \oldstylenums{3}\textsc{s}-run-{\CL}\\
\glt  ‘He/she runs.’            
\z
\clearpage
\ea \label{ex:10:29}
Ahəmay  \textbf{bay.}\\
\gll  a-həm-aj     \textbf{baj}\\
      \oldstylenums{3}\textsc{s}-run-{\CL}    {\NEG}\\
\glt  ‘He/she doesn’t run.’
\z

THe negation of a transitive clause is shown in (\ref{ex:10:30}--\ref{ex:10:35}).

\ea \label{ex:10:30}
Amənjar  Hawa. \\
\gll  a-mənzar   Hawa\\
      \oldstylenums{3}\textsc{s}-see  Hawa\\
\glt  ‘He/she sees Hawa.’           
\z

\ea \label{ex:10:31}
Amənjar  Hawa  \textbf{bay.}\\
\gll  a-mənzar   Hawa   \textbf{baj}\\
      \oldstylenums{3}\textsc{s}-see  Hawa  {\NEG}\\
\glt  ‘He/she doesn’t see Hawa.’
\z

\ea \label{ex:10:32}
Akaɗ  awak.\\
\gll  a-kaɗ   awak\\
      \oldstylenums{3}\textsc{s}-kill  goat\\
\glt  ‘He/she kills a goat.’          
\z

\ea \label{ex:10:33}
Akaɗ  awak  \textbf{bay.}\\
\gll  a-kaɗ   awak   \textbf{baj}\\
      \oldstylenums{3}\textsc{s}-kill  goat  {\NEG}\\
\glt  ‘He/she doesn’t kill a goat.’
\z

\ea \label{ex:10:34}
Asaw sese.\\
\gll  a-s=aw  ʃɛʃɛ\\
      \oldstylenums{3}\textsc{s}-please={\oneS}.{\IO}  meat\\
\glt  ‘I want meat.’            
\z

\ea \label{ex:10:35}
Asaw  sese  \textbf{bay.}\\
\gll  a-s=aw     ʃɛʃɛ   \textbf{baj}\\
      \oldstylenums{3}\textsc{s}-please={\oneS}.{\IO}   meat  {\NEG}\\
\glt  ‘I do not want meat.’
\z

The   negation of existentials is shown in (\ref{ex:10:36}--\ref{ex:10:39}). 

\ea \label{ex:10:36}
Babəza  əwla  ahay  aba. \\
\gll  babəza=uwla=ahaj  aba \\
      children={\oneS}.{\POSS}=Pl  {\EXT}\\
\glt  ‘I have children.’            
\z

\ea \label{ex:10:37}
Babəza  əwla  ahay  \textbf{abay.}\\
\gll  babəza=uwla=ahaj  \textbf{abaj}\\
      children={\oneS}.{\POSS}=Pl  {\EXT}+{\NEG}\\
\glt  ‘I have no children.’  
\z

\ea \label{ex:10:38}
Dala  anaw  aka. \\
\gll  dala   an=aw     aka\\
      money  {\DAT}={\oneS}  {\EXT}+on \\
\glt  ‘I have money.’           
\z

\ea \label{ex:10:39}
Dala  anaw  aka \textbf{bay.}\\
\gll  dala   an=aw     aka \textbf{baj}\\
      money  {\DAT}={\oneS}  {\EXT}+on    {\NEG}\\
\glt  ‘I have no money.’
\z

The  negation of a predicate adjective is illustrated in (\ref{ex:10:40}--\ref{ex:10:43}). 

\ea \label{ex:10:40}
Ndahan  zləle  ga.\\
\gll  ndahaŋ   ɮɪlɛ   ga\\
      \oldstylenums{3}\textsc{s}    richness  {\ADJ}\\
\glt  ‘He/she is rich.’           
\z

\ea \label{ex:10:41}
Ndahan  zləle  ga  \textbf{bay.}\\
\gll  ndahaŋ  ɮɪlɛ   ga   \textbf{baj}\\
      \oldstylenums{3}\textsc{s}    richness  {\ADJ}  {\NEG}\\
\glt  ‘He/she is not rich.’
\z

\ea \label{ex:10:42}
Ndahan  gədan  ga.\\
\gll  ndahaŋ   gədaŋ   ga\\
      \oldstylenums{3}\textsc{s}    strength  {\ADJ}\\
\glt  ‘He/she is strong.’            
\z

\ea \label{ex:10:43}
Ndahan  gədan  ga  \textbf{bay.}\\
\gll  ndahaŋ   gədaŋ   ga   \textbf{baj}\\
      \oldstylenums{3}\textsc{s}    strength  {\ADJ}  {\NEG}\\
\glt  ‘He/she is not strong.’
\z

\subsection{Constituent negation}\label{sec:10.2.3}
\hypertarget{RefHeading1212961525720847}{}
Most frequently, it seems that the element closest to the negative that is under the scope of negation, even though a clause-final negative marker can have scope over the whole verb phrase or even over the entire clause. To negate only one constituent in a clause, the clause is sometimes rearranged so that the constituent that is negated is placed in the clause-final position adjacent to the negation particle. Examples (\ref{ex:10:44}--\ref{ex:10:46}) show a question \REF{ex:10:44} with two responses (\ref{ex:10:45}--\ref{ex:10:46}) where each of the two ambiguous elements is negated. The subject (\textit{Mana}) is part of the presupposition (marked off by \textit{na} in the question, see \sectref{sec:11.2}). In \REF{ex:10:45} the oblique is negated and in \REF{ex:10:46} the entire predicate. The clauses were not restructured since the elements in question were already clause-final. In the following examples, the element that is negated is delimited by square brackets and the negative is bolded.

\ea \label{ex:10:44}
Mana  na,  olo  [a  kosoko  ava]  ɗaw?\\
\gll  Mana   na   ɔ{}-lɔ   [a   kɔsɔkʷɔ   ava]  ɗaw\\
      Mana  {\PSP}  \oldstylenums{3}\textsc{s}-go  at  market  in  {\textsc{q}}\\
\glt  ‘As for Mana, is he going to the market?’
\z

\ea \label{ex:10:45}
Ehe,  olo  [a  kosoko  ava]  \textbf{bay};  olo  afa  bahay.\\
\gll  ɛhɛ     ɔ-lɔ   [a   kɔsɔkʷɔ   ava]  \textbf{baj}    ɔ-lɔ   afa     bahaj\\
      no    \oldstylenums{3}\textsc{s}-go  at  market  in  {\NEG}  \oldstylenums{3}\textsc{s}-go  {house of}    chief\\
\glt  ‘No, he isn’t going to the market; rather he is going to the chief’s house.’
\z

\ea \label{ex:10:46}
Ehe,  olo  [a  kosoko  ava]  \textbf{bay};  enjé  a  mogom.\\
\gll  ɛhɛ,   [ɔ-lɔ   a   kɔsɔkʷɔ   ava]  \textbf{baj}   ɛ-nʒ-ɛ     a   mɔgʷɔm\\
      no    \oldstylenums{3}\textsc{s}-go  at  market  in  {\NEG}  \oldstylenums{3}\textsc{s}-stay-{\CL}  at  home\\
\glt  ‘No, he isn’t going to the market; rather he is staying at home (or going to the chief’s house).’
\z

Examples (\ref{ex:10:47}--\ref{ex:10:50}) show some restructuring when different constituents are negated. Example \REF{ex:10:47} illustrates a question and \REF{ex:10:48} to \REF{ex:10:50} illustrate three possible answers, each negating a different constituent. Normal SVO structure is maintained for all answers. The responses each use two clauses. The first clause expresses the negation of the element in final position, and the second restates the clause giving the corrected information. In each case the first clause is restructured so as to move the element to be negated to the clause-final position. The response in \REF{ex:10:48} indicates that the hearer accepts 'that Mana gave the guitar to someone,' but it was not his father. In this clause, \textit{kəndew} ‘guitar’ is realised as the \oldstylenums{3}\textsc{s} \DO pronominal. The response in \REF{ex:10:49} indicates 'that Mana gave something to his father,' but not a guitar. In this case, the adpositional phrase \textit{ana baba ahan} ‘to his father’ is replaced by the indirect object pronominal so that the negated element \textit{kəndew} ‘guitar’ can be placed next to the negative. 

\ea \label{ex:10:47}
Mana  avəlan  kəndew  ana  baba  ahan  ɗaw?\\
\gll  Mana  à-vəl=aŋ     kɪndɛw  ana   baba=ahaŋ    ɗaw\\
      Mana     \oldstylenums{3}\textsc{s}+{\PFV}-give=\oldstylenums{3}\textsc{s}.{\IO}  guitar  {\DAT} father=\oldstylenums{3}\textsc{s}.{\POSS}  {\textsc{q}}\\
\glt  ‘Did Mana give the guitar to his father?’
\z

\ea \label{ex:10:48}
Ehe,  avəlan  na  [ana  baba  ahan]  \textbf{bay},  \\
\gll  ɛhɛ à-vəl=aŋ na [ana baba=ahaŋ] \textbf{baj}\\    
      no    \oldstylenums{3}\textsc{s}+{\PFV}-give=\oldstylenums{3}\textsc{s}.{\IO}  \oldstylenums{3}\textsc{s}.{\DO}  {\DAT} father=\oldstylenums{3}\textsc{s}.{\POSS}  {\NEG}\\ 

      \medskip
 avəlan  na  ana  gəmsodo  ahan.\\
\gll à-vəl=aŋ na ana gʊmsɔdɔ=ahaŋ\\
     \oldstylenums{3}\textsc{s}+{\PFV}-give=\oldstylenums{3}\textsc{s}.{\IO}  \oldstylenums{3}\textsc{s}.{\DO} {\DAT} {mother’s brother}=\oldstylenums{3}\textsc{s}.{\POSS} \\
\glt  ‘No, he didn’t give it t\textit{o his father}, he gave it to his mother’s brother.’
\z

\ea \label{ex:10:49}
Ehe,  avəlan  [kəndew]  \textbf{bay},  avəlan  cecewk.\\
\gll  ɛhɛ    à-vəl=aŋ   [kɪndɛw]   \textbf{baj} à-vəl=aŋ tʃɛtʃœkʷ\\
      no    \oldstylenums{3}\textsc{s}+{\PFV}-give=\oldstylenums{3}\textsc{s}.{\IO}  guitar    {\NEG} \oldstylenums{3}\textsc{s}+{\PFV}-give=\oldstylenums{3}\textsc{s}.{\IO}  flute\\
\glt  ‘No, he didn’t give a \textit{guitar} to his father, he gave him a flute.’ 
\z

The fourth possible reply to the question in \REF{ex:10:47} negates the subject. Moloko clause structure does not allow the subject to occupy the clause-final position; to specifically negate the subject of a clause \REF{ex:10:52}, a predicate nominal clause structure is used. The predicate is recast as a relative clause (see \sectref{sec:5.4.3}) with the presupposed information that someone gave a guitar to his father marked with \textit{na}. The nominal is the negated subject \textit{Mana bay} ‘not Mana.’ 

% \clearpage
\ea \label{ex:10:50}
Ehe,  aməvəlan  kəndew  ana  baba  ahan  na,  [Mana]  bay;\\  
\gll  ɛhɛ    amə-vəl=aŋ  kɪndɛw  ana  baba=ahaŋ    na [Mana]  baj \\ 
      no    {\DEP}-give=\oldstylenums{3}\textsc{s}.{\IO}  guitar  {\DAT} father=\oldstylenums{3}\textsc{s}.{\POSS}  {\PSP} Mana     {\NEG}  \\    
\glt ‘No, \textit{Mana} didn’t give the guitar to his father.  (lit. the one that gave guitar to his father, not Mana)’ \\    
           
      \clearpage
aməvəlan  na,  Majay.\\      
\gll amə-vəl=aŋ na Madzaj\\
     {\DEP}-give=\oldstylenums{3}\textsc{s}.{\IO}  {\PSP}  Madzay\\
\glt  ‘The person that gave [it was] Madzay.’ 
\z

Examples (\ref{ex:10:51}--\ref{ex:10:52}) show a similar restructuring of a verbal clause into a predicate nominal in order to negate the subject of a clause. A question with a verbal clause structure is shown in \REF{ex:10:51}. In order to negate the subject, the clause is restructured to put all of the known information in a predicate that is a relative clause delimited by\textit{ na,} and the negated subject becomes the final nominal \REF{ex:10:52}. 

\ea \label{ex:10:51}
Hawa  adan  ɗaf  ana  Mana  ɗaw?\\
\gll  Hawa   à-d=aŋ    ɗaf  ana  Mana  ɗaw\\
      Hawa  \oldstylenums{3}\textsc{s}+{\PFV}-prepare=\oldstylenums{3}\textsc{s}.{\IO}  {millet loaf}  {\DAT} Mana  {\textsc{q}}\\
\glt  ‘Did Hawa prepare food for Mana?’
\z

\ea \label{ex:10:52}
Amadan  ɗaf  ana  Mana  na,  [Hawa]  \textbf{bay}.\\
\gll  ama-d=aŋ    ɗaf  ana  Mana  na  [Hawa]   \textbf{baj}\\
      {\DEP}-prepare=\oldstylenums{3}\textsc{s}.{\IO}  {millet loaf}  {\DAT} Mana  {\PSP}  Hawa   {\NEG}\\
\glt  ‘The one that prepared the millet loaf for Mana [was] not \textit{Hawa}.’
\z

\section{Interrogative constructions}\label{sec:10.3}
\hypertarget{RefHeading1212981525720847}{}
The syntax of interrogative constructions is remarkable in that all interrogative particles except \textit{wɛlɛj} ‘which one' occur clause finally. In certain constructions, the clause itself is rearranged so that the interrogative particle can remain clause final. Interrogative constructions are superimposed on top of the other clausal construction types. Like the case for the negation construction (see \sectref{sec:10.2.3}), the element closest to the interrogative pronoun or question word seems most frequently under the scope of interrogation. Types of interrogative constructions include content questions (see \sectref{sec:10.3.1}), yes/no questions (see \sectref{sec:10.3.2}), tag question construction, to clarify a particular statement (see \sectref{sec:10.3.3}), rhetorical question constructions (see \sectref{sec:10.3.4}), and emphatic question constructions (see \sectref{sec:10.3.5}).  

\subsection{Content question construction}\label{sec:10.3.1}
\hypertarget{RefHeading1213001525720847}{}
Information questions use interrogative pronouns which must be clause-final. The interrogative pronouns (see \sectref{sec:3.1.4}) each fill a slot in the clause according to the element they each are questioning. All elements in a clause can be questioned including subject, direct object, indirect object, verb, oblique, and noun modifier. The clause structure will always be arranged such that the element questioned is clause-final. Three main clause structures are employed in order to achieve clause-final interrogative pronouns. \tabref{tab:82}. shows the interrogative forms used for content questions. 

\begin{table}
\resizebox{\textwidth}{!}{\begin{tabular}{ll}
\lsptoprule
{Construction} & {Structure and example}\\\midrule
\textbf{Verbal clause structure} & clause –  interrogative word \\
Questions clausal element & \textit{zar   a-mənjar   \textbf{way}} \\
& man  \oldstylenums{3}\textsc{s}+{\PFV}-see  who \\
& ‘Who did the man see?’\\\midrule
\textbf{Predicate nominal} & dependent clause marked with \textit{na} – interrogative word\\
Questions subject & \textit{hor  amə-d-əye  ɗaf   na  \textbf{way}}\\
& woman  {\DEP}-make-{\CL}  {millet loaf}  {\PSP}  who\\
& ‘Who is making millet loaf?’ \\
&  (lit. the woman that is making millet loaf [is] who?)\\\midrule
\textbf{Right-shifted \textit{na} marked element} & clause – interrogative word – right-shifted \textit{na} marked element\\
Questions internal element & \textit{Mala  a-vəl=an  \textbf{almay} ana  məlama=ahan  na}\\
& Mala  \oldstylenums{3}\textsc{s}-give=\oldstylenums{3}\textsc{s}.{\IO}  what  {\DAT} sibling=\oldstylenums{3}\textsc{s}.{\POSS}  {\PSP}\\
& ‘Mala gave what to his brother?’\\
\lspbottomrule
\end{tabular}}
\caption{Content information constructions\label{tab:82}}
\end{table}

The first clause structure that is employed is the verbal clause structure (SVO), but with substitution of a question word. The verbal clause structure is rearranged in the same manner as for constituent negation (see \sectref{sec:10.2.3}) in order to position the questioned element in the clause-final position so that it is replaced by the interrogative pronoun. Information questions in verbal clauses are paired with a response in (\ref{ex:10:53}--\ref{ex:10:65}) so that the structure of the interrogative clause can be compared with that of the declarative. Examples in this section are given in pairs.  The first example in the pair shows the interrogative construction. The second example is the clause with the information filled in for comparison.

\largerpage The direct object is questioned in \REF{ex:10:53}. The presupposed information is that the man saw someone. Note that there are no other elements that follow the direct object in the verb phrase. The interrogative pronoun fills the direct object slot (identified by square brackets). 


\ea \label{ex:10:53}
Zar  amənjar  [\textbf{way}]?\\
\gll  zar     à-mənzar   [\textbf{waj}]\\
      man    \oldstylenums{3}\textsc{s}+{\PFV}-see  who\\
\glt  ‘Who did the man see?’ 
\z

\ea \label{ex:10:54}
Zar  amənjar  [Mana].\\
\gll  zar     à-mənzar   [Mana]\\
      man    \oldstylenums{3}\textsc{s}+{\PFV}-see  Mana\\
\glt  ‘The man saw Mana.’ 
\z

A noun modifier is questioned in \REF{ex:10:55}. The presupposed information is that the woman made some kind of sauce, and the question seeks to find out what kind of sauce. The interrogative pronoun \textit{weley} ‘which’ is within the noun phrase delimited by square brackets in the example. Even though the interrogative pronoun is inside a noun phrase, that noun phrase is clause-final so the interrogative pronoun is the final word in the clause. 

\ea \label{ex:10:55}
Hor  ede  [elele  \textbf{weley}]?\\
\gll  hʷɔr    ɛ-dɛ    [ɛlɛlɛ  \textbf{wɛlɛj}]\\
      woman  \oldstylenums{3}\textsc{s}-prepare  sauce  which\\
\glt  ‘The woman is making which kind of sauce?’
\z

\ea \label{ex:10:56}
Hor  ede  [elele  kəlef].\\
\gll  hʷɔr    ɛ{}-d-ɛ    [ɛlɛlɛ  kɪlɛf]\\
      woman  \oldstylenums{3}\textsc{s}-prepare-{\CL}  sauce  fish\\
\glt  ‘The woman is making fish sauce.’
\z

Example \REF{ex:10:57} questions the direct object of a subordinate clause, in this case a purpose adverbial clause (delimited by square brackets). The presupposed information is that the listener has come to do something. The interrogative pronoun \textit{almay}  ‘what’ is clause-final since the adverbial clause has no other elements following the direct object. Two possible responses are shown in \REF{ex:10:58}--\REF{ex:10:59}.

\ea \label{ex:10:57}
Kəlala  [aməgəye  \textbf{almay}]?\\
\gll  k\`{ə}-l=ala   [amɪ-g-ijɛ   \textbf{almaj}]\\
      {\twoS}+{\PFV}-go=to  {\DEP}-do-{\CL}  what\\
\glt  ‘What have you come to do?’ (lit. you have come to do what?)
\z

% % \clearpage
\ea \label{ex:10:58}
Nəlala  [aməgəye  slərele].\\
\gll  n\`{ə}-l=ala     [amɪ-g-ijɛ   ɬɪrɛlɛ]\\
      {\oneS}+{\PFV}-go=to  {\DEP}-do{}-{\CL}  work\\
\glt  ‘I came to do work.’ 
\z

\clearpage
\ea \label{ex:10:59}
Nəlala  [aməjənok].\\
\gll  n\`{ə}-l=ala     [amə-dzən-ɔkʷ]\\
      {\oneS}+{\PFV}-go=to  {\DEP}-help-{\twoS}\\
\glt  ‘I came to help you.’ 
\z

In \REF{ex:10:60}, the indirect object is questioned. The presupposed information is that Mala gave a book to someone. The interrogative pronoun \textit{way} ‘who,’ is located within a prepositional phrase identified by square brackets.  That prepositional phrase is clause-final, so that again the interrogative pronoun is the final element in the clause. 

\ea \label{ex:10:60}
Mala  avəlan  ɗeləywer  [ana  \textbf{way}]?\\
\gll  Mala   à-vəl=aŋ     ɗɛlijwɛr     [ana   \textbf{waj ]}\\
      Mala  \oldstylenums{3}\textsc{s}+{\PFV}-give=\oldstylenums{3}\textsc{s}.{\IO}  paper    {\DAT} who\\
\glt  ‘Mala gave the book to whom?’ 
\z

\ea \label{ex:10:61}
Mala  avəlan  ɗeləywer  [ana  Hawa].\\
\gll  Mala   à-vəl=aŋ     ɗɛlijwɛr     [ana   Hawa]\\
      Mala  \oldstylenums{3}\textsc{s}+{\PFV}-give=\oldstylenums{3}\textsc{s}.{\IO}  paper    {\DAT} Hawa\\
\glt  ‘Mala gave the book to Hawa.’ 
\z

In \REF{ex:10:62} and \REF{ex:10:64}, an oblique is questioned. The presupposed information is that the woman plans to go to market sometime. The interrogative pronoun is the temporal element in the clause in \REF{ex:10:62}. While temporal noun phrases can occur clause-initially, the interrogative pronoun is again found in the clause-final position. 

\ea \label{ex:10:62}
Hor  olo  a  kosoko  ava  [\textbf{epeley}]?\\
\gll  hʷɔr    ɔ-lɔ  a  kɔsɔkʷɔ  ava  [\textbf{ɛpɛlɛj}]\\
      woman  \oldstylenums{3}\textsc{s}-go  at  market  in  when\\
\glt  ‘When is the woman going to market?’
\z

\ea \label{ex:10:63}
Hor  olo  a  kosoko  ava  [hajan].\\
\gll  hʷɔr    ɔ{}-lɔ  a  kɔsɔkʷɔ  ava  [hadzaŋ]\\
      woman  \oldstylenums{3}\textsc{s}-go  at  market  in  tomorrow\\
\glt  ‘The woman is going to market tomorrow.’
\z

The elements within non-core adpositional phrases are questioned using the generic location question word \textit{amtamay }‘where’ \REF{ex:10:64}. This generic location question word does not need to be located inside an adpositional phrase, eliminating the possibility that the locational postposition would follow the interrogative pronoun in the clause allowing the interrogative pronoun to be clause-final. The presupposed information is that the hearer is going somewhere. 

\ea \label{ex:10:64}
Kolo  [\textbf{amtamay}]?\\
\gll  k\'{ɔ}-lɔ  [\textbf{amtamaj}] \\
      {\twoS}+{\PFV}-go  where\\
\glt  ‘Where did you go?’
\z

\ea \label{ex:10:65}
Nolo  [a  kosoko  ava].\\
\gll  n\'{ɔ}-lɔ  [\textbf{a} kɔsɔkʷɔ    \textbf{ava}] \\
      {\oneS}+{\PFV}-go  at  market    in\\
\glt  ‘I went to market.’
\z

The second clause structure that is employed for interrogative constructions is the predicate nominal. The predicate nominal structure is employed for questioning an element of a predicate nominal clause. (\ref{ex:10:66}–\ref{ex:10:71}) are example pairs where the first of each pair is a question and the second is a possible response. In \REF{ex:10:66} an aspect of the nominal is questioned with the interrogative pronoun in a prepositional phrase. The prepositional phrase is delimited by square brackets.

\ea \label{ex:10:66}
Mogom  nehe  [anga  \textbf{way}]?\\
\gll mɔgʷɔm  nɛhɛ  [aŋga  \textbf{waj}]\\
      house  {\DEM}  {\POSS}  who\\
\glt  ‘This house here belongs to whom?’
\z

\ea \label{ex:10:67}
Mogom  nehe  [anga  Mana].\\
\gll  mɔgʷɔm  nɛhɛ  [aŋga  Mana]\\
      house  {\DEM}  {\POSS}  Mana\\
\glt  ‘This house here belongs to Mana.’ (the house here, belonging to Mana)
\z

In \REF{ex:10:68} and \REF{ex:10:70}, the interrogative word itself is the predicate. 

\ea \label{ex:10:68}
Mogom  ango  [\textbf{amtamay}]?\\
\gll  mɔgʷɔm=aŋgʷɔ     [\textbf{amtamaj}]\\
      home={\twoS}.{\POSS}  where\\
\glt  ‘Where is your home?’
\z

\clearpage
\ea \label{ex:10:69}
Mogom  əwla  [a Laway].\\
\gll  mɔgʷɔm=uwla     [a Lawaj]\\
      home={\oneS}.{\POSS}  to  Lalawaj\\
\glt  ‘My home is in Lalaway.’
\z

\ea \label{ex:10:70}
 Bahay  a  slala  aləkwəye na  [\textbf{way}]?\\
\gll  bahaj  a  ɬala=alʊkʷøjɛ  na  [\textbf{waj}]\\
      chief  {\GEN}  village={\twoP}.{\POSS}  {\PSP}  who\\
\glt  ‘The chief of your village is who?’
\z

\ea \label{ex:10:71}
Bahay  a  slala  əwla  na  [Ajəva].\\
\gll  bahaj  a  ɬala=uwla    na  [Adzəva]\\
      chief  {\GEN}  village={\oneS}.{\POSS}  {\PSP}  Adziva\\
\glt  ‘The chief of my village is Adziva.’
\z

The predicate nominal clause is also used for questioning the subject in what would otherwise be a normal verbal clause (paralleling the case for the negative, see \sectref{sec:10.2.3}). The subject of what would be a verbal clause in a declarative speech act cannot be questioned using the SVO verbal clause construction in Moloko, because the clause can never be simply rearranged so that the subject is clause-final. For example, it is impossible to question the subject in \REF{ex:10:72} using the SVO verbal clause construction.\footnote{Unless the emphatic question construction is used \sectref{sec:10.3.5}.} 

\ea \label{ex:10:72}
Hor  ede  ɗaf.\\
\gll  hʷɔr  ɛ-d-ɛ    ɗaf\\
      woman  \oldstylenums{3}\textsc{s}-make-{\CL}  {millet loaf}\\
\glt  ‘The woman is making millet loaf.’
\z

To question the subject (\ref{ex:10:73}–\ref{ex:10:74}), the verbal clause must be reformed into a predicate nominal interrogative construction. The clause is reformed into a noun phrase with a relative clause so that the interrogative pronoun questioning the subject can be in clause-final position. 

\ea \label{ex:10:73}
Hor  amədəye  ɗaf  na  \textbf{way}?\\
\gll  hʷɔr    amɪ-d-ijɛ  ɗaf   na  \textbf{waj}\\
      woman  {\DEP}-make-{\CL}  {millet loaf}  {\PSP}  who\\
\glt  ‘Who is making millet loaf?’ (lit. the woman that is making millet loaf [is] who?)
\z

\ea \label{ex:10:74}
 Hor  amədəye  ɗaf  na  \textbf{weley}?\\
\gll  hʷɔr    amə-d-ijɛ  ɗaf  na  \textbf{wɛlɛj}\\
      woman  {\DEP}-make-{\CL}  {millet loaf}  {\PSP}  which\\
\glt  ‘Which woman is making millet loaf?’ (lit. the woman that is making millet loaf [is] which one?)
\z

\REF{ex:10:75} and \REF{ex:10:77} show two other predicate nominal clauses that question what would be the subject of an otherwise verbal clause. \REF{ex:10:76} and \REF{ex:10:78} are possible responses to these questions. 

\ea \label{ex:10:75}
Məze  amanday  aməzəme  ɗaf  na  \textbf{way}?\\
\gll  mɪʒɛ   ama-ndaj   amɪ-ʒum-ɛ  ɗaf    na   \textbf{waj}\\
      person  {\DEP}-{\PRG}  {\DEP}-eat-{\CL}  {millet loaf}  {\PSP}  who\\
\glt  ‘Who is eating loaf?’ (lit. the man that is eating millet loaf [is] who?)
\z

\ea \label{ex:10:76}
Mana  anday  ozom  ɗaf. \\
\gll Mana   a-ndaj     a-zɔm    ɗaf \\
      person  \oldstylenums{3}\textsc{s}-{\PRG}    \oldstylenums{3}\textsc{s}-eat    {millet loaf}\\
\glt  ‘Mana is eating millet loaf.’ 
\z

\ea \label{ex:10:77}
Aməzəɗe  dəray  na  \textbf{way}?\\
\gll  amɪ-ʒɪɗ-ɛ    dəraj    na   \textbf{waj}\\
      {\DEP}-take-{\CL}  head  {\PSP}  who\\
\glt  ‘Who will win?’ (lit. the one that takes the head [is] who?)
\z

\ea \label{ex:10:78}
Mana  azaɗ  dəray.\\
\gll  Mana  a-zaɗ  dəraj\\
      Mana  \oldstylenums{3}\textsc{s}-take  head\\
\glt  ‘Mana won.’ (lit. Mana took head)
\z

The third structure for content information questions uses a right-shifted \textit{na}{}-marked element (see \sectref{sec:11.3}). This structure is employed in cases where it is impossible for a questioned verb phrase element to be clause-final. In \REF{ex:10:79}, the direct object is questioned. In this case the direct object cannot be clause-final since it is necessary to include the information \textit{ana məlama ahan}\textit{ }‘to his brother,’ and the prepositional phrase must follow the direct object in the verb phrase (\chapref{chap:8}).  Thus in the interrogative structure, the interrogative pronoun replaces the direct object and the rest of the clause is put into a post-posed \textit{na}{}-marked phrase (underlined in this example). A possible response is shown in \REF{ex:10:80}.

\ea \label{ex:10:79}
Mala  avəlan  \textbf{almay} \underline{ana  məlama  ahan  na}?\\
\gll  Mala  a-vəl=aŋ    \textbf{almaj} \ulp{ana}{~}  \ulp{məlama}{~}\ulp{=ahaŋ}{~~~~} \ule{na}\\
      Mala  \oldstylenums{3}\textsc{s}-give=\oldstylenums{3}\textsc{s}.{\IO}    what  {\DAT} sibling=\oldstylenums{3}\textsc{s}.{\POSS}  {\PSP}\\
\glt  ‘Mala gave what to his brother?’
\z

\ea \label{ex:10:80}
Mala  avəlan  dala  ana  məlama  ahan.\\
\gll  Mala  a-vəl=aŋ    dala ana  məlama=ahaŋ\\
      Mala  \oldstylenums{3}\textsc{s}-give=\oldstylenums{3}\textsc{s}.{\IO}    money  {\DAT} sibling=\oldstylenums{3}\textsc{s}.{\POSS}\\
\glt  ‘Mala gave money to his brother.’
\z

\subsection{Yes-No question construction}\label{sec:10.3.2}
\hypertarget{RefHeading1213021525720847}{}
Yes/no questions are interrogative clauses which can be answered by a simple ‘yes’ or ‘no’ – they are not asking for content in the reply. Moloko uses the interrogative marker \textit{ɗaw} at the end of what is otherwise a declarative clause to create yes/no interrogatives. Pure yes-no questions can be answered with either yes or no, but in Moloko there is often a degree of expectation to the question.\footnote{Expectation is a central element in understanding Moloko grammar (see \sectref{sec:7.4.3}), as is what constitutes shared information with the hearer (see \chapref{chap:11}). Questions are constructed in Moloko with that knowledge and expectation in mind, even when seeking new information. Tag questions are discussed in \sectref{sec:10.3.3}.} When a speaker asks a yes/no question (\ref{ex:10:81}–\ref{ex:10:83}), they are usually expecting an affirmative reply. 

\ea \label{ex:10:81}
Zar na  ndahan  baba  a  Mala  \textbf{ɗaw}?\\
\gll  zar   na   ndahaŋ   baba   a   Mala   \textbf{ɗaw}\\
      man    {\PSP}  \oldstylenums{3}\textsc{s}  father  {\GEN}  Mala  {\textsc{q}}\\
\glt  ‘That man, is he Mala’s father?’
\z

In \REF{ex:10:82}, the speaker expects that Mana is on his way; he is asking for confirmation (but a negative response is always possible). Likewise in \REF{ex:10:83}, he expects that the referent \textit{zar ango} ‘your husband’ is well.

\ea \label{ex:10:82}
Mana  na  álala  \textbf{ɗaw}?\\
\gll  Mana   na  á-l=ala    \textbf{ɗaw}\\
      Mana  {\PSP}  \oldstylenums{3}\textsc{s}+IPV-go=to  {\textsc{q}}\\
\glt  'Mana, is he coming?'
\z

\clearpage
\ea \label{ex:10:83}
Zar  ango  ndahan  aba  \textbf{ɗaw}?\\
\gll  zar=aŋgʷɔ    ndahaŋ  aba   \textbf{ɗaw}\\
      man={\twoS}.{\POSS}  \oldstylenums{3}\textsc{s}  {\EXT}  {\textsc{q}}\\
\glt  ‘Is your husband well?’ (part of a greeting; lit. your husband, does he exist?) 
\z

There is often an even stronger affirmative expectation when the question is negated. Compare the positive and negative pairs of questions (\ref{ex:10:84}–\ref{ex:10:89}). Some of the negated questions can be used rhetorically (see \sectref{sec:10.3.4}), since the speaker already knows that the answer is yes. In the examples, the interrogative markers and the negative particles are bolded.

\ea \label{ex:10:84}
Baba  ango,  ndahan  ava  a  mogom  \textbf{ɗaw}?\\
\gll  baba=aŋgʷɔ    ndahaŋ   ava  a  mɔgʷɔm    \textbf{ɗaw}\\
      father={\twoS}.{\POSS}  \oldstylenums{3}\textsc{s}    {\EXT}+in  at  home    {\textsc{q}}\\
\glt  ‘Is your father in?’
\z

\ea \label{ex:10:85}
Baba  ango,  ndahan  ava  a  mogom  \textbf{bay}  \textbf{ɗaw}?\\
\gll  baba=aŋgʷɔ    ndahaŋ ava  a  mɔgʷɔm   \textbf{baj}  \textbf{ɗaw}\\
      father={\twoS}.{\POSS}  \oldstylenums{3}\textsc{s}    {\EXT}+in  at  home    {\NEG}  {\textsc{q}}\\
\glt  ‘Is your father not in?’
\z

\ea \label{ex:10:86}
Ólo  a  kosoko  ava  \textbf{ɗaw}?\\
\gll  \'{ɔ}-lɔ    a  kɔsɔkʷɔ  ava  \textbf{ɗaw}\\
      \oldstylenums{3}\textsc{s}+{\IFV}-go  at  market  in  {\textsc{q}}\\
\glt  ‘Is he going to the market?’
\z

\ea \label{ex:10:87}
Ólo  a  kosoko  ava  \textbf{bay}  \textbf{ɗaw}?\\
\gll  \'{ɔ}-lɔ    a  kɔsɔkʷɔ  ava  \textbf{baj}  \textbf{ɗaw}\\
      \oldstylenums{3}\textsc{s}+{\IFV}-go  at  market  in  {\NEG}  {\textsc{q}}\\
\glt  ‘Is he not going to the market?’
\z

\ea \label{ex:10:88}
Məlama  ango  álala  \textbf{ɗaw}?\\
\gll  məlama=aŋgʷɔ     á-l=ala     \textbf{ɗaw}\\
      sibling={\twoS}.{\POSS}  \oldstylenums{3}\textsc{s}+{\IFV}-go=to  {\textsc{q}}\\
\glt  ‘Is your brother coming?’
\z

\largerpage
\ea \label{ex:10:89}
Məlama  ango  álala  \textbf{bay}  \textbf{ɗaw}?\\
\gll  məlama=aŋgʷɔ     á-l=ala     \textbf{baj}  \textbf{ɗaw}\\
      sibling={\twoS}.{\POSS}  \oldstylenums{3}\textsc{s}+{\IFV}-go=to  {\NEG}  {\textsc{q}}\\
\glt  ‘Is your brother not coming?’
\z  

As is the case for the negation construction (see \sectref{sec:10.2.3}), it could be that the entire proposition in the clause is being questioned. However, it is often the case that only the final constituent is being questioned. Often the clause is restructured when a constituent of the clause is questioned so that the constituent is in final position. In \REF{ex:10:90} the direct object is fronted and marked as presupposed (it is the topic of discussion) so that the other elements in the clause are questioned (see \sectref{sec:10.3.2}). See also \REF{ex:10:82} where the subject is marked as presupposed and it is whether or not he is coming that is being questioned.

\ea \label{ex:10:90}
Awak  ango  na,  káaslay  na  \textbf{ɗaw}?\\
\gll  awak=aŋgʷɔ    na  káá-ɬ{}-aj    na  \textbf{ɗaw}\\
      goat={\twoS}.{\POSS}  {\PSP}  {\twoS}+{\POT}-slay{}-{\CL}  \oldstylenums{3}\textsc{s}.{\DO}  {\textsc{q}}\\
\glt  ‘Your goat, are you going to slaughter it?’
\z 

\subsection{Tag question construction}\label{sec:10.3.3}
\hypertarget{RefHeading1213041525720847}{}
Question tags can be attached at the end of what would otherwise be the construction used for a declarative clause to seek confirmation of a particular statement. In Moloko, a question tag is \textit{kəyga bay ɗaw} ‘is that not so?’ The affirmative response is \textit{kəyga} ‘it is so.' The negative response is \textit{kəyga bay }‘it is not so’ with a statement to explain why the negative answer. Some rhetorical questions have a special question tag \textit{esəmey} ‘isn’t that so’ (see \sectref{sec:10.3.4}). In the examples below, what is under the scope of questioning is put in square brackets. 

\ea \label{ex:10:91}
[Kolo  a  Marva  hajan]  kəyga  bay  daw?   \hspace{65pt}      \\
\gll  {}[k\'{ɔ}-lɔ   a   Marva  hadzaŋ]   \textbf{kijga}    \textbf{baj}   \textbf{ɗaw} \hspace{5pt}  \\
      {\twoS}+{\IFV}-go  at  Maroua  tomorrow  {like that}    {\NEG}  {\textsc{q}}  {} \\
\glt  ‘You are going to Maroua tomorrow, not so?’  \hspace{35pt}         
\z 

\ea \label{ex:10:92}
{}[Apazan  kolo  a  kosoko  ava]  \textbf{kəyga  bay  ɗaw?}\\
\gll  [apazaŋ  k\`{ɔ}-lɔ  a  kɔsɔkʷɔ  ava]  \textbf{kijga}  \textbf{baj}  \textbf{ɗaw}\\
      yesterday  {\twoS}+{\PFV}-go  at  market  in  {like that}  {\NEG}  {\textsc{q}}\\
\glt  ‘You went to the market yesterday, right?’
\z 

\ea \label{ex:10:93}
Nə  alməmar  na,  [avar  abay]  \textbf{kəyga}  \textbf{bay}  \textbf{ɗaw?}\\
\gll  nə    alməmar   na  [avar   abaj]     \textbf{kijga}   \textbf{baj}   \textbf{ɗaw}\\
      with      {dry season}  {\PSP}  rain  {\EXT}+{\NEG}  {like this}  {\NEG}  {\textsc{q}}\\ 
\glt  ‘In dry season, there is no rain, right?’  
\z 

Other question tags are evaluative. Example \REF{ex:10:94} is a question tag asked in a context where the speaker is examining something physically (perhaps at the market as he is considering to buy it) or analysing and evaluating an event. 

\ea \label{ex:10:94}
{}[Səlom ga]  \textbf{ɗaw}?\\
\gll  [sʊlɔm  ga] \textbf{ɗaw}\\
      goodness  {\ADJ}  {\textsc{q}}\\
\glt  ‘[Is that] good?’              
\z 

\subsection{Rhetorical question construction}\label{sec:10.3.4}
\hypertarget{RefHeading1213061525720847}{}
In a rhetorical question, the speaker is not pragmatically asking for information. Rather, the questions can be evaluative, may carry an element of reproach, or may be a mild command. The context gives the rhetorical force. Some rhetorical questions have a special emphatic structure (see \sectref{sec:10.3.5}) but many have the normal interrogative structure for a content question (\ref{ex:10:95}–\ref{ex:10:96}, see \sectref{sec:10.3.1}). For example, the speaker is not seeking an explanation when he asks \textit{kamay} ‘why’ in \REF{ex:10:95}. More probably he is making a strong statement, ‘the people had no reason to do this bad thing to me.’ Likewise in \REF{ex:10:96}, the speaker is saying that the listener will listen to no one. 

\ea \label{ex:10:95}
Məze  ahay  tagaw  ele  lala  bay  \textbf{kamay}?\\
\gll  mɪʒɛ=ahaj  ta-g=aw  ɛlɛ  lala  baj  \textbf{kamaj}\\
      person=Pl  \oldstylenums{3}\textsc{p}-do={\oneS}.{\IO}  thing  good  {\NEG}  why\\
\glt  ‘The people had no reason to do this bad thing to me.’ (lit. the people did the bad thing to me why?)
\z 

\ea \label{ex:10:96}
\corpussource{Values, 29}\\
Hərmbəlom  na,  amaɗaslava  ala  məze  na,  ndahan  ese  na,\\   
\gll  Hʊrmbʊlɔm  na  ama-ɗaɬ=ava=ala  mɪʒɛ   na  ndahaŋ  ɛʃɛ na \\ 
      God  {\PSP}    {\DEP}-multiply=in=to   person   {\PSP}     \oldstylenums{3}\textsc{s}     again    {\PSP} \\     
\glt ‘God, the one who multiplied the people, him again,’\\
      
      \medskip
kagas  ma  Hərmbəlom  na,  asabay  na,\\       
\gll ka-gas ma Hʊrmbʊlɔm na asa-baj na \\ 
     {\twoS}-catch   word     God             {\PSP}  again-{\NEG}  {\PSP} \\ 
\glt ‘[if] you no longer accept the word of God,’\\ 

\clearpage
     \medskip
káagas  na  anga  \textbf{way}?\\
\gll káá-gas na aŋga \textbf{waj}\\
     {\twoS}+{\POT}-catch  {\PSP}  {\POSS} who\\
\glt  ‘you won't listen to anyone.' (lit. ‘you will catch it [word] of whom?’) 
\z 

Other rhetorical questions have the same structure as a tag question (\ref{ex:10:97}–\ref{ex:10:98}, see \sectref{sec:10.3.3}). However either there is no expected answer or the expected answer is the opposite of that for a normal yes/no tag question. For example, during the telling of the text from which \REF{ex:10:97} is taken, when the storyteller asked the rhetorical question \textit{lala  ɗaw} ‘[is that] good?’ the people in the audience replied \textit{lala bay} ‘[it is] not good.’ (even though the answer was obvious from the story). Likewise, in \REF{ex:10:98}, the audience replied \textit{səlom ga} ‘[it is] good’ to the rhetorical question \textit{səlom ga bay ɗaw} ‘[is that] not good?’

\ea \label{ex:10:97}
Kólo  kagas  anga  məze  kək,  lala  \textbf{ɗaw}?\\
\gll  k\'{ɔ}-lɔ  kà-gas    aŋga  mɪʒɛ  kək      lala  \textbf{ɗaw}\\
      {\twoS}+{\IFV}-go  {\twoS}+{\PFV}-catch  {\POSS}  person  {\textsc{id}:catch by throat}    good  {\textsc{q}}\\
\glt  ‘[If] you catch [something] belonging to someone else [and steal it], [is that] good?’
\z 

\ea \label{ex:10:98}
Kólo  ele  ango,  səlom  ga  bay  \textbf{ɗaw}?\\
\gll  k\'{ɔ}-lɔ  ɛlɛ=aŋgʷɔ    sʊlɔm  ga  baj  \textbf{ɗaw}\\
      {\twoS}+{\IFV}-go  thing={\twoS}.{\POSS}  good  {\ADJ}  {\NEG}  {\textsc{q}}\\
\glt  ‘[If] you mind your own business (lit. go to your things), [is that] not good?’
\z 

A particular question tag, \textit{esəmey} ‘isn’t that so’ carries an element of reproach. There is no expected answer to the question in \REF{ex:10:99}. The message is a strong declaration that the speaker had already told something to the hearer. 

\ea \label{ex:10:99}
[Nahok ma  fan] \textbf{esəmey}?\\
\gll  [nà-h=ɔkʷ     ma  faŋ]  \textbf{ɛʃɪmɛj}\\
      {\oneS}+{\PFV}-tell={\twoS}.{\IO}  word  already  {isn’t that so}\\
\glt  ‘I already told you, didn’t I?’
\z 

\subsection{Emphatic question construction}\label{sec:10.3.5}
\hypertarget{RefHeading1213081525720847}{}
Emphatic questions do not ask for information, but rather make an emphatic statement or carry imperatival force. As such they are a sub-type of rhetorical questions (see \sectref{sec:10.3.4}). The emphatic question construction uses two interrogative pronouns, a reduced emphatic pronoun within the clause in the normal slot for the element questioned, and the other a sometimes reduced pronoun at the end of the clause. 

These reduced interrogative pronouns are \textit{wa} (from \textit{way} ‘who’) in \REF{ex:10:100}, \REF{ex:10:102}, \REF{ex:10:103}, \textit{may} and \textit{alma} (from \textit{almay} ‘what’) in \REF{ex:10:101} and \REF{ex:10:104}, respectively, \textit{malma} (from \textit{malmay} ‘what’) in \REF{ex:10:105}, and \textit{meme} and \textit{mey} (from \textit{memey} ‘how’) in \REF{ex:10:106}.

\ea \label{ex:10:100}
\textbf{Wa}  aməgok  na  \textbf{way}?\\
\gll  \ \ \textbf{wa}    amə-g=ɔk  na  \textbf{waj}\\
      \ \ who    {\DEP}-do={\twoS}.{\IO}  \oldstylenums{3}\textsc{s}.{\DO}  who\\
\glt  \ \ ‘\textit{What} is wrong?’ / ‘Stop crying!’ (lit. who to do it to you, who) 
\z 

\ea \label{ex:10:101}
Kege  \textbf{may} ana  war  ga  \textbf{may}?\\
\gll  ka-gɛ  \textbf{maj} ana  war  ga  \textbf{maj}\\
      {\twoS}-do  what  {\DAT} child  {\ADJ}  what\\
\glt  \textit{‘What} are you doing to the child, \textit{what}?’ / ‘Stop doing that!’
\z 

\ea \label{ex:10:102}
\corpussource{Cicada, S. 18}\\
Náanjakay  na  \textbf{wa}  [amazaw  ala  agwazla  ana  ne   na]  \textbf{way}?\\
\gll  náá-nzak-aj     na  \textbf{wa}    [ama-z=aw   =ala  agʷaɮa    ana   nɛ    na] \\    
      {\oneS}+{\POT}{}-find{}-{\CL} {\PSP} who   {\DEP}-take={\oneS}.{\IO}   =to    {spp. of tree}     {\DAT} {\oneS}    {\PSP}\\   
      
      \medskip
\gll \textbf{waj}\\
     who\\ 
\glt  ‘\textit{Who} can I find to bring to me this tree for me?  \textit{Who}?’ / ‘\textit{Someone} should be able to bring me this tree.’
\z 

\ea \label{ex:10:103}
\textbf{Wa}  andaɗay  \textbf{way}?\\
\gll  \textbf{wa}    a-ndaɗ-aj   \textbf{waj}\\
      who    \oldstylenums{3}\textsc{s}-love{}-{\CL}  who\\
\glt  ‘Who loves whom?' / ‘No one loves him.'
\z 

\ea \label{ex:10:104}
\textbf{Alma}  amədəvala  okfom  na  \textbf{may}?\\
\gll  \textbf{alma}  amə-dəv=ala    ɔkʷfɔm  na  \textbf{maj}\\
      what  {\DEP}-trip=to    mouse  {\PSP}  what\\
\glt  ‘\textit{What} was it that made that mouse fall? \textit{What}?' / ‘What else [but a snake] makes a mouse fall?’
\z 

\clearpage
\ea \label{ex:10:105}
\textbf{Malma}  awəlok{  }\textbf{may}?\\
\gll  \textbf{malma}   a-wəl=ɔkʷ   \textbf{maj}\\
      what  \oldstylenums{3}\textsc{s}-hurt={\twoS}.{\IO}  what\\
\glt  ‘\textit{What} is bothering (hurting) you? \textit{What}?’ / ‘\textit{Nothing} should be bothering you.’
\z 

\ea \label{ex:10:106}
\textbf{Meme}  ege  \textbf{mey}?\\
\gll  \textbf{mɛmɛ}   ɛ{}-g-ɛ     \textbf{mɛj}\\
      how    \oldstylenums{3}\textsc{s}-do-{\CL}  how?\\
\glt  ‘\textit{What} happened?’ / ‘Why did you do that?’ / ‘Stop the foolishness.’ (lit. how did it do?)
\z 

\section{Imperative constructions}\label{sec:10.4}
\hypertarget{RefHeading1213101525720847}{}
There are several types of imperative constructions in Moloko, which are used in different situations, sometimes to express different degrees of obligation.  So far six different constructions have been identified, each with a different force of exhortation. They are shown in \tabref{tab:83}. Some constructions use the imperative mood form of the verb (see \sectref{sec:7.2}), others use Imperfective aspect\is{Tense, mood, and aspect!Imperfective aspect|(} or irrealis mood\is{Tense, mood, and aspect!Irrealis mood|(} or are in the form of a rhetorical question (see \sectref{sec:10.3.4}). \tabref{tab:83} illustrates all of the imperative constructions for the verb /\textit{lo}/ ‘go.’ The verb forms are also shown in Perfective and Imperfective aspect (lines 1 and 2) for comparison.

\begin{table}
\begin{tabularx}{\textwidth}{lXp{4cm}p{4cm}}
\lsptoprule
\multicolumn{2}{l}{Line}                & {{\twoS} forms}      & {\oldstylenums{3}\textsc{s} forms}\\\midrule
{1} & {Declarative, Perfective aspect} & \textit{ka-l=ala }\newline{\twoS}+{\PFV}-go=to\newline‘You came.’  & \textit{a-l=ala}\newline\oldstylenums{3}\textsc{s}+{\PFV}-go=to \newline ‘He/she came.’\\\midrule
{2} & {Declarative, Imperfective aspect} & \textit{ká-l=ala  }\newline {\twoS}+{\IFV}-go=to\newline ‘You come.’  & \textit{á-l=ala  }\newline \oldstylenums{3}\textsc{s}+{\IFV}-go=to\newline ‘He/she comes.’\\\midrule
{3} & {Imperative} & \textit{l=ala} \newline go[{\twoS}.{\IMP}]=to  \newline ‘Come (now)!’  \\\midrule
{4} & {Polite request} & \textit{ká-l=ala} \textit{ete  ɗaw}\newline \mbox{{\twoS}+{\IFV}-go=to polite \textsc{q}} \newline ‘Please come.’  \\\midrule
{5} & {Negative expectation} & \textit{ká-l=ala} \textit{bay} \newline {\twoS}+{\IFV}-go=to  {\NEG} \newline ‘Don’t come.’ \newline (I don’t expect you to come)  & \textit{á-l=ala    bay}\newline \oldstylenums{3}\textsc{s}+{\IFV}-go=to  {\NEG}\newline ‘He/she is not coming.’\newline (I don’t expect him to come) \\\midrule
{6} & {Hortative\is{Tense, mood, and aspect!Irrealis mood|)}} & \textit{kaa-l=ala} \newline {\twoS}+{\HOR}-go=to \newline  ‘You come now!’\newline (I want you to come) & \textit{m{ə}-l=ala}\newline \oldstylenums{3}\textsc{s}+{\HOR}-go=to \newline ‘He/she should come.’\newline (I want him to come)  \\\midrule
{7} & {Adverb of obligation} & \textit{səy} \textit{  k{ə}-l=ala=va }\newline only    {\twoS}+{\PFV}-go=to={\PRF}\newline ‘You must come.’ & \textit{səy  m{ə}-l=ala }\newline only     \oldstylenums{3}\textsc{s}+{\HOR}-go=to\newline ‘He/she must come.’\\\midrule
{8} & {Rhetorical question} & \textit{ká-l=ala  bay  ɗaw}\newline {\twoS}+{\IFV}-go=to  {\NEG} \textsc{q}\newline ‘You should come.’\newline (lit. Are you not coming?) & \textit{á-l=ala    bay   ɗaw}\newline  \oldstylenums{3}\textsc{s}+{\IFV}-go=to  {\NEG} \textsc{q}\newline ‘He should come.’\newline (lit. Is he not coming?)\\\lspbottomrule
\end{tabularx}
\caption{Imperative constructions \label{tab:83}}
\end{table}

The imperative form of the verb is used for an immediate command (\ref{ex:10:107}--\ref{ex:10:109}, line 3 of \tabref{tab:83}). The verb is in the imperative mood (see \sectref{sec:7.2}) and can be preceded by a vocative. The addressee is expected to carry out the order in the immediate future as opposed to commands that demand reflection before carrying them out. In hortatory texts, imperatives are not usually found in the body of the exhortation since the hearer is expected to wait until the discourse is finished before carrying out the instructions.  


\ea \label{ex:10:107}
Lohom  a  mogom.\\
\gll  lɔhʷ-ɔm  a  mɔgʷɔm\\
      go-{\twoP}  at  home\\
\glt  ‘Go home!’
\z 

\clearpage
\ea \label{ex:10:108}
Zəmok  ɗaf.\\
\gll  \ zʊm-ɔkʷ  ɗaf\\
      eat-\oldstylenums{1}\textsc{Pin} {millet loaf}\\
\glt  ‘Let’s eat!’
\z 

\ea \label{ex:10:109}
Cəke.\\
\gll  tʃɪk-ɛ\\
      stand[{\twoS}.{\IMP}]-{\CL}\\
\glt  ‘Stand up!’
\z 

The word \textit{etey} or \textit{ete} ‘please’ can be added to other clause types (\ref{ex:10:110}--\ref{ex:10:111}, line 5 in \tabref{tab:83}) to achieve a milder pragmatic imperative force than the use of the construction without the polite adverb.

\ea \label{ex:10:110}
Nde na  asaw  na,  gaw  na \textbf{etey}?\\
\gll  ndɛ  na   a-s=aw     na   g=aw     na   \textbf{ɛtɛj}\\
      so  {\PSP}   \oldstylenums{3}\textsc{s}-please={\oneS}.{\IO} {\PSP}   do={\oneS}.{\IO}  \oldstylenums{3}\textsc{s}.{\DO}  please\\
\glt  ‘So I want that you do that for me, please.’
\z 

\ea \label{ex:10:111}
N\'{ə}njakay  yam \textbf{ete} ɗaw?\\
\gll  n\'{ə}-nzak-aj    jam  \textbf{ɛtɛ}  ɗaw\\
      {\oneS}+{\IFV}-find{}-{\CL}  water  please  {\textsc{q}}\\
\glt  ‘Could you please get me some water?’ (lit. can I find water please)
\z 

A negated clause in the Imperfective aspect\is{Tense, mood, and aspect!Imperfective aspect|)} expresses a negative exhortation or statement of expectation (\ref{ex:10:112}--\ref{ex:10:113}, line 5 in \tabref{tab:83}). In second person \REF{ex:10:112}, the negative expectation carries a weak hortative force. The speaker is expressing that he/she expects the addressee not to carry out the action. In third person \REF{ex:10:113} the negative expectation is not hortatory, but rather simply expresses that the speaker does not expect that the action will be performed. 

\ea \label{ex:10:112}
Kámənjar  fabay.\\
\gll  ká-mənz\={a}r     fá-bàj\\
      {\twoS}+{\IFV}-see    already-{\NEG}\\
\glt  ‘Don’t look at it yet.’ (I don’t expect you to look at it).
\z 

\ea \label{ex:10:113}
Á-mənjar     fabay.\\
\gll  á-mənz\={a}r     fá-bàj\\
      \oldstylenums{3}\textsc{s}+{\IFV}{}-see    already-{\NEG}\\
\glt  ‘I don’t think he looked at it.’ (I don’t expect that he looked at it).
\z 

A clause with a verb in the Hortative mood\is{Tense, mood, and aspect!Irrealis mood} (line 6 in \tabref{tab:83}, see \sectref{sec:7.4.3}) concentrates on the will of the speaker -- the speaker wishes the action done. This form is illustrated for \oldstylenums{3}\textsc{s} in \REF{ex:10:114}. 

\ea \label{ex:10:114}
Mamənjar  fabay. \\
\gll  mà-mənz\={a}r     fá-bàj\\
      {\twoS}+{\HOR}-see    already-{\NEG}\\
\glt  ‘He/she shouldn’t look at it yet.’ / ‘Don’t let him/her look at it.’ (I don’t expect him/her to look at it).
\z 

An even stronger deontic form is made by the addition of an adverb of obligation (\textit{dewele} ‘obligation’ \REF{ex:10:116}, \textit{səy} ‘only’ \ref{ex:10:115}–\ref{ex:10:117}) preceding the clause, with the verb in Hortative mood (line 7 in \tabref{tab:83}). Imperative forms with an adverb of obligation indicate that the hearer is obligated to do something (he/she has no choice, there is no other way). These forms are used to give an order with insistence, a strong counsel.

\ea \label{ex:10:115}
Səy  koogom  endeɓ.\\
\gll  sij     k\`{ɔ}\`{ɔ}-gʷ-ɔm   ɛndɛɓ\\
      only    {\twoP}-do-{\twoP}   wisdom\\
\glt  ‘You must be wise (lit. do only wisdom).’
\z 

\ea \label{ex:10:116}
Dewele  səy  keege  na.\\
\gll  dɛwɛlɛ   sij   k\`{ɛ}\`{ɛ}-gɛ     na\\
      obligation  only   {\twoS}+{\HOR}-do   \oldstylenums{3}\textsc{s}.{\DO}\\
\glt  ‘You are obligated to do that.’ (lit. obligation: you must only do it)
\z 

\ea \label{ex:10:117}
Səy  keege  anga  dewele.\\
\gll  sij     k\`{ɛ}\`{ɛ}-g-ɛ     aŋga   dɛwɛlɛ\\
      only   {\twoS}+{\HOR}-do-{\CL}   {\POSS}   obligation\\
\glt  ‘You must do that obligation.’ (lit. you must only do the thing that belongs to obligation)
\z 

\section{Exclamatory constructions}\label{sec:10.5}
\hypertarget{RefHeading1213121525720847}{}
Exclamatory sentences have either an interjection at the initial position \REF{ex:10:118} or one of several exclamatory adverbs at the final position (\ref{ex:10:119}–\ref{ex:10:122}). In the examples, the interjections and exclamatory adverbs are bolded. 

\ea \label{ex:10:118}
\textbf{Kay},  nege  na  bay!\\
\gll  \textbf{kaj}    n\`{ɛ}-g-ɛ    na  baj\\
      interj.  {\oneS}+{\PFV}-do-{\CL}  \oldstylenums{3}\textsc{s}.{\DO}  {\NEG}\\
\glt  ‘No, I didn’t do it!’
\z  

\ea \label{ex:10:119}
Apazan  nok  awəy  Məwsa  álala;\\  
\gll  apazaŋ   nɔkʷ  awij  Muwsa  á-l=ala \\ 
      yesterday  {\twoS}  said  Moses  \oldstylenums{3}\textsc{s}+{\IFV}-go=to \\     
\glt ‘Yesterday you said that Moses would come;’      
      
   \medskip
macakəmbay  aməlala  na  ndahan  bay  \textbf{nəy}!\\   
\gll matsakəmbaj amə-l=ala  na ndahaŋ baj \textbf{nij}\\
     meanwhile    {\DEP}-go=to  {\PSP}  \oldstylenums{3}\textsc{s}  {\NEG}  exclamation\\
\glt  ‘but the one that came was not him after all!’
\z 

\ea \label{ex:10:120}
Enje  bay  ɗeɗen  \textbf{dey}!\\
\gll  \`{ɛ}{}-nʒ-ɛ    baj  ɗɛɗɛŋ  \textbf{dɛj}\\
      \oldstylenums{3}\textsc{s}+{\PFV}-suffice-{\CL}  {\NEG}  truth  exclamation\\
\glt  ‘It really wasn’t enough!’
\z  

\ea \label{ex:10:121}
Gaw  endeɓ  \textbf{dey}!\\
\gll  g=aw     ɛndɛɓ   \textbf{dɛj}\\
      do[{\twoS}.{\IMP}]={\oneS}.{\IO}  brain  exclamation\\
\glt  ‘Be careful!’ (lit. do brain for me)
\z 

\ea \label{ex:10:122}
\corpussource{Values, 50}\\
Epele  epele  na me,  Hərmbəlom  anday  agas  ta  a  ahar  ava \textbf{re}!\\
\gll  {ɛpɛlɛ ɛpɛlɛ}   na   mɛ  Hʊrmbʊlɔm   a-ndaj   a-gas         ta   \\  
      {\ID}{in the future}  {\PSP}  opinion    God      \oldstylenums{3}\textsc{s}-{\PROG}  \oldstylenums{3}\textsc{s}-catch    \oldstylenums{3}\textsc{p}.{\DO}  \\   
      
      \medskip
\gll a   ahar  ava \textbf{rɛ}\\
     at  hand  in {in spite}\\
\glt  ‘In the future in my opinion, God is going to accept them [the elders] in his hands, in spite [of what anyone says]!’
\z 
